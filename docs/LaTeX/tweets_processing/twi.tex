\documentclass[11pt]{article}
\usepackage[utf8]{inputenc}
%\usepackage{latin1}
\usepackage[T1]{fontenc} % fuer accent circonflex ueber i, Trennung mit Umlaut
\usepackage{german}
\usepackage{scrpage2}
\usepackage{epsf}
\usepackage{graphicx}
\usepackage{amssymb}
\usepackage{makeidx}
\usepackage{booktabs}
\usepackage{rotating}
\usepackage{tabularx}
%\usepackage{covingtn}
\usepackage{enumitem}
\usepackage{ae}
\usepackage[comma,round]{natbib}
\usepackage{url}
%\usepackage{a4nicer}
%\usepackage{palatino}


%%%%%%%%%%%%%%%%%%%%%%%%%%%%%%%%%%%%%%%

\title{Automatische Verarbeitung deutschsprachiger Tweets:\\ Eine Fallstudie}

%\author{{Manfred Stede\\ Applied Computational
%  Linguistics\\Department of Linguistics\\ University of Potsdam
%  (Germany)\\{\tt stede@uni-potsdam.de}} \and {Kristin Irsig\\ID Berlin}}

\author{}
%Manfred Stede\\ Applied Computational
%  Linguistics\\ University of Potsdam\\ {\tt
%  stede@uni-potsdam.de} \and Kristin Irsig\\ ID Information und
%  Dokumentation\\ im Gesundheitswesen GmbH \&Co. KGAA\\ Berlin\\ {\tt K.Irsig@id-berlin.de}}

\date{}

\pagestyle{empty}


\begin{document}

\maketitle

\thispagestyle{empty}

%\begin{abstract}
%\noindent Abstrakt.
%\end{abstract}



\section{Einführung}

TODO: DIE BESONDERHEITEN DES TWEET PROCESSING
\begin{itemize}
  \item micro-blogging: max. 140 Zeichen (Schwierigkeit für z.B.
  Sprachidentifikation, Koreferenzerkennung wenn Referent in vielen verstreuten
  tweets auftaucht, Parsing)
  \item Sehr hoher ``noise'': Alle möglichen Arten von Sonderzeichen (user
  mark-up, smilies, emoticons etc.)
  \item sehr niedrige Grammatikalität
  \item Orthographiekonfusion: ``Sprechschrift'', ad-hoc Abkürzungen,
  Buchstaben-Zahl-Wörter (z.B. 2b, b4, r8, sk8er), Prolongationen (guuuuuut),
  Netzjargon
  \item Hashtags, Retweets
  \item \ldots
\end{itemize}

\section{Zielsetzung}

Das Projekt ``Diskurse in Social Media'' untersucht aus
kommunikationswissenschaftlicher Perspektive den Verlauf von
politischen Diskursen in drei verschiedenen Social Media: Twitter,
Facebook und Blogs. Es wird untersucht, inwieweit sich zwischen diesen
drei Medien Unterschiede finden lassen im Hinblick auf
\begin{itemize}
\item den zeitlichen Verlauf von Debatten: Wie entwickelt sich das
  Volumen einer Diskussion?
\item die Dialogizität der Diskurse: Wie gehen Teilnehmer auf die
  Beiträge anderer Teilnehmer ein?
\item die Meinungsführerschaft: Sind bestimmte Akteure in den
  Diskussionen ``tonangebend''?
\item die Themen: Welche Aspekte des Themenkreises werden diskutiert?
\item die Meinung: Welche Stimmungslage kommt in einer Diskussion zum Ausdruck?
\end{itemize}
Die qualitativ hochwertige Analyse dieser Fragestellungen setzt das
sachkundige menschliche Urteil voraus, das heißt: Am Ende der
Bemühungen steht eine Inhaltsanalyse durch ausgebildete Analysten. Die
Projektpartner aus der Wirtschaftsinformatik und der
Computerlinguistik sollen diesen Prozess aber maßgeblich unterstützen,
um bei einem relativ umfangreichen Datensatz die menschliche Analyse
auf die relevanten Beiträge konzentrieren zu können. Als erstes
Arbeitskorpus wurden dazu Daten ausgewählt, die im Jahr 2011 über
einen Zeitraum von zwei Wochen hinweg (TODO: stimmt das? - wenn die
Angabe 12.12-17.02 im Dateinamen des Korpus den Zeitraum des
Textsammelns bedeutet, so d\"urften es gut 2 Monate sein) das
Stichwort `Wulff' beinhalten, also höchstwahrscheinlich Äußerungen zur
Affäre um den seinerzeitigen Bundespräsidenten beinhalten. Im
vorliegenden Beitrag beschränken wir uns auf die Twitter-Daten, das
sind 119 455 einzelne Tweets, von denen wir zunächst 28 818 als
Duplikate idetifiziert haben; somit verbleibt eine Grundmenge von 90
637 zu verarbeitenden Tweets.

Im Projekt sind die Wirtschaftsinformatiker für die Beschaffung der
Datensätze und die Konstruktion von Graphstrukturen zuständig, die die
Verweise zwischen Diskussionsbeiträgen abbilden. Der
Computerlinguistik obliegt die inhaltliche Analyse der einzelnen
Beiträge im Hinblick auf behandelte Themen und die Stimmungslage: Es
wird eine ``Vorsortierung'' der Beiträge durchgeführt, um im Idealfall
die Aufgabe der Analysten auf eine zügige Durchsicht beschränken zu
können. Die Vorsortierung erfolgt anhand dreier Dimensionen:
\begin{itemize}
\item Themenklassifikation: Durch unüberwachte Verfahren wird ein
  Clustering von Beiträgen im Hinblick auf (Unter-) Themen
  vorgenommen.\\ Hier einige Beispiele aus dem Wulff-Korpus, die das
  Adressieren verschiedener Themen illustrieren: TODO
\item Sentimentklassifikation: Für jeden Einzelbeitrag soll erkannt
  werden, ob eine positive, negative oder neutrale Haltung ausgedrückt
  wird. Korpus-Beispiele: TODO\\ Zusätzlich soll im Falle von Verweisen auf andere Beiträge
  festgestellt werden, ob diese zustimmend, kritisch, oder neutral
  ausfallen. Korpusbeispiele: TODO
\item Diskursqualitätklassifikation: Textbeiträge können inhaltlich
  fundiert sein und das Potenzial haben, eine Diskussion fruchtbar
  voranzubringen, oder lediglich kurze ``Einwürfe''
  darstellen. Korpusbeispiele: TODO
\end{itemize}


\section{Die Pipeline zur Vorverarbeitung}

TODO:
\subsection{Sprachidentifikation}

Die von den Wirtschaftsinformatikern zusammengestellten Daten
enthalten auch Tweets, die nicht in deutscher Sprache abgefasst
sind. Um diese zu filtern, wurde mit drei {\em off-the-shelf} Werkzeugen zur Sprachidentifikation
experimentiert. Alle drei Sprachidentifizierer arbeiten auf der Basis von N-Grammen, welche zu je
einem Sprachmodell für jede zu erkennende Sprache zusammengefasst sind und anhand deren das jeweilige Werkzeug versucht die Sprache(n) zu erkennen. \\ 
In folgender Übersicht sind die wichtigsten Eigenschaften der jeweiligen Werkzeuge zusammengefasst:  \\

\underline{TextCat}\footnote{\url{http://odur.let.rug.nl/~vannoord/TextCat/}} \newline
\begin{itemize}
\item Eigentlich Textkategorisierer, lässt sich aber auch zur Sprachidentifikation nutzen
\item Perl Programm (klein, schnell, robust) mit zahlreichen Kommandozeilenoptionen
\item Auf 76 Sprachen trainiert; entsprechende Sprachmodelle mitgeliefert
\item Trainingsdatenset: Usenet newsgroups; in einem Test 98,8\% korrekte Klassifikation
\item Vorgehen: 
\begin{itemize}
\item Tokenisiere Trainingsdaten (verwerfe Zahlen und Interpunktion) 
\item Generiere aus Trainingsdatenset alle möglichen N-Gramme (N=1 bis 5) inkl. Hinzufügen von blanks am Wortanfang und -ende
\item Zähle alle N-Gramme und sortiere sie nach Frequenz in umgekehrter Reihenfolge = N-Gramm-Frequenzprofil
\item Beobachtung: erste 300 N-Gramme spiegeln Sprachspezifika wider
\item Vergleiche Positionen von N-Grammen im Frequenzprofil des Trainingsdatensets und des Dokuments dessen Sprache erraten werden soll ("'out-of-place"'-measure als Distanzmaß) 
\item Geringer Distanzwert lässt auf gleiche Sprache schließen
\end{itemize}
\item von 569 deutschen Tweets aus dem Wulff-Korpus 220 Tweets nicht als Deutsch erkannt \newline
\end{itemize}

\underline{langid.py}\footnote{\url{https://github.com/saffsd/langid.py}} \newline
\begin{itemize}
\item als Kommandozeilentool, Python-Modul oder Web service verwendbar, da einzelne Datei mit  minimalen Abhängigkeiten (somit auch schnell)
\item mitgeliefertes Sprachmodell wurde auf 97 Sprachen aus 5 verschiedenen Domänen trainiert
\item Trainingsdatensets: Regierungsdokumente, Software Dokumentationen, Nachrichten,  Texte aus Online Enzyklopädien und Internet crawl (TODO: Wie das übersetzen?) (= multi-domain language identification corpus, Lui and Baldwin (2011))
\item domänenspezifische Features (z.B. HTML, XML, markdown) beeinflussen Sprachidentifikation nicht
\item  naiver Bayes Klassifizierer mit multinomialem event model; trainiert auf Mischung von N-Grammen $(1\leq n \leq4)$
\item Integration von domänenspezifischen Informationen durch sog. "'LD (=Language Domain) feature selection"'
\item Tokenizierung und Feature Auswahl werden im Eingabedokument mit einem Schritt erledigt (via Aho-Corasick string matching)
\item schlägt TextCat bei 7 verschiedenen Test-Korpora (inkl. Micro-Blogs)  hinsichtlich Exaktheit und Geschwindigkeit
\item von 569 deutschen Tweets aus dem Wulff-Korpus nur 40 Tweets nicht als Deutsch erkannt \newline
\end{itemize}



\underline{guess\_language}\footnote{\url{http://code.google.com/p/guess-language/}} \newline
\begin{itemize}
\item Python-Modul (klein, relativ schnell); basierend auf einer C++-Implementation
\item Erkennt 64 Sprachen
\item Trainingsdatenset: k.A.
\item Sprachmodelle mit jeweils 300 Trigrammen
\item Benutzt Heuristiken basierend auf Zeichensatz und Trigrammen
\item Bei längeren Beispielen höhere Erkennungsrate
\item Keine Behandlung von markup (HTML o.ä. verschlechtert Erkennungsrate)
\item Kaum dokumentiert: Keine Angaben über Trainingsdaten, Tests o.ä., gelegentliche Erweiterungen/ Verbesserungen (letztes Update 16.02.2013)
\item von 569 deutschen Tweets aus dem Wulff-Korpus 367 Tweets nicht als Deutsch erkannt
\end{itemize}


Um die Präzision der Sprachidentifikation der o.g. Tools einzuschätzen, wurde folgenderweise vorgegangen: Alle
Tweets, die einstimmig von allen drei Programmen, als deutsch bzw. nicht-deutsch
interpretiert wurden, sind bewusst außer Betracht gelassen worden, da diese keinen Einfluss auf 
die Entscheidung für die jeweilige Software-Lösung gehabt hätten. Von den übrig
gebliebenen 5962 Tweets wurden 10\%{} (d.h. 596 Tweets) zufällig ausgewählt und die Richtigkeit der
geratenen Sprachen manuell überprüft. Die Ergebnisse dieser
Evaluierung sind in folgender Tabelle unten zusammengefasst: \newline


\begin{table}[h]
\begin{center}
\begin{tabular}{|c|c|c|}
\hline
\multicolumn{3}{|c|}{Korrekte Entscheidungen}\\\hline\hline
%\multicolumn{3}{|c|c|c|}\newline
LangId & TextCat & guess\_language \\\hline
92.97\% & 61.34\% & 35.5\% \\\hline
\end{tabular}
\caption{Evaluierung Sprachidentifikation}
\end{center}
\end{table}



\subsection{Satzgrenzenerkennung}

Die generelle Arbeitsweise eines ``sentence splitters'' besteht darin,
Punkte am Ende eines Wortes dahingehend zu disambiguieren, ob es sich
um einen Satzende-Punkt oder den Bestandteil einer Abkürzung (wie
z.B. in ``Nr. 3''), einer Datumsangabe o.ä. handelt. Bei
Tweet-Processing wurden aßerdem noch zusätzliche zum Teil störende
Faktoren festgestellt, die eine Anpassung der Modulregeln an dieses
Textgenre notwendig machten, wie z.B:
\begin{itemize}
  \item Retweet-Markierungen, die in vielen Fällen in Tweets anstelle eines
    Punktes Sätze voneinander abgrenzen;
  \item Missachtung von Groß- und Kleinschreibungsregeln;
  \item Abgrenzung von Sätzen durch Gedanken- und Schrägstriche.
\end{itemize}

\subsection{Normalisierung und Tokenisierung}
TODO: Normalisierung \newline
Für die Tokenisierung wurde ein in Perl geschriebener Tokenizer verwendet, der
als Vorverarbeitungstool zusammen mit dem Tree-Tagger erhältlich ist.  

\subsubsection{Zu behandelnde Phänomene}

TODO: Klassifizierte Liste von Dingen, die man behandeln muss

\subsubsection{Vorgehen}


\subsection{Part-of-speech Tagging} 

Für die automatische Wortartenbestimmung in den Tweets wurden der
TNT-Tagger\footnote{\url{http://www.coli.uni-saarland.de/~thorsten/tnt/}} und
der
Tree-Tagger\footnote{\url{http://www.ims.uni-stuttgart.de/projekte/corplex/TreeTagger/}}
zur Evaluation herangezogen. Beide Tagger sind probabilistische
Part-of-speech-Tagger, jedoch unterscheiden sie sich in ihrer
genauen Funktionsweise. \newline 
Die allgemeine Funktionsweise von probabilistischen Taggern besteht darin
Folgewahrscheinlichkeiten von Wortarten anhand eines manuell annotierten, 
einsprachigen Korpus zu schätzen (Training) und auf Basis
dieses sog. Sprachmodells Wortarten in Texten automatisch zu bestimmen
(Tagging). Die Trefferquote eines probabilistischen POS-Taggers hängt
im allgemeinen stark von der Qualität und Quantität des annotierten
Korpus und des darauf errechneten Sprachmodells ab. Eine weitere Grundlage für
das erfolgreiche Tagging ist ein sog. Tagset, welches genau vorgibt zu welcher Wortart ein bestimmtes Wort gehört. 
Für die Untersuchung deutschsprachiger Tweets wurde von beiden Taggern das STTS-Tagset 
benutzt\footnote{\url{http://www.ims.uni-stuttgart.de/projekte/corplex/TagSets/stts-table.html}}.
\newline
Der TNT-Tagger (Kurzform von ``Trigrams'n'Tags'') ist ein sehr effizienter POS-Tagger, 
der in einfacher Weise auf beliebige Sprachen und beliebige Tagsets
trainiert werden kann. Im TNT-Tagger ist der Viterbi-Algorithmus für Markov-Modelle zweiter Ordnung
implementiert, welcher dafür verwendet wird
die wahrscheinlichsten Wortartenfolgen (Uni-, Bi-, und Trigramme)
zu berechnen. Die Anzahl der vorkommenden Trigramme reicht aber nicht aus um
Folgewahrscheinlichkeiten aller Wortarten verlässlich zu schätzen, was zu
unerwünschten Ergebnissen beim Tagging führt. Aus diesem Grund werden zusätzlich
Trigrammwahrscheinlichkeiten aus den anderen N-Grammen berechnet. Danach wird
aus diesen kontextuellen Frequenzen die höchste Wahrscheinlichkeit für ein
bestimmtes Trigramm geschätzt, welches dann in das Sprachmodell aufgenommen
wird. Um unbekannte Wörter ebenfalls behandeln zu können,
besitzt der TNT-Tagger eine Routine, die anhand von unterschiedlich langen Wortendungen Tags vergibt. 
Dieses Verfahren beruht auf der Annahme, dass Wortendungen ein starker Indikator
für die jeweilige Wortart sind. Diese lexikalischen und die zuvor erwähnten
kontextuellen Frequenzen bilden das Sprachmodell des TNT-Taggers. Für das
Tagging des  Wulff-Korpus wurde das mitgelieferte deutsche Sprachmodell
verwendet, welches auf dem
NEGRA-corpus\footnote{\url{http://www.coli.uni-saarland.de/projects/sfb378/negra-corpus/}}
trainert wurde. Die Arbeitsweise des Taggers besteht nun im wesentlichen darin,
den zu taggenden Wulff-Korpus mit dem trainierten Sprachmodell abzugleichen und
daraufhin die POS-Tags zu vergeben.  \newline
Die Qualität eines Markov-Modell basierten Taggers hängt maßgeblich von der Größe des Trainingskorpus ab.
Je kleiner das Trainingskorpus, desto geringer ist die Trefferquote des Taggers. Ein größeres Korpus erhöht 
jedoch die Zeit, die für das Training aufgewendet werden muss erheblich. Sehr große Korpora sind außerdem 
mit einem enormen Arbeitsaufwand für die Erstellung und Annotation verbunden. 
Zudem ist bei sehr großen Korpora nicht zwangsläufig garantiert, dass sehr
seltene Wortartenfolgen später richtig getaggt werden. \newline
Der Tree-Tagger vermeidet diese Probleme indem er zum Schätzen
der Folgewahrscheinlichkeiten sog. binäre Entscheidungsbäume benutzt. Diese
Entscheidungsbäume werden aus einer Menge von Trigrammen konstruiert und
besitzen an ihren Blättern eine Tabelle mit Wahrscheinlichkeiten für bestimmte
Wortarten. Diese Methode führt (insbesondere bei kleineren Trainingssets) zu
einer höheren Genauigkeit des Taggers. Des Weiteren arbeitet der Tree-Tagger mit
diversen eingebauten Lexika (Vollformlexikon, Abkürzungslexikon [, Präfix- u. Suffixlexikon???]), die es
ermöglichen Lemma-Informationen auszugeben. \newline
Zusätzlich bietet der Tree-Tagger den Vorteil eines eingebauten
Tokenizers, welcher zum tokenisieren des Wulff-Korpus verwendet wurde.
Somit entfiel eine Evaluation eines zusätzlichen Tokenizers. Auf dem Output des Tree-Tagger internen Tokenizers
wurden beide Tagger getestet. Für die anschließende Evaluation wurden aus
allen Tweets 1000 Sätze zufällig ausgewählt (dies entspricht 19649
Token) und getaggt. Danach wurde die Korrektheit der vergebenen Tags manuell
überprüft. Die Ergebnisse waren in ca. 83\% der Fälle kongruent, lediglich in
ca. 17\% (dies entspricht 3338 Token) der Fälle wurden unterschiedliche Tags
zugewiesen. Das Ergebnis der Analyse der unterschiedlich getaggten Token ist in unten stehender Tabelle
zusammengefasst. \newline



%In our evaluation experiments we compared TreeTagger and TNT by randomly
% choosing 1000 tweets from Twitter Wulff corpus. These tweets were subsequently run through the taggers pre-processed with same tokenizer (TreeTagger's Perl Tokenizer). Taggers showed difference in 3351 from input 19649 words (what roughly made 17\% of total number of tokens). The results of evaluation are depicted in the table below:
\begin{table}[h]
\begin{center}
\begin{tabular}{|*{6}{p{0.15\textwidth}|}}\hline
\multicolumn{6}{|c|}{Tagger-Vergleich}\\\hline\hline
\multicolumn{3}{|c|}{Tree Tagger}&\multicolumn{3}{c|}{TNT Tagger}\\\hline
von Tree-Tagger gewählter besserer Tag & von TNT-Tagger gewählter schlechterer
Tag & Anzahl & von TNT-Tagger gewählter besserer Tag & von Tree-Tagger
gewählter schlechterer Tag & Anzahl\\\hline 
NE & NN & 312 & NE & NN & 387\\
NN & ADJD & 122 & VVINF & VVFIN & 48\\
NN & VVFIN & 81 & NN & ADJD & 45\\
NN & ADJA & 80 & NE & ADJA & 37\\
NN & NE & 65 & NN & ADJA & 35\\
NE & ADJD & 64 & NN & NE & 32\\
NE & VVFIN & 51 & \$( & ADJA & 31\\
NE & XY & 48 & FM & NE & 21\\
VVFIN & VVPP & 33 & FM & NN & 17\\
NN & CARD & 33 & PWS & PIS & 17\\
... & ... & ... & ... & ... & ...\\
\hline
\multicolumn{2}{|c|}{Bessere Zuweisungen:} & \textbf{1530} &
\multicolumn{2}{|c|}{Bessere Zuweisungen:} & \textbf{1083}\\\hline\hline
\multicolumn{5}{|l|}{Anzahl gefundener Unterschiede:}&
\textbf{3338}\newline \small{(16.99\% aller Token)}\\\hline
\multicolumn{5}{|l|}{Irrelevante Unterschiede:}&\textbf{725}\\\hline\multicolumn{5}{|l|}{Anzahl
getesteter Sätze:}& \textbf{1000}\\\hline
\end{tabular}
\end{center}
\caption{Evaluierung POS-Tagger}
\end{table}




%In our evaluation experiments we compared TreeTagger and TNT by randomly choosing 1000 tweets from Twitter Wulff corpus. These tweets were subsequently run through the taggers %pre-processed with same tokenizer (TreeTagger's Perl Tokenizer). Taggers showed difference in 3351 from input 19649 words (what roughly made 17\% of total number of tokens). The %results of evaluation are depicted in the table below:

%\begin{tabular}{|*{6}{p{0.15\textwidth}|}}\hline
%\multicolumn{6}{|c|}{Tagger Comparison}\\\hline\hline
%\multicolumn{3}{|c|}{Tree Tagger}&\multicolumn{3}{c|}{TNT Tagger}\\\hline
%Chosen Better Tag&Opposite Worse Tag & \# of Cases & Chosen Better Tag & Opposite Worse Tag & \# of Cases\\\hline
%NE & NN & 314 & NE & NN & 389\\
%NN & ADJD & 122 & VVINF & VVFIN & 48\\
%NN & VVFIN & 81 & NN & ADJD & 45\\
%NN & ADJA & 80 & NE & ADJA & 37\\
%NN & NE & 65 & NN & ADJA & 36\\
%NE & ADJD & 64 & NN & NE & 32\\
%NE & VVFIN & 51 & \$( & ADJA & 31\\
%NE & XY & 48 & FM & NE & 21\\
%VVFIN & VVPP & 33 & FM & NN & 17\\
%NN & CARD & 33 & PWS & PIS & 17\\
%... & ... & ... & ... & ... & ...\\

%\multicolumn{2}{|c|}{Total better choices:} & \textbf{1535} & \multicolumn{2}{|c|}{Total better choices:} & \textbf{1089}\\\hline\hline
%\multicolumn{5}{|l|}{Differences found:}& \textbf{3351}\newline \small{(17.05\% of tokens)}\\\hline
%\multicolumn{5}{|l|}{Differences considered irrelevant:}& \textbf{727}\\\hline
%\multicolumn{5}{|l|}{Sentences tested:}& \textbf{1000}\\\hline
%\end{tabular}

\section{Inhaltsklassifikation / Auswertung}

\subsection{Koreferenz}
Für eine genauere Analyse des Tweet-Inhalts auf der Satzebene soll
eine Koreferenzresolution vorgenommen werden.

TODO:
\begin{itemize}
\item Wieviele Pronomen finden wir?
\item Wieviele sonstige Korefs?
\item Wie ist die Performanz von PoCoRes?
\item Performanz auf normalisiertem Input?
\item Was sind die typischen Fehler?
\item Perspektive: wie weiter? Ist twitter-Koref einfacher als
  generelle Koref? Was dürfte ein vielversprechendes Verfahren sein?
\end{itemize}

\subsection{``Off-topic''}
Die Datenerhebung geschah im Wesentlichen durch keyword matching:
Beinhaltet ein tweet das Wort `Wulff'? Dadurch können gelegentlich
tweets in die Suchmenge gelangen, die thematisch nicht relevant sind,
weil sie sich auf eine andere Person gleichen Namens beziehen.\\
TODO: Beispiel\\
TODO: Kommt das häufig vor?\\
TODO: Wie gehen wir damit um?

\subsection{Subtopiks}

\subsection{Sentiment}

\subsection{Diskursqualität}

\section{Zusammenfassung}





%-------------------------------------------------------
\bibliographystyle{named}
\begin{small}
\bibliography{ALLmar130309}
\bibliography{BRANTS, Thorsten. TnT: a statistical part-of-speech tagger. In: Proceedings of the sixth conference on Applied natural language processing. Association for Computational Linguistics, 2000. S. 224-231.}
\bibliography{SCHMID, Helmut. Probabilistic part-of-speech tagging using decision trees. In: Proceedings of international conference on new methods in language processing. 1994. S. 44-49.}
\bibliography{SCHMID, Helmut. Improvements in part-of-speech tagging with an application to German. In: In Proceedings of the ACL SIGDAT-Workshop. 1995.}
\bibliography{LUI, Marco; BALDWIN, Timothy. langid. py: An off-the-shelf language identification tool. In: Proceedings of the 50th Annual Meeting of the Association for Computational Linguistics (ACL 2012), Demo Session, Jeju, Republic of Korea. 2012.}
\bibliography{VRL, NICTA. Cross-domain Feature Selection for Language Identification.}




\end{small}

\end{document}

% some bibtex chunks for future citation
LanguageGuessing:

@inproceedings{lui2012langid,
  title={langid. py: An off-the-shelf language identification tool},
  author={Lui, Marco and Baldwin, Timothy},
  booktitle={Proceedings of the 50th Annual Meeting of the Association for Computational Linguistics (ACL 2012), Demo Session, Jeju, Republic of Korea},
  year={2012}
}

@article{vrlcross,
  title={Cross-domain Feature Selection for Language Identification},
  author={VRL, NICTA}
}


@article{cavnar1994n,
  title={N-gram-based text categorization},
  author={Cavnar, William B and Trenkle, John M and others},
  journal={Ann Arbor MI},
  volume={48113},
  number={2},
  pages={161--175},
  year={1994},
  publisher={Citeseer}
}

TAGGING:

@inproceedings{brants2000tnt,
  title={TnT: a statistical part-of-speech tagger},
  author={Brants, Thorsten},
  booktitle={Proceedings of the sixth conference on Applied natural language processing},
  pages={224--231},
  year={2000},
  organization={Association for Computational Linguistics}
}



@inproceedings{schmid1994probabilistic,
  title={Probabilistic part-of-speech tagging using decision trees},
  author={Schmid, Helmut},
  booktitle={Proceedings of international conference on new methods in language processing},
  volume={12},
  pages={44--49},
  year={1994},
  organization={Manchester, UK}
}

@inproceedings{schmid1995improvements,
  title={Improvements in part-of-speech tagging with an application to German},
  author={Schmid, Helmut},
  booktitle={In Proceedings of the ACL SIGDAT-Workshop},
  year={1995},
  organization={Citeseer}
}


