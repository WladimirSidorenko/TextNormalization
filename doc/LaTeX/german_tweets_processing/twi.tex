\documentclass[11pt]{article}
\usepackage[utf8]{inputenc}
\usepackage{latin1}
\usepackage[T1]{fontenc} % fuer accent circonflex ueber i, Trennung mit Umlaut
\usepackage{german}
\usepackage{scrpage2}
\usepackage{epsf}
\usepackage{graphicx}
\usepackage{amssymb}
\usepackage{makeidx}
\usepackage{booktabs}
\usepackage{rotating}
\usepackage{tabularx}
\usepackage{covingtn}
\usepackage{enumitem}
\usepackage{ae}
\usepackage[comma,round]{natbib}
\usepackage{url}
\usepackage{a4nicer}
%\usepackage{palatino}


%%%%%%%%%%%%%%%%%%%%%%%%%%%%%%%%%%%%%%%

\title{Automatische Verarbeitung deutschsprachiger Tweets:\\ Eine Fallstudie}

%\author{{Manfred Stede\\ Applied Computational
%  Linguistics\\Department of Linguistics\\ University of Potsdam
%  (Germany)\\{\tt stede@uni-potsdam.de}} \and {Kristin Irsig\\ID Berlin}}

\author{}
%Manfred Stede\\ Applied Computational
%  Linguistics\\ University of Potsdam\\ {\tt
%  stede@uni-potsdam.de} \and Kristin Irsig\\ ID Information und
%  Dokumentation\\ im Gesundheitswesen GmbH \&Co. KGAA\\ Berlin\\ {\tt K.Irsig@id-berlin.de}}

\date{}

\pagestyle{empty}


\begin{document}

\maketitle

\thispagestyle{empty}

%\begin{abstract}
%\noindent Abstrakt.
%\end{abstract}



\section{Einführung}

TODO: DIE BESONDERHEITEN DES TWEET PROCESSING


% ======================================================================
\section{Zielsetzung}

Das Projekt ``Diskurse in Social Media'' untersucht aus
kommunikationswissenschaftlicher Perspektive den Verlauf von
politischen Diskursen in drei verschiedenen Social Media: Twitter,
Facebook und Blogs. Es wird untersucht, inwieweit sich zwischen diesen
drei Medien Unterschiede finden lassen im Hinblick auf
\begin{itemize}
\item den zeitlichen Verlauf von Debatten: Wie entwickelt sich das
  Volumen einer Diskussion?
\item die Dialogizität der Diskurse: Wie gehen Teilnehmer auf die
  Beiträge anderer Teilnehmer ein?
\item die Meinungsführerschaft: Sind bestimmte Akteure in den
  Diskussionen ``tonangebend''?
\item die Themen: Welche Aspekte des Themenkreises werden diskutiert?
\item die Meinung: Welche Stimmungslage kommt in einer Diskussion zum Ausdruck?
\end{itemize}
Die qualitativ hochwertige Analyse dieser Fragestellungen setzt das
sachkundige menschliche Urteil voraus, das heißt: am Ende der
Bemühungen steht eine Inhaltsanalyse durch ausgebildete Analysten. Die
Projektpartner aus der Wirtschaftsinformatik und der
Computerlinguistik sollen diesen Prozess aber maßgeblich unterstützen,
um bei einem relativ umfangreichen Datensatz die menschliche Analyse
auf die relevanten Beiträge konzentrieren zu können. Als erstes
Arbeitskorpus wurden dazu Daten ausgewählt, die im Jahr 2011 über
einen Zeitraum von zwei Wochen hinweg (TODO: stimmt das? - wenn die
Angabe 12.12-17.02 im Dateinamen des Korpus den Zeitraum des
Textsammelns bedeutet, so d\"urften es gut 2 Monate sein) das
Stichwort `Wulff' beinhalten, also höchstwahrscheinlich Äußerungen zur
Affäre um den seinerzeitigen Bundespräsidenten beinhalten. Im
vorliegenden Beitrag beschränken wir uns auf die Twitter-Daten, das
sind TODO: wieviele? einzelne Tweets, von denen wir zunächst (TODO:
wieviele 28 818 ?) als Duplikate idetifiziert haben; somit verbleibt eine
Grundmenge von 90 637 zu verarbeitenden Tweets.

Im Projekt sind die Wirtschaftsinformatiker für die Beschaffung der
Datensätze und die Konstruktion von Graphstrukturen zuständig, die die
Verweise zwischen Diskussionsbeiträgen abbilden. Der
Computerlinguistik obliegt die inhaltliche Analyse der einzelnen
Beiträge im Hinblick auf behandelte Themen und die Stimmungslage: Es
wird eine ``Vorsortierung'' der Beiträge durchgeführt, um im Idealfall
die Aufgabe der Analysten auf eine zügige Durchsicht beschränken zu
können. Die Vorsortierung erfolgt anhand drier Dimensionen:
\begin{itemize}
\item Themenklassifikation: Durch unüberwachte Verfahren wird ein
  Clustering von Beiträgen im Hinblick auf (Unter-) Themen
  vorgenommen.\\ Hier einige Beispiele aus dem Wulff-Korpus, die das
  Adressieren verschiedener Themen illustrieren: TODO
\item Sentimentklassifikation: Für jeden Einzelbeitrag soll erkannt
  werden, ob eine positive, negative oder neutrale Haltung ausgedrückt
  wird. Korpus-Beispiele: TODO\\ Zusätzlich soll im Falle von Verweisen auf andere Beiträge
  festgestellt werden, ob diese zustimmend, kritisch, oder neutral
  ausfallen. Korpusbeispiele: TODO
\item Diskursqualitätklassifikation: Textbeiträge können inhaltlich
  fundiert sein und das Potenzial haben, eine Diskussion fruchtbar
  voranzubringen, oder lediglich kurze ``Einwürfe''
  darstellen. Korpusbeispiele: TODO
\end{itemize}


% ======================================================================
\section{Die Pipeline zur Vorverarbeitung}

TODO:
\subsection{Sprachidentifikation}

Die von den Wirtschaftsinformatikern zusammengestellten Daten
enthalten noch tweets, die nicht in deutscher Sprache abgefasst
sind. Um diese zu filtern, haben wir mit drei {\em off-the-shelf}
Werkzeugen zur Sprachidentifikation experimentiert.

\begin{itemize}
\item Google Language Identifier. TODO: weiß man wie er arbeitet?
\item lang-ident
\item textcat
\end{itemize}
TODO: Vergleichende Evaluation: Methode, Durchführung.


\subsection{Satzgrenzenerkennung}

Die generelle Arbeitsweise eines ``sentence splitters'' besteht darin,
Punkte am Ende eines Wortes dahingehend zu disambiguieren, ob es sich
um einen Satzende-Punkt oder den Bestandteil einer Abkürzung (wie
z.B. in ``Nr. 3''), einer Datumsangabe o.ä. handelt.

TODO: Besonderheiten bei Tweets?


\subsection{Normalisierung und Tokenisierung}

\subsubsection{Zu behandelnde Phänomene}

TODO: Klassifizierte Liste von Dingen, die man behandeln muss

\subsubsection{Vorgehen}


\subsection{Part-of-speech tagging}

TODO: Tree Tagger versus TNT.

% ======================================================================
\section{Inhaltsklassifikation / Auswertung}

\subsection{Koreferenz}
Für eine genauere Analyse des Tweet-Inhalts auf der Satzebene soll
eine Koreferenzresolution vorgenommen werden.

TODO:
\begin{itemize}
\item Wieviele Pronomen finden wir?
\item Wieviele sonstige Korefs?
\item Wie ist die Performanz von PoCoRes?
\item Performanz auf normalisiertem Input?
\item Was sind die typischen Fehler?
\item Perspektive: wie weiter? Ist twitter-Koref einfacher als
  generelle Koref? Was dürfte ein vielversprechendes Verfahren sein?
\end{itemize}

\subsection{``Off-topic''}
Die Datenerhebung geschah im Wesentlichen durch keyword matching:
Beinhaltet ein tweet das Wort `Wulff'? Dadurch können gelegentlich
tweets in die Suchmenge gelangen, die thematisch nicht relevant sind,
weil sie sich auf eine andere Person gleichen Namens beziehen.\\ 
TODO: Beispiel\\
TODO: Kommt das häufig vor?\\
TODO: Wie gehen wir damit um?

\subsection{Subtopiks}

\subsection{Sentiment}

\subsection{Diskursqualität}

% ======================================================================
\section{Zusammenfassung}





%-------------------------------------------------------
\bibliographystyle{named}
\begin{small}
\bibliography{ALLmar130309}
\end{small}

\end{document}
