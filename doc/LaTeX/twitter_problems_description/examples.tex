\documentclass[a4paper,11pt]{article}

%%%
\usepackage[latin1]{inputenc}
\usepackage[top=2cm,bottom=2cm,left=2cm,right=2cm]{geometry}

%%%

\begin{document}

\begin{enumerate}
\item{} \textbf{Morphologische Probleme}
  \begin{itemize}
  \item{} Twitter-spezifische Ph\"anomene:
    \begin{itemize}
    \item{} \textbf{Hashtags} (wie z.B. \textit{\#{}Wulff}).\\
      Hashtags sind spezielle W\"orter mit dem \#-Zeichen an ihrem Anfang. Die Aufgabe dieser Tags besteht darin, besonders interessante Twitter-Themen zu markieren. Jedoch k\"onnen viele Computerprogramme ohne spezielle Anpassung oft nicht verstehen, dass es sich bei ``\#{}Wulff'' und ``Wulff'' um dasselbe Wort und dieselbe Person handelt.\\
      \textit{Zu keinem Zeitpunkt in meiner Amtszeit habe ich unberechtigte Vorteile gew\"ahrt" richtig Herr \#{}Wulff - haben alle f\"ur die Vorteile bezahlt!}\\
    \item{} \textbf{AT-Tokens} (wie z.B. \textit{@Merkel}).\\
      Diese Tokens markieren in der Regel die Zielperson, an die die Nachricht gerichtet ist, oder die urspr\"ungliche Autorin des Tweets. Jedoch ist der Umgang damit in Twitter-Texten sehr lose.\\
      \textit{"@Merkel spricht @Wulff ihr vollstes Vertrauen aus"... ja, hat sie damals bei @Guttenberg auch gemacht o.O}\\
    \item{} \textbf{Smileys/Emoticons} (wie z.B. $>\_<, ::(((, -\_-$).\\
      Bei den Smileys geht es um spezielle Zeichenketten, die zum Ausdruck von Emotionen dienen. Viele Computerprogramme k\"onnen aber \"ofters nicht einmal verstehen, dass es sich dabei um ein einzelnes Wort handelt und interpretieren das ``$::((($'' als w\"aren es 5 separate voneinander unabh\"angige Zeichen.\\
      \textit{\#Wulff so:"Mimimimi!" \#Merkel so:" $>\_<$ " \#Twitter so:"Gnihihi!"\\
        Juhhhhhhhhhhhhhhuuuuu Hab ich doch endlich lange genug an deinem Thron ges\"agt...  WULFF $::((($\\
        Kann ich im \#GoogleReader einstellen, dass mir Artikel mit dem Stichwort "Wulff" nicht mehr angezeigt werden...? $-.-$\\
      }
    \end{itemize}
  \end{itemize}

\item{} \textbf{Rechtschreibfehler}
  \begin{itemize}
  \item{} \textbf{Probleme mit Eingabe/Encoding} (wie z.B. \textit{Rucktritt, Loesung}).\\
    \textit{\#Politiker sind Teil des Problems Rucktritt ein Teil der Loesung!!!}
  \item{} \textbf{Emotionale Entstellung von W\"ortern} (wie z.B. \textit{sooooo}).\\
    \textit{"Ich hatte keine Zeit, um zum Ministerpr�sidenten zum Bundespr�sidenten zu werden. Ich hatte eine soooo schwierige Kindheit"}
  \item{} \textbf{Typische Rechtschreibfehler} (wie z.B. \textit{posetiv}).\\
    \textit{Kann mir jemand sagen, warum der Wulff nich an R\"ucktritt denkt? W\"are eigentlich sogar posetiv.}
  \end{itemize}

\item{} \textbf{Lexikalische Probleme}
  \begin{itemize}
  \item{} Verbreitung von \textbf{Abk\"urzungen}\\
    Da Twitter-Texte eine maximale L\"ange von 140 Zeichen haben, versuchen die Leute ihre Nachrichten m\"oglichst kompakt zu halten und greifen deswegen zu Abk\"urzungen. Die meisten Textverarbeitungsprogramme k\"onnen aber \"ofters nicht verstehen, dass hinter vielen Abk\"urzungen andere ganz konkrete W\"orter stehen.\\
    \textit{Missbrauch einer Vertrauensstellung zur Erlangung eines mat. Vorteils, auf den kein rechtl. begr�ndeter Anspruch besteht}
  \item{} \textbf{Umgangssprachliche und Slang-W\"orter} (wie z.B. \textit{isses})\\
    Da Twitter meist f\"ur Alltagsgespr\"ache und Meinungsaustausch verwendet wird, kommen hier wie in keiner anderen Textsorte sehr viele umgangssprachliche W\"orter und Wendungen vor.\\
    \textit{und dann isses gut?}
  \item{} \textbf{Neologismen} (wie z.B. \textit{guttenbergen})\\
    Der lockere Umgang mit der Sprache in Twitter-Texten f\"uhrt nicht selten zur Schaffung von vielen neuen W\"ortern, deren Bedeutung f\"ur ein Computerprogramm jedoch meist unklar ist.\\
    \textit{Kevin hat meine Hausaufgaben geguttenbergt. Habe ihm sowas von auf die Mailbox gewulfft! Soll merkeln, dass wir keine Br\"uderle sind.}
  \end{itemize}


\item{} \textbf{Syntaktische Probleme}
  \begin{itemize}
  \item{} \textbf{Retweets}\\
    \textit{Retweets} sind spezielle syntaktische Konstruktionen, die f\"ur Twitter-Nachrichten typisch sind, in gew\"ohnlichen geschriebenen Texten aber nicht vorkommen. Ein typischer Retweet hat meist die Form ``$RT + AT-Token$'' und wird automatisch vom Twitter eingef\"ugt werden, wenn jemand die Nachricht einer anderen Person teilen m\"ochte.\\
    \textit{@RegSprecher RT @msmfun Es gibt nur eine Amtshandlung an der ich \#Wulff messe: Fertigt er das \#ESM-Erm�chtigungsgesetz aus, oder nicht?!}

  \item{} \textbf{Syntaktisches Rauschen} (Interjektonen, Sonderzeichen oder Onomatopoetika usw.)\\
    Beim syntaktischen Rauschen geht es um Textartefakte, die mitten im Satz vorkommen und dadurch automatische syntaktische Ana\-ly\-se von Texten erheblich erschweren bzw. unm\"oglich machen. Typische Beispiele f\"ur solche Artefakte sind W\"orter wie \textit{lol, rofl oder das Sonderzeichen $+$}.\\
    \textit{"Der mit dem Wulff tanzt" loool jaja ich bin sp�t mit dem witz. Ich weiss\\
      Genau abschaffen, DEU-BundesPr�sident ist eh nur Durchwinker + Grinse-Kefer-Lackaffe + EigeneVorteileVerschaffer \#S21 \#Wulff \#LGNPCK}
  \end{itemize}


\item{} \textbf{Semantische Probleme}
  \begin{itemize}
  \item{} \textbf{Versteckte Sentimente}\\
    Unter versteckten Sentimenten versteht man Meinungs\"au\ss{}erungen, die keine offensichtlichen Merkmale von Subjektivit\"at tragen, deren illokutive Funktion aber erst durch komplettes Verst\"andnis des gesamten Sinns des Satzes ersichtlich wird.\\
    \textit{"Ich mach' den Wulff!" ist das neue "Sag' ich dir nicht!"}

  \item{} \textbf{Sarkasmus}\\
    Als sarkastisch werden solche \"Au\ss{}erungen bezeichnet, die von ihrer Form her scheinbar positiv sind, in der Tat aber eine negative Einsch\"atzung beinhalten.\\
    \textit{Um was zur Sexismusdebatte beizutragen. Recht so, die Frauen sind immer an allem Schuld. Gutes Vorbild! Herr \#Wulff}

  \item{} \textbf{Satz\"ubergreifende Ironie}\\
    Satz\"ubergreifende Ironie liegt dann vor, wenn 2 oder mehrere gedankliche \"Au\ss{}erungen zusammen einen ironischen Sinn ergeben.\\
    \textit{\#Wulff sollte im Amt bleiben. Im Sozialamt." \#heuteshow\\
      Ich lese immer Frau Merkel stellt sich hinter \#Wulff ..." Er steht am Abgrund, da ist dahinter besser :-)}
  \end{itemize}
\end{enumerate}
\end{document}
