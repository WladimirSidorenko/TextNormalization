\documentclass[11pt,a4paper]{article}

%%%%%%%%%%%%%%%%%%%%%%%%%%%%%%%%%%%%%%%%%%%%%%%%%%%%%%%%%%%%%%%%%%
%% Libraries
\usepackage[driver=pdftex,vmargin=2cm,hmargin=2cm]{geometry}
\usepackage[usenames,dvipsnames]{xcolor}
\usepackage{amsmath}
\usepackage{array}
\usepackage{booktabs}
\usepackage{color}
\usepackage{framed}
\usepackage[colorlinks=true]{hyperref}
\usepackage{multirow}
\usepackage{natbib}
\usepackage{paralist}
\usepackage{url}
\usepackage{tikz}
%% \usepackage{gb4e}
\usepackage{xargs}
\usepackage{eurosym}

\hypersetup{
  colorlinks,
  citecolor=Violet,
  linkcolor=Red,
  urlcolor=Blue}

%%%%%%%%%%%%%%%%%%%%%%%%%%%%%%%%%%%%%%%%%%%%%%%%%%%%%%%%%%%%%%%%%%
%% Commands
\definecolor{dodgerblue4}{RGB}{16,78,139}
\definecolor{orange3}{RGB}{205,133,0}
\definecolor{DarkSlateBlue}{RGB}{72,61,139}

\newlength{\cwidth} \setlength{\cwidth}{0.18\textwidth}
\newenvironment{example}{\begin{center}\begin{exe}\ex}{\end{exe}\end{center}}
\newcommand{\xmltag}[1]{\textcolor{black}{{\small$<$#1$>$}}}
\newcommand{\sentiment}[2][negative]{$<$sentiment
  sentiment\_polarity="#1"$>$\textcolor{dodgerblue4}{#2}$<$/sentiment$>$}
\newcommand{\source}[1]{\xmltag{source}\textcolor{orange3}{#1}\xmltag{/source}}
\newcommand{\target}[1]{\xmltag{target}\textbf{#1}\xmltag{/target}}
\newcommand{\negation}[1]{\xmltag{negation}\textcolor{red}{#1}\xmltag{/negation}}
\newcommand{\intensifier}[2][1]{\xmltag{intensifier
    intensity="#1"}\textcolor{DarkSlateBlue}{#2}\xmltag{/intensifier}}
\newcommandx{\emoexpression}[3][1=negative, 2=1]{
  \xmltag{emo-expression polarity="#1" intensity="#2"}
  \textcolor{green}{#3}\xmltag{/emo-expression}}

\renewenvironment{example}{\begin{center}\itshape}{\upshape\end{center}}



%%%%%%%%%%%%%%%%%%%%%%%%%%%%%%%%%%%%%%%%%%%%%%%%%%%%%%%%%%%%%%%%%%
%%%%%%%%%%%%%%%%%%%%%%%%%%%%%%%%%%%%%%%%%%%%%%%%%%%%%%%%%%%%%%%%%%
%% Commands
\author{Wladimir Sidorenko}
\date{\today}
\title{Guidelines for the Annotation of a Sentiment Corpus}

%%%%%%%%%%%%%%%%%%%%%%%%%%%%%%%%%%%%%%%%%%%%%%%%%%%%%%%%%%%%%%%%%%
%% Main
\begin{document}
\maketitle{}
\section{Introduction}
Welcome. In this assignment your task is to annotate sentiments in a
corpus of German Twitter messages.

\subsection{Annotation Tool}

For the annotation you will use \texttt{MMAX2} -- a freely available
annotation tool that can be downloaded under the following link:

\url{http://sourceforge.net/projects/mmax2/files/mmax2/mmax2_1.13.003/MMAX2_1.13.003b.zip/download}

After you have downloaded and unpacked the archive, change to the
newly created directory \texttt{1.13.003/MMAX2} in your shell and
execute the following commands: \newline

\texttt{> chmod u+x ./mmax2.sh}

\texttt{> nohup ./mmax2.sh \&} \newline

{\setlength{\parindent}{0pt} If you have never used \texttt{MMAX2}
  before, please read the document \texttt{mmax2quickstart.pdf}
  you can find in the subdirectory \texttt{MMAX2/Docs}. At the end of 
  this document the section "MMAX Techniques" will explain how to use the most 
  important techniques for annotating our corpus in MMAX.}

\subsection{Corpus Files}

You should also have received a copy of the corpus' project files as a
tar-gzipped archive.  Please unpack this archive using the
command: \newline

\texttt{> tar -xzf twitter-sentiment.tgz} \newline

{\setlength{\parindent}{0pt} A directory called
  \texttt{mmax-prj} will appear in your current folder. After having unpacked the files, change to the 
	\texttt{MMAX2} window, click on \texttt{File -> Load} and select the path\footnote{Make sure the path itself contains no
    white spaces. Otherwise \texttt{MMAX2} will fail to load the
    project file with the error message \emph{java.netMalformedURLException.no protocol: words.dtd}} 
    to the unpacked \texttt{mmax-prj} folder and load one of the \texttt{*.mmax} files there. 
    Now the data to annotate will appear in \texttt{MMAX}.}

%%%%%%%%%%%%%%%%%%%%%%%%%%%%%%%%%%%%%%%%%%%%%%%%%%%%%%%%%%%%%%%%%%

\section{Tags and Attributes}
Here is an overview of all tags and their corresponding attributes you will use in this annotation-task:

\begin{enumerate}
\item \textit{sentiment}-tag with attributes:
  \begin{enumerate}
  \item polarity,
  \item intensity,
  \item sarcasm,
  \item sentiment-ref;
  \end{enumerate}
\item \textit{source}-tag with attributes:
  \begin{enumerate}
  \item anaph-ref,
  \item sentiment-ref;
  \end{enumerate}
\item \textit{target}-tag with attributes:
  \begin{enumerate}
  \item anaph-ref,
  \item sentiment-ref;
  \end{enumerate}
\item \textit{emo-expression}-tag with attributes:
  \begin{enumerate}
  \item polarity,
  \item sarcasm, 
  \item intensity,
  \item emo-expression-ref,
  \item sentiment-ref;
  \end{enumerate}
\item \textit{intensifier}-tag with attributes:
  \begin{enumerate}
  \item degree,
  \item sentiment-ref;
  \end{enumerate}
\item \textit{diminisher}-tag with attributes:
  \begin{enumerate}
  \item degree,
  \item sentiment-ref;
  \end{enumerate}
\item and the \textit{negation}-tag.
\end{enumerate}
A more detailed description of the meaning of these tags and
attributes is given in the next subsections.



\subsection{sentiment}
\texttt{sentiment}s are the most important markables. With this tag
you should annotate statements that express some \emph{polar opinion} about
a subject/object as well as comparisons of two subjects/objects. 
We DO NOT regard as sentiments opinions that do not convey any polarity. 
Additionally the polar expressions have to be related to (or aimed at) a particular subject/object. 
In other words, simple informative statements, apologies, congratulations, salutations, requests, politeness, commands or 
hesitations \textcolor{red}{Was meinst Du mit (Ver)Z\"ogerungen?} should not be marked with this tag.

To determinine the boundaries of a sentiment span look how the sentiment
is expressed grammatically. If it is expressed by a single
noun phrase (e.g. ``\textit{Auf dem Tresor lag \xmltag{sentiment}ein
  h\"assliches Buch\xmltag{/sentiment}}.''), mark this noun phrase with the sentiment tag. If
it is expressed by a clause (e.g. ``\textit{\xmltag{sentiment}Ich hasse B\"ucher
  ohne Inhaltsangabe.\xmltag{/sentiment}}''), mark the whole clause as sentiment.
In case a sentiment is expressed by multiple phrases, clauses or sentences
(e.g. ``\textit{\xmltag{sentiment}Sie denken, reden, riechen, lieben,
  schmecken, ficken Plastik. Sie haben das so gelernt in der
  Plastik-Werbewelt.\xmltag{/sentiment}}''), include all relevant phrases/clauses/sentences
 in one single sentiment tag. 

The attributes of the \texttt{sentiment} tag with their respective
values and meanings are specified in the table below: \\
\begin{tabular}[t]{|l|c|p{0.6\textwidth}|}\hline
  Attribute & Value & Value's Meaning\\\hline
  %%%%%%%%%%%%%%%%%%%
  \multirow{3}{*}{polarity} & \textit{positive} & sentiment
  expresses a positive attitude to its respective target\\\cline{2-3}

  & \textit{negative\newline(default)} & sentiment
  expresses a negative attitude to its respective target\\\cline{2-3}

  & \textit{comparison} & sentiment expresses a comparison of two
  targets with a preference for one of them\\\hline
  %%%%%%%%%%%%%%%%%%%
  \multirow{3}{*}{intensity} & \textit{0} & stylistically weak
  expression tending to be a neutral phrase/clause/sentence\\\cline{2-3}

  & \textit{1\newline(default)} & middle stylistic
  expressivity\\\cline{2-3}

  & \textit{2} & stylistically very expressive sentiment\\\hline
  %%%%%%%%%%%%%%%%%%%
  \multirow{2}{*}{sarcasm} & \textit{true} & actual polarity is the 
  opposite of its apparent form, e.g. an apparent positive expression
  in fact expresses a negative opinion or vice versa; often found in derisions (The actual sense can only be derived
  on the basis of outside context or common knowledge.)\\\cline{2-3}

  & \textit{false\newline(default)} & no sarcasm present - polar
  attitude has literal meaning\\\hline
  %%%%%%%%%%%%%%%%%%%
  sentiment-ref & \textit{$->$\newline(directed edge)} & in case one
  single sentiment is split into parts, you can draw an edge to connect the parts\\\hline 
\end{tabular}
\vspace{1.5cm}

\underline{Things you SHOULD mark as sentiments:}
\begin{itemize}
  \item Phrases, clauses, sentences and statements expressing a polar opinion
    (e.g. \textit{Der neue Film \"uber Superman war knorke!} -
    \textit{The new superman movie was fantastic});
  \item Interrogative clauses containing a polar opinion
    and do not call this opinion into question but rather ask for its
    reasons or other related aspects (e.g. \textit{Warum findet die
      Mehrheit der Bev\"olkerung CDU so toll?} - \textit{Why does the
      majority of the population consider CDU great?});
  \item Exclamations expressing support or disapproval of
    something (e.g. \textit{Der Sommer ist da. Super!} -
    \textit{Summer has come. Great!}).
  \item Comparisons expressing some preference of one subject/object over another 
      (e.g. \textit{Canon EOS 550d macht bessere Aufnahmen als 600d} - \textit{Canon EOS 550d takes
      better pictures than 600d}).
\end{itemize}


\underline{Things you SHOULD NOT mark as sentiments:}
\begin{itemize}
  \item Statements that do not have an annotatable target:
      (e.g. \textit{F\"uhle mich so traurig :(} - \textit{Feeling so blue :(});
  \item Statements describing some objective facts, even if possible
    consequences of these facts can be assumed to have negative
    influence on the author (e.g. \textit{Wenn ein Floh einen Menschen
      bei\ss{}t und ihn mit erbrochenem Blut infiziert, werden die
      Pestbakterien ins Gewebe \"ubertragen.} - \textit{When a flea
      bites a human and contaminates the wound with regurgitated
      blood, the plague carrying bacteria are passed into the
      tissue.});
  \item Interrogative clauses in case they ask whether some polar
    opinion is true or not. (e.g. \textit{Findet die Mehrheit
      der Bev\"olkerung CDU toll?} - \textit{Does the majority
      of the population consider CDU great?});
\end{itemize}

%%%%%%%%%%%%%%%%%%%%%%%%%%%%%%%%%%%%%%%%%%%%%%%%%%%%%%%%%%%%%%%%%%%%%%%%%%%%%%%


\subsection{source}
The \texttt{source} tag is used to mark the immediate author(s) or
experiencer(s) of a polar expression.  For example, in the
sentence -- \textit{Sie mag ihr neues Outfit nicht} (\textit{She
  doesn't like her new outfit}) -- ``Sie'' (``She'') should be
considered as opinon's origin and marked as \textit{source} of the
sentiment relation.  Sources are usually expressed by pronouns, nouns, 
multiword expressions (e.g. \textit{Neue Z\"uricher Zeitung}) or usernames 
(e.g. \textit{@John1991}). 
When marking \texttt{source}s, please always include complete noun phrases 
in this tag, i.e. nouns with all their respective dependent attributes. 

The attributes and attribute values of the \texttt{source} tag are shown in the table below: \\

\begin{tabular}{|l|c|p{0.6\textwidth}|}\hline
  Attribute & Value & Value's Meaning\\\hline
  sentiment-ref & \textit{$->$\newline(directed edge)} & directed edge
  pointing from an outside \texttt{source} element to its respective sentiment;
  in case of multiple overlapping sentiments (e.g. one \texttt{source}
  element is related to two or more sentiments) this
  edge should point from the \texttt{source} element to its respective sentiment
  spans\\\hline

  anaph-ref & \textit{$->$\newline(directed edge)} & directed edge
  pointing from a \texttt{source} element that is expressed by a pronoun or pronominal
  adverb to their respective antecedents\\\hline
\end{tabular}
\vspace{0.5cm} 

\underline{Things you SHOULD mark as sources:}
\begin{itemize}
  \item Original author(s) of polar opinions, i.e. persons,
    groups or officials who express sympathy or antipathy for
    something or somebody e.g. \textit{\textbf{Michael} meinte, das w\"are die
      beste L\"osung.}  (\textit{\textbf{Michael} thought, it would be the
      best solution.});
\end{itemize}


\underline{Things you SHOULD NOT mark as sources:}
\begin{itemize}
  \item Persons how are neutrally citing someone else's opinion,
    e.g. in the sentence \textit{Nach Tatjanas Worten war Michael sehr
      dar\"uber ver\"argert.} (\textit{According to Tatjana, Michael
      was very angry about that.}) you should only mark
    \textit{Michael} as source and not \textit{Tatjana}. Please note, that in case an author
    supports or contradicts someone else's opinion, two sentiment
    relations should be made - one with the original author as source
    and one with the supporter/opponent of the opinion as the other
    source.\footnote{In that case sentiments will overlap, and
      you also should mark these sources with \textit{sentiment-ref}-tag.}
\end{itemize}



%%%%%%%%%%%%%%%%%%%%%%%%%%%%%%%%%%%%%%%%%%%%%%%%%%%%%%%%%%%%%%%%%%%%%%%%%%%%%%%%%%%%%%%%%%
\subsection{target}
\texttt{target}s are the counterpart of \texttt{source}s.
As \texttt{target}s you should mark subjects/objects a particular polar
opinion is about. That means, \texttt{target}s are the topics an attitude is expressed about. 
These topics are usually represented
by nouns, pronouns or also complete clauses/sentences.  You are asked to include
complete noun phrases and complete clauses/sentences in this tag (i.e. the main
target word with all its dependent attributes).

In coordinations or sentiments spanning multiple phrases/clauses/sentences it 
is possible to have more than one target. Please annotate every single target seperately. 
The same applies to comparisons. Mark each target seperately and do not forget to annotate 
the corresponding \textit{preference} attribute for the respective target.

The attributes and attribute values of the \texttt{target} tag almost
fully correspond to the attributes of the \texttt{source} tags. The only 
additional attribute for targets is the \textit{preference} attribute that should 
be used in comparisons only. It has the values \textit{preferred} and \textit{dispreferred} to annotate 
the preference of one target over another. \newline

\underline{Things you SHOULD mark as targets:}
\begin{itemize}
  \item Persons, things or objects the opinion is made about,
    e.g. \textit{Ich hasse \textbf{lange Schlangen}} (\textit{I hate
      \textbf{long queues}});

  \item Descriptions of actions or events, e.g. \textit{Ich hasse es,
    \textbf{wenn St\"arkere Schw\"achere drangsalieren}} (\textit{I
    hate it \textbf{when stronger people bully weaker ones}});

  \item Elements of a comparison,
    e.g. \textit{\textbf{\"Altere Handys} halten l\"anger als
      \textbf{moderne Smartphones}} (\textit{\textbf{Older mobile
        phones} outwear \textbf{modern smartphones}}).
\end{itemize}
%%%%%%%%%%%%%%%%%%%%%%%%%%%%%%%%%%%%%%%%%%%%%%%%%%%%%%%%%%%%%%%%%%%

\subsection{emo-expression}
\texttt{emo-expression}s are lexical elements that bear some
polar meaning on their own. Usually these are words like
\textit{gut} (\textit{good}), \textit{schlechter} (\textit{worse}),
\textit{m\"ogen} (\textit{like}) or \textit{hassen}
(\textit{hate}). Such elements can be expressed either as single words
or more complex expressions (in case of idiomatic expressions or support
verbs). That means emo-expressions can be single nouns, noun-phrases, single 
verbs, verb phrases, single adjectives, adjective-phrases or adverbials. 

Please annotate smileys and other emoticons, too.

Since sarcasm can be expressed by a single emo-expression outside of a sentiment span there is an attribute for this, too. 
For instance "LOL" expresses sarcasm in the following tweet: \textit{vermerk auf rentenwisch : 
bitte beachten sie die inflationsrate. ein betrag von {100\euro} heute ist in 25  jahren nur noch etwa {69\euro} wert . LOL (comment on pension note:
please notice the inflation rate. An ammount of {100\euro} today is worth only {69\euro} in 25 years. LOL)}

Please mark all emo-expressions you can find in your dataset regardless of a corresponding sentiment bearing phrase, clause or sentence. 

Notice that a lot of emo-expressions can be ambiguous. A positive emo-expression in one tweet can express a negative attitude in another. Please annotate them accordingly.

\textcolor{red}{Wichtige Frage: Sollen Interjektionen (oh, aha, achso, OMG etc.) auch als emo-expressions annotiert werden? Ich habe das n\"amlich nicht gemacht, weil sie in so gut wie allen F\"allen nicht als wirklich positiv bzw. negativ zu identifizieren waren. Wie z.B. \textit{Lichgestaaaaaaaaaaalt !!!!!! in deren Schatten ich mich drehe uuuooooooh oooooh uuuuuooohhhh oooooohhhhh xD} oder \textit{ooh du hasts gut , h\"att auch gern urlaub ...}. Ich finde Interjektionen auch grenzwertig, da sie unter gewissen Umst\"anden auch als intensifier dienen können, z.B. oh wie gut, oh ja etc. . Falls sie doch annotiert werden sollen, bitte Bescheid sagen und/oder eine entsprechende Anmerkung in den Guidelines machen. W\"are gut explizit zu schreiben ob sie annotiert werden sollen oder nicht.}


emo-expressions have the following attributes with their corresponding values: \newline

\begin{tabular}{|l|c|p{0.6\textwidth}|}\hline
  Attribute & Value & Value's Meaning\\\hline
  %%%%%%%%%%%%%%%%%%%

  \multirow{2}{*}{polarity} & \textit{positive} & this emotional
  expression has positive polar meaning about sentiment's
  target\\\cline{2-3}

  & \textit{negative\newline(default)} & this emotional expression has
  negative polar meaning\\\hline

  %%%%%%%%%%%%%%%%%%%

  \multirow{2}{*}{sarcasm} & \textit{true} & actual polarity is
  opposite of apparent form \\\cline{2-3}

  & \textit{false\newline(default)} & no sarcasm present - 
  emo-expression has literal meaning\\\hline
  
  %%%%%%%%%%%%%%%%%%%%%%%%%%%%%%%

  \multirow{3}{*}{intensity} & \textit{0} & stylistically weak
  emotional expression\\\cline{2-3}

  & \textit{1\newline(default)} & middle stylistic
  expressivity\\\cline{2-3}

  & \textit{2} & stylistically very expressive emotional
  expression\\\hline

  %%%%%%%%%%%%%%%%%%%

  emo-expression-ref & \textit{$->$\newline(directed edge)} & in cases
  a single emotional expression is split into parts this edge
  should point from the parts to the nucleus of the emo-expression (choose
  whatever part you want as nucleus and connect all other parts to it
  - it does not matter what part you consider as main)\\\hline

  %%%%%%%%%%%%%%%%%%%

  sentiment-ref & \textit{$->$\newline(directed edge)} & in cases
  multiple sentiments overlap and the emo-expression is located in an overlapping span this edge
  should point to the sentiment on which this emo-expression has immediate
  impact\\\hline
\end{tabular}
\vspace{0.5cm}

\underline{Things you SHOULD mark as emo-expressions:}
\begin{itemize}
  \item Adjectives and adverbs bearing polar attitudes \textit{Peter
    hatte \textbf{bessere} Noten in der Schule als sein Bruder}
    (\textit{Peter had \textbf{better} grades at school than his
    brother.});

  \item Verbs expressing attitude of a speaker to target,
    e.g. \textit{Mir \textbf{gefiel} die neue House-Staffel}
    (\textit{I \textbf{liked} the new House series});

  \item Idiomatic expression including support verbs
    e.g. \textit{\textbf{Zum Teufel} soll die neue Regierung
      \textbf{gehen}} (\textit{The new government should \textbf{go to
        hell}}).

  \item Smileys in case they really express an emotional attitude and
    are not used for politeness or without any particular meaning
    e.g. \textit{Gleich in Braunschweig mit Kameraden treffen
        \textbf{:)}} (\textit{Will soon meet friends in Braunschweig
        \textbf{:)}}).
\end{itemize}



%%%%%%%%%%%%%%%%%%%%%%%%%%%%%%%%%%%%%%%%%%%%%%%%%%%%%%%%%%%%%%%%%%

\subsection{negation}
\texttt{negation}s are lexical or syntactic elements that
reverse the primary polar meaning of emo-expressions to the opposite
so that the overall polarity of the whole sentiment is different to
the polarity of emo-expressions belonging to it. A typical example of
negation is \textit{nicht} in sentences like \textit{Ein guter Schritt
  war diese Entscheidung \textbf{nicht}.} (\textit{This decision was
  \textbf{not} a good move.}).

\vspace{0.5cm}
You SHOULD only mark negative elements that have an impact on the
sentiment polarity. Negating elements having no such impact
shold not be marked. Negation elements are usually represented by:
\begin{itemize}
\item Negation particle \textit{nicht} (\textit{not}),
  e.g. \textit{Klug ist dieser Hund sicherlich \textbf{nicht}}
  (\textit{This dog is certainly \textbf{not} clever});

\item Negative article \textit{kein} e.g. \textit{Er war
  \textbf{kein} Vorbild f\"ur seine Kinder} (\textit{He was
  \textbf{not} a good role model for his children});

\item Indefinite pronouns like \textit{niemand}, \textit{keiner}
  etc., e.g. \textit{\textbf{Niemand} hielt ihn f\"ur einen
    ehrlichen Menschen.} (\textit{\textbf{Nobody} considered him an
    honest man});

\item Any lexical or idiomatic unit in case they turn the sentiment
  polarity to the opposite, e.g. \textit{Ich \textbf{zweifle}, dass
    das neue iPhone ein besseres Display hat.} (\textit{I
    \textbf{doubt} the new iPhone has a better display}).
\end{itemize}

DO NOT mark elements as negations that have no effect on the sentiment
polarity. For example, in the sentence \textit{Ich mag Leute, die nicht
  nur an sich selbst denken.} (\textit{I like people who not only care
  about themselves.}) \textit{nicht} (\textit{not}) should \underline{not} be
marked as negation since it does not change the positive polarity
expressed by \textit{m\"ogen} (\textit{like}).
\vspace{0.5cm}

Negations only have one possible attribute, namely
\textit{sentiment-ref} that is a directed edge pointing from a negation
to the sentiment it belongs to. You should only draw this edge in cases multiple sentiment 
relations overlap and it is not obvious to which of these sentiments the negation belongs to,
i.e. in cases the negation is used for both sentiment spans.

%%%%%%%%%%%%%%%%%%%%%%%%%%%%%%%%%%%%%%%%%%%%%%%%%%%%%%%%%%%%%%%%%%%%%%%%%%%%%%%%%%

\subsection{intensifier}
\texttt{intensifier}s are elements that increase the polar
meaning of emotional expressions. Intensifiers are usually expressed
by adjectives or adverbs like \textit{sehr} (\textit{very}),
\textit{ziemlich} (\textit{rather}) and the like. In intensifier-chains 
\textit{(e.g. sehr, sehr gut)} please annotate each intensifier separately.
As mentioned above you should annotate every single emo-expression you can 
find regardless of a corresponding sentiment span. 
Please do not forget to annotate the corresponding intensifier if there is one. \newline

Intensifiers have the following attributes and values: \newline

\begin{tabular}{|l|c|p{0.6\textwidth}|}\hline
  \multirow{2}{*}{degree} & \textit{1 (default)} & intensifier 
  slightly increases polarity of emotional
  expression, e.g. \textit{ziemlich},
  \textit{recht} etc.\\\cline{2-3}

  & \textit{2} & intensifier strongly increases polarity  of
  emotional expression, e.g. \textit{sehr}, \textit{super},
  \textit{stark} etc.\\\hline

  %%%%%%

  sentiment-ref & \textit{$->$\newline(directed edge)} & Directed
  edge pointing from the intensifier to the sentiment it belongs
  to. By analogy to negations, you should only draw this edge in cases
  multiple sentiment relations overlap and it is not obvious
  to which of these sentiments the intensifier belongs to,
  i.e. in cases when the intensifier is used for both
  sentiment spans.\\\hline
\end{tabular}

%%%%%%%%%%%%%%%%%%%%%%%%%%%%%%%%%%%%%%%%%%%%%%%%%%%%%%%%%%%%%%%%%%

\subsection{diminisher}
\texttt{diminisher}s are the counterpart to \texttt{intensifier}s. 
These are elements that decrease the polar meaning of emotional expressions. 
Like intensifiers diminishers are usually expressed by adjectives or adverbs. Typical examples of such
adverbs are \textit{wenig}, \textit{kaum}, \textit{ein bisschen} etc.. 
\textcolor{red}{Gibt es eigtl. diminisher-chains??? Soweit ich mich erinnere sind mir in den Tweets keine begegnet und spontan fallen mir auch keine ein. }\newline

Diminishers have the same attributes as intensifiers. The only
difference are the values for the attribute. Instead of positive values you have 
negative ones. \newline 

Here is a table of the diminisher attributes: \newline

\begin{tabular}{|l|c|p{0.6\textwidth}|}\hline

  \multirow{2}{*}{degree} & \textit{-1 (default)} & diminisher
  slightly decreases polarity of emotional
  expression, e.g. \textit{wenig},
  \textit{bisschen} etc.\\\cline{2-3}

  & \textit{-2} & diminisher strongly decreases polarity of
  emotional expression, e.g. \textit{kaum} etc.\\\hline


  sentiment-ref & \textit{$->$\newline(directed edge)} & Directed
  edge pointing from the diminisher to the sentiment it belongs
  to. You should only draw this edge in cases
  multiple sentiment relations overlap and it is not obvious
  to which of these sentiments the diminisher belongs to,
  i.e. in cases when the diminisher is used for both
  sentiment spans.\\\hline

\end{tabular}

%%%%%%%%%%%%%%%%%%%%%%%%%%%%%%%%%%%%%%%%%%%%%%%%%%%%%%%%%%%%%%%%%%%%%%%%%%%%
%%%%%%%%%%%%%%%%%%%%%%%%%%%%%%%%%%%%%%%%%%%%%%%%%%%%%%%%%%%%%%%%%%%%%%%%%%%%

\section{MMAX Techniques}
In this section the most important techniques for annotating our corpus in MMAX will be described.

\subsection{Turn on Auto-Save and Auto-apply}
After setting up MMAX load a .mmax file as described in the beginning of this document. You will be asked to validate the annotations. Please always do that. 
Before you start the annotation it is recommended to turn on "Auto-Save" and "Auto-apply". Go to \texttt{File -> Auto-Save -> Every} and select a time value. 
Then change to the window for the attributes (the window in the upper left corner of your screen) and go to \texttt{Settings} and check the \texttt{Auto-apply} box. 
Your choice will be confirmed by a red "Auto-apply is ON" at the bottom of the window. \newline

\subsection{Annotate Markables and Attributes}
The window in the middle of your screen now should show the tweets. To annotate a markable choose a word, phrase, clause or sentence by clicking the left mouse-button and mark the respective span while holding the mouse-button. After releasing the left mouse-button a menu will pop up showing the list of markables. Left-click on the corresponding markable and the span will turn to the respective color of the markable. Then change to the attribute window in the upper left corner, click on the tab for the markable you just annotated and set the value(s) for the attributes. (In case you did not turn on "Auto-apply" press "Apply".)

\subsection{Deleting Markables}
Just do a \underline{right-click} on a colored markable and left-click \texttt{Delete this markable}.

\subsection{Annoating Discontinuous Markables}
Sometimes you will find discontinuous emo-expressions. The most prominent example are German particle verbs (e.g. weh tun). To annotate them as one single element first annotate one element (it will turn to the respective markable color) as usual, left-click to highlight it and then mark the other element. A menu will pop up saying "Add to this markable". Left-click on that.

\subsection{Annotating anaphref}
Imaginge you have a sentence like: \textit{Peter ist schlau und gut aussehen tut er auch noch. (Peter is clever and he looks good, too.)}. Here you should annotate \textit{er} as anaphref to \textit{Peter}. To do so first annotate \textit{Peter} and \textit{er} as single targets. Then left-click on \textit{Peter} to highlight the markable and then do a \underline{right-click} on \textit{er}. A menu will pop up saying \texttt{"Mark as anaphric antecendent of target"}. Left-click on that.

\subsection{Annotating sentiment\_ref}
The same technique as above applies. First mark a sentiment span as usual. Second annotate the other markable (whatever it may be) as usual. Left-click on the first markable to highlight it then \underline{right-click} on the second markable and in the pop-up menu left-click on "Mark sentiment which this span belongs to".

%%%%%%%%%%%%%%%%%%%%%%%%%%%%%%%%%%%%%%%%%%%%%%%%%%%%%%%%%%%%%%%%%%%%%%%%%%%%
%%%%%%%%%%%%%%%%%%%%%%%%%%%%%%%%%%%%%%%%%%%%%%%%%%%%%%%%%%%%%%%%%%%%%%%%%%%%

\section{FAQ}
In this section some questions that may arise during the annotation process are answered. Generally the best guide for the annotation is your intuition. If you think there is a sentiment in a tweet then first search for a target. If there is no target you do not need to annotate the tweet.

\subsection{I have difficulties determining the overall sentiment of a tweet. Is there a common modus operandi?}
Please keep in mind that sentiments should always be annotated \underline{target-oriented}. Look at this tweet: \textit{Ich hab Maria ja schon so vermisst und so , ne ? (I kind of did miss Maria, right?)}. Here you might think the tweet has a negative sentiment, because the author missed Maria which is not a good thing. But we are always only concerned about the sentiment relation between the source and the target! And here it is a positive sentiment, because the author missed Maria because he likes her.

\subsection{How to annotate sarcasm?}
If the apparent form of a sentiment is positive (and therefore should be annotated as a positive sentiment) but you can detect sarcasm please annotate the polarity as negative and additionally set \texttt{sarcasm} to \texttt{true} in the attribute window. This applies to the sentiment tag as well as to emo-expressions.

\subsection{Should subjunctives be annotated?}
Yes. Have a look on the following tweet: \textit{W\"are toll wenn das der n\"achste " Call of Duty " - Teil sein w\"urde , liebe @GameStar. (Dear @gamestar, it would be great if this would be the sequel of "Call of Duty".)}. Although this is a subjunctive sentiment and is not "reality" right now you should annotate this, too.

\subsection{Should multiple sentiment layers be annotated?}
Yes. For example in the tweet \textit{72 jähriger Leser regt sich bei mir über YouTube und Gema auf :) (72 year old reader troubles over YouTube and Gema at me :) )} you can find two layers of sentiments. The first layer is the fact that the reader troubles over YouTube and Gema. The second layer is the fact that the author of the tweet is amused about the first fact. So here you have to annotate two different sentiments. The first is negative and has "YouTube" and "Gema" as target (with "troubles over" as emo-expression). The second is positive and has "72 jähriger Leser" as target and "mir" as source. Do not forget to use the \texttt{sentiment\_ref} attribute to annotate which sentiment refers to which target.

\subsection{What about English emo-expressions?}
Although we annotate German tweets it is not unusual to find English emo-expressions. For example: \textit{Das Erzgebirge rockt DSDS ! (The Erzgebirge rocks DSDS !)}. You should annotate English emo-expressions, too, if they are common in German.

\subsection{How to cope with self-reference inside a tweet?}
Have a look at this tweet: \texttt{Mein Bauch mag mich nicht weil ich Käse gegessen hab :-) (My belly does not like me because I ate cheese :-) )}. In this tweet the source and the target are the same one single person. Do not get irritated by that. As long as you can annotate a target (and a source) everything is fine. In this example "Mein Bauch" is the source and "mich" is the target.

\subsection{I can find the same source two times in one tweet. What to do?}
Look at this tweet: \texttt{Ich hab übrigens ne 1-  in Englisch . Find ich gut ! (By the way, I got an A- in English. I Like it!)}
Here the "Ich" in the first sentence needs not to be annotated. It is sufficient to annotate just the "Ich" in the second sentence. Since there is no anaphora present you will not need to mark an anaphoric antecedent as well. \newline
Here you should annotate both sentences as one sentiment with "ne 1- in Englisch" as target, the "I" in the second sentence as source and "gut" as emo-expression.


%%%%%%%%%%%%%%%%%%%%%%%%%%%%%%%%%%%%%%%%%%%%%%%%%%%%%%%%%%%%%%%%%%%%%%%%
\end{document}





