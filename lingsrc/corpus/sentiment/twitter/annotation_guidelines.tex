\documentclass[11pt,a4paper]{article}

%%%%%%%%%%%%%%%%%%%%%%%%%%%%%%%%%%%%%%%%%%%%%%%%%%%%%%%%%%%%%%%%%%
%% Libraries
\usepackage[driver=pdftex,vmargin=2cm,hmargin=2cm]{geometry}
\usepackage[usenames,dvipsnames]{xcolor}
\usepackage{amsmath}
\usepackage{array}
\usepackage{booktabs}
\usepackage{color}
\usepackage{framed}
\usepackage[colorlinks=true]{hyperref}
\usepackage{multirow}
\usepackage{natbib}
\usepackage{paralist}
\usepackage{url}
\usepackage{tikz}
%% \usepackage{gb4e}
\usepackage{xargs}

\hypersetup{
  colorlinks,
  citecolor=Violet,
  linkcolor=Red,
  urlcolor=Blue}

%%%%%%%%%%%%%%%%%%%%%%%%%%%%%%%%%%%%%%%%%%%%%%%%%%%%%%%%%%%%%%%%%%
%% Commands
\definecolor{dodgerblue4}{RGB}{16,78,139}
\definecolor{orange3}{RGB}{205,133,0}
\definecolor{DarkSlateBlue}{RGB}{72,61,139}

\newlength{\cwidth} \setlength{\cwidth}{0.18\textwidth}
\newenvironment{example}{\begin{center}\begin{exe}\ex}{\end{exe}\end{center}}
\newcommand{\xmltag}[1]{\textcolor{black}{{\small$<$#1$>$}}}
\newcommand{\sentiment}[2][negative]{$<$sentiment
  sentiment\_polarity="#1"$>$\textcolor{dodgerblue4}{#2}$<$/sentiment$>$}
\newcommand{\source}[1]{\xmltag{source}\textcolor{orange3}{#1}\xmltag{/source}}
\newcommand{\target}[1]{\xmltag{target}\textbf{#1}\xmltag{/target}}
\newcommand{\negation}[1]{\xmltag{negation}\textcolor{red}{#1}\xmltag{/negation}}
\newcommand{\intensifier}[2][1]{\xmltag{intensifier
    intensity="#1"}\textcolor{DarkSlateBlue}{#2}\xmltag{/intensifier}}
\newcommandx{\emoexpression}[3][1=negative, 2=1]{
  \xmltag{emo-expression polarity="#1" intensity="#2"}
  \textcolor{green}{#3}\xmltag{/emo-expression}}

\renewenvironment{example}{\begin{center}\itshape}{\upshape\end{center}}

%%%%%%%%%%%%%%%%%%%%%%%%%%%%%%%%%%%%%%%%%%%%%%%%%%%%%%%%%%%%%%%%%%
%% Commands
\author{Wladimir Sidorenko}
\date{\today}
\title{Guidelines for the Annotation of the Sentiment Corpus}

%%%%%%%%%%%%%%%%%%%%%%%%%%%%%%%%%%%%%%%%%%%%%%%%%%%%%%%%%%%%%%%%%%
%% Main
\begin{document}
\maketitle{}
\section{Introduction}
Welcome. In this assignment your task is to annotate sentiments in a
corpus of Twitter messages.

\subsection{Annotation Tool}

For annotation, you will use \texttt{MMAX2} -- a freely available
annotation tool which can be downloaded under the following link:

\url{http://sourceforge.net/projects/mmax2/files/mmax2/mmax2_1.13.003/MMAX2_1.13.003b.zip/download}

After you have downloaded and unpacked the archive, change to the
newly appeared directory \texttt{1.13.003/MMAX2} in your shell and
execute the following commands:

\texttt{> chmod u+x ./mmax2.sh}

\texttt{> nohup ./mmax2.sh \&}

{\setlength{\parindent}{0pt} If you have never used \texttt{MMAX2}
  before, please read the document \texttt{mmax2quickstart.pdf} first,
  which you can find in the subdirectory \texttt{MMAX2/Docs}.}

\subsection{Corpus Files}

You should also have received a copy of corpus'project files as a
tar-gzipped archive.  Please unpack this archive using the command:

\texttt{> tar -xzf twitter-sentiment.tgz}

{\setlength{\parindent}{0pt} After that, a directory called
  \texttt{mmax-prj} will appear in your current folder.  Change to
  your \texttt{MMAX2} window and click on the menu \texttt{File ->
    Load}.  Select the path to the unpacked \texttt{mmax-prj} folder
  in the appeared popup menu\footnote{Make sure that the path does not
    contain any white spaces. Otherwise, \texttt{MMAX2} will fail to
    load the project file with the error message
    \emph{java.netMalformedURLException.no protocol: words.dtd}} and
  select one of the \texttt{*.mmax} files there. Now, source data will
  be loaded into \texttt{MMAX} program.}

%%%%%%%%%%%%%%%%%%%%%%%%%%%%%%%%%%%%%%%%%%%%%%%%%%%%%%%%%%%%%%%%%%

\section{Tags and Attributes}
In this annotation, we will use following tags with following
attributes:
\begin{enumerate}
\item \textit{sentiment}-tag with attributes:
  \begin{enumerate}
  \item polarity,
  \item intensity,
  \item sarcasm,
  \item sentiment-ref;
  \end{enumerate}
\item \textit{source}-tag with attributes:
  \begin{enumerate}
  \item anaph-ref,
  \item sentiment-ref;
  \end{enumerate}
\item \textit{target}-tag with attributes:
  \begin{enumerate}
  \item anaph-ref,
  \item sentiment-ref;
  \end{enumerate}
\item \textit{emo-expression}-tag with attributes:
  \begin{enumerate}
  \item polarity,
  \item intensity,
  \item emo-expression-ref,
  \item sentiment-ref;
  \end{enumerate}
\item \textit{intensifier}-tag with attributes:
  \begin{enumerate}
  \item degree,
  \item sentiment-ref;
  \end{enumerate}
\item \textit{diminisher}-tag with attributes:
  \begin{enumerate}
  \item degree,
  \item sentiment-ref;
  \end{enumerate}
\item and the \textit{negation}-tag.
\end{enumerate}
A more detailed description of the meaning of these tags and
attributes is given in the next subsections.

\subsection{sentiment}
\texttt{sentiment}s are our most important markables.  With these tags
you should mark statements which express some polar assessing opinions
about particular subjects.  Opinions which are polar but which do not
assess anything like, for example, greetings or vague emotional
statements, should not be marked as sentiments.

Sentiment tags can include:
\begin{itemize}
\item single noun phrases, possibly with their prepositional
  attributes, e.g. ``\textit{Auf dem Tresor lag
    \xmltag{sentiment}ein h\"assliches Buch\xmltag{/sentiment}}.'';
\item clauses, e.g. ``\textit{\xmltag{sentiment}Ich hasse B\"ucher
  ohne Inhaltsangabe.\xmltag{/sentiment}}'';
\item multiple sentences in cases when these sentences jointly form
  a sentiment, e.g. ``\textit{\xmltag{sentiment}Sie denken, reden,
    riechen, lieben, schmecken, ficken Plastik. Sie haben das so
    gelernt in der
    Plastik-Werbewelt.\xmltag{/sentiment}}''\footnote{In this
    example, \textit{Plastik-Werbewelt} is being assessed.  It is,
    however, impossible to state that a sentiment is expressed by
    looking at only one of both sentences.}.
\end{itemize}

Sentiment tags can have following attributes with following possible
values:\\
\begin{tabular}[t]{|l|c|p{0.5\textwidth}|}\hline
  Attribute & Value & Value's Meaning\\\hline
  %%%%%%%%%%%%%%%%%%%

  \multirow{3}{*}{polarity} & \textit{positive} & this sentiment
  expresses positive attitude to its respective target\\\cline{2-3}

  & \textit{negative\newline(default)} & this sentiment
  expresses negative attitude to its respective target\\\cline{2-3}

  & \textit{comparison} & this sentiment expresses comparison of two
  targets with preference given to one of them\\\hline
  %%%%%%%%%%%%%%%%%%%

  \multirow{3}{*}{intensity} & \textit{weak} & stylistically weakly
  marked expression which tends more towards a neutral
  sentence\\\cline{2-3}

  & \textit{middle\newline(default)} & middle stylistic
  expressivity\\\cline{2-3}

  & \textit{strong} & stylistically very expressive sentiment\\\hline
  %%%%%%%%%%%%%%%%%%%

  \multirow{2}{*}{sarcasm} & \textit{true} & this polar attitude is
  derisive, i.e. its actual polarity is the opposite of its apparent
  form (that means that an apparent praise is in fact meant as rebuke
  and vice versa - but this actual sense can only be derived on the
  basis of outside context or common knowledge)\\\cline{2-3}

  & \textit{false\newline(default)} & no sarcasm present - polar
  attitude has literal meaning\\\hline

  %%%%%%%%%%%%%%%%%%%
  sentiment-ref & \textit{$->$\newline(directed edge)} & in case a
  single sentiment is not a contiguous span of text but is rather split into
  parts, you can draw an edge with this label from additional parts of
  a sentiment to its main part\\\hline
\end{tabular}

\subsection{source}
The \texttt{source} tag is used to mark the immediate author(s) or
experiencer(s) of a sentiment expression.  These are usually speakers
or writers of an assessment or persons whose assessing attitude is
being cited.  In cases when a source is not explicitly present in a
message, it is implicitly assumed to be the author who posted the
tweet.

An example of a \texttt{source} expression is the pronoun ``Sie''
(``she'') in the following sentence:
\begin{example}
  \textit{\source{Sie} mag die neue Farbe
    nicht}\\ (\textit{\textbf{She} doesn't like the new color})
\end{example}
Sources are usually expressed by noun phrases.  In cases, when a
sentiment contains multiple sources joined by coordinative
conjunctions, you should mark each conjoined source with a separate
tag (see the example below):
\begin{example}
  \textit{\source{Ihr} und \source{ihrer Mutter} gef\"allt die neue
    Farbe nicht}\\ (\textit{\textbf{She} and \textbf{her mother} do not
    like the new color})
\end{example}

The attributes of the \texttt{source} tag with their possible values
and meanings are listed in the table below:\\
\begin{tabular}{|l|c|p{0.5\textwidth}|}\hline
  Attribute & Value & Value's Meaning\\\hline sentiment-ref &
  \textit{$->$\newline(directed edge)} & directed edge pointing from
  source element to its respective sentiment span.  You only need to
  draw this edge if two different sentiment relations are overlapping
  on the same span and we need to disambiguate to which sentiment
  relation given source belongs.  Another case when you should draw
  this edge is when source is expressed outside the sentiment
  span.\\\hline

  anaph-ref & \textit{$->$\newline(directed edge)} & directed edge
  pointing from a source element expressed by a pronoun or pronominal
  adverb to its respective antecedent\\\hline
\end{tabular}

\subsection{target}
With \texttt{target}-tags you should mark objects that are being
assessed in a sentiment expression.  One sentiment relation should
always have at least one target.  In cases when multiple targets are
conjoined in a single expression, you should mark each of these
targets separately in the same way as it is done for sources.

An example of a sentiment target is given in the sentence below:
\begin{example}
  \textit{Mein Bruder ist nicht gerade begeistert von \target{Call of
      Duty}.}\\ (\textit{My brother is not exactly impressed by
    \textbf{Call of Duty}.})
\end{example}

In cases when a sentiment is represented by a comparison, you should
mark with a separate target tag each of the compared objects.
Additionally, for object which is being preferred in the comparison,
you should set the \textit{preferred} attribute to \textit{true} in a
popup menu of the tag (see the example below):
\begin{example}
  \textit{Ich mag $<$target preferred=``true''$>$\textbf{Domino-Eis}$<$/target$>$ lieber als \target{Magnum}.}\\
    (\textit{I like \textbf{Domino ice cream} more than \textbf{Magnum}.})
\end{example}

The attributes and attribute values of the \texttt{target} tags almost
fully correspond to the respective attributes of the \texttt{source}
tag.  The only additional attribute which targets can take on is the
\textit{preferred} attribute described above.

\subsection{emo-expression}
With the \texttt{emo-expression} tag, you should mark words or phrases
which have some polar connotation in their lexical meaning in a given
context.  Typical examples of emotional expressions are adjectives and
adverbs (e.g. ``gut'', ``sch\"on'', ``traurig'' etc.), nouns
(e.g. ``Held'', ``Vorbild'', ``Schurke'' etc.), verbs
(e.g. ``lieben'', ``hassen'', ``beschimpfen'' etc.), idiomatic
expressions (e.g. ``auf die Nerven gehen'' etc.), smileys,
interjections etc.

Please note that, depending on the context in which it is used, one
and the same word can have different lexical polarities or also loose
its polar meaning completely in certain settings.  For example, the
word ``Edelstein'' (``jewel'') in the following two sentences only
becomes a polar term and, correspondingly, an \texttt{emo-expression}
in the second example when its meaning is metaphoric:
\begin{example}
  \textit{Koh-i-Noor ist das teuerste Juwel heutzutage.}\\
  \textit{Koh-i-Noor is the most expensive jewel nowadays.}
\end{example}
\begin{example}
  \textit{Dieser Wein ist ein wahres \emoexpression{Juwel} in meiner Kollektion.}\\
  \textit{This wine is a true \textbf{jewel} in my collection.}
\end{example}

You should only mark a term as an emo-expression, if the meaning of
this term in given context is polar.  The polarity attribute of the
tag should reflect the \textit{lexical} or also called \textit{prior}
polarity of the marked term disregarding possible negations in the
context.  It means that in a sentence like ``\textit{Das war keine
  gute Idee}'' (``\textit{It was not a good idea}''), the polarity of
the emotional expression ``\textit{gute}'' (``\textit{good}'') should
still be positive, even though the negative article ``\textit{keine}''
turns its polarity to the opposite.

Furthermore, you should also mark emotional expressions in cases when
these expressions do not evoke any sentiment.  If an emo-expression is
a part of a sentiment however, its prior polarity should be determined
from the perspective of sentiment's target.  It means, that in a
sentence like ``\textit{Ich vermisse meinen Freund.}''  (``\textit{I
  am missing my boy friend.}''), the polarity of the term
``\textit{vermisse}'' (``\textit{missing}'') should be positive
because it expresses a positive attitude to the target
``\textit{Freund}'' (``\textit{boy friend}'') whose absence is
perceived as discomfort.

A complete list of possible attributes for emo-expressions is listed
in the table below:

\begin{tabular}{|l|c|p{0.5\textwidth}|}\hline
  Attribute & Value & Value's Meaning\\\hline
  %%%%%%%%%%%%%%%%%%%

  \multirow{2}{*}{polarity} & \textit{positive} & this emotional
  expression has positive polar meaning\\\cline{2-3}

  & \textit{negative\newline(default)} & this emotional expression has
  negative polar meaning\\\hline

  %%%%%%%%%%%%%%%%%%%

  \multirow{3}{*}{intensity} & \textit{weak} & stylistically weakly
  marked emotional expression\\\cline{2-3}

  & \textit{middle\newline(default)} & middle stylistic
  expressivity\\\cline{2-3}

  & \textit{strong} & stylistically very expressive emotional
  expression\\\hline

  %%%%%%%%%%%%%%%%%%%

  emo-expression-ref & \textit{$->$\newline(directed edge)} & in cases
  when a single emotional expression is represented not as a
  contiguous span of text but rather split into separate parts, this
  edge should point from the separated parts to the main part of the
  emo-expression\\\hline
\end{tabular}

\subsection{intensifier}
Intensifiers are elements that increase the polar meaning of an
emotional expression.  Intensifiers are usually expressed by adverbs
or adjectives like \textit{sehr} (\textit{very}), \textit{ziemlich}
(\textit{rather}) etc.  Other ways of expressing intensifiers are
however still possible.

Intensifier elements can have following attributes with following
values:

\begin{tabular}{|l|c|p{0.5\textwidth}|}\hline
  \multirow{2}{*}{degree} & \textit{medium (default)} & intensifier
  moderately increases the polar sense of an emotional expression,
  e.g. \textit{ziemlich}, \textit{recht} etc.\\\cline{2-3}

  & \textit{strong} & intensifier strongly increases polar sense of
  emotional expression, e.g. \textit{sehr}, \textit{super},
  \textit{stark} etc.\\\hline

  %%%%%%

  sentiment-ref\footnote{TODO: do we need it?} &
  \textit{$->$\newline(directed edge)} & a directed edge pointing from
  the intensifier to the sentiment it belongs to. By analogy to
  negations, you should only draw this edge in cases multiple
  sentiment relations overlap each other and it's not obvious to which
  of these sentiments the given intensifier belongs to, i.e. in cases
  when the given intensifier is included in both sentiment
  spans\\\hline
\end{tabular}

\subsection{diminisher}
Diminishers are elements that decrease the polar lexical meaning of an
emotional expression.  Like intensifiers, diminishers are usually
expressed by adjectives or adverbs (e.g. ``\textit{wenig}'',
``\textit{kaum}'', ``\textit{ein bisschen}''), however other means of
expressing diminishing notions are still possible.

Diminishers have the same set of attributes as intensifiers with the
only difference in the interpretation of the attribute
\textit{degree}:

\begin{tabular}{|l|c|p{0.5\textwidth}|}\hline
  \multirow{2}{*}{degree} & \textit{weak (default)} & degree by which
  this diminisher decreases polar sense of respective emotional
  expression (1 means slight decrease), e.g. \textit{wenig},
  \textit{bisschen} etc.\\\cline{2-3}

  & \textit{strong} & this diminisher strongly decreases polar sense
  of emotional expression, e.g. \textit{kaum} etc.\\\hline

  sentiment-ref\footnote{TODO: Do we need it???} &
  \textit{$->$\newline(directed edge)} & a directed edge pointing from
  diminisher to sentiment it pertains to. You should only draw this
  edge in cases when multiple sentiment relations overlap with each
  other and it's not obvious to which of these sentiments given
  diminisher belongs to, i.e. in cases when given diminisher is
  included in both sentiment spans\\\hline
\end{tabular}

\subsection{negation}
Negations are elements which change the polarity of an emotional
expression to the opposite.  Negations can, for example, be
represented by the negation particle ``\textit{nicht}'', the negative
article ``\textit{kein}'' or any other means.

Negations are closely related to the diminishers in that they reduce
the prior polar sense of an emotional expression.  The difference
between these two elements is that negations reduce this polar sense
completely turning it to the opposite (e.g. ``\textit{nicht
  verst\"andlich}'' -- ``\textit{not comprehensible}'') whereas
diminishers still allow some part of the prior polar meaning to remain
(compare ``\textit{weniger verst\"andlich}'' -- ``\textit{less
  comprehensible}'').

The only attribute that a negation element can take on is
\textit{sentiment-ref} described below:\footnote{TODO: Do we need it?}

\begin{tabular}{|l|c|p{0.5\textwidth}|}\hline
  sentiment-ref\footnote{TODO: Do we need it???} &
  \textit{$->$\newline(directed edge)} & a directed edge pointing from
  negation to the sentiment it pertains to. You should only draw this
  edge in cases when multiple sentiment relations overlap each other
  and it is not obvious to which of these sentiments given negation
  belongs to, i.e. in cases when given negation is included in both
  sentiment spans\\\hline
\end{tabular}

Please note, that negations, diminishers, and intensifiers should only
be marked as such if they have an immediate affect on the insity or
polarity of a sentiment.  It means that these elements should not be
marked if no sentiment relation is present.

\section{FAQ}
This section provides some examples and more elaborate descriptions of
difficult and controversial annotation cases.  It is subdivided into
subsection depending on which annotation element caused difficulties:

\subsection{sentiment}
\subsection{source}
\subsection{target}
\subsection{emo-expression}
\subsection{intensifier}
\subsection{diminisher}
\subsection{negation}
\end{document}


\section{Examples}
Please DO NOT mark as sentiments following cases:
\begin{itemize}
\item Statements describing some emotional states for which no
  target can be derived or found (e.g. \textit{Ich f\"uhle mich so
    traurig :(} - \textit{I am feeling so blue :(});
\item Statements describing some objective facts, even if possible
  consequences of these facts can be assumed to have negative
  influence on the author (e.g. \textit{Wenn ein Floh einen Menschen
    bei\ss{}t und ihn mit erbrochenem Blut infiziert, werden die
    Pestbakterien ins Gewebe \"ubertragen.} - \textit{When a flea
    bites a human and contaminates the wound with regurgitated
    blood, the plague carrying bacteria are passed into the
    tissue.});
\item Interrogative clauses in case they ask whether some polar
  opinion is true or not. (e.g. \textit{Findet die Mehrheit
    der Bev\"olkerung CDU toll?} - \textit{Does the majority
    of population consider CDU great?});
\end{itemize}

\subsection{negation}
\texttt{\xmltag{negation}}s are lexical or syntactic elements that
reverse the primary polar meaning of emo-expressions to the opposite,
so that the overall polarity of the whole sentiment is different to
the polarity of emo-expressions belonging to it. A typical example of
negation is \textit{nicht} in sentences like \textit{Ein guter Schritt
  war diese Entscheidung \textbf{nicht}.} (\textit{This decision was
  \textbf{not} a good move.}).

\subsubsection{Examples}
You SHOULD only mark negative elements that have impact on
sentiment's polarity. Negating elements having no such impact
shold not be marked. Negation elements are usually represented by:
\begin{itemize}
\item Negation particle \textit{nicht} (\textit{not}),
  e.g. \textit{Klug ist dieser Hund sicherlich \textbf{nicht}}
  (\textit{This dog is certainly \textbf{not} clever});

\item Negative article \textit{kein} e.g. \textit{Er war
  \textbf{kein} Vorbild f\"ur seine Kinder} (\textit{He was
  \textbf{not} a good role model for his children});

\item Indefinite pronouns like \textit{niemand}, \textit{keiner}
  etc., e.g. \textit{\textbf{Niemand} hielt ihn f\"ur einen
    ehrlichen Menschen.} (\textit{\textbf{Nobody} considered him an
    honest man});

\item Any lexical or idiomatic unit in case they turn sentiment's
  polarity to the opposite, e.g. \textit{Ich \textbf{zweifle}, dass
    das neue iPhone ein besseres Display hat.} (\textit{I
    \textbf{doubt} the new iPhone has a better display}).
\end{itemize}

\subsubsection{Counter-Examples}
DO NOT mark as negations elements that have no effect on sentiment's
polarity. For example, in the sentence \textit{Ich mag Leute, die nicht
  nur an sich selbst denken.} (\textit{I like people who not only care
  about themselves.}) \textit{nicht} (\textit{not}) should not be
marked as negation, since it does not change the positive polarity
expressed by \textit{m\"ogen} (\textit{like}).

\subsubsection{Attributes}
Negations only have one possible attribute, namely
\textit{sentiment-ref} that is a directed edge pointing from negation
to sentiment it belongs to. You should only draw this edge in cases multiple sentiment relations overlap each other and it's not
obvious to which of these sentiments the given negation belongs to,
i.e. in cases the given negation is included in both sentiment spans.

%%%%%%%%%%%%%%%%%%%%%%%%%%%%%%%%%%%%%%%%%%%%%%%%%%%%%%%%%%%%%%%%%%



%%%%%%%%%%%%%%%%%%%%%%%%%%%%%%%%%%%%%%%%%%%%%%%%%%%%%%%%%%%%%%%%%%


%%%%%%%%%%%%%%%%%%%%%%%%%%%%%%%%%%%%%%%%%%%%%%%%%%%%%%%%%%%%%%%%%%

%%%%%%%%%%%%%%%%%%%%%%%%%%%%%%%%%%%%%%%%%%%%%%%%%%%%%%%%%%%%%%%%%%

\section{FAQ and Examples}
Things you SHOULD mark as sources:
\begin{itemize}
\item Original author(s) of polar opinions, i.e. those persons,
  groups or officials who experience sympathy or aversion for
  something e.g. \textit{\textbf{Michael} meinte, das w\"are die
    beste L\"osung.}  (\textit{\textbf{Michael} thought, it would be the
    best solution.});
\end{itemize}

\subsubsection{Counter-Examples}
Things you SHOULD NOT mark as sources:
\begin{itemize}
\item Persons how are neutrally citing someone else's opinion,
  e.g. in the sentence \textit{Nach Tatjanas Worten war Michael sehr
    dar\"uber ver\"argert.} (\textit{According to Tatjana, Michael
    was very angry about that.}) Here you should only mark
  \textit{Michael} as sentiment's source and not \textit{Tatjana}. Please note, that in case an author
  supports or contradicts someone else's opinion, two sentiment
  relations should be made - one with the original author as source,
  and one with the supporter/opponent of the opinion enclosed in
  source tags.\footnote{In that case sentiments will overlap, and
    you also should mark these sources with \textit{sentiment-ref}-tag.}
\end{itemize}


\subsubsection{Examples}
Elements you SHOULD mark as targets:
\begin{itemize}
\item Persons, things or objects the opinion is made about,
  e.g. \textit{Ich hasse \textbf{lange Schlangen}} (\textit{I hate
    \textbf{long queues}});

\item Descriptions of actions or events, e.g. \textit{Ich hasse es,
  \textbf{wenn St\"arkere Schw\"achere drangsalieren}} (\textit{I
  hate it \textbf{when stronger people bully weaker ones}});

\item Compared elements of a comparison,
  e.g. \textit{\textbf{\"Altere Handys} halten l\"anger als
    \textbf{moderne Smartphones}} (\textit{\textbf{Older mobile
      phones} outwear \textbf{modern smartphones}}).
\end{itemize}

\subsubsection{Attributes}
\textit{target} attributes fully correspond to the attributes of
\textit{source} elements. So please refer to section \ref{src-attrs}
for the description of possible attributes for targets.

\subsection{\xmltag{emo-expression}}
\texttt{\xmltag{emo-expression}}s are lexical elements that bear some
polarity meaning on their own. Usually this are words like
\textit{gut} (\textit{good}), \textit{schlechter} (\textit{worse}),
\textit{m\"ogen} (\textit{like}), or \textit{hassen}
(\textit{hate}). Such elements can be expressed either as single words
or as complex expressions (in case of idiomatic expressions or support
verbs).

\subsubsection{Examples}
Elements you SHOULD mark as emo-expressions:
\begin{itemize}
\item Adjectives and adverbs bearing polar attitudes \textit{Peter
  hatte \textbf{bessere} Noten in der Schule als sein Bruder}
  (\textit{Peter had \textbf{better} grades at school than his
  brother.});

\item Verbs expressing attitude of a speaker to target,
  e.g. \textit{Mir \textbf{gefiel} die neue House-Staffel}
  (\textit{I \textbf{liked} the new House series});

\item Idiomatic expression including support verbs
  e.g. \textit{\textbf{Zum Teufel} soll die neue Regierung
    \textbf{gehen}} (\textit{The new government should \textbf{go to
      hell}}).

\item Smileys in case they really express an emotional attitude and
  are not used for politeness or without any particular meaning
  e.g. \textit{Gleich in Braunschweig mit Kameraden treffen
    \textbf{:)}} (\textit{Will soon meet friends in Braunschweig
    \textbf{:)}}).
\end{itemize}

\subsubsection{Attributes}
\xmltag{emo-expression} tags have the following attributes with their
corresponding values:

%%%%%%%%%%%%%%%%%%%%%%%%%%%%%%%%%%%%%%%%%%%%%%%%%%%%%%%%%%%%%%%%%%

