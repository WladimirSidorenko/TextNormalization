\documentclass[11pt,a4paper]{article}

%%%%%%%%%%%%%%%%%%%%%%%%%%%%%%%%%%%%%%%%%%%%%%%%%%%%%%%%%%%%%%%%%%
%% Libraries
\usepackage[driver=pdftex,vmargin=2cm,hmargin=2cm]{geometry}
\usepackage[usenames,dvipsnames]{xcolor}
\usepackage{amsmath}
\usepackage{array}
\usepackage{booktabs}
\usepackage{color}
\usepackage{framed}
\usepackage[colorlinks=true]{hyperref}
\usepackage{multirow}
\usepackage{natbib}
\usepackage{paralist}
\usepackage{url}
\usepackage{tikz}
%% \usepackage{gb4e}
\usepackage{xargs}

\hypersetup{
  colorlinks,
  citecolor=Violet,
  linkcolor=Red,
  urlcolor=Blue}

%%%%%%%%%%%%%%%%%%%%%%%%%%%%%%%%%%%%%%%%%%%%%%%%%%%%%%%%%%%%%%%%%%
%% Commands
\definecolor{dodgerblue4}{RGB}{16,78,139}
\definecolor{orange3}{RGB}{205,133,0}
\definecolor{DarkSlateBlue}{RGB}{72,61,139}

\newlength{\cwidth} \setlength{\cwidth}{0.18\textwidth}
\newenvironment{example}{\begin{center}\begin{exe}\ex}{\end{exe}\end{center}}
\newcommand{\xmltag}[1]{\textcolor{black}{{\small$<$#1$>$}}}
\newcommand{\sentiment}[2][negative]{$<$sentiment
  sentiment\_polarity="#1"$>$\textcolor{dodgerblue4}{#2}$<$/sentiment$>$}
\newcommand{\source}[1]{\xmltag{source}\textcolor{orange3}{#1}\xmltag{/source}}
\newcommand{\target}[1]{\xmltag{target}\textbf{#1}\xmltag{/target}}
\newcommand{\negation}[1]{\xmltag{negation}\textcolor{red}{#1}\xmltag{/negation}}
\newcommand{\intensifier}[2][1]{\xmltag{intensifier
    intensity="#1"}\textcolor{DarkSlateBlue}{#2}\xmltag{/intensifier}}
\newcommandx{\emoexpression}[3][1=negative, 2=1]{
  \xmltag{emo-expression polarity="#1" intensity="#2"}
  \textcolor{green}{#3}\xmltag{/emo-expression}}

\renewenvironment{example}{\begin{center}\itshape}{\upshape\end{center}}

%%%%%%%%%%%%%%%%%%%%%%%%%%%%%%%%%%%%%%%%%%%%%%%%%%%%%%%%%%%%%%%%%%
%% Commands
\author{Wladimir Sidorenko}
\date{\today}
\title{Guidelines for Annotation of Sentiment Corpus}

%%%%%%%%%%%%%%%%%%%%%%%%%%%%%%%%%%%%%%%%%%%%%%%%%%%%%%%%%%%%%%%%%%
%% Main
\begin{document}
\maketitle{}
\section{Introduction}
Welcome. In this assignment your task is to annotate sentiments in a
corpus of Twitter messages.

\subsection{Annotation Tool}

For annotation, you will use \texttt{MMAX2} -- a freely available
annotation tool which can be downloaded under the following link:

\url{http://sourceforge.net/projects/mmax2/files/mmax2/mmax2_1.13.003/MMAX2_1.13.003b.zip/download}

After you have downloaded and unpacked the archive, change to the
newly appeared directory \texttt{1.13.003/MMAX2} in your shell and
execute the following commands:

\texttt{> chmod u+x ./mmax2.sh}

\texttt{> nohup ./mmax2.sh \&}

{\setlength{\parindent}{0pt} If you have never used \texttt{MMAX2}
  before, please read the document \texttt{mmax2quickstart.pdf} which
  you can find in the subdirectory \texttt{MMAX2/Docs}.}

\subsection{Corpus Files}

You should also have received a copy of corpus' project files as a
tar-gzipped archive.  Please unpack this archive using the
command:

\texttt{> tar -xzf twitter-sentiment.tgz}

{\setlength{\parindent}{0pt} After that, a directory called
  \texttt{mmax-prj} will appear in your current folder.  Change to
  your \texttt{MMAX2} window and click on the menu \texttt{File ->
    Load}.  Select path to the unpacked \texttt{mmax-prj} folder in
  the appeared popup menu\footnote{Be sure the path itself contains no
    white spaces. Otherwise \texttt{MMAX2} will fail to load the
    project file with the error message
    \emph{java.netMalformedURLException.no protocol: words.dtd}} and
  select one of the \texttt{*.mmax} files there. Now, source data will
  be loaded into \texttt{MMAX} program.}

%%%%%%%%%%%%%%%%%%%%%%%%%%%%%%%%%%%%%%%%%%%%%%%%%%%%%%%%%%%%%%%%%%

\section{Tags and Attributes}
In this annotation, we will use following tags with following
attributes:
\begin{enumerate}
\item \textit{sentiment}-tag with attributes:
  \begin{enumerate}
  \item polarity,
  \item intensity,
  \item sarcasm,
  \item sentiment-ref;
  \end{enumerate}
\item \textit{source}-tag with attributes:
  \begin{enumerate}
  \item anaph-ref,
  \item sentiment-ref;
  \end{enumerate}
\item \textit{target}-tag with attributes:
  \begin{enumerate}
  \item anaph-ref,
  \item sentiment-ref;
  \end{enumerate}
\item \textit{emo-expression}-tag with attributes:
  \begin{enumerate}
  \item polarity,
  \item intensity,
  \item emo-expression-ref,
  \item sentiment-ref;
  \end{enumerate}
\item \textit{intensifier}-tag with attributes:
  \begin{enumerate}
  \item degree,
  \item sentiment-ref;
  \end{enumerate}
\item \textit{diminisher}-tag with attributes:
  \begin{enumerate}
  \item degree,
  \item sentiment-ref;
  \end{enumerate}
\item and the \textit{negation}-tag.
\end{enumerate}
A more detailed description of the meaning of these tags and
attributes is given in the next subsections.

\subsection{sentiment}
\texttt{sentiment}s are our most important markables.  In these tags
you should include statements which express somer polar opinions about
some subject as well as comparisons of two subjects.  We DO NOT regard
as sentiments opinions which do not convey any polarity nor do we
consider as such polar expressions which are not related to any
particular subject.  In other words, simple informative statements,
apologies, comands or hesitations should not be marked with these
tags.

For determining the boundaries of a sentiment, look by which
grammatical means given sentiment is expressed -- if it is a single
noun phrase (e.g. ``\textit{Auf dem Tresor lag \xmltag{sentiment}ein
  h\"assliches Buch\xmltag{/sentiment}}.''), mark this noun phrase; if
it's a clause (e.g. ``\textit{\xmltag{sentiment}Ich hasse B\"ucher
  ohne Inhaltsangabe.\xmltag{/sentiment}}''), mark the whole clause;
if a sentiment is expressed by multiple clauses
(e.g. ``\textit{\xmltag{sentiment}Sie denken, reden, riechen, lieben,
  schmecken, ficken Plastik. Sie haben das so gelernt in der
  Plastik-Werbewelt.\xmltag{/sentiment}}''), include all pertaining
clauses in the span.

The attributes of the \texttt{sentiment} tag with their respective
values and meanings are specified in the table below:\\
\begin{tabular}[t]{|l|c|p{0.6\textwidth}|}\hline
  Attribute & Value & Value's Meaning\\\hline
  %%%%%%%%%%%%%%%%%%%
  \multirow{3}{*}{polarity} & \textit{positive} & this sentiment
  expresses positive attitude to its respective target\\\cline{2-3}

  & \textit{negative\newline(default)} & this sentiment
  expresses negative attitude to its respective target\\\cline{2-3}

  & \textit{comparison} & this sentiment expresses comparison of two
  targets with preference given to some of them\\\hline
  %%%%%%%%%%%%%%%%%%%
  \multirow{3}{*}{intensity} & \textit{0} & stylistically weakly
  marked expression which tends more towards a neutral
  sentence\\\cline{2-3}

  & \textit{1\newline(default)} & middle stylistic
  expressivity\\\cline{2-3}

  & \textit{2} & stylistically very expressive sentiment\\\hline
  %%%%%%%%%%%%%%%%%%%
  \multirow{2}{*}{sarcasm} & \textit{true} & this polar attitutde is
  meant as a derision, i.e. its actual polarity is the opposite of its
  apparent form (that means that an apparent praise is in fact meant
  as rebuke and vice versa - but this actual sense can only be derived
  on the basis of outside context or common knowledge)\\\cline{2-3}

  & \textit{false\newline(default)} & no sarcasm present - polar
  attitude has literal meaning\\\hline
  %%%%%%%%%%%%%%%%%%%
  sentiment-ref & \textit{$->$\newline(directed edge)} & in case a
  single sentiment is not a contiguous text span but rather split into
  parts, you can draw an edge with this label from additional parts of
  a sentiment to its main part\\\hline
\end{tabular}

\subsection{source}
The \texttt{source} tag is used to mark the immediate author(s) or
experiencer(s) of a respective polar expression.  For example, in
sentence -- \textit{Sie mag ihr neues Outfit nicht} (\textit{She
  doesn't like her new outfit}) -- ``Sie'' (``She'') should be
considered as opinon's origin and marked as \textit{source} of this
sentiment relation.  Sources are usually expressed by pronouns, nouns
or multiword expressions (e.g. \textit{Neue Z\"uricher Zeitung}).
When marking \texttt{source}s, please always include complete
respective noun phrases in this tag, i.e. nouns with all their
respective dependent attributes.

The attributes of the \texttt{source} tag with their possible values
and meanings are specified in the table below:\\
\begin{tabular}{|l|c|p{0.6\textwidth}|}\hline
  Attribute & Value & Value's Meaning\\\hline
  sentiment-ref & \textit{$->$\newline(directed edge)} & directed edge
  pointing from an outside source element to its respective sentiment,
  or in case of multiple overlapping sentiments when one source
  element is included in both of sentiment element - in that case this
  edge should point from source to its respective sentiment
  span\\\hline

  anaph-ref & \textit{$->$\newline(directed edge)} & directed edge
  pointing from source element expressed by pronoun or pronominal
  adverb to their respective antecedents\\\hline
\end{tabular}

\subsection{target}
\texttt{target}-tags should mark subjects which a particular polar
opinion is about.  It means, \texttt{target}s are the topics to which
someone expresses her attitude.  These topics are usually represented
by nouns, pronouns or also complete clauses.  You are asked to include
complete noun phrases and complete clauses in this tag (i.e. the main
target word with all its dependent attributes).

Coordinatively connected noun phrases both of which represent targets
of a sentiment should be marked separately.  In cases the sentiment is
a comparison, you should also mark two separate targets.  One target
in such case should comprise the object that is preferred in the
comparison and the other one should encompass the dispreferred
object. In case of comparison, you should also set corresponding
\textit{preference} attributes in the respective target tags.

The attributes and attribute values of the \texttt{target} tags almost
fully correspond to the respective attributes of \texttt{source} tags,
namely targets also have attributes \textit{sentiment-ref} and
\textit{anaph-ref} just as \texttt{source}s.  The only additional
attribute which targets can take on is the \textit{preference}
attribute used only in comparisons -- this attribute takes the value
\textit{preferred} or \textit{dispreferred} depending on the
preference of one target over another in comparison.

\subsection{emo-expression}
\subsection{intensifier}
\subsection{diminisher}
\subsection{negation}

%%%%%%%%%%%%%%%%%%%%%%%%%%%%%%%%%%%%%%%%%%%%%%%%%%%%%%%%%%%%%%%%%%



%%%%%%%%%%%%%%%%%%%%%%%%%%%%%%%%%%%%%%%%%%%%%%%%%%%%%%%%%%%%%%%%%%

\begin{tabular}{|l|c|p{0.6\textwidth}|}\hline
  Attribute & Value & Value's Meaning\\\hline
  %%%%%%%%%%%%%%%%%%%

  \multirow{2}{*}{polarity} & \textit{positive} & this emotional
  expression has positive polar meaning about sentiment's
  target\\\cline{2-3}

  & \textit{negative\newline(default)} & this emotional expression has
  negative polar meaning\\\hline

  %%%%%%%%%%%%%%%%%%%

  \multirow{3}{*}{intensity} & \textit{0} & stylistically weakly
  marked emotional expression\\\cline{2-3}

  & \textit{1\newline(default)} & middle stylistic
  expressivity\\\cline{2-3}

  & \textit{2} & stylistically very expressive emotional
  expression\\\hline

  %%%%%%%%%%%%%%%%%%%

  emo-expression-ref & \textit{$->$\newline(directed edge)} & in cases
  when a single emotional expression is not represented as a
  contiguous span but rather split into separate parts, this edge
  should point from parts to the main part of emo-expression (choose
  whatever part you want as main and connect all the other parts to it
  - it does not matter which part you consider as main)\\\hline

  %%%%%%%%%%%%%%%%%%%

  sentiment-ref & \textit{$->$\newline(directed edge)} & in cases
  multiple sentiments overlap each other and the emo-expression
  element is located in an overlapping span of both of them this edge
  should point to sentiment on which this emo-expression has immediate
  impact\\\hline
\end{tabular}

%%%%%%%%%%%%%%%%%%%%%%%%%%%%%%%%%%%%%%%%%%%%%%%%%%%%%%%%%%%%%%%%%%

\subsection{\xmltag{negation}}
\texttt{\xmltag{negation}}s are lexical or syntactic elements that
reverse the primary polar meaning of emo-expressions to the opposite,
so that the overall polarity of the whole sentiment is different to
the polarity of emo-expressions belonging to it. A typical example of
negation is \textit{nicht} in sentences like \textit{Ein guter Schritt
  war diese Entscheidung \textbf{nicht}.} (\textit{This decision was
  \textbf{not} a good move.}).

\subsubsection{Examples}
You SHOULD only mark negative elements that have impact on
sentiment's polarity. Negating elements having no such impact
shold not be marked. Negation elements are usually represented by:
\begin{itemize}
\item Negation particle \textit{nicht} (\textit{not}),
  e.g. \textit{Klug ist dieser Hund sicherlich \textbf{nicht}}
  (\textit{This dog is certainly \textbf{not} clever});

\item Negative article \textit{kein} e.g. \textit{Er war
  \textbf{kein} Vorbild f\"ur seine Kinder} (\textit{He was
  \textbf{not} a good role model for his children});

\item Indefinite pronouns like \textit{niemand}, \textit{keiner}
  etc., e.g. \textit{\textbf{Niemand} hielt ihn f\"ur einen
    ehrlichen Menschen.} (\textit{\textbf{Nobody} considered him an
    honest man});

\item Any lexical or idiomatic unit in case they turn sentiment's
  polarity to the opposite, e.g. \textit{Ich \textbf{zweifle}, dass
    das neue iPhone ein besseres Display hat.} (\textit{I
    \textbf{doubt} the new iPhone has a better display}).
\end{itemize}

\subsubsection{Counter-Examples}
DO NOT mark as negations elements that have no effect on sentiment's
polarity. For example, in the sentence \textit{Ich mag Leute, die nicht
  nur an sich selbst denken.} (\textit{I like people who not only care
  about themselves.}) \textit{nicht} (\textit{not}) should not be
marked as negation, since it does not change the positive polarity
expressed by \textit{m\"ogen} (\textit{like}).

\subsubsection{Attributes}
Negations only have one possible attribute, namely
\textit{sentiment-ref} that is a directed edge pointing from negation
to sentiment it belongs to. You should only draw this edge in cases multiple sentiment relations overlap each other and it's not
obvious to which of these sentiments the given negation belongs to,
i.e. in cases the given negation is included in both sentiment spans.

\subsection{\xmltag{intensifier}}
\texttt{\xmltag{intensifier}}s are elements that increase the polar
meaning of emotional expressions. Intensifiers are usually expressed
by adjectives or adverbs like \textit{sehr} (\textit{very}),
\textit{ziemlich} (\textit{rather}) etc.

\subsubsection{Attributes}
Intensifiers have following attributes and values:

\begin{tabular}{|l|c|p{0.6\textwidth}|}\hline
  \multirow{2}{*}{degree} & \textit{1 (default)} & degree by which
  this intensifier increases polar sense of respective emotional
  expression (1 means slight increase), e.g. \textit{ziemlich},
  \textit{recht} etc.\\\cline{2-3}

  & \textit{2} & this intensifier strongly increases polar sense of
  emotional expression, e.g. \textit{sehr}, \textit{super},
  \textit{stark} etc.\\\hline

  %%%%%%

  sentiment-ref & \textit{$->$\newline(directed edge)} & a directed
  edge pointing from the intensifier to the sentiment it belongs
  to. By analogy to negations, you should only draw this edge in cases
  multiple sentiment relations overlap each other and it's not obvious
  to which of these sentiments the given intensifier belongs to,
  i.e. in cases when the given intensifier is included in both
  sentiment spans\\\hline
\end{tabular}

%%%%%%%%%%%%%%%%%%%%%%%%%%%%%%%%%%%%%%%%%%%%%%%%%%%%%%%%%%%%%%%%%%

\subsection{\xmltag{diminisher}}
\texttt{\xmltag{diminisher}}s are elements that decrease the polar
meaning of emotional expressions. Like intensifiers diminishers are
usually expressed by adjectives or adverbs. Typical examples of such
adverbs are \textit{wenig}, \textit{kaum}, \textit{ein bisschen} etc.

\subsubsection{Attributes}
Diminishers have the same attributes as intensifiers with the only
difference in values of \textit{degree} attribute. So here is once
again a summary table of diminisher's attributes:

\begin{tabular}{|l|c|p{0.6\textwidth}|}\hline
  sentiment-ref & \textit{$->$\newline(directed edge)} & a directed
  edge pointing from diminisher to sentiment it pertains to. You
  should only draw this edge in cases when multiple sentiment
  relations overlap with each other and it's not obvious to which of
  these sentiments given diminisher belongs to, i.e. in cases when
  given diminisher is included in both sentiment spans\\\hline

  %%%%%%

  \multirow{2}{*}{degree} & \textit{-1 (default)} & degree by which
  this diminisher decreases polar sense of respective emotional
  expression (1 means slight decrease), e.g. \textit{wenig},
  \textit{bisschen} etc.\\\cline{2-3}

  & \textit{-2} & this diminisher strongly decreases polar sense of
  emotional expression almost turning the sense of emo-expression to
  the opposite, e.g. \textit{kaum} etc.\\\hline
\end{tabular}
\end{document}


\section{Examples}
Please DO NOT mark as sentiments following cases:
\begin{itemize}
\item Statements describing some emotional states for which no
  target can be derived or found (e.g. \textit{Ich f\"uhle mich so
    traurig :(} - \textit{I am feeling so blue :(});
\item Statements describing some objective facts, even if possible
  consequences of these facts can be assumed to have negative
  influence on the author (e.g. \textit{Wenn ein Floh einen Menschen
    bei\ss{}t und ihn mit erbrochenem Blut infiziert, werden die
    Pestbakterien ins Gewebe \"ubertragen.} - \textit{When a flea
    bites a human and contaminates the wound with regurgitated
    blood, the plague carrying bacteria are passed into the
    tissue.});
\item Interrogative clauses in case they ask whether some polar
  opinion is true or not. (e.g. \textit{Findet die Mehrheit
    der Bev\"olkerung CDU toll?} - \textit{Does the majority
    of population consider CDU great?});
\end{itemize}

\section{FAQ and Examples}
Things you SHOULD mark as sources:
\begin{itemize}
\item Original author(s) of polar opinions, i.e. those persons,
  groups or officials who experience sympathy or aversion for
  something e.g. \textit{\textbf{Michael} meinte, das w\"are die
    beste L\"osung.}  (\textit{\textbf{Michael} thought, it would be the
    best solution.});
\end{itemize}

\subsubsection{Counter-Examples}
Things you SHOULD NOT mark as sources:
\begin{itemize}
\item Persons how are neutrally citing someone else's opinion,
  e.g. in the sentence \textit{Nach Tatjanas Worten war Michael sehr
    dar\"uber ver\"argert.} (\textit{According to Tatjana, Michael
    was very angry about that.}) Here you should only mark
  \textit{Michael} as sentiment's source and not \textit{Tatjana}. Please note, that in case an author
  supports or contradicts someone else's opinion, two sentiment
  relations should be made - one with the original author as source,
  and one with the supporter/opponent of the opinion enclosed in
  source tags.\footnote{In that case sentiments will overlap, and
    you also should mark these sources with \textit{sentiment-ref}-tag.}
\end{itemize}


\subsubsection{Examples}
Elements you SHOULD mark as targets:
\begin{itemize}
\item Persons, things or objects the opinion is made about,
  e.g. \textit{Ich hasse \textbf{lange Schlangen}} (\textit{I hate
    \textbf{long queues}});

\item Descriptions of actions or events, e.g. \textit{Ich hasse es,
  \textbf{wenn St\"arkere Schw\"achere drangsalieren}} (\textit{I
  hate it \textbf{when stronger people bully weaker ones}});

\item Compared elements of a comparison,
  e.g. \textit{\textbf{\"Altere Handys} halten l\"anger als
    \textbf{moderne Smartphones}} (\textit{\textbf{Older mobile
      phones} outwear \textbf{modern smartphones}}).
\end{itemize}

\subsubsection{Attributes}
\textit{target} attributes fully correspond to the attributes of
\textit{source} elements. So please refer to section \ref{src-attrs}
for the description of possible attributes for targets.

\subsection{\xmltag{emo-expression}}
\texttt{\xmltag{emo-expression}}s are lexical elements that bear some
polarity meaning on their own. Usually this are words like
\textit{gut} (\textit{good}), \textit{schlechter} (\textit{worse}),
\textit{m\"ogen} (\textit{like}), or \textit{hassen}
(\textit{hate}). Such elements can be expressed either as single words
or as complex expressions (in case of idiomatic expressions or support
verbs).

\subsubsection{Examples}
Elements you SHOULD mark as emo-expressions:
\begin{itemize}
\item Adjectives and adverbs bearing polar attitudes \textit{Peter
  hatte \textbf{bessere} Noten in der Schule als sein Bruder}
  (\textit{Peter had \textbf{better} grades at school than his
  brother.});

\item Verbs expressing attitude of a speaker to target,
  e.g. \textit{Mir \textbf{gefiel} die neue House-Staffel}
  (\textit{I \textbf{liked} the new House series});

\item Idiomatic expression including support verbs
  e.g. \textit{\textbf{Zum Teufel} soll die neue Regierung
    \textbf{gehen}} (\textit{The new government should \textbf{go to
      hell}}).

\item Smileys in case they really express an emotional attitude and
  are not used for politeness or without any particular meaning
  e.g. \textit{Gleich in Braunschweig mit Kameraden treffen
    \textbf{:)}} (\textit{Will soon meet friends in Braunschweig
    \textbf{:)}}).
\end{itemize}

\subsubsection{Attributes}
\xmltag{emo-expression} tags have the following attributes with their
corresponding values:

%%%%%%%%%%%%%%%%%%%%%%%%%%%%%%%%%%%%%%%%%%%%%%%%%%%%%%%%%%%%%%%%%%

