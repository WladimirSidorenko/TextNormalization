\documentclass[11pt,a4paper]{article}

%%%%%%%%%%%%%%%%%%%%%%%%%%%%%%%%%%%%%%%%%%%%%%%%%%%%%%%%%%%%%%%%%%
%% Libraries
\usepackage[driver=pdftex,vmargin=2cm,hmargin=2cm]{geometry}
\usepackage[usenames,dvipsnames]{xcolor}
\usepackage{amsmath}
\usepackage{array}
\usepackage{booktabs}
\usepackage{color}
\usepackage{framed}
\usepackage[colorlinks=true]{hyperref}
\usepackage{multirow}
\usepackage{natbib}
\usepackage{paralist}
\usepackage{url}
\usepackage{tikz}
\usepackage{gb4e}
\usepackage{xargs}

%%%%%%%%%%%%%%%%%%%%%%%%%%%%%%%%%%%%%%%%%%%%%%%%%%%%%%%%%%%%%%%%%%
%% Commands
\definecolor{dodgerblue4}{RGB}{16,78,139}
\definecolor{orange3}{RGB}{205,133,0}
\definecolor{DarkSlateBlue}{RGB}{72,61,139}

\newlength{\cwidth} \setlength{\cwidth}{0.18\textwidth}
\newenvironment{example}{\begin{center}\begin{exe}\ex}{\end{exe}\end{center}}
\newcommand{\xmltag}[1]{\textcolor{black}{{\small$<$#1$>$}}}
\newcommand{\sentiment}[2][negative]{$<$sentiment
  sentiment\_polarity="#1"$>$\textcolor{dodgerblue4}{#2}$<$/sentiment$>$}
\newcommand{\source}[1]{\xmltag{source}\textcolor{orange3}{#1}\xmltag{/source}}
\newcommand{\target}[1]{\xmltag{target}\textbf{#1}\xmltag{/target}}
\newcommand{\negation}[1]{\xmltag{negation}\textcolor{red}{#1}\xmltag{/negation}}
\newcommand{\intensifier}[2][1]{\xmltag{intensifier
    intensity="#1"}\textcolor{DarkSlateBlue}{#2}\xmltag{/intensifier}}
\newcommandx{\emoexpression}[3][1=negative, 2=1]{
  \xmltag{emo-expression polarity="#1" intensity="#2"}
  \textcolor{green}{#3}\xmltag{/emo-expression}}

%%%%%%%%%%%%%%%%%%%%%%%%%%%%%%%%%%%%%%%%%%%%%%%%%%%%%%%%%%%%%%%%%%
%% Commands
\author{Wladimir Sidorenko}
\date{\today}
\title{Guidelines for Annotation of Sentiment Corpus}

%%%%%%%%%%%%%%%%%%%%%%%%%%%%%%%%%%%%%%%%%%%%%%%%%%%%%%%%%%%%%%%%%%
%% Main
\begin{document}
\maketitle{}
\section{Introduction}
Welcome. In this assignment your task is to annotate sentiments in a
corpus of Twitter messages.

\subsection{Annotation Tool}

For the annotation you will be using \texttt{MMAX2} - a freely
available annotation tool which can be downloaded under:

\url{http://sourceforge.net/projects/mmax2/files/mmax2/mmax2_1.13.003/MMAX2_1.13.003b.zip/download}

If you have never used \texttt{MMAX2} before, you should first read
chapters 1 through 5 of document mmax2quickstart.pdf which you can
find in subdirectory \texttt{Docs} of your unpacked \texttt{MMAX2}
archive. After you have read this document, please change in your
terminal to the subfolder \texttt{MMAX2} of your unpacked archive and
type the commands\\ \texttt{chmod u+x ./mmax2.sh}\\ \texttt{nohup
  ./mmax2.sh \&}\\ to start \texttt{MMAX2}. Upon doing this an
\texttt{MMAX2} should appear on your desktop.

\subsection{Corpus Files}

You should also have received a copy of \texttt{mmax2} project files
with corpus data as a tar-gzipped archive. Unpack this archive using
command\\\texttt{tar -xzf twitter-sentiment.tgz}\\After that, a
directory called \texttt{mmax-prj} should appear in your current
folder. Now, change to your \texttt{MMAX2} window and click on the
menu \texttt{File -> Load}. In the appeared popup menu, select a path
to the unpacked \texttt{mmax-prj} folder and choose one of the
\texttt{*.mmax} files in that directory, say
\texttt{general.1.mmax}. Source data will be loaded into
\texttt{MMAX2} program.

\section{Tags and Attributes}
In this annotation we will use following tags with following
attributes.

\subsection{\xmltag{sentiment}}
\texttt{\xmltag{sentiment}}s are our most important markables.  For
the purpose of this assignment we will only regard as sentiments
statetements which express some polar attitudes of a writer with
regard to some topics, activities or events. We WILL NOT count as
sentiments vague emotional states of an author which are not related
to any particular subject nor will we regard as sentiments expressions
of uncertainty or communicative acts like apologies or commands,
though one can find such definition in the literature. As a rule of
thumb for annotating sentiments you should always act like following:
first ask yourself whether a sentence you are looking at is neutral or
expresses some polar attitude, in case it has some polar opinion, ask
yourself whether this polarity is distinctly targeted at some thing or
event which is visible or assumable from context and only if it does -
mark relevant span with sentiment tag.

\subsubsection{Meaning}

\texttt{\xmltag{sentiment}}s encompass spans of text which express
polar opinion about some distinctly visible or at least assumable
targets. These spans should only comprise information pertaining to
particular attitude and can be expressed as:
\begin{itemize}
  \item single noune phrases (e.g. \textit{eine herrliche Blume} -
    \textit{a wonderful flower});
  \item single clause (e.g. \textit{Ich mag diese Fernsehshow} -
    \textit{I like this TV show});
  \item multiple clauses (e.g. \textit{Diese Regierung is die beste
    aller Zeiten. April, April!} - \textit{This government is the best
    ever. April fool!}).
\end{itemize}

\subsubsection{Examples}
Things you SHOULD mark as sentiments:
\begin{itemize}
  \item Noun phrases and statements expressing polar opinion
    (e.g. \textit{Der neue Film \"uber Superman war knorke!} -
    \textit{The new superman movie was fantastic});
  \item Interrogative clauses in case they express some polar opinion
    and don't call it into question but rather ask for its reasons
    etc. (e.g. \textit{Warum findet die Mehrheit der Bev\"olkerung CDU
      so toll?} - \textit{Why does the majority of population consider
      CDU great?});
  \item Exclamations in case they express support or disapproval of
    something (e.g. \textit{Der Sommer ist da. Super!} -
    \textit{Summer has come. Great!}).
  \item Comparisons (e.g. \textit{Canon EOS 550d macht bessere
    Aufnahmen als 600d} - \textit{Canon EOS 550d takes better pictures
    than 600d}).\footnote{Comparison sentiments are a special case of
    sentiment relations and get a special attribute and have some
    peculiarities with targets. Read on.}

\end{itemize}

\subsubsection{Counter-Examples}
Things you SHOULD NOT mark as sentiments:
\begin{itemize}
  \item Statements describing some emotional states for which no
    target can be derived or found (e.g. \textit{Ich f\"uhle mich so
      traurig :(} - \textit{Feeling so blue :(});
  \item Interrogative clauses in case they ask whether some polar
    opinion is true or not. (e.g. \textit{Findet die Mehrheit
      der Bev\"olkerung CDU toll?} - \textit{Does the majority
      of population consider CDU great?});
\end{itemize}


\subsubsection{Attributes}
\xmltag{sentiment} tags have following attributes with following
values:

\begin{tabular}{|l|c|p{0.6\textwidth}|}\hline
  Attribute & Value & Value's Meaning\\\hline
  %%%%%%%%%%%%%%%%%%%

  \multirow{3}{*}{polarity} & \textit{positive} & this sentiment
  expresses positive attitude to its respective target\\\cline{2-3}

  & \textit{negative\newline(default)} & this sentiment
  expresses negative attitude to its respective target\\\cline{2-3}

  & \textit{comparison} & this sentiment expresses comparison of two
  things with preference given to some of them\\\hline
  %%%%%%%%%%%%%%%%%%%

  \multirow{2}{*}{sarcasm} & \textit{true} & this polar attitutde is
  meant as a derision, i.e. its actual polarity is the opposite of its
  apparent form (that means that an apparent praise is in fact meant
  as rebuke and vice versa - but this actual sense can only be
  derieved on the basis of outside context or )\\\cline{2-3}

  & \textit{false\newline(default)} & no sarcasm present - polar
  attitude has literal meaning\\\hline
  %%%%%%%%%%%%%%%%%%%

  \multirow{3}{*}{intensity} & \textit{0} & stylictically weakly
  marked expression which tends more towards a neutral
  sentence\\\cline{2-3}

  & \textit{1\newline(default)} & middle stylistic
  expressivity\\\cline{2-3}

  & \textit{2} & stylistically very expressive sentiment\\\hline
\end{tabular}

\subsection{\xmltag{source}}
With tag \texttt{\xmltag{source}} you should mark original author(s)
of respective sentiment opinions, e.g. in sentence \textit{Sie mag
  Ihren neuen Outfit} (\textit{She likes her new outfit}) - ``Sie''
(``She'') should be marked as source.

\subsubsection{Meaning}

\texttt{\xmltag{source}} tag should encompass the author of a polar
attitude. In case this author is cited by a third person she
nevertheless should be marked as single source. In case this citing is
made in form of affirmation or contradiction to an opinion, two
independent sentiment relations should be made - one with the original
author as a source, and one with the affirming (contradicting) author
marked with source tag (note that in case of contradiction the
polarity of sentiment may change to the opposite).

Source tags should encompass spans denoting the author, that will
usually be pronouns, nouns, multiword expressions in case they form a
single entity, e.g. \textit{Neue Z\"uricher Zeitung}.

\subsubsection{Counter-Examples}
Things you SHOULD NOT mark as sentiments:
\begin{itemize}
  \item Statements describing some emotional states for which no
    target can be derived or found (e.g. \textit{Ich f\"uhle mich so
      traurig :(} - \textit{Feeling so blue :(});
  \item Interrogative clauses in case they ask whether some polar
    opinion is true or not. (e.g. \textit{Findet die Mehrheit
      der Bev\"olkerung CDU toll?} - \textit{Does the majority
      of population consider CDU great?});
\end{itemize}


\subsubsection{Attributes}
\xmltag{source} tags have following attributes with following
values:

\begin{tabular}{|l|c|p{0.6\textwidth}|}\hline
  Attribute & Value & Value's Meaning\\\hline
  %%%%%%%%%%%%%%%%%%%

  \multirow{3}{*}{polarity} & \textit{positive} & this sentiment
  expresses positive attitude to its respective target\\\cline{2-3}

  & \textit{negative\newline(default)} & this sentiment
  expresses negative attitude to its respective target\\\cline{2-3}

  & \textit{comparison} & this sentiment expresses comparison of two
  things with preference given to some of them\\\hline

  %%%%%%%%%%%%%%%%%%%

  \multirow{2}{*}{sarcasm} & \textit{true} & this polar attitutde is
  meant as a derision, i.e. its actual polarity is the opposite of its
  apparent form (that means that an apparent praise is in fact meant
  as rebuke and vice versa - but this actual sense can only be
  derieved on the basis of outside context or )\\\cline{2-3}

  & \textit{false\newline(default)} & no sarcasm present - polar
  attitude has literal meaning\\\hline
  %%%%%%%%%%%%%%%%%%%

  \multirow{3}{*}{intensity} & \textit{0} & stylictically weakly
  marked expression which tends more towards a neutral
  sentence\\\cline{2-3}

  & \textit{1\newline(default)} & middle stylistic
  expressivity\\\cline{2-3}

  & \textit{2} & stylistically very expressive sentiment\\\hline
\end{tabular}

\subsection{\xmltag{target}}
\texttt{\xmltag{sentiment}}

\subsubsection{Meaning}

\texttt{\xmltag{sentiment}}s encompass spans of text which express
polar opinion about some distinctly visible or at least assumable
targets. These spans should only comprise information pertaining to
particular attitude and can be expressed as:
\begin{itemize}
  \item single noune phrases (e.g. \textit{eine herrliche Blume} -
    \textit{a wonderful flower});
  \item single clause (e.g. \textit{Ich mag diese Fernsehshow} -
    \textit{I like this TV show});
  \item multiple clauses (e.g. \textit{Diese Regierung is die beste
    aller Zeiten. April, April!} - \textit{This government is the best
    ever. April fool!}).
\end{itemize}

\subsubsection{Examples}
Things you SHOULD mark as sentiments:
\begin{itemize}
  \item Noun phrases and statements expressing polar opinion
    (e.g. \textit{Der neue Film \"uber Superman war knorke!} -
    \textit{The new superman movie was fantastic});
  \item Interrogative clauses in case they express some polar opinion
    and don't call it into question but rather ask for its reasons
    etc. (e.g. \textit{Warum findet die Mehrheit der Bev\"olkerung CDU
      so toll?} - \textit{Why does the majority of population consider
      CDU great?});
  \item Exclamations in case they express support or disapproval of
    something (e.g. \textit{Der Sommer ist da. Super!} -
    \textit{Summer has come. Great!}).
  \item Comparisons (e.g. \textit{Canon EOS 550d macht bessere
    Aufnahmen als 600d} - \textit{Canon EOS 550d takes better pictures
    than 600d}).\footnote{Comparison sentiments are a special case of
    sentiment relations and get a special attribute and have some
    peculiarities with targets. Read on.}

\end{itemize}

\subsubsection{Counter-Examples}
Things you SHOULD NOT mark as sentiments:
\begin{itemize}
  \item Statements describing some emotional states for which no
    target can be derived or found (e.g. \textit{Ich f\"uhle mich so
      traurig :(} - \textit{Feeling so blue :(});
  \item Interrogative clauses in case they ask whether some polar
    opinion is true or not. (e.g. \textit{Findet die Mehrheit
      der Bev\"olkerung CDU toll?} - \textit{Does the majority
      of population consider CDU great?});
\end{itemize}


\subsubsection{Attributes}
\xmltag{sentiment} tags have following attributes with following
values:

\begin{tabular}{|l|c|p{0.6\textwidth}|}\hline
  Attribute & Value & Value's Meaning\\\hline
  %%%%%%%%%%%%%%%%%%%

  \multirow{3}{*}{polarity} & \textit{positive} & this sentiment
  expresses positive attitude to its respective target\\\cline{2-3}

  & \textit{negative\newline(default)} & this sentiment
  expresses negative attitude to its respective target\\\cline{2-3}

  & \textit{comparison} & this sentiment expresses comparison of two
  things with preference given to some of them\\\hline

  %%%%%%%%%%%%%%%%%%%

  \multirow{2}{*}{sarcasm} & \textit{true} & this polar attitutde is
  meant as a derision, i.e. its actual polarity is the opposite of its
  apparent form (that means that an apparent praise is in fact meant
  as rebuke and vice versa - but this actual sense can only be
  derieved on the basis of outside context or )\\\cline{2-3}

  & \textit{false\newline(default)} & no sarcasm present - polar
  attitude has literal meaning\\\hline
  %%%%%%%%%%%%%%%%%%%

  \multirow{3}{*}{intensity} & \textit{0} & stylictically weakly
  marked expression which tends more towards a neutral
  sentence\\\cline{2-3}

  & \textit{1\newline(default)} & middle stylistic
  expressivity\\\cline{2-3}

  & \textit{2} & stylistically very expressive sentiment\\\hline
\end{tabular}

\subsection{\xmltag{emo-expression}}
\texttt{\xmltag{sentiment}}s are our most important markables in this
annotations. As already stated above, we regard as sentiments only
polar opinions like judgements or suggestions with regard to some
particular objects or events. We DO NOT count as sentiments vague
emotional states of an author which are not related to any particular
reason nor do we regard as sentiments expressions of uncertainty or
communicative acts like apologies or commands, though one can find
such definition in the literature. As a rule of thumb for annotating
sentiments you should always act like following: first ask yourself
whether a sentence is neutral or expresses some polar attitude, in
case it has some polarity, ask yourself whether this polarity is
distinctly targeted at some thing or event which is visible or
assumable from context and if it does - mark this span as sentiment.

\subsubsection{Meaning}

\texttt{\xmltag{sentiment}}s encompass spans of text which express
polar opinion about some distinctly visible or at least assumable
targets. These spans should only comprise information pertaining to
particular attitude and can be expressed as:
\begin{itemize}
  \item single noune phrases (e.g. \textit{eine herrliche Blume} -
    \textit{a wonderful flower});
  \item single clause (e.g. \textit{Ich mag diese Fernsehshow} -
    \textit{I like this TV show});
  \item multiple clauses (e.g. \textit{Diese Regierung is die beste
    aller Zeiten. April, April!} - \textit{This government is the best
    ever. April fool!}).
\end{itemize}

\subsubsection{Examples}
Things you SHOULD mark as sentiments:
\begin{itemize}
  \item Noun phrases and statements expressing polar opinion
    (e.g. \textit{Der neue Film \"uber Superman war knorke!} -
    \textit{The new superman movie was fantastic});
  \item Interrogative clauses in case they express some polar opinion
    and don't call it into question but rather ask for its reasons
    etc. (e.g. \textit{Warum findet die Mehrheit der Bev\"olkerung CDU
      so toll?} - \textit{Why does the majority of population consider
      CDU great?});
  \item Exclamations in case they express support or disapproval of
    something (e.g. \textit{Der Sommer ist da. Super!} -
    \textit{Summer has come. Great!}).
  \item Comparisons (e.g. \textit{Canon EOS 550d macht bessere
    Aufnahmen als 600d} - \textit{Canon EOS 550d takes better pictures
    than 600d}).\footnote{Comparison sentiments are a special case of
    sentiment relations and get a special attribute and have some
    peculiarities with targets. Read on.}

\end{itemize}

\subsubsection{Counter-Examples}
Things you SHOULD NOT mark as sentiments:
\begin{itemize}
  \item Statements describing some emotional states for which no
    target can be derived or found (e.g. \textit{Ich f\"uhle mich so
      traurig :(} - \textit{Feeling so blue :(});
  \item Interrogative clauses in case they ask whether some polar
    opinion is true or not. (e.g. \textit{Findet die Mehrheit
      der Bev\"olkerung CDU toll?} - \textit{Does the majority
      of population consider CDU great?});
\end{itemize}


\subsubsection{Attributes}
\xmltag{sentiment} tags have following attributes with following
values:

\begin{tabular}{|l|c|p{0.6\textwidth}|}\hline
  Attribute & Value & Value's Meaning\\\hline
  %%%%%%%%%%%%%%%%%%%

  \multirow{3}{*}{polarity} & \textit{positive} & this sentiment
  expresses positive attitude to its respective target\\\cline{2-3}

  & \textit{negative\newline(default)} & this sentiment
  expresses negative attitude to its respective target\\\cline{2-3}

  & \textit{comparison} & this sentiment expresses comparison of two
  things with preference given to some of them\\\hline

  %%%%%%%%%%%%%%%%%%%

  \multirow{2}{*}{sarcasm} & \textit{true} & this polar attitutde is
  meant as a derision, i.e. its actual polarity is the opposite of its
  apparent form (that means that an apparent praise is in fact meant
  as rebuke and vice versa - but this actual sense can only be
  derieved on the basis of outside context or )\\\cline{2-3}

  & \textit{false\newline(default)} & no sarcasm present - polar
  attitude has literal meaning\\\hline
  %%%%%%%%%%%%%%%%%%%

  \multirow{3}{*}{intensity} & \textit{0} & stylictically weakly
  marked expression which tends more towards a neutral
  sentence\\\cline{2-3}

  & \textit{1\newline(default)} & middle stylistic
  expressivity\\\cline{2-3}

  & \textit{2} & stylistically very expressive sentiment\\\hline
\end{tabular}

\subsection{\xmltag{negation}}
\texttt{\xmltag{sentiment}}s are our most important markables in this
annotations. As already stated above, we regard as sentiments only
polar opinions like judgements or suggestions with regard to some
particular objects or events. We DO NOT count as sentiments vague
emotional states of an author which are not related to any particular
reason nor do we regard as sentiments expressions of uncertainty or
communicative acts like apologies or commands, though one can find
such definition in the literature. As a rule of thumb for annotating
sentiments you should always act like following: first ask yourself
whether a sentence is neutral or expresses some polar attitude, in
case it has some polarity, ask yourself whether this polarity is
distinctly targeted at some thing or event which is visible or
assumable from context and if it does - mark this span as sentiment.

\subsubsection{Meaning}

\texttt{\xmltag{sentiment}}s encompass spans of text which express
polar opinion about some distinctly visible or at least assumable
targets. These spans should only comprise information pertaining to
particular attitude and can be expressed as:
\begin{itemize}
  \item single noune phrases (e.g. \textit{eine herrliche Blume} -
    \textit{a wonderful flower});
  \item single clause (e.g. \textit{Ich mag diese Fernsehshow} -
    \textit{I like this TV show});
  \item multiple clauses (e.g. \textit{Diese Regierung is die beste
    aller Zeiten. April, April!} - \textit{This government is the best
    ever. April fool!}).
\end{itemize}

\subsubsection{Examples}
Things you SHOULD mark as sentiments:
\begin{itemize}
  \item Noun phrases and statements expressing polar opinion
    (e.g. \textit{Der neue Film \"uber Superman war knorke!} -
    \textit{The new superman movie was fantastic});
  \item Interrogative clauses in case they express some polar opinion
    and don't call it into question but rather ask for its reasons
    etc. (e.g. \textit{Warum findet die Mehrheit der Bev\"olkerung CDU
      so toll?} - \textit{Why does the majority of population consider
      CDU great?});
  \item Exclamations in case they express support or disapproval of
    something (e.g. \textit{Der Sommer ist da. Super!} -
    \textit{Summer has come. Great!}).
  \item Comparisons (e.g. \textit{Canon EOS 550d macht bessere
    Aufnahmen als 600d} - \textit{Canon EOS 550d takes better pictures
    than 600d}).\footnote{Comparison sentiments are a special case of
    sentiment relations and get a special attribute and have some
    peculiarities with targets. Read on.}

\end{itemize}

\subsubsection{Counter-Examples}
Things you SHOULD NOT mark as sentiments:
\begin{itemize}
  \item Statements describing some emotional states for which no
    target can be derived or found (e.g. \textit{Ich f\"uhle mich so
      traurig :(} - \textit{Feeling so blue :(});
  \item Interrogative clauses in case they ask whether some polar
    opinion is true or not. (e.g. \textit{Findet die Mehrheit
      der Bev\"olkerung CDU toll?} - \textit{Does the majority
      of population consider CDU great?});
\end{itemize}


\subsubsection{Attributes}
\xmltag{sentiment} tags have following attributes with following
values:

\begin{tabular}{|l|c|p{0.6\textwidth}|}\hline
  Attribute & Value & Value's Meaning\\\hline
  %%%%%%%%%%%%%%%%%%%

  \multirow{3}{*}{polarity} & \textit{positive} & this sentiment
  expresses positive attitude to its respective target\\\cline{2-3}

  & \textit{negative\newline(default)} & this sentiment
  expresses negative attitude to its respective target\\\cline{2-3}

  & \textit{comparison} & this sentiment expresses comparison of two
  things with preference given to some of them\\\hline

  %%%%%%%%%%%%%%%%%%%

  \multirow{2}{*}{sarcasm} & \textit{true} & this polar attitutde is
  meant as a derision, i.e. its actual polarity is the opposite of its
  apparent form (that means that an apparent praise is in fact meant
  as rebuke and vice versa - but this actual sense can only be
  derieved on the basis of outside context or )\\\cline{2-3}

  & \textit{false\newline(default)} & no sarcasm present - polar
  attitude has literal meaning\\\hline
  %%%%%%%%%%%%%%%%%%%

  \multirow{3}{*}{intensity} & \textit{0} & stylictically weakly
  marked expression which tends more towards a neutral
  sentence\\\cline{2-3}

  & \textit{1\newline(default)} & middle stylistic
  expressivity\\\cline{2-3}

  & \textit{2} & stylistically very expressive sentiment\\\hline
\end{tabular}

\subsection{\xmltag{intensifier}}
\texttt{\xmltag{sentiment}}s are our most important markables in this
annotations. As already stated above, we regard as sentiments only
polar opinions like judgements or suggestions with regard to some
particular objects or events. We DO NOT count as sentiments vague
emotional states of an author which are not related to any particular
reason nor do we regard as sentiments expressions of uncertainty or
communicative acts like apologies or commands, though one can find
such definition in the literature. As a rule of thumb for annotating
sentiments you should always act like following: first ask yourself
whether a sentence is neutral or expresses some polar attitude, in
case it has some polarity, ask yourself whether this polarity is
distinctly targeted at some thing or event which is visible or
assumable from context and if it does - mark this span as sentiment.

\subsubsection{Meaning}

\texttt{\xmltag{sentiment}}s encompass spans of text which express
polar opinion about some distinctly visible or at least assumable
targets. These spans should only comprise information pertaining to
particular attitude and can be expressed as:
\begin{itemize}
  \item single noune phrases (e.g. \textit{eine herrliche Blume} -
    \textit{a wonderful flower});
  \item single clause (e.g. \textit{Ich mag diese Fernsehshow} -
    \textit{I like this TV show});
  \item multiple clauses (e.g. \textit{Diese Regierung is die beste
    aller Zeiten. April, April!} - \textit{This government is the best
    ever. April fool!}).
\end{itemize}

\subsubsection{Examples}
Things you SHOULD mark as sentiments:
\begin{itemize}
  \item Noun phrases and statements expressing polar opinion
    (e.g. \textit{Der neue Film \"uber Superman war knorke!} -
    \textit{The new superman movie was fantastic});
  \item Interrogative clauses in case they express some polar opinion
    and don't call it into question but rather ask for its reasons
    etc. (e.g. \textit{Warum findet die Mehrheit der Bev\"olkerung CDU
      so toll?} - \textit{Why does the majority of population consider
      CDU great?});
  \item Exclamations in case they express support or disapproval of
    something (e.g. \textit{Der Sommer ist da. Super!} -
    \textit{Summer has come. Great!}).
  \item Comparisons (e.g. \textit{Canon EOS 550d macht bessere
    Aufnahmen als 600d} - \textit{Canon EOS 550d takes better pictures
    than 600d}).\footnote{Comparison sentiments are a special case of
    sentiment relations and get a special attribute and have some
    peculiarities with targets. Read on.}

\end{itemize}

\subsubsection{Counter-Examples}
Things you SHOULD NOT mark as sentiments:
\begin{itemize}
  \item Statements describing some emotional states for which no
    target can be derived or found (e.g. \textit{Ich f\"uhle mich so
      traurig :(} - \textit{Feeling so blue :(});
  \item Interrogative clauses in case they ask whether some polar
    opinion is true or not. (e.g. \textit{Findet die Mehrheit
      der Bev\"olkerung CDU toll?} - \textit{Does the majority
      of population consider CDU great?});
\end{itemize}


\subsubsection{Attributes}
\xmltag{sentiment} tags have following attributes with following
values:

\begin{tabular}{|l|c|p{0.6\textwidth}|}\hline
  Attribute & Value & Value's Meaning\\\hline
  %%%%%%%%%%%%%%%%%%%

  \multirow{3}{*}{polarity} & \textit{positive} & this sentiment
  expresses positive attitude to its respective target\\\cline{2-3}

  & \textit{negative\newline(default)} & this sentiment
  expresses negative attitude to its respective target\\\cline{2-3}

  & \textit{comparison} & this sentiment expresses comparison of two
  things with preference given to some of them\\\hline

  %%%%%%%%%%%%%%%%%%%

  \multirow{2}{*}{sarcasm} & \textit{true} & this polar attitutde is
  meant as a derision, i.e. its actual polarity is the opposite of its
  apparent form (that means that an apparent praise is in fact meant
  as rebuke and vice versa - but this actual sense can only be
  derieved on the basis of outside context or )\\\cline{2-3}

  & \textit{false\newline(default)} & no sarcasm present - polar
  attitude has literal meaning\\\hline
  %%%%%%%%%%%%%%%%%%%

  \multirow{3}{*}{intensity} & \textit{0} & stylictically weakly
  marked expression which tends more towards a neutral
  sentence\\\cline{2-3}

  & \textit{1\newline(default)} & middle stylistic
  expressivity\\\cline{2-3}

  & \textit{2} & stylistically very expressive sentiment\\\hline
\end{tabular}

\subsection{\xmltag{diminisher}}
\texttt{\xmltag{sentiment}}s are our most important markables in this
annotations. As already stated above, we regard as sentiments only
polar opinions like judgements or suggestions with regard to some
particular objects or events. We DO NOT count as sentiments vague
emotional states of an author which are not related to any particular
reason nor do we regard as sentiments expressions of uncertainty or
communicative acts like apologies or commands, though one can find
such definition in the literature. As a rule of thumb for annotating
sentiments you should always act like following: first ask yourself
whether a sentence is neutral or expresses some polar attitude, in
case it has some polarity, ask yourself whether this polarity is
distinctly targeted at some thing or event which is visible or
assumable from context and if it does - mark this span as sentiment.

\subsubsection{Meaning}

\texttt{\xmltag{sentiment}}s encompass spans of text which express
polar opinion about some distinctly visible or at least assumable
targets. These spans should only comprise information pertaining to
particular attitude and can be expressed as:
\begin{itemize}
  \item single noune phrases (e.g. \textit{eine herrliche Blume} -
    \textit{a wonderful flower});
  \item single clause (e.g. \textit{Ich mag diese Fernsehshow} -
    \textit{I like this TV show});
  \item multiple clauses (e.g. \textit{Diese Regierung is die beste
    aller Zeiten. April, April!} - \textit{This government is the best
    ever. April fool!}).
\end{itemize}

\subsubsection{Examples}
Things you SHOULD mark as sentiments:
\begin{itemize}
  \item Noun phrases and statements expressing polar opinion
    (e.g. \textit{Der neue Film \"uber Superman war knorke!} -
    \textit{The new superman movie was fantastic});
  \item Interrogative clauses in case they express some polar opinion
    and don't call it into question but rather ask for its reasons
    etc. (e.g. \textit{Warum findet die Mehrheit der Bev\"olkerung CDU
      so toll?} - \textit{Why does the majority of population consider
      CDU great?});
  \item Exclamations in case they express support or disapproval of
    something (e.g. \textit{Der Sommer ist da. Super!} -
    \textit{Summer has come. Great!}).
  \item Comparisons (e.g. \textit{Canon EOS 550d macht bessere
    Aufnahmen als 600d} - \textit{Canon EOS 550d takes better pictures
    than 600d}).\footnote{Comparison sentiments are a special case of
    sentiment relations and get a special attribute and have some
    peculiarities with targets. Read on.}

\end{itemize}

\subsubsection{Counter-Examples}
Things you SHOULD NOT mark as sentiments:
\begin{itemize}
  \item Statements describing some emotional states for which no
    target can be derived or found (e.g. \textit{Ich f\"uhle mich so
      traurig :(} - \textit{Feeling so blue :(});
  \item Interrogative clauses in case they ask whether some polar
    opinion is true or not. (e.g. \textit{Findet die Mehrheit
      der Bev\"olkerung CDU toll?} - \textit{Does the majority
      of population consider CDU great?});
\end{itemize}


\subsubsection{Attributes}
\xmltag{sentiment} tags have following attributes with following
values:

\begin{tabular}{|l|c|p{0.6\textwidth}|}\hline
  Attribute & Value & Value's Meaning\\\hline
  %%%%%%%%%%%%%%%%%%%

  \multirow{3}{*}{polarity} & \textit{positive} & this sentiment
  expresses positive attitude to its respective target\\\cline{2-3}

  & \textit{negative\newline(default)} & this sentiment
  expresses negative attitude to its respective target\\\cline{2-3}

  & \textit{comparison} & this sentiment expresses comparison of two
  things with preference given to some of them\\\hline

  %%%%%%%%%%%%%%%%%%%

  \multirow{2}{*}{sarcasm} & \textit{true} & this polar attitutde is
  meant as a derision, i.e. its actual polarity is the opposite of its
  apparent form (that means that an apparent praise is in fact meant
  as rebuke and vice versa - but this actual sense can only be
  derieved on the basis of outside context or )\\\cline{2-3}

  & \textit{false\newline(default)} & no sarcasm present - polar
  attitude has literal meaning\\\hline
  %%%%%%%%%%%%%%%%%%%

  \multirow{3}{*}{intensity} & \textit{0} & stylictically weakly
  marked expression which tends more towards a neutral
  sentence\\\cline{2-3}

  & \textit{1\newline(default)} & middle stylistic
  expressivity\\\cline{2-3}

  & \textit{2} & stylistically very expressive sentiment\\\hline
\end{tabular}

\section{Tags Summary}

\end{document}
%%%%%%%%%%%%%%%%%%%%%%%%%%%%%%%%%%%%%%%%%%%%%%%%%%%%%%%%%%%%%%%%%%
\section{Tags and Attributes}

Below is a short overview of tags and elements we will use

\subsection{sentiment}
What do we mark as sentiment?

Here is a short summary of cases which should be marked as sentiments.
\begin{enumerate}
\item Clauses and noun phrases expressing polar opinions about some
  things or actions except for cases listed in section
  \ref{sec:not-sentiment}. Examples:
  \begin{itemize}
  \item Das war ein sehr sympathischer und professioneller Auftritt
    und tat der Sendung gut.
  \end{itemize}

\item Exclamations expressing author's emotional attitude to
  particular subject:
  \begin{itemize}
  \item Meister ist nur der \#{}fcbayern!!!
  \item \#{}Merkel hat Deutschland ausgedient, werft sie auf die M\"ullhalde
    der Geschichte!
  \end{itemize}

\item wh-questions in cases when they don't cast doubt on the polar
  opinion (see exceptions in \ref{no-sent:question}):
  \begin{itemize}
  \item Warum die \#{}CDU freiwillige Selbstverpflichtungen so toll findet?
  \end{itemize}
\end{enumerate}


%%%%%%%%%%%%%%%%%%%%%%%%%%%%%%%%%%%%%%%%%%%%%%%%%%%%%%%%%%%%%%%%%%
What do we not mark as sentiment?\label{sec:not-sentiment}

Here is a summary of cases which should \textbf{not} be marked as
sentiments.
\begin{enumerate}
\item \label{no-sent:question} yes-no- and wh-questions which cast doubt on the presence or validity of a sentiment:
  \begin{itemize}
  \item Findet die \#{}CDU freiwillige Selbstverpflichtungen toll?
  \item Wie toll findet die \#{}CDU freiwillige Selbstverpflichtungen?
  \end{itemize}

\item spam:
  \begin{itemize}
  \item Gr\"o\ss{}er, d\"unner, bunter - die neue Ger\"ateflut bei
    \#Samsung: http://on.wsj.com/19Xhdk3
  \end{itemize}

\item fact statements, even if described facts could potentially be
  assumed as funny, dangerous, tragic or sad:
  \begin{itemize}
  \item H\"oheres Unfall-Risiko bei E-Bikes: Elektro-Fahrr\"ader
    erleben einen Boom. Laut Sch\"atzungen des
    VC\"O... http://t.co/yMZZCe3QKM \#{}vorarlberg
  \end{itemize}

\item sentences stating indifference to some facts:
  \begin{itemize}
  \item Mir doch Latte ob \#{}BVB gewinnt... welche Auswirkungen wird
    das Ergebnis n\"achste Woche f\"ur mich haben? \#{}Fu\ss{}ball ist
    wie DSDS von RTL...
  \end{itemize}

\item sentences in which smileys express politeness rahther than attitude:
  \begin{itemize}
  \item Vielleich k\"onnen wie sp\"ater zusammen Kaffe trinken :)
  \item Hat die \#{}spd jetzt einen Kanzlerkandidaten? :-)
  \end{itemize}
\end{enumerate}

\subsection{source}
\subsection{target}
\subsection{emotional-expression}
\subsection{negation}
\subsection{intensifier}
\subsection{diminisher}

%%%%%%%%%%%%%%%%%%%%%%%%%%%%%%%%%%%%%%%%%%%%%%%%%%%%%%%%%%%%%%%%%%
\section{Summary of Elements and Attributes}
To help you remember all the introduced tags, attributes and
relations, we provided a short summary of elements and attributes
along with their respective meanings in table \ref{tab:tags} on page
\pageref{tab:tags}.
\begin{table}
  \begin{tabular}{|c|*{5}{>{\centering}p{\cwidth}|}}
    \hline
    Tag & Tag's Meaning & Attributes & Attributes' Values & Examples\tabularnewline\hline

    \multirow{3}{*}{sentiment} &
    \multirow{3}{*}{\parbox{\cwidth}{Polar opinion about some persons,
        facts or actions}} & \textit{sentiment\_polarity} &
    positive/negative &\tabularnewline\cline{3-5}

    & & \textit{sentiment\_sarcasm} & true/false\newline (only for
    negative sentiments) &\tabularnewline\cline{3-5}

    & & \textit{sentiment\_intensity} & 0 (weak)\newline 1 (middle -
    the default)\newline 2 (strong) &\tabularnewline\hline

    \multirow{2}{*}{source} & \multirow{2}{*}{\parbox{\cwidth}{Person who
        experiences the feeling}} & \textit{source\_anaphref} & pointer from
    pronoun inside a sentiment to an antecedent noun outside the sentiment
    relation &\tabularnewline\cline{3-5}

    & & \textit{source\_ref} & pointer from an outside noun to sentiment
    relation &\tabularnewline\hline

    \multirow{2}{*}{target} & \multirow{2}{*}{\parbox{\cwidth}{Subject
        or action about which the opinion is}} &
    \textit{target\_anaphref} & pointer from pronoun inside a
    sentiment to an antecedent noun outside the sentiment relation &
    \tabularnewline\cline{3-5}

    & & \textit{target\_ref} & pointer from an outside noun to sentiment
    relation &\tabularnewline\hline

    \multirow{3}{*}{emo-expression} &
    \multirow{2}{*}{\parbox{\cwidth}{Lexical item with an inherent
        subjective meaning}} & \textit{emo\_polarity} &
    positive/negative & \tabularnewline\cline{3-5}

    & & \textit{emo\_intensity} & 0 (weak)\newline 1 (middle - the
    default)\newline 2 (strong) & \tabularnewline\cline{3-5}

    & & \textit{eexpression\_target} & pointer from a part of an
    emo-expression to another emo-expression which this part belongs
    to & \tabularnewline\hline

    negation & Lexical item which changes the polarity of an
    emo-expression to the opposite & \textit{negation\_target} &
    pointer from negating element to emo-expression which it negates &
    \tabularnewline\hline

    \multirow{2}{*}{intensifier} &
    \multirow{2}{*}{\parbox{\cwidth}{Usually an adverbial or
        adjectival item which increase or decreases the polar meaning
        of an emotional expression\newline (if an intensifier is
        negated, its negation should also be included in tag)}} &
    \textit{int\_intensity} & -2 (strongly decreases polar meaning -
    say almost turns to the opposite)\newline -1 (slightly decreases
    polar meaning)\newline 1 (slightly increases polar
    meaning)\newline 2 (strongly increases polar meaning) &
    \tabularnewline\cline{3-5}

    & & \textit{intensifier\_{}target} & pointer from intensifier to
    emo-expression which it relates to & \tabularnewline\hline
  \end{tabular}
  \caption{A prototype of the Job Information dialog}
  \label{tab:tags}
\end{table}

%%%%%%%%%%%%%%%%%%%%%%%%%%%%%%%%%%%%%%%%%%%%%%%%%%%%%%%%%%%%%%%%%%

%%%%%%%%%%%%%%%%%%%%%%%%%%%%%%%%%%%%%%%%%%%%%%%%%%%%%%%%%%%%%%%%%%
\section{Open questions}

This scheme still raises some question which should be clarified
before we start off with annotation. These questions are:
\begin{enumerate}
\item Shall we mark wishes as a special kind of sentiment relations?
  \begin{itemize}
  \item Will jetz Radl fahrn oder Inliner :(
  \item Nen weisser papst - . - ICH WOLLTE NEN SCHWARZEN :'(
  \end{itemize}

\item Shall we capture agreement/disagreement with some sentiment
  statements, agreement/disagreement in general?
  \begin{itemize}
  \item Die Konklave w\"ahlt den Papst und dann sagen sie Gott war es
    - Wollt ihr mich verarschen ?!
  \item Lisa singt Hat sie das Zeug zum Superstar?" jaaaaaaa!!!! :D
    NEIN! Riccardo :)
  \item Das geht nun aber wirklich zu weit! :-)
  \item was soll denn der Schei\ss!!
  \item :D das macht allerdings sinn
  \item oookay cool !
  \item "Diese \#{}Bundesregierung ist die erfolgreichste seit der
    \#{}Wiedervereinigung." April, April
  \end{itemize}

\item Shall we try to detect sentiment in cases like below? Possible
  targets are marked in bold:
  \begin{itemize}
  \item Kann man ja keine \textbf{Tickets} kaufen und habe keine gewonnen :((((((
  \item Gleich in Braunschweig mit \textbf{Kamaraden} Treffen :)
  \item In einer halben Stunde zum \textbf{Training} :-)
  \end{itemize}

\item Shall we capture sadness or excitement feelings which don't have
  any particular target?
  \begin{itemize}
  \item Leute ich f\"uhle mich so Hangover m\"a\ss{}ig... WARUM?
  \item hab mir auf die Zunge gebissen :-(
  \end{itemize}
\end{enumerate}

%%%%%%%%%%%%%%%%%%%%%%%%%%%%%%%%%%%%%%%%%%%%%%%%%%%%%%%%%%%%%%%%%%
\section{Examples}

To give you some feeling about how sentiments should be marked, we
provided a couple of examples with full annotation:
\begin{example}
\sentiment[positive]{\source{Ich} habe ein \target{@YouTube-Video} von
  @mathehilfe24 \emoexpression[positive]{positiv} bewertet}
\end{example}
\begin{example}
  \tikzstyle{every picture}+=[remember picture]
  \tikzstyle{na} = [shape=rectangle,inner sep=0pt]
  \tikz\node[na](word1){Clara}; (13) frisst weil \sentiment{\source{\tikz\node[na](word2){sie};} \emoexpression{traurig}
  ist, das sie so \target{fett} ist}.
  \begin{tikzpicture}[overlay]
    \path[->,red,thick](word2) edge [in=10, out=175] node
         {target\_anaphref} (word1);
  \end{tikzpicture}
\end{example}
\begin{example}
\sentiment{\source{Mir} ist \emoexpression{schlecht}, ich sollte keine
  \target{Fefe-Links} klicken am Morgen.}
\end{example}
\begin{example}
'\sentiment{Mit \target{Peer} wird's \emoexpression{schwer}}'.
\end{example}
\begin{example}
  \tikzstyle{every picture}+=[remember picture]
  \tikzstyle{na} = [shape=rectangle,inner sep=0pt]
  \sentiment[positive]{\source{Ich} bin grade\\
    \intensifier{\tikz\node[na](word1){voll};}
    \emoexpression{\tikz\node[na](word2){traurig};}..
    Weil ich nicht bei justin \target{\tikz\node[na](word3){bieber};}
    sein kann}..

  \sentiment[positive]{\source{Ich}
    \emoexpression[positive]{\tikz\node[na](word4){vermisse};}
    \target{\tikz\node[na](word5){ihn};}\\\intensifier{\tikz\node[na](word6){sooo};}..}
  \begin{tikzpicture}[overlay]
    \path[->,red,thick](word1) edge [out=175, in=10] node {intensifier\_anaphref}
    (word2);
    \path[->,red,thick](word5) edge [out=20, in=-10] node {target\_anaphref}
    (word3);
    \path[->,red,thick](word6) edge [out=20, in=-10] node {intensifier\_target}
    (word4);
  \end{tikzpicture}
\end{example}

%%%%%%%%%%%%%%%%%%%%%%%%%%%%%%%%%%%%%%%%%%%%%%%%%%%%%%%%%%%%%%%%%%
\end{document}


\begin{example}
  72 j\"ahriger Leser regt sich bei mir \"uber YouTube und Gema auf :)\\
  (\textit{A 72-year-old reader got worked up over YouTube and Gema
    :)})\label{ex1}
\end{example}
\begin{example}
  \tikzstyle{every picture}+=[remember picture]
  \tikzstyle{na} = [shape=rectangle,inner sep=0pt]

  Die \tikz\node[na](word1){Merkel};. \sentiment[positive]{Das einzig
    sympathische an \target{\tikz\node[na](word2){ihr};} , ist dass
    sie die Beschl\"usse ihrer eigenen Partei so ernst nimmt wie
    wir.}\label{ex-2}
  \begin{tikzpicture}[overlay]
    \path[->,red,thick](word2) edge [in=10, out=175] node
         {target\_anaphref} (word1);
  \end{tikzpicture}
\end{example}

\begin{example}
  \tikzstyle{every picture}+=[remember picture]
  \tikzstyle{na} = [shape=rectangle,inner sep=0pt]
  Georgina \target{\tikz\node[na](word1){Fleur};}, Am Flughafen in
  Frankfurt L\"acheln\\ \sentiment[positive]{\tikz\node[na](word2){Ist
      garnicht so eine Zicke};}!
  \begin{tikzpicture}[overlay]
    \path[->,red,thick](word1) edge [out=20, in=160] node {target\_{}ref}
    (word2);
  \end{tikzpicture}
\end{example}

%%%%%%%%%%%%%%%%%%%%%%%%%%%%%%%%%%%%%%%%%%%%%%%%%%%%%%%%%%%%%%%%%%
\section{The Emo-Expression, the Negation and the Intensifier}
\begin{example}
  \sentiment{\target{Boa} ist seid heute\\ \emoexpression{langweilig}}
  !\\
  (\textit{From today, Boa is boring!})\label{ex-3}
\end{example}
\begin{example}
  ..., weil \sentiment{ich sehr daran \negation{zweifle} dass er ein
    \emoexpression{guter} Vater w\"are}.\\ \label{ex-5} (\textit{...,
    because I doubt it very much that he would be a good father.})
\end{example}

\begin{example}
  \tikzstyle{every picture}+=[remember picture]
  \tikzstyle{na} = [shape=rectangle,inner sep=0pt]
  ..., weil \sentiment{ich sehr daran
    \negation{\tikz\node[na](word1){zweifle};} dass er ein
    \emoexpression{guter} Vater
    \tikz\node[na](word2){w\"are};}.\\ \label{ex-5} (\textit{...,
    because I doubt it very much that he would be a good father.})
  \begin{tikzpicture}[overlay]
    \path[->,red,thick](word1) edge [in=20, out=175] node
         {negation\_{}target} (word2);
  \end{tikzpicture}
\end{example}

\begin{example}
  \tikzstyle{every picture}+=[remember picture]
  \tikzstyle{na} = [shape=rectangle,inner sep=0pt]
  \sentiment[positive]{Ein \intensifier[2]{\tikz\node[na](word1){sehr};}
    \emoexpression[positive]{\tikz\node[na](word2){kluger};}
    \target{Hund} f\"uhrt viele verschiedene Befehle aus}.\\ (\textit{A
    very clever dog can execute many different commands.})\label{ex-6}
  \begin{tikzpicture}[overlay]
    \path[->,red,thick](word1) edge [in=20, out=165] node {intensifier\_target} (word2);
  \end{tikzpicture}
\end{example}

\begin{example}
  \tikzstyle{every picture}+=[remember picture]
  \tikzstyle{na} = [shape=rectangle,inner sep=0pt]
  \sentiment{Ein \intensifier[-2]{\tikz\node[na](word1){nicht sehr};}
    \emoexpression[positive]{\tikz\node[na](word2){kluger};}
    \target{Hund} f\"uhrt nur wenige Befehle aus}.\\ (\textit{A not very
    clever dog can execute only few commands.})\label{ex-6}
  \begin{tikzpicture}[overlay]
    \path[->,red,thick](word1) edge [in=20, out=165] node {intensifier\_target} (word2);
  \end{tikzpicture}
\end{example}

\begin{example}
  \tikzstyle{every picture}+=[remember picture]
  \tikzstyle{na} = [shape=rectangle,inner sep=0pt]
  \sentiment{72 j\"ahriger \source{Leser}\\
    \emoexpression{\tikz\node[na](word2){regt sich};} bei mir\\
    \"uber \target{YouTube} und \target{Gema}\\
    \emoexpression{\tikz\node[na](word1){auf};}} :)\\ (\textit{A
    72-year-old reader got worked up over YouTube and Gema :)})
  \begin{tikzpicture}[overlay]
    \path[->,red,thick](word1) edge [out=20, in=-10, yshift=-2cm] node
         {eexpression\_{}target} (word2);
  \end{tikzpicture}
\end{example}
