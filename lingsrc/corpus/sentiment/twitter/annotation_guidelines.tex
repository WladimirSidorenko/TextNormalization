\documentclass[11pt,a4paper]{article}

%%%%%%%%%%%%%%%%%%%%%%%%%%%%%%%%%%%%%%%%%%%%%%%%%%%%%%%%%%%%%%%%%%
%% Libraries
\usepackage[driver=pdftex,vmargin=2cm,hmargin=2cm]{geometry}
\usepackage[usenames,dvipsnames]{xcolor}
\usepackage{amsmath}
\usepackage{array}
\usepackage{booktabs}
\usepackage{color}
\usepackage{framed}
\usepackage{hyperref}
\usepackage{multirow}
\usepackage{natbib}
\usepackage{paralist}
\usepackage{url}
\usepackage{tikz}
\usepackage{gb4e}
\usepackage{xargs}

%%%%%%%%%%%%%%%%%%%%%%%%%%%%%%%%%%%%%%%%%%%%%%%%%%%%%%%%%%%%%%%%%%
%% Commands
\definecolor{dodgerblue4}{RGB}{16,78,139}
\definecolor{orange3}{RGB}{205,133,0}
\definecolor{DarkSlateBlue}{RGB}{72,61,139}

\newlength{\cwidth} \setlength{\cwidth}{0.18\textwidth}
\newenvironment{example}{\begin{center}\begin{exe}\ex}{\end{exe}\end{center}}
\newcommand{\xmltag}[1]{\textcolor{black}{{\small$<$#1$>$}}}
\newcommand{\sentiment}[2][negative]{$<$sentiment
  sentiment\_polarity="#1"$>$\textcolor{dodgerblue4}{#2}$<$/sentiment$>$}
\newcommand{\source}[1]{\xmltag{source}\textcolor{orange3}{#1}\xmltag{/source}}
\newcommand{\target}[1]{\xmltag{target}\textbf{#1}\xmltag{/target}}
\newcommand{\negation}[1]{\xmltag{negation}\textcolor{red}{#1}\xmltag{/negation}}
\newcommand{\intensifier}[2][1]{\xmltag{intensifier
    intensity="#1"}\textcolor{DarkSlateBlue}{#2}\xmltag{/intensifier}}
\newcommandx{\emoexpression}[3][1=negative, 2=1]{
  \xmltag{emo-expression polarity="#1" intensity="#2"}
  \textcolor{green}{#3}\xmltag{/emo-expression}}

%%%%%%%%%%%%%%%%%%%%%%%%%%%%%%%%%%%%%%%%%%%%%%%%%%%%%%%%%%%%%%%%%%
%% Commands
\author{Wladimir Sidorenko}
\date{\today}
\title{The Good, the Bad and the Ugly\\
  Guidelines for Annotation of Sentiment Corpus}

%%%%%%%%%%%%%%%%%%%%%%%%%%%%%%%%%%%%%%%%%%%%%%%%%%%%%%%%%%%%%%%%%%
%% Main
\begin{document}
\maketitle{}
\section{Introduction}
Welcome. In this assignment your task is to annotate sentiments in a
corpus of Twitter messages.  As sentiments we regard polar attitudes
of some people to other people, events or things. The main goal of
detecting and extracting a polar attitude is to get to know, whether a
particular person approves or disapproves someone or something or
maybe has some valuable suggestions for particular things.

%%%%%%%%%%%%%%%%%%%%%%%%%%%%%%%%%%%%%%%%%%%%%%%%%%%%%%%%%%%%%%%%%%
\section{The Source, the Target and the Sarcasm}
A typical example of a sentiment looks like this:
\begin{example}
  72 j\"ahriger Leser regt sich bei mir \"uber YouTube und Gema auf :)\\
  (\textit{A 72-year-old reader got worked up over YouTube and Gema
    :)})\label{ex1}
\end{example}
From this sentence one can see that the reader is dissatisfied with
Youtube and Gema and he loudly expresses his disapproval of them. The
reader here is the holder of the opinion and we call him the
\textit{source} of sentiment. Youtube and Gema are things which this
opinion is aimed at and we refer to them as sentiment's
\textit{target}s. The whole sentence forms a \textit{sentiment} and as
we already said it has a \textit{negative} polarity. XML annotation
for this sentence would look like this:
\begin{example}
  \sentiment{72 j\"ahriger \source{Leser}
    regt sich bei mir \"uber \target{YouTube} und \target{Gema} auf} :)\\
\end{example}
From this example, you can see, that we only included main words in
source and target tags and we didn't include the smiley ``:)'' at the
end of the message. The reason for omission of smiley was that it
apparently belonged to another sentiment relation - not the one which
showed the attitude of the reader to Gema or YouTube but rather the
one which expressed the opinion of the author of message about the
72-year-old reader. But since it is not quite clear whether this
opinion was positive or rather sarcastic, we didn't mark any sentiment
relation at all in order not to introduce mistakes.

To make things a little bit more interesting, we also added one more
attribute to the $<$sentiment$>$ tag called ``sentiment\_intensity''
which is supposed to express how vigorous the expression of a
sentiment was. Allowed values for this attribute are ``0'' (``very
weak''), ``1'' (``middle intensity'', which is the default), and ``2''
(``enormously intense''). You are free to assign any level of severity
which you assume is correct for a particular sentiment - we don't
impose any restrictions here and give you full freedom. The single
purpose of this attribute is to get to know how different people judge
different expressions. So, don't spend too much time asking yourself
whether a particular sentence was vigorous or not - just assign the
first value which will come into your mind. But do not assign ``1'' to
everything until all sentiments really share same degree of intensity.

Now, coming back to source and target, we need to clarify a couple of
things. Unfortunately, it is not always the case that both of these
elements are specified within a single sentiment relation. It may
happen that, for example, either one or both of them are expressed as
a pronominal adverb or personal pronoun. And since neither of these
parts of speech has any significant lexical meaning by itself due to
their referring nature, we need to encode the nouns which these
pronouns actually refer to, so that we could see who is the actual
source and what is the actual target of a sentiment and don't content
ourselves with their placeholders. Thankfully, \texttt{MMAX2} provides
a possibility for drawing edges between elements, so you can easilly
mark antecedents of target and source in case when it's needed. To do
that, just mark pronouns / pronominal adverbs enclosed in $<$source$>$
or $<$target$>$ tags with the left mouse button, after that click with
the right mouse button on their respective antecedents and you will be
automatically offered a menu suggesting to establish an
``source\_{}anaphref'' or ````target\_{}anaphref'' relation.
\begin{example}
  \tikzstyle{every picture}+=[remember picture]
  \tikzstyle{na} = [shape=rectangle,inner sep=0pt]

  Die \tikz\node[na](word1){Merkel};. \sentiment[positive]{Das
    einzig sympathische an \target{\tikz\node[na](word2){ihr};}
    , ist dass sie die Beschl\"usse ihrer eigenen Partei so
    ernst nimmt wie wir.}\label{ex-2}
  \begin{tikzpicture}[overlay]
    \path[->,red,thick](word2) edge [in=10, out=175] node
         {target\_anaphref} (word1);
  \end{tikzpicture}
\end{example}
Though, in some cases neither actual source (S) nor target (T) nor
even their pronouns will be present in a sentiment, though they could
be present somewhere else in the outside context. In this cases, you
can mark S and T outside the sentiment relation and then establish an
edge from them to the pertaining sentiment, just as we did with
pronouns in the previous example, but now the edge will be called
``source\_ref'' or ``target\_ref'' and it will be pointing from the
outside to sentiment relation. Also this edge will not point at a
single noun or noun phrase but rather at whole sentiment:
\begin{example}
  \tikzstyle{every picture}+=[remember picture]
  \tikzstyle{na} = [shape=rectangle,inner sep=0pt]
  Georgina \target{\tikz\node[na](word1){Fleur};}, Am Flughafen in
  Frankfurt L\"acheln\\ \sentiment[positive]{\tikz\node[na](word2){Ist
      garnicht so eine Zicke};}!
  \begin{tikzpicture}[overlay]
    \path[->,red,thick](word1) edge [out=20, in=160] node {target\_{}ref}
    (word2);
  \end{tikzpicture}
\end{example}
Returning back to the polarity attribute, we should also mention that
there are cases which express an apparently positive sentiment but
their actual meaning in fact is negative. We can derive this meaning
on the basis of the surrounding context or from our world
knowledge. If I, for example, would say ``It was a great decision by
Bin Laden to take shelter in Pakistan which was full of American
informers'', one in fact could see that I'm actually mocking at that
decision since it was counterintuitive and finally lead to dramatic
consequences for Bin Ladden. In cases like that, you should mark such
sentiment relations as ones with negative polarity. After you do this,
you will see an option in MMAX called ``sentiment\_sarcasm''. This
option is set to ``false'' by default, but set it to ``true'' if you
see that a sentiment with an apparently positive meaning is actually
meant as a derision.

%%%%%%%%%%%%%%%%%%%%%%%%%%%%%%%%%%%%%%%%%%%%%%%%%%%%%%%%%%%%%%%%%%
\section{The Emo-Expression, the Negation and the Intensifier}
Since a sentiment usually does not come to us from Heaven as a whole,
but is made up of single elements, we need to identify some of those
elements which might help us detect sentimental relations and
determine their respective polarity. We are especially interested in 3
kinds of such clue elements which are:
\begin{itemize}
\item \textit{emo-expression}s;\label{eexpression};
\item \textit{negation};\label{negation}s;
\item and \textit{intensifier}.\label{intensifier}s.
\end{itemize}
The first group encompasses words which convey some polarity
connotation on their own. Typical examples of such lexical expressions
are: ``gut'' (\textit{good}), ``schlecht'' (\textit{bad}),
``gl\"ucklich'' (\textit{happy}), ``traurig'' (\textit{sad}),
``begeistert'' (\textit{excited}), ``verrufen'' (\textit{infamous})
etc. For them we introduced a special tag called $<$emoexpression$>$
with attributes \textit{emo\_polarity} and \textit{emo\_intensity}. By
analogy with \textit{sentiment} relation, \textit{emo\_polarity} can
be either \textit{positive} or
\textit{negative}. \textit{emo\_intensity} can take any integer value
from 0 to 2 inclusively with 0 denoting the weakest intensity and 2
marking a very strong feeling (by default this attribute is set to
1). So, a typical annotation of an emo-expression would look like
this:
\begin{example}
  \sentiment{\target{Boa} ist seid heute\\ \emoexpression{langweilig}}
  !\\
  (\textit{From today, Boa is boring!})\label{ex-3}
\end{example}
These emotional expressions will serve as primary indicators of a
sentimental relation. Their presence or absence alone will indicate
whether some polar opinion exists and what its possible polarity
is. However we should take into account that these indicators can be
embedded in a context, which may change their inherent polarity to the
opposite. If one, for example, would say ``From today, Boa is not
boring!'', it would mean that author's attitude to Boa is positive
rather than negative in contrast to the previous example. Two main
things which can act as such polarity switcher and which you are asked
to mark are \textit{negation}s and \textit{intensifier}s. The former
are primarily represented by either the negation particle ``nicht'' or
the negation article ``kein'', though forms of negation can be pretty
varying as you can see from the example below:
\begin{example}
  ..., weil \sentiment{ich sehr daran \negation{zweifle} dass er ein
    \emoexpression{guter} Vater w\"are}.\\ \label{ex-5} (\textit{...,
    because I doubt it very much that he would be a good father.})
\end{example}
But since there might be several emo-expression within a single
sentiment and therefore several negations would be possible, we need
to denote their correlation via an edge. So, draw a
\textit{negation\_{}target}-edge from the negation to the word which
it negates (which not obligatory has to be an emo-expression, but can
be any word within sentiment), in order to show its syntactic
dependency. The only restriction imposed on annotating negations is
that it should turn the polarity of sentiment to the opposite. See
example below.
\begin{example}
  \tikzstyle{every picture}+=[remember picture]
  \tikzstyle{na} = [shape=rectangle,inner sep=0pt]
  ..., weil \sentiment{ich sehr daran
    \negation{\tikz\node[na](word1){zweifle};} dass er ein
    \emoexpression{guter} Vater
    \tikz\node[na](word2){w\"are};}.\\ \label{ex-5} (\textit{...,
    because I doubt it very much that he would be a good father.})
  \begin{tikzpicture}[overlay]
    \path[->,red,thick](word1) edge [in=20, out=175] node
         {negation\_{}target} (word2);
  \end{tikzpicture}
\end{example}
Negations which have no effects on sentiment's polarity should not be
marked.

Almost the same annotation conventions as for negations apply to
\textit{intensifier}s. \textit{Intensifier}s are usually adverbial or
adjectival modifiers which may increase or conversely decrease the
polar meaning of an emotional expression. Typical examples of
intensifiers are ``sehr'' (``very''), ``ziemlich'' (``quite''),
``kaum'' (``hardly''), ``wenig'' (``little'') etc. Since we need to
capture the relation between an intensifier and the emo-expression
which it relates to, you also have to establish a
\textit{intensifier\_target} edge between them with intensifier being
the source of this edge and emo-expression being the target.

Intensifiers also have an attribute called \textit{int\_intensity}
whose possible values range from -2 to 2 and which shall show whether
an intesifier increases (1 or 2) or decreases (-2 or -1) the primary
polar sense of an emo-expression. No zero value is available
here. Please note that if an intensifier is negated by ``nicht'' you
should also include negation into intensifier's tag and intensifier's
primary intensity may change in that case too. See following examples:
\begin{example}
  \tikzstyle{every picture}+=[remember picture]
  \tikzstyle{na} = [shape=rectangle,inner sep=0pt]
  \sentiment{Ein \intensifier[2]{\tikz\node[na](word1){sehr};}
    \emoexpression[positive]{\tikz\node[na](word2){kluger};}
    \target{Hund} f\"uhrt viele verschiedene Befehle aus}.\\ (\textit{A
    very clever dog can execute many different commands.})\label{ex-6}
  \begin{tikzpicture}[overlay]
    \path[->,red,thick](word1) edge [in=20, out=165] node {intensifier\_target} (word2);
  \end{tikzpicture}
\end{example}
\begin{example}
  \tikzstyle{every picture}+=[remember picture]
  \tikzstyle{na} = [shape=rectangle,inner sep=0pt]
  \sentiment{Ein \intensifier[-2]{\tikz\node[na](word1){nicht sehr};}
    \emoexpression[positive]{\tikz\node[na](word2){kluger};}
    \target{Hund} f\"uhrt nur wenige Befehle aus}.\\ (\textit{A not very
    clever dog can execute only few commands.})\label{ex-6}
  \begin{tikzpicture}[overlay]
    \path[->,red,thick](word1) edge [in=20, out=165] node {intensifier\_target} (word2);
  \end{tikzpicture}
\end{example}
The last thing we need to mention here is the fact that even
semantically and syntactically coherent emo-expressions can be
represented as not contiguous strings if an emo-expression is for
example a verb with a separable particle. In these cases you are asked
to draw a \textit{eexpression\_target}-edge from particle or auxiliary
verb to the main verb of emo-expression. So, going back to our example
\ref{ex1}, the final variant of annotation should look like that:
\begin{example}
  \tikzstyle{every picture}+=[remember picture]
  \tikzstyle{na} = [shape=rectangle,inner sep=0pt]
  \sentiment{72 j\"ahriger \source{Leser}\\
    \emoexpression{\tikz\node[na](word2){regt sich};} bei mir\\
    \"uber \target{YouTube} und \target{Gema}\\
    \emoexpression{\tikz\node[na](word1){auf};}} :)\\ (\textit{A
    72-year-old reader got worked up over YouTube and Gema :)})
  \begin{tikzpicture}[overlay]
    \path[->,red,thick](word1) edge [out=20, in=-10, yshift=-2cm] node
         {intensifier\_{}target} (word2);
  \end{tikzpicture}
\end{example}

%%%%%%%%%%%%%%%%%%%%%%%%%%%%%%%%%%%%%%%%%%%%%%%%%%%%%%%%%%%%%%%%%%
\section{Summary of Elements and Attributes}
To help you remember all the introduced tags, attributes and
relations, we provided a short summary of elements and attributes
along with their respective meanings in table \ref{tab:tags} on page
\pageref{tab:tags}.
\begin{table}
  \begin{tabular}{|c|*{5}{>{\centering}p{\cwidth}|}}
    \hline
    Tag & Tag's Meaning & Attributes & Attributes' Values & Examples\tabularnewline\hline

    \multirow{3}{*}{sentiment} &
    \multirow{3}{*}{\parbox{\cwidth}{Polar opinion about some persons,
        facts or actions}} & \textit{sentiment\_polarity} &
    positive/negative &\tabularnewline\cline{3-5}

    & & \textit{sentiment\_sarcasm} & true/false\newline (only for
    negative sentiments) &\tabularnewline\cline{3-5}

    & & \textit{sentiment\_intensity} & 0 (weak)\newline 1 (middle -
    the default)\newline 2 (strong) &\tabularnewline\hline

    \multirow{2}{*}{source} & \multirow{2}{*}{\parbox{\cwidth}{Person who
        experiences the feeling}} & \textit{source\_anaphref} & pointer from
    pronoun inside a sentiment to an antecedent noun outside the sentiment
    relation &\tabularnewline\cline{3-5}

    & & \textit{source\_ref} & pointer from an outside noun to sentiment
    relation &\tabularnewline\hline

    \multirow{2}{*}{target} & \multirow{2}{*}{\parbox{\cwidth}{Subject
        or action about which the opinion is}} &
    \textit{target\_anaphref} & pointer from pronoun inside a
    sentiment to an antecedent noun outside the sentiment relation &
    \tabularnewline\cline{3-5}

    & & \textit{target\_ref} & pointer from an outside noun to sentiment
    relation &\tabularnewline\hline

    \multirow{3}{*}{emo-expression} &
    \multirow{2}{*}{\parbox{\cwidth}{Lexical item with an inherent
        subjective meaning}} & \textit{emo\_polarity} &
    positive/negative & \tabularnewline\cline{3-5}

    & & \textit{emo\_intensity} & 0 (weak)\newline 1 (middle - the
    default)\newline 2 (strong) & \tabularnewline\cline{3-5}

    & & \textit{eexpression\_target} & pointer from a part of an
    emo-expression to another emo-expression which this part belongs
    to & \tabularnewline\hline

    negation & Lexical item which changes the polarity of an
    emo-expression to the opposite & \textit{negation\_target} &
    pointer from negating element to emo-expression which it negates &
    \tabularnewline\hline

    \multirow{2}{*}{intensifier} &
    \multirow{2}{*}{\parbox{\cwidth}{Usually an adverbial or
        adjectival item which increase or decreases the polar meaning
        of an emotional expression\newline (if an intensifier is
        negated, its negation should also be included in tag)}} &
    \textit{int\_intensity} & -2 (strongly decreases polar meaning -
    say almost turns to the opposite)\newline -1 (slightly decreases
    polar meaning)\newline 1 (slightly increases polar
    meaning)\newline 2 (strongly increases polar meaning) &
    \tabularnewline\cline{3-5}

    & & \textit{intensifier\_{}target} & pointer from intensifier to
    emo-expression which it relates to & \tabularnewline\hline
  \end{tabular}
  \caption{A prototype of the Job Information dialog}
  \label{tab:tags}
\end{table}

%%%%%%%%%%%%%%%%%%%%%%%%%%%%%%%%%%%%%%%%%%%%%%%%%%%%%%%%%%%%%%%%%%
\section{What do we mark as sentiment?}

Here is a short summary of cases which should be marked as sentiments.
\begin{enumerate}
\item Clauses and noun phrases expressing polar opinions about some
  things or actions except for cases listed in section
  \ref{sec:not-sentiment}. Examples:
  \begin{itemize}
  \item Das war ein sehr sympathischer und professioneller
    Auftritt und tat der Sendung gut.
  \end{itemize}

\item Exclamations expressing author's emotional attitude to
  particular subject:
  \begin{itemize}
  \item Meister ist nur der \#{}fcbayern!!!
  \item \#{}Merkel hat Deutschland ausgedient, werft sie auf die M\"ullhalde
    der Geschichte!
  \end{itemize}

\item wh-questions in cases when they don't cast doubt on the polar
  opinion (see exceptions in \ref{no-sent:question}):
  \begin{itemize}
  \item Warum die \#{}CDU freiwillige Selbstverpflichtungen so toll findet?
  \end{itemize}
\end{enumerate}


%%%%%%%%%%%%%%%%%%%%%%%%%%%%%%%%%%%%%%%%%%%%%%%%%%%%%%%%%%%%%%%%%%
\section{What do we not mark as sentiment?}\label{sec:not-sentiment}

Here is a summary of cases which should \textbf{not} be marked as
sentiments.
\begin{enumerate}
\item \label{no-sent:question} yes-no- and wh-questions which cast doubt on the presence or validity of a sentiment:
  \begin{itemize}
  \item Findet die \#{}CDU freiwillige Selbstverpflichtungen toll?
  \item Wit toll findet die \#{}CDU freiwillige Selbstverpflichtungen?
  \end{itemize}

\item spam:
  \begin{itemize}
  \item Gr\"o\ss{}er, d\"unner, bunter - die neue Ger\"ateflut bei
    \#Samsung: http://on.wsj.com/19Xhdk3
  \end{itemize}

\item fact statements, even if described facts could potentially be
  assumed as funny, dangerous, tragic or sad:
  \begin{itemize}
  \item H\"oheres Unfall-Risiko bei E-Bikes: Elektro-Fahrr\"ader
    erleben einen Boom. Laut Sch\"atzungen des
    VC\"O... http://t.co/yMZZCe3QKM \#{}vorarlberg
  \end{itemize}

\item sentences stating indifference to some facts:
  \begin{itemize}
  \item Mir doch Latte ob \#{}BVB gewinnt... welche Auswirkungen wird
    das Ergebnis n\"achste Woche f\"ur mich haben? \#{}Fu\ss{}ball ist
    wie DSDS von RTL...
  \end{itemize}

\item sentences in which smileys express politeness rahther than attitude:
  \begin{itemize}
  \item Vielleich k\"onnen wie sp\"ater zusammen Kaffe trinken :)
  \item Hat die \#{}spd jetzt einen Kanzlerkandidaten? :-)
  \end{itemize}
\end{enumerate}

%%%%%%%%%%%%%%%%%%%%%%%%%%%%%%%%%%%%%%%%%%%%%%%%%%%%%%%%%%%%%%%%%%
\section{Open questions}

This scheme still raises some question which should be clarified
before we start off with annotation. These questions are:
\begin{enumerate}
\item Shall we mark wishes as a special kind of sentiment relations?
  \begin{itemize}
  \item Will jetz Radl fahrn oder Inliner :(
  \item Nen weisser papst - . - ICH WOLLTE NEN SCHWARZEN :'(
  \end{itemize}

\item Shall we capture agreement/disagreement with some sentiment
  statements, agreement/disagreement in general?
  \begin{itemize}
  \item Die Konklave w\"ahlt den Papst und dann sagen sie Gott war es
    - Wollt ihr mich verarschen ?!
  \item Lisa singt Hat sie das Zeug zum Superstar?" jaaaaaaa!!!! :D
    NEIN! Riccardo :)
  \item Das geht nun aber wirklich zu weit! :-)
  \item was soll denn der Schei\ss!!
  \item :D das macht allerdings sinn
  \item oookay cool !
  \item "Diese \#{}Bundesregierung ist die erfolgreichste seit der
    \#{}Wiedervereinigung." April, April
  \end{itemize}

\item Shall we try to detect sentiment in cases like below? Possible
  targets are marked in bold:
  \begin{itemize}
  \item Kann man ja keine \textbf{Tickets} kaufen und habe keine gewonnen :((((((
  \item Gleich in Braunschweig mit \textbf{Kamaraden} Treffen :)
  \item In einer halben Stunde zum \textbf{Training} :-)
  \end{itemize}

\item Shall we capture sadness or excitement feelings which don't have
  any particular target?
  \begin{itemize}
  \item Leute ich f\"uhle mich so Hangover m\"a\ss{}ig... WARUM?
  \item hab mir auf die Zunge gebissen :-(
  \end{itemize}
\end{enumerate}

%%%%%%%%%%%%%%%%%%%%%%%%%%%%%%%%%%%%%%%%%%%%%%%%%%%%%%%%%%%%%%%%%%
\section{Examples}

To give you some feeling about how sentiments should be marked, we
provided a couple of examples with full annotation:
\begin{example}
\sentiment[positive]{\source{Ich} habe ein \target{@YouTube-Video} von
  @mathehilfe24 \emoexpression[positive]{positiv} bewertet}
\end{example}
\begin{example}
  \tikzstyle{every picture}+=[remember picture]
  \tikzstyle{na} = [shape=rectangle,inner sep=0pt]
  \tikz\node[na](word1){Clara}; (13) frisst weil \sentiment{\source{\tikz\node[na](word2){sie};} \emoexpression{traurig}
  ist, das sie so \target{fett} ist}.
  \begin{tikzpicture}[overlay]
    \path[->,red,thick](word2) edge [in=10, out=175] node
         {target\_anaphref} (word1);
  \end{tikzpicture}
\end{example}
\begin{example}
\sentiment{\source{Mir} ist \emoexpression{schlecht}, ich sollte keine
  \target{Fefe-Links} klicken am Morgen.}
\end{example}
\begin{example}
'\sentiment{Mit \target{Peer} wird's \emoexpression{schwer}}'.
\end{example}
\begin{example}
  \tikzstyle{every picture}+=[remember picture]
  \tikzstyle{na} = [shape=rectangle,inner sep=0pt]
  \sentiment[positive]{\source{Ich} bin grade\\
    \intensifier{\tikz\node[na](word1){voll};}
    \emoexpression{\tikz\node[na](word2){traurig};}..
    Weil ich nicht bei justin \target{\tikz\node[na](word3){bieber};}
    sein kann}..

  \sentiment[positive]{\source{Ich} \emoexpression[positive]{vermisse}
    \target{\tikz\node[na](word4){ihn};}\\\intensifier{sooo}..}
  \begin{tikzpicture}[overlay]
    \path[->,red,thick](word1) edge [out=175, in=10] node {intensifier\_anaphref}
    (word2);
    \path[->,red,thick](word4) edge [out=20, in=-10] node {target\_anaphref}
    (word3);
  \end{tikzpicture}
\end{example}

%%%%%%%%%%%%%%%%%%%%%%%%%%%%%%%%%%%%%%%%%%%%%%%%%%%%%%%%%%%%%%%%%%
\end{document}
