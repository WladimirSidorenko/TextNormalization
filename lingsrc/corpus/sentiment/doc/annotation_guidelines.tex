\documentclass[11pt,a4paper]{article}

%%%%%%%%%%%%%%%%%%%%%%%%%%%%%%%%%%%%%%%%%%%%%%%%%%%%%%%%%%%%%%%%%
%% Libraries
\usepackage[usenames,dvipsnames]{xcolor}
\usepackage[driver=pdftex,vmargin=2cm,hmargin=2cm]{geometry}
\usepackage{amsthm}
\usepackage{array}
\usepackage{color}
\usepackage{comment}
\usepackage{changepage}
\usepackage{etoolbox}
\usepackage{eurosym}
\usepackage{framed}
\usepackage{hyperref}
\usepackage{multicol}
\usepackage{multirow}
\usepackage{nameref}
\usepackage{paralist}
\usepackage{soul}
\usepackage{url}
\usepackage{wasysym}   % smiley symbols
\usepackage{xparse}
%\usepackage{gb4e}

\hypersetup{colorlinks, citecolor=Violet, linkcolor=Blue,urlcolor=Blue}

%%%%%%%%%%%%%%%%%%%%%%%%%%%%%%%%%%%%%%%%%%%%%%%%%%%%%%%%%%%%%%%%%%
%% Lengths
\newlength{\cwidth}
\setlength{\cwidth}{0.18\textwidth}

\newlength{\clmnwidth}
\setlength{\clmnwidth}{0.65\textwidth}

\newlength{\exmpindent}
\setlength{\exmpindent}{70pt}

%%%%%%%%%%%%%%%%%%%%%%%%%%%%%%%%%%%%%%%%%%%%%%%%%%%%%%%%%%%%%%%%%%
%% Commands
\definecolor{gray80}{RGB}{204,204,204}
\definecolor{dodgerblue4}{RGB}{16,78,139}
\definecolor{orange3}{RGB}{205,133,0}
\definecolor{DarkSlateBlue}{RGB}{72,61,139}

\newcommand{\corpusDir}{\texttt{sentiment}}
\newcommand{\authorAddress}{\texttt{sidarenk@uni-potsdam.de}}
%% Code Environment
\NewDocumentCommand{\Colorbox}{O{\dimexpr\linewidth-2\fboxsep} m m}{%
  \colorbox{#2}{\makebox[#1][l]{#3}}}
\newcommand{\code}[1]{

\smallskip\noindent\Colorbox{gray80}{\parbox{\textwidth}{\texttt{#1}}}\smallskip

\noindent}
%% Examples Environment
\newtheoremstyle{mytheoremstyle} % name
    {\topsep}                    % Space above
    {\topsep}                    % Space below
    {\parindent=\exmpindent\hangindent=\exmpindent} % Body font
    {}                           % Indent amount
    {\scshape}                   % Theorem head font
    {.}                          % Punctuation after theorem head
    {.5em}                       % Space after theorem head
    {}  % Theorem head spec (can be left empty, meaning `normal')

\theoremstyle{mytheoremstyle}
\newtheorem*{exmp*}{Example}
\newtheorem{exmp}{Example}[section]
%% Annotattion Tags
\newcommand{\mtag}[2]{{\upshape[\emph{#2}\upshape]$_{\textrm{\bfseries\emph{\tiny
        #1}}}$}}

\newcommand{\sentiment}[2][]{\mtag{sen\-ti\-ment\ifstrempty{#1}{}{:#1}}{#2}}
\newcommand{\source}[2][]{\mtag{source\ifstrempty{#1}{}{:#1}}{#2}}
\newcommand{\target}[2][]{\mtag{tar\-get\ifstrempty{#1}{}{:#1}}{#2}}
\newcommand{\emoexpression}[2][]{\mtag{emo-\-ex\-pression\ifstrempty{#1}{}{:#1}}{#2}}
\newcommand{\intensifier}[2][]{\mtag{in\-ten\-si\-fier\ifstrempty{#1}{}{:#1}}{#2}}
\newcommand{\diminisher}[2][]{\mtag{di\-mi\-ni\-sher\ifstrempty{#1}{}{:#1}}{#2}}
\newcommand{\negation}[2][]{\mtag{ne\-ga\-tion\ifstrempty{#1}{}{:#1}}{#2}}

%%%%%%%%%%%%%%%%%%%%%%%%%%%%%%%%%%%%%%%%%%%%%%%%%%%%%%%%%%%%%%%%%%
%% Title
\author{Uladzimir Sidarenka}
\date{\today}
\title{Guidelines for the Annotation of the Sentiment Corpus}

%%%%%%%%%%%%%%%%%%%%%%%%%%%%%%%%%%%%%%%%%%%%%%%%%%%%%%%%%%%%%%%%%%
%% Main
\begin{document}
\maketitle{}
\section{Overview}
\subsection{Introduction}

In this assignment, your task is to annotate sentiments in a corpus of
Twitter messages.  We define \emph{sentiments} as polar (either
positive or negative) evaluative subjective opinions about some
persons, subjects, or events.  In this assignment, you have to
annotate both -- text spans denoting the opinions (\emph{sentiments})
and text spans denoting the subjects and events being evaluated
(\emph{sentiment targets}).  Additionally, you should also mark
opinions' holders (\emph{sentiment sources}) and lexical elements
which might significantly change the polarity or the intensity of a
sentiment.  These elements are:
\begin{itemize}
  \item \emph{emotional expressions}, which are words or phrases that
    unequivocally possess some evaluative lexical meaning by
    themselves (these are typically words like \emph{hassen}
    (\emph{hate}), \emph{bewundern} (\emph{admire}), \emph{sch\"on}
    (\emph{nice}) etc.);
  \item \emph{negations}, which are words or expressions that might
    completely flip the polarity of an emotional expression or a
    sentiment to the opposite (e.g. \emph{nicht} gut (\emph{not}
    good), \emph{kein} Talent (\emph{not} a talent) etc.);
  \item \emph{intensifiers} and \emph{diminishers} (or
    \emph{downtoners}), which are words and expressions that might
    increase or decrease the evaluative sense of an emotional
    expression, respectively.  Examples of intensifiers include words
    like \emph{sehr} (\emph{very}), \emph{besonders}
    (\emph{especially}), \emph{insbesondere} (\emph{particularly})
    etc.  Typical examples of diminishers are words like \emph{ein
      wenig} (\emph{a little}), \emph{ein bisschen} (\emph{a bit}),
    \emph{gewisserma\ss{}en} (\emph{to a certain degree}) etc.
\end{itemize}

After marking these elements, you should also specify the values of
their attributes.  A complete list of all elements along with the
description of their possible attributes is given in Section
\ref{sec:markables}.  In Section \ref{sec:summary} we also provide a
short summary of the task.  In Section \ref{sec:faq}, you then may
find answers to some questions which caused particular difficulties
during the previous runs of annotation.  Finally, Section
\ref{sec:examples} gives a couple of complete annotation examples for
some sentences from our corpus.

\subsection{Terminology and Format}
\textbf{Terminology.} Throughout this document, we use the term
\emph{markable} to denote an annotated span of text.  The term \emph{markable
  type} (or \emph{markable tag}) refers to the tag assigned to that markable.
Additional attributes associated with the annotated text spans are called
\emph{markable attributes}.

We do not make a distinction between the terms \emph{opinions} and
\emph{sentiments} and use both words interchangeably throughout this text.

\smallskip\noindent\textbf{Format.} In these guidelines, we rely on
the following conventions regarding the text format.

We specify shell commands in gray boxes in \texttt{typesetting} font
as shown in the example below: \code{echo 'Hello world!'}  The
\texttt{typesetting} font is also used for literal mentions of
markable types, markable attributes, file names, directory paths, and
executable commands.

Examples of words and phrases are given in \textit{italics} and their
respective English translation is provided in parentheses.

Examples of sentence annotations are shown in regular font. Text
enclosed in markables is \emph{emphasized} and surrounded by square
brackets (e.g. [\emph{markable text}]).  The type of the markables is
given as a subscript after the closing right bracket; optional
markable attributes are specified after the type, separated from it by
a colon, e.g.:
\begin{exmp}
  \sentiment[polarity=positive]{\target{Der neue Papst} gilt als
    \emoexpression{bescheidener}, \emoexpression{zur\"uckgenommener}
    Typ.}

{\footnotesize(\sentiment[polarity=positive]{\target{The new Pope} is
    believed to be a \emoexpression{sober}, \emoexpression{modest}
    man.})}\label{exmp:1}
\end{exmp}

\subsection{Annotation Tool}

For annotating this corpus, you should use \texttt{MMAX2} -- a freely
available annotation tool -- which you can download under the
following link:

{\setlength{\parindent}{0pt}\small\url{http://sourceforge.net/projects/mmax2/files/mmax2/mmax2_1.13.003/MMAX2_1.13.003b.zip/download}}

After you have downloaded this file, you should unpack the received
archive, change to the newly created directory \texttt{1.13.003/MMAX2}
in your terminal shell and execute the following commands: \code{chmod
  u+x ./mmax2.sh\\ nohup ./mmax2.sh \&} An \texttt{MMAX2} window
should then appear on your screen.  If you have never used
\texttt{MMAX2} before, please read the document
\texttt{mmax2quickstart.pdf} which you can find in the subdirectory
\texttt{MMAX2/Docs} of the downloaded archive.

\subsection{Corpus Files}

You should also have received a copy of corpus files either as a
tar-gzipped archive or via the version control system \texttt{Git}.

If you got a \texttt{.tgz} archive of the corpus, then
unpack it using the command: \code{tar -xzf archive-name.tgz} After
that, a directory called \corpusDir{} should appear in your current
folder.

If you received access to the \texttt{Git} repository of the project,
you should clone the project to your local computer using the command:

\code{git clone
  ssh://hebe.ling.uni-potsdam.de/var/local/git/Depot/socmedia
  socmedia}

\noindent{}A directory called \texttt{socmedia} should then appear in
the current folder on your local computer.  You can find your
annottation files in the directory
\texttt{lingsrc/corpus/\corpusDir{}/annotator-ANNOTATOR\_ID} in the
newly appeared \texttt{socmedia} folder (ANNOTATOR\_ID is the ID
number which was previously assigned to you by the author of these
guidelines).

In order to load an annotation file into your \texttt{MMAX2} program,
please change to the \texttt{MMAX2} window and click on the menu
\texttt{File -> Load}.  In the displayed pop-up window, select the
path to the \corpusDir{}\texttt{/anno\-ta\-tor-ANNOTATOR\_ID}
folder\footnote{Please, make sure that the path to the \corpusDir{}
  folder does not contain any white spaces.  Otherwise, \texttt{MMAX2}
  might fail to load the project.} and click on one of the
\texttt{*.mmax} files found there.  The chosen project should then be
loaded into your \texttt{MMAX2} editor.

If you have any difficulties with launching \texttt{MMAX2} or loading
project files into it, please contact the author of these guidelines
via e-mail (\authorAddress{}).

%%%%%%%%%%%%%%%%%%%%%%%%%%%%%%%%%%%%%%%%%%%%%%%%%%%%%%%%%%%%%%%%%%

\section{Tags and Attributes}\label{sec:markables}
In the following, we provide a short list of all markables and their possible
attributes that will be used in this annotation:

\begin{multicols}{2}
  \begin{enumerate}
  \item \texttt{sentiment}-markable with the attributes:
    \begin{enumerate}
    \item \texttt{polarity},
    \item \texttt{intensity},
    \item \texttt{sarcasm};
    \end{enumerate}
  \item \texttt{target}-markable with the attributes:
    \begin{enumerate}
    \item \texttt{preferred},
    \item \texttt{anaph-ref},
    \item \texttt{sentiment-ref};
    \end{enumerate}
  \item \texttt{source}-markable with the attributes:
    \begin{enumerate}
    \item \texttt{anaph-ref},
    \item \texttt{sentiment-ref};
    \end{enumerate}
  \item \texttt{emo-expression}-markable with the attributes:
    \begin{enumerate}
    \item \texttt{polarity},
    \item \texttt{intensity},
    \item \texttt{sarcasm},
    \item \texttt{sentiment-ref};
    \end{enumerate}
  \item \texttt{intensifier}-markable with the attributes:
    \begin{enumerate}
    \item \texttt{degree},
    \item \texttt{emo-expression-ref};
    \end{enumerate}
  \item \texttt{diminisher}-markable with the attributes:
    \begin{enumerate}
    \item \texttt{degree},
    \item \texttt{emo-expression-ref};
    \end{enumerate}
  \item and, finally, the \texttt{negation}-markable with the
    attribute:
    \begin{enumerate}
    \item \texttt{emo-expression-ref}.
    \end{enumerate}
  \end{enumerate}
\end{multicols}
\noindent{}A more detailed description o these markables and the
values of their respective attributes is given in the next
subsections.

\subsection{sentiment}\label{sec:sentiment}
\noindent\textbf{Definition.} \emph{Sentiments} are polar subjective
evaluative opinions about people, subjects, or events.

According to our definition, a sentiment must always satisfy the
following three conditions:
\begin{itemize}
\item it has to be \textbf{polar}, i.e. a sentiment should always
  reflect either positive or negative attitude to its respective
  target.  Cases like \textit{Ich glaube, er wird heute fr\"uher
    kommen} (\textit{I think he will be earlier today}) should not be
  marked as \texttt{sentiment} because the attitude of the author is
  neither positive nor negative but neutral;

\item it has to be \textbf{subjective}, i.e. you should not mark as
  \texttt{sentiment}s mere statements of objective facts like, for
  example, \textit{Beim Angriff wurden 14 Glasscheiben besch\"adigt}
  (\textit{14 glass plates were broken during the attack}), even if
  you have your personal negative associations with the reported
  events.  Sentiment instead should always unequivocally show
  \emph{the personal opinion of the immediate author of an
    expression};

\item a sentiment has to be \textbf{evaluative}, which means that it
  should always refer to an explicit target and judge about its
  properties.  You should not mark as \texttt{sentiment}s cases like
  \textit{Ich bin heute so gl\"ucklich} (\textit{I am so happy today})
  because these statements do not evaluate anything in particular but
  simply express emotions.
\end{itemize}

\noindent\textbf{Example.} Typical examples of sentiments are evaluative
sentences like the one show in Example \ref{ex:sentiment}.
\begin{exmp}
  \sentiment{Ich mag den neuen James Bond Film nicht.}

  (\sentiment{I don't like the new James Bond movie.})\label{ex:sentiment}
\end{exmp}
\noindent This sentence expresses a personal subjective opinion of the author,
this opinion is polar and strictly negative, and it also has an explicit
evaluation target -- the \textit{movie}.  So, we put \texttt{sentiment} tags
around this sentence.

We also consider contrastive comparisons to be a special type of
evaluative opinions.  But unlike other types of sentiments,
comparisons typically express a relative subjective judgement, i.e. an
object is regarded to be either better or worse than another compared
object, but we usually do not know if the author actually likes or
dislikes either of them.  For this type of evaluations, we have
introduced a special value of the \texttt{polarity} attribute --
called \texttt{comparison} (cf. Table \ref{tbl:sentiment}).

You should NOT mark as sentiments polar opinions for which its unknown
if they are true or not.  These typically are sentences like
\textit{Ich wei\ss{} nicht, ob ich meinen Bruder mag} (\textit{I don't
  know if I like my brother}).  In this example, neither we nor the
author actually know if the author likes or dislikes her brother.
Exceptions from this rule are cases like \textit{Ich zweifle, dass er
  ein guter Mensch ist} (\textit{I doubt that he is a good man}) or
\textit{Ich glaube nicht, dass er diesen Preis verdient hat}
(\textit{I don't think that he has deserved this award}) which express
author's disagreement with some positive evaluations and, therefore,
act themselves as negative judgements about the targets.  Special care
should be taken when dealing with questions and irrealis
(cf. Questions \ref{qstn:interrogative}, \ref{qstn:wish},
\ref{qstn:condition}, and \ref{qstn:irrealis} in \nameref{sec:faq}
Section of this document).

\noindent\textbf{Boundaries.} \texttt{sentiment} markables should
encompass both the object being evaluated (the target) and the actual
phrase phragment which expresses the evaluation (typically an
emo-expression, if it exists).  After determining these two elements,
you should put the \texttt{sentiment} tags around the \emph{minimal
  complete syntactic or discourse-level unit in which both target and
  evaluation expression appear together}.

In Example \ref{exmp:book}, for instance, the evaluated object is
\textit{Buch} (\textit{book}), the evaluative expression is
\textit{langweiliges} (\textit{boring}), and the minimal syntactic unit which
simultaneously covers both of these elements is the noun phrase \textit{ein
  langweiliges Buch} (\textit{a boring book}).  We therefore put
\texttt{sentiment} tags around this noun phrase but do not put anything else
inside them.
\begin{exmp}
  Auf dem Tisch lag \sentiment{ein langweiliges Buch}.

  (There was \sentiment{a boring book} on the table.)\label{exmp:book}
\end{exmp}
Sentiments are not restricted to just noun phrases, they can also be
expressed by complete clauses or even multiple sentences
(i.e. discourse-level units).  The main point is that a
\texttt{sentiment} span has to be \emph{complete}, i.e. it should
capture the common syntactic or discourse-level ancestor element of
both evaluation and target and also include all other decendants of
that common ancestor.  Furthermore, a \texttt{sentiment} markable has
to be \emph{minimal}, i.e. it should only cover the lowest possible
ancestor element of evaluation and target but should not include
parents or siblings of this ancestor.

Example \ref{exmp:petterson} shows how a sentiment relation can be
expressed by a clause:
\begin{exmp}
  Wir akzeptieren das, weil \sentiment{wir alle ein bisschen in
    Petterson verliebt sind}.

  (We accept this because \sentiment{we all are a little bit in love
    with Petterson}.)\label{exmp:petterson}
\end{exmp}
\noindent In this sentence, the evaluative statement is made about
\textit{Petterson} who acts as sentiment's target.  The author says
that they all \textit{in ihn verliebt sind} (\textit{are in love with
  him}) which is her subjective evaluative opinion.  Both target and
evaluative expression appear together in one verb phrase with the head
verb \textit{sein} (\textit{to be}).  So, we mark this complete verb
phrase including its grammatical subject \textit{wir} (\textit{we})
which is the syntactic descendant of the head verb.

%% \begin{itemize}
%% \item single noun phrases, possibly with their prepositional
%%   attributes, e.g. \textit{Auf dem Tisch lag \sentiment{ ein
%%       langweiliges Buch} (There was \sentiment{a boring book} on the
%%     table)};
%% \item clauses, e.g. \textit{\sentiment{Ich hasse B\"ucher ohne
%%   Inhaltsangabe}. (\sentiment{I hate books without table of
%%   contents})};
%% \item multiple sentences in cases when these sentences jointly form a
%%   sentiment relation, e.g.\\\textit{\sentiment{Sie denken, reden,
%%       riechen, lieben, schmecken, f*cken Plastik. Sie haben\\das so
%%       gelernt in der Plastik-Werbewelt}.\\ (\sentiment{They think and
%%       speak about, smell at, love, taste, f*ck plastic.  They
%%       have\\ learned it so in the advertising world of plastic}.)}.
%% \end{itemize}

\noindent\textbf{Attributes.} After you have marked a \texttt{sentiment} span,
you should next set the values of its attributes.  Acceptable attributes with
their meanings and values are given in Table \ref{tbl:sentiment}.
\begin{center}
  \begin{table}[ht]
    \caption{Attributes and values of \texttt{sentiment}s.}
    \begin{tabular}{|l|c|p{\clmnwidth}|}\hline
      Attribute & Value & Value's Meaning\\\hline
      %%%%%%%%%%%%%%%%%%%

      & \textit{positive} & sentiment expresses positive attitude about
      its respective target, e.g. \textit{Es war ein fantastischer Abend
        (It was a fantastic evening)};\\\cline{2-3}

      & \textit{negative\newline(default)} & sentiment expresses
      negative attitude about its respective target, e.g. \textit{Seine
        Schwester ist einfach unausstehlich (His sister is simply
        obnoxious)}\\\cline{2-3}

      \multirow{-3}{*}{polarity} & \textit{comparison} & sentiment
      expresses a comparison of two objects with preference given to one
      of them, e.g. \textit{Mir gef\"allt das rote Kleid mehr als das
        blaue (I like the red dress more than the blue one)}\\\hline

      %%%%%%%%%%%%%%%%%%%

      & \textit{weak} & sentiment expresses a weak evaluative opinion,
      e.g. \textit{Der Auftritt war mehr oder weniger gut (The
        appearance was more or less good)}\\\cline{2-3}

      & \textit{medium\newline(default)} & sentiment has a middle
      emotional expressivity, e.g. \textit{Mir hat das neue Album gut
        gefallen (I enjoyed the new album)}\\\cline{2-3}

      \multirow{-3}{*}{intensity} & \textit{strong} & this sentiment
      expresses a very emotional polar statement, e.g. \textit{Dieses
        Festival war einfach umwerfend!!! (This festival was simply
        terrific!!!)}\\\hline
      %%%%%%%%%%%%%%%%%%%

      \multirow{2}{*}{sarcasm} & \textit{true} & this polar attitude is
      derisive, i.e. its actual polarity is the opposite of its apparent
      form. (An apparent praise, for example, could be meant as a rebuke
      and vice versa. The actual sense, however, can only be inferred on
      the basis of world knowledge or reasoning.)  An example of a
      sarcastic sentiment is the following passage: \textit{Mein
        J\"ungerer ist in der Pr\"ufung durchgefallen.  Klasse! (My
        youngest has failed his exam.  Well done!)}  In this case, you
      should set the polarity attribute of the sentiment to
      \texttt{negative} and the value of the \texttt{sarcasm} attribute
      to \textit{true}.\\\cline{2-3}

      & \textit{false\newline(default)} & no sarcasm is present -- the
      polar attitude has its literal meaning; this is the default.
      setting\\\hline
    \end{tabular}
    \label{tbl:sentiment}
  \end{table}
\end{center}

%%%%%%%%%%%%%%%%%%%%%%%%%%%%%%%%%%%%%%%%%%%%%%%%%%%%%%%%%%%%%%%%%%%%%%%%%%%%%%%%%%%%%%%%%%
\subsection{target}
\noindent\textbf{Definition.} \emph{Targets} are objects or events
that are being evaluated by a sentiment expression.

\noindent{} Because sentiments are required to be evaluative, there
MUST always be at least one target for each sentiment relation.

\noindent\textbf{Example.} An example of a sentiment target is given in
Sentence \ref{exmp:target}:
\begin{exmp}
Mein Bruder ist nicht begeistert von \target{dem neuen Call of Duty}.

(My brother is not impressed by \target{the new Call of
  Duty}.)\label{exmp:target}
\end{exmp}
\noindent In this sentence, the author is telling us about the subjective
opinion of her brother regarding the new version of a computer game.  This new
computer game is the object of the evaluation and we annotate it as
\texttt{target}.

\noindent\textbf{Boundaries.} Similar to \texttt{sentiment}s, you should put
the \texttt{target} tags around the minimal complete syntactic or
discourse-level units which denote the objects or events being evaluated.
These are usually noun phrases (e.g. \textit{Mir wird's schlecht, wenn ich
  \target{diese Werbung} im Fernsehen sehe} (\textit{I feel sick when I see
  this \target{ad} on TV})) or clauses (e.g. \textit{Ich hasse wenn
  \target{Voldemort mein Shampoo benutzt}.} (\textit{I hate when
  \target{Voldemort is using my shampoo}})).

If a sentiment has multiple targets, you should mark each one of them
separately (cf. Example \ref{exmp:trg-conj}).
\begin{exmp}
  Meiner Mutter haben \target{Nelken} und \target{Dahlien} immer gefallen.

  (My mother has always liked \target{carnations} and
  \target{dahlias}.)\label{exmp:trg-conj}
\end{exmp}

Similar, in comparisons, you should also annotate each compared object
separately.  Additionally, for the object which is being dispreferred,
you should also set the value of the \texttt{preferred} attribute to
\texttt{false} (cf. Example \ref{exmp:trg-comp}).
\begin{exmp}
  Ich mag \target[preferred=true]{Domino-Eis} mehr als
  \target[preferred=false]{Magnum}.

  (I like \target[preferred=true]{Domino ice cream} more than
  \target[preferred=false]{Magnum}.)\label{exmp:trg-comp}
\end{exmp}

\noindent\textbf{Attributes.} Further possible attributes of \texttt{target}s
are shown in Table \ref{tbl:target}.
\begin{center}
  \begin{table}[h]
    \caption{Attributes and values of \texttt{target}s.}
    \begin{tabular}{|l|c|p{0.94\clmnwidth}|}\hline
      Attribute & Value & Value's Meaning\\\hline

      & \textit{true} (default) & in comparisons, this value means that
      the respective target is being considered better than another
      compared object, e.g. \textit{\emph{Die neue Frisur} passt ihr
        garantiert besser als die alte (\emph{The new hairstyle} suits
        her definitely better than the old one)};\\\cline{2-3}

      \multirow{-2}{*}{preferred} & \textit{false} & in comparisons,
      this value marks the target element which is being considered
      worse than its counterpart, e.g. \textit{Die zweite Saison von
        Breaking Bad war viel spannender als \emph{die dritte} (The
        second season of Breaking Bad was much more exciting than
        \emph{the third one})};\\\hline

      sentiment-ref & \textit{$\longrightarrow$\newline(directed
        edge)} & directed edge pointing from \texttt{target} to its
      respective \texttt{sentiment}.  You need to draw this edge in
      two cases:
      \begin{itemize}
      \item when the \texttt{target} is located at intersection of two
        different \texttt{sentiment}s (in this case, you should draw
        an edge from \texttt{target} to \texttt{sentiment}, which this
        \texttt{target} actually belongs to),

      \item when the target of an opinion is expressed outside the
        \texttt{sentiment} span;
      \end{itemize}\\\hline

      anaph-ref & \textit{$\longrightarrow$\newline(directed edge)} &
      directed edge pointing from \texttt{target} expressed by a
      pronoun or pronominal adverb to its respective non-pronominal
      antecedent (in order to draw this edge, you also need to mark
      the antecedent as \texttt{target})\\\hline
    \end{tabular}
    \label{tbl:target}
  \end{table}
\end{center}

%%%%%%%%%%%%%%%%%%%%%%%%%%%%%%%%%%%%%%%%%%%%%%%%%%%%%%%%%%%%%%%%%%
%% Source
\subsection{source}
\noindent\textbf{Definition.} Sentiment \emph{sources} are immediate author(s)
or holder(s) of evaluative opinions.  These are typically writers of the
messages or persons or institutions whose opinion is being cited.

If sentiment's source of is not explicitly mentioned in the message,
we assume it to be the author of the tweet. You need not annotate
anything as \texttt{source} in this case.

\noindent\textbf{Example.} An example of an explicit sentiment source
is the pronoun \textit{Sie} (\textit{she}) in Example
\ref{exmp:source}.
\begin{exmp}
  \source{Sie} mag die neue Farbe nicht

  (\source{She} doesn't like the new color)\label{exmp:source}
\end{exmp}

Note that in case of citations you should only mark the immediate
person or the institution whose original opinion is being cited, but
you should not mark the citing person as a \texttt{source}
(cf. Example \ref{exmp:source-citation}).
\begin{exmp}
  Laut Staatsanwalt soll die \source{Angeklagte} sich missbilligend \"uber
  ihren Vorgesetzten ge\"au\ss{}ert haben.

  (According to the attorney, the \source{defendant} had made
  disapproving remarks about her boss.)\label{exmp:source-citation}
\end{exmp}

\noindent\textbf{Boundaries.} For determining the boundaries of
\texttt{source}s, you should proceed in similar fashion as we did for
\texttt{target}s and \texttt{sentiment}s and only mark complete
minimal syntactic units.  Sources are most commonly expressed by noun
phrases.  And, similar to \texttt{target}s, if the source of a
sentiment is expressed by multiple separate noun phrases, you should
mark each of them separately (cf. Example \ref{exmp:source2}).

\begin{exmp}
  \source{Ihr} und \source{ihrer Mutter} gef\"allt die neue Farbe
  nicht.\\ (Neither \source{she} and \source{her mother} likes the new
  color)\label{exmp:source2}
\end{exmp}

\noindent\textbf{Attributes.} The attributes of the \texttt{source}
tag are listed in Table \ref{tbl:source}.  They are fully identical to
the attributes of the \texttt{target} markables.
\begin{center}
  \begin{table}[h]
    \caption{Attributes and values of \texttt{source}s.}
    \begin{tabular}{|l|c|p{0.935\clmnwidth}|}\hline
      Attribute & Value & Value's Meaning\\\hline

      sentiment-ref & \textit{$\longrightarrow$\newline(directed
        edge)} & cf. Table \ref{tbl:target}\\\hline

      anaph-ref & \textit{$\longrightarrow$\newline(directed edge)} &
      cf. Table \ref{tbl:target}\\\hline
    \end{tabular}\label{tbl:source}
  \end{table}
\end{center}

%%%%%%%%%%%%%%%%%%%%%%%%%%%%%%%%%%%%%%%%%%%%%%%%%%%%%%%%%%%%%%%%%%
%% Emo-expression
\subsection{emo-expression}
\noindent\textbf{Definition.} \emph{Emo-expressions} are words or
phrases that have a polar evaluative lexical meaning by themselves.

\noindent\textbf{Example.} An example of an emo-expression is the word
\textit{ekelhaft} (\textit{disgusting}) in Sentence
\ref{exmp:emo-expr1}.
\begin{exmp}
  Beim Aufr\"aumen des Zimmers haben wir einen
  \emoexpression{ekelhaften} Teller mit verschimmeltem Essen unter dem
  Bett gefunden.

  (When we cleaned the room, we found a \emoexpression{disgusting}
  plate with moldy food under the bed.)\label{exmp:emo-expr1}
\end{exmp}

In contrast to \texttt{source}s and \texttt{target}s which should only
be marked in the presence of a \texttt{sentiment}, you should always
annotate emotional expressions in text no matter if a target-oriented
sentiment exists or not.

Note, however, that because many words and idioms are ambiguous and
can take on many different lexical meanings, it can often be the case
that only some of these lexical meanings are evaluative and
subjective.  In such cases, you should only mark as
\texttt{emo-expression}s words whose actual sense in the given context
is polar.  If these words have an objective meaning in other contexts,
you should not annotate them as \texttt{emo-expression}s in that
cases.
\begin{exmp}
  Dieser Wein ist ein echtes \emoexpression{Juwel} in meiner
  Kollektion.

  (This wine is a real \emoexpression{jewel} in my collection.)

  Koh-i-Noor ist das teuerste Juwel heutzutage.

  (Koh-i-Noor is the most expensive jewel nowadays.)\label{exmp:emo-expression-jewel}
\end{exmp}
\noindent{}In example \ref{exmp:emo-expression-jewel}, for instance,
the meaning of the word \textit{Juwel} (\textit{jewel}) is metaphoric
and subjective in the first sentence, but literal and objective in the
second.  So you should only annotate this word as
\texttt{emo-expression} in the former case but not annotate it in the
latter.

\noindent\textbf{Boundaries.} \texttt{emo-expression}s are typically
expressed by:
\begin{itemize}
  \item nouns, e.g. \textit{Held (hero)}, \textit{Ideal (ideal)},
    \textit{Betr\"uger (fraudster)} etc.;

  \item adjectives or adverbs, e.g. \textit{sch\"on (nice)},
    \textit{zuverl\"assig (reliably)}, \textit{hinterh\"altig
      (devious)}, \textit{heimt\"uckisch (insidiously)} etc.;

  \item verbs, e.g. \textit{lieben (to love)}, \textit{bewundern (to
    admire)}, \textit{hassen (to hate)} etc.;

  \item idioms, e.g. \textit{auf die Nerven gehen (to get on one's
    nerves)} etc.;

  \item smileys, e.g. :), :-(, \smiley{}, \frownie{} etc.
\end{itemize}
If an \texttt{emo-expression} is formed by an idiomatic phrase, you should
always annotate the complete idiom.  For verbs which take on an evaluative
sense only if used with certain prepositions (e.g. \textit{to go for sth.} in
the sense of \textit{to like}), you should annotate both the verb and the
preposition as a single markable (please refer to the \texttt{MMAX} manual to
see how to annotate discontinuous spans).

\noindent\textbf{Attributes.} When determining the value of the
\texttt{polarity} attribute of an \texttt{emo-expression}, you should
disregard any possible contextual modifiers like intensifiers or
negations and set the value of this attribute to the lexical (or also
called \emph{prior}) polarity of the phrase (the one it would have
without any negations and other modifiers) (cf. Example
\ref{exmp:emo-expression-polarity}).
\begin{exmp}
Es war keine \emoexpression[polarity=positive]{gute} Idee.

(It was not a \emoexpression[polarity=positive]{good} idea.)\label{exmp:emo-expression-polarity}
\end{exmp}

Also, when determining the value of the \texttt{polarity} attribute of
an \texttt{emo-expression}, you should analyze its polarity from the
perspective of the subject or event which is being evaluated (in case
when such subject is present in the context).  This means that in
cases like \textit{Ich vermisse meine Freundin} (\textit{I miss my
  girlfriend}), the polarity of the emo-expression \textit{vermissen}
(\textit{to miss}) is still positive because the author evidently has
a positive attitude to the girlfriend even if he experiences sadness
because of her absence.

Further attributes of \texttt{emo-expression}s include \texttt{intensity},
\texttt{sarcasm}, and \texttt{sentiment-ref}.  Possible values and
descriptions of these attributes are summarized in Table
\ref{tbl:emo-expression}.
\begin{center}
  \begin{table}[ht]
    \caption{Attributes and values of \texttt{emo-expression}s.}
    \begin{tabular}{|l|c|p{0.935\clmnwidth}|}\hline
      Attribute & Value & Value's Meaning\\\hline
      %%%%%%%%%%%%%%%%%%%

      & \textit{positive} & emotional expression has a positive
      evaluative meaning, e.g. \textit{gut (good), verhimmeln (to
        ensky), Prachtkerl (corker)} etc.\\\cline{2-3}

      \multirow{-2}{*}{polarity} & \textit{negative\newline(default)}
      & emotional expression has a negative evaluative meaning towards
      its target, e.g. \textit{versauen (to botch up), rotzig
        (snotty), Dreckskerl (scum)} etc.\\\hline

      %%%%%%%%%%%%%%%%%%%

      & \textit{weak} & emo-expression has a weak evaluative sense,
      e.g. \textit{solala (so-so), nullachtf\"unfzehn (vanilla),
        durchschnittlich (mediocre)} etc.\\\cline{2-3}

      & \textit{medium\newline(default)} & emo-expression has middle
      stylistic expressivity, e.g. \textit{gut (good), schlecht (bad),
        robust (tough)} etc.\\\cline{2-3}

      \multirow{-3}{*}{intensity} & \textit{strong} & emo-expression
      expresses a very strong positive or negative evaluation,
      e.g. \textit{allerbeste (bettermost), zum Kotzen (to make one
        puke), Kacke (shit)} etc.\\\hline

      %%%%%%%%%%%%%%%%%%%%

      \multirow{2}{*}{sarcasm} & \textit{true} & emo-expression is
      derisive, i.e. its actual polarity is the opposite of its
      apparent form. (This means that an apparent praise which appears
      in text is in fact meant as a rebuke and vice versa. The actual
      sense, however, can only be inferred on the basis of world
      knowledge or reasoning.)\\\cline{2-3}

      & \textit{false\newline(default)} & no sarcasm is present -- the
      polar attitude has its literal meaning; this is the default
      setting\\\hline

      %%%%%%%%%%%%%%%%%%%

      sentiment-ref & \textit{$\longrightarrow$\newline(directed
        edge)} & arrow pointing to the \texttt{sentiment} which this
      \texttt{emo-expression} belongs to.  You should only draw this
      edge if an \texttt{emo-expression} is located in the overlapping
      of two \texttt{sentiment} spans or outside of the
      \texttt{sentiment} span which it belongs to\\\hline
    \end{tabular}
    \label{tbl:emo-expression}
  \end{table}
\end{center}

%%%%%%%%%%%%%%%%%%%%%%%%%%%%%%%%%%%%%%%%%%%%%%%%%%%%%%%%%%%%%%%%%%
%% Intensifier
\subsection{intensifier}
\noindent\textbf{Definition.} \emph{Intensifiers} are elements which increase
the expressivity and the polar sense of an emotional expression.

\noindent\textbf{Example.} An example of an intensifier is the word
\textit{sehr} (\textit{very}) in Example \ref{exmp:intensifier}.
\begin{exmp}
  Wir suchen eine \intensifier{sehr} zuverl\"assige Polin als
  Haushaltshilfe.

  (We are looking for a \intensifier{very} reliable Polish woman as
  domestic help.)\label{exmp:intensifier}
\end{exmp}
\noindent\textbf{Boundaries.} Intensifiers are usually expressed by
adverbs or adjectives like \textit{sehr} (\textit{very}), sicherlich
(\textit{certainly}) etc., but other ways of expressing them are still
possible (cf. Example \ref{exmp:intensifier-comp}).
\begin{exmp}
  Dieser Junge ist stark \intensifier{wie ein Pferd}.

  (This boy is strong \intensifier{as a
    horse}.)\label{exmp:intensifier-comp}
\end{exmp}

\noindent\textbf{Attributes.} An \texttt{intensifier} should always
relate to some \texttt{emo-expression} and you should also always
explicitly show that relation by drawing an edge attribute from
\texttt{intensifier} to its respective \texttt{emo-expression}
markable.

Further possible attributes of \texttt{intensifier}s are shown in
Table \ref{tbl:intensifier}.
\begin{center}
  \begin{table}[htb]
    \caption{Attributes and values of \texttt{intensifier}s.}
    \begin{tabular}{|l|c|p{0.875\clmnwidth}|}\hline

      & \textit{medium (default)} & the intensifier moderately
      increases the polar sense of the emotional expression,
      e.g. \textit{ziemlich (quite), recht (fairly)} etc.\\\cline{2-3}

      \multirow{-2}{*}{degree} & \textit{strong} & the intensifier
      strongly increases the polar sense and stylistic markedness of the
      emotional expression, e.g. \textit{sehr (very), super (super),
        stark (strongly)} etc.\\\hline

      %%%%%%

      emo-expression-ref & \textit{$\longrightarrow$\newline(directed
        edge)} & a directed edge pointing from the intensifier to the
      \texttt{emo-expression} whose meaning is being intensified\\\hline
    \end{tabular}
    \label{tbl:intensifier}
  \end{table}
\end{center}

\subsection{diminisher}
\noindent\textbf{Definition.} \emph{Diminisher}s are words or phrases
that decrease the polar lexical sense of an \texttt{emo-expression}.

\noindent\textbf{Example.} In Example \ref{exmp:diminisher}, the
diminisher is expressed by the adverb \textit{weniger}
(\textit{less}).
\begin{exmp}
  \diminisher{Weniger} erfolgreiche Unternehmen verzichten auf externe
  Berater.\label{exmp:diminisher}

  The \diminisher{less} successful companies do not use external
  consultants.
\end{exmp}
\noindent\textbf{Attributes.} Similar to intensifiers, diminishers
should always relate to some emotional expression and you should also
explicitly show this relation by drawing an edge attribute.

The attributes of diminishers mainly correspond to that of
intensifiers.  The only difference concerns the \texttt{degree}
attribute which shows how strong an intensifier \emph{increases} but a
diminisher \emph{decreases} the lexical sense of an emo-expression.  A
list of possible attributes for the \texttt{diminisher}s is summarized
in Table \ref{tbl:diminisher}.
\begin{center}
  \begin{table}[hb]
    \caption{Attributes and values of \texttt{diminisher}s.}
    \begin{tabular}{|l|c|p{0.88\clmnwidth}|}\hline

      & \textit{medium (default)} & diminisher moderately decreases
      the polar sense of its respective \texttt{emo-expression},
      e.g. \textit{wenig (few), bisschen (little)} etc.\\\cline{2-3}

      \multirow{-2}{*}{degree} & \textit{strong} & diminisher strongly
      decreases the polar sense of the \texttt{emo-expression},
      e.g. \textit{kaum (hardly)} etc.\\\hline

      emo-expression-ref & \textit{$\longrightarrow$\newline(directed
        edge)} & see Table \ref{tbl:intensifier}\\\hline
    \end{tabular}
    \label{tbl:diminisher}
  \end{table}
\end{center}

\subsection{negation}
\noindent\textbf{Definition.} \emph{Negation}s are elements which turn
the polarity of an \texttt{emo-expression} to the complete opposite.

\noindent\textbf{Example.} In Example \ref{exmp:negation}, for
instance, the negative article \textit{kein} (\textit{not}) makes the
\emph{contextual} polarity of the word \textit{interessant}
(\textit{interesting}) to be negative, even though the prior polarity
of this word is unequivocally positive.
\begin{exmp}
Diese Geschichte war \"uberhaupt nicht \negation{interessant}!

This story was \negation{not} interesting at all!\label{exmp:negation}
\end{exmp}

The role and the meaning of negations are closely related to that of
diminishers.  In order to help you better differentiate between these
elements, we have listed the most obvious differences between the two
classes:
\begin{itemize}
  \item\textit{Semantic differences}.  While diminishers only decrease
    the lexical sense of an emo-expression, a part of this original
    sense still remains active (i.e. \textit{a hardly understandable
      speech} is still understandable); negations, on the contrary,
    fully deny that meaning and turn it to the complete opposite
    (\textit{a not understandable speech} is absolutely
    unintelligible);

  \item\textit{Part-of-speech differences}.  While diminishers are
    usually expressed by adjectives or adverbs, negations are
    typically represented by the negative article \textit{kein (no)},
    the negation particle \textit{nicht (not)}, or adjectives or
    verbs, e.g.  \textit{Es ist sehr zweifelhaft, dass die neue
      Version von Windows besser wird} (\textit{It is very doubtful
      that the new Windows version will be any better})

  %% \item\textit{Syntactic differences}.  While diminishers are usually
  %%   expressed by either adjectives or adverbs; negations, on the contrary,
  %%   are commonly represented by either the negative article \textit{kein
  %%   (no)} or the negation particle \textit{nicht (not)}, or even complete
  %%   clauses (\textit{Er ist nicht einverstanden, dass die neue Version von
  %%   Windows besser ist (He disagrees that the new Windows version is any
  %%   better)}).
\end{itemize}

\noindent\textbf{Attributes.} The only attribute of negations is the
mandatory edge \texttt{emo-expression-ref}.  You should draw this edge
from the \texttt{negation} to the \texttt{emo-expression} being
negated.  Like intensifiers and diminishers, negations should always
relate to at least one \texttt{emo-expression}.
\begin{center}
  \begin{table}
    \caption{Attributes and values of \texttt{negation}s.}
    \begin{tabular}{|l|c|p{0.875\clmnwidth}|}\hline
      emo-expression-ref & \textit{$\longrightarrow$\newline(directed
        edge)} & an edge from \texttt{negation} to the
      \texttt{emo-expression} being negated\\\hline
    \end{tabular}
    \label{tbl:negation}
  \end{table}
\end{center}

\section{Summary}\label{sec:summary}
To summarize, your task in this assignment is to find subjective
evaluative opinions about some subjects or events.  You should
annotate these opinions with the \texttt{sentiment} tags and determine
the polarity and the intensity of the expressed attitudes.  After
that, you should annotate subjects and events which are being
evaluated and mark them as \texttt{target}s.  The holders of the
opinions should be annotated as \texttt{source}s.  Both,
\texttt{source}s and \texttt{target}s can only exist in the presence
of a \texttt{sentiment}.

Another important task is to annotate words and phrases which convey a
polar evaluative meaning by themselves.  We call these words
\texttt{emo-expression}s and you should always annotate them
regardless of whether a sentiment relation is present or not.  If an
\texttt{emo-expression} is intensified, diminished, or negated by
another word or phrase, you should also annotate this modifying
element as \texttt{intensifier}, \texttt{diminisher}, or
\texttt{negation}, respectively.

\section{FAQ}\label{sec:faq}
This section provides some examples of difficult and controversial
annotation cases and gives possible solutions to them.  Please read
them carefully before you start doing the annotation.

\begin{enumerate}
\item\textbf{Q: Should I annotate sentiments in questions?}\label{qstn:interrogative}

  \textbf{A:} It primarily depends on the type of the question.  You
  should typically distinguish two cases:
  \begin{itemize}
    \item If it is a \textit{yes-no-question} or \textit{wh-question}
      which asks whether a particular sentiment statement is true or
      not, then you should not annotate this sentence as
      \texttt{sentiment} because the validity state of this evaluation
      is unknown.  In Example \ref{exm:sentiment-question1}, for
      instance, we do not know whether the asked person actually likes
      or dislikes her new skirt, so we do not annotate sentiment in
      this case;
      \begin{exmp}
        Gef\"allt dir der neue Rock?

        (Do you like the new skirt?)\label{exm:sentiment-question1})
      \end{exmp}
    \item If this is a \textit{wh-question} which asks about the
      reasons or some extra aspects of a polar opinion but does not
      raise the truth of this opinion to question, then you should
      mark a sentiment relation in this case (cf. Example
      \ref{exm:sentiment-question2}).
      \begin{exmp}
        \sentiment{Warum hasst du deine Schwester?}

        (\sentiment{Why do you hate your
          sister?})\label{exm:sentiment-question2})
      \end{exmp}
  \end{itemize}

\item\textbf{Q: Should I annotate sentiments in wishes?}\label{qstn:wish}
%% 1 I would like to get this car -- + <to get this car>;
%% 2 I wish this car would consume less fuel -- + <this car would consume less fuel>;
%% 4 I wish my boyfriend would go with me to the zoo. -- + <my boyfriend would go with me to the zoo>;

  \textbf{A:} Basically, yes.  If someone expresses a wish to get or
  to do something, then this person typically also has a positive
  attitude to the desired object or activity (cf. Examples
  \ref{exmp:desire0} and \ref{exmp:desire1}).

  \begin{exmp}
    \sentiment[polarity=positive]{Habe sooooo Lust auf \target{einen
        Dattel / Bananen Milchshake} .. :-)*\_{}*}

    ( \sentiment[polarity=positive]{Am sooooo up for \target{a date /
        banana milk shake} .. :-) *\_{}*} )\label{exmp:desire0}
  \end{exmp}

  \begin{exmp}
    \sentiment[polarity=positive]{Ich will \target{jetzt nach Hause
        gehen}}

    (\sentiment[polarity=positive]{I now want \target{to get
        home}})\label{exmp:desire1}
  \end{exmp}

  This rule also applies to cases, when the author wants another
  person or object to get a particular property or to do something
  (cf. Example \ref{exmp:desire2}).

  \begin{exmp}
    \sentiment[polarity=positive]{Ich m\"ochte, dass \target{das neue
        Modell weniger Kraftstoff verbrauchen w\"urde}.}

    (\sentiment[polarity=positive]{I want that \target{the new model
        consumed less fuel}.})\label{exmp:desire2}
  \end{exmp}

  If you think that the sentence also expresses an evaluation of the
  object for which a praticular property is wished (in the above case,
  it would be \textit{the new model}), you are also free to annotate
  an additional sentiment with that object as a target (cf. Example
  \ref{exmp:desire3}).

  \begin{exmp}
    \sentiment[polarity=negative]{Ich will, dass \target{das Auto}
      weniger Kraftstoff verbrauchen w\"urde.}

    (\sentiment[polarity=negative]{I want that \target{the new model
        consumes less fuel}.})\label{exmp:desire3}
  \end{exmp}

  It might however not always be the case that the object for which
  some action or property is desired is actually being
  evaluated(cf. \textit{Bruder} (\textit{brother}) in Example
  \ref{exmp:desire3.1}).

  \begin{exmp}
    Ich will, dass mein Bruder mit mir in den Zoo geht.

    (I wish my brother would go with me to the
    zoo.)\label{exmp:desire3.1}
  \end{exmp}

  You should also take special care when dealing with suggestions and
  recommendations.  While recommendations might presuppose an
  appraisal in some cases (cf. Example \ref{exmp:desire4}), they can
  also be completely legitimate objective sentences as well
  (cf. Example \ref{exmp:desire4}).
  \begin{exmp}
    \sentiment[polarity=positive]{\target{Die Regierung
        \emoexpression[polarity=positive]{soll} mehr f\"ur die Umwelt\\
        tun}.}

    (\sentiment[polarity=positive]{\target{The government
        \emoexpression[polarity=positive]{should} do more for the\\
        environment}.})\label{exmp:desire4}
  \end{exmp}

  \begin{exmp}
    An der n\"achsten Kreuzung sollst du nach links abbiegen.

    (You should turn left at the next crossing)\label{exmp:desire4}
  \end{exmp}

  You should not consider as sentiments recommendations in advertising
  slogans, since these not necessarily express the real opinion of the
  authors:
  \begin{exmp}
    Kauft jetzt den neuen Staubsauger von Bosch.

    (Buy the new vacuum cleaner from Bosch now)\label{exmp:desire5}
  \end{exmp}

\item\textbf{Q: Should I annotate sentiments in conditional
  sentences?}\label{qstn:condition}

  If conditional sentence describes some \emph{external} condition,
  under which the author would like or dislike a particular thing or
  event, then you should annotate the whole expression as a
  \texttt{sentiment} and the (dis-)liked thing as a \texttt{target}
  (cf. Examples \ref{exmp:cond1} -- \ref{exmp:cond4}).

  \begin{exmp}
    \sentiment[polarity=positive]{Wenn es regnet, mag ich es immer,
      \target{vor dem Fenster zu sitzen}.}

    (\sentiment[polarity=positive]{I always like \target{to sit in
        front of the window}, if it is rainy.})\label{exmp:cond1}
  \end{exmp}

  \begin{exmp}
    \sentiment[polarity=positive]{Wenn das Wetter besser w\"are,
      w\"urde ich gern \target{joggen gehen}.}

    (\sentiment[polarity=positive]{If the weather was better, I would
      like \target{to go jogging}.})\label{exmp:cond2}
  \end{exmp}

  \begin{exmp}
    \sentiment[polarity=negative]{Selbst wenn es keine Alternative
      g\"abe, w\"urde mir \target{dieses Auto} nicht gefallen.}

    (\sentiment[polarity=negative]{Even if there was no other
      alternative, I would not like \target{this
        car}.})\label{exmp:cond3}
  \end{exmp}

  \begin{exmp}
    \sentiment[polarity=positive]{Wenn ich gesund w\"are, w\"urde ich
      gern \target{mit euch campen}.}

    (\sentiment[polarity=positive]{If I wasn't sick, I would gladly
      \target{go camping with you}.})\label{exmp:cond4}
  \end{exmp}

  If, on the other hand, the condition describes some \emph{internal}
  change of the object or an action which the object inherently should
  do, in order that the author liked or disliked it, then you should
  only annotate that condition as a \texttt{target} and the whole
  expression as a \texttt{sentiment} (cf. Examples \ref{exmp:cond5} --
  \ref{exmp:cond7})\footnote{In Example \ref{exmp:cond6}, we have
    annotated all predicates of the target sentence with one
    \texttt{target} tag because only the joint action of the Pope is a
    sufficient condition for the author to love him.  In other words,
    the author has positive attitude not to the each potential deed of
    the Pope but to all of these deeds as a whole.  This is the only
    possible exception from our rule that we mark each conjoined
    target separately.}.

  \begin{exmp}
    \sentiment[polarity=positive]{Wenn \target{dieses Auto weniger
        Kraftstoff verbrauchen w\"urde}, w\"urde ich es gerne kaufen.}

    (\sentiment[polarity=positive]{If \target{this car consumed less
        fuel}, I would definitely buy it.})\label{exmp:cond5}
  \end{exmp}

  \begin{exmp}
    \sentiment[polarity=positive]{Wenn \target{dieses Auto weniger
        Kraftstoff verbrauchen w\"urde}, w\"urde ich es gerne kaufen.}

    (\sentiment[polarity=positive]{If \target{this car consumed less
        fuel}, I would definitely buy it.})\label{exmp:cond5}
  \end{exmp}

  \begin{exmp}
    \sentiment[polarity=positive]{RT @VanessaLeii: Wenn \target{er
        jetzt raus kommt, die Arme hebt und "Don't cry for me
        Argentina" singt}, mag ich ihn.} \#Papst

    (\sentiment[polarity=positive]{RT @VanessaLeii: If \target{he now
        comes out, raises his hands, and starts singing "Don't cry for
        me Argentina"}, I will love him.} \#Pope)\label{exmp:cond6}
  \end{exmp}

  \begin{exmp}
    \sentiment[polarity=positive]{Wenn das Camping nicht so viel
      Aufwand machen w\"urde, w\"urde ich es gerne machen.}

    (\sentiment[polarity=positive]{If \target{camping wouldn't mean
        so much work}, I would love it.})\label{exmp:cond7}
  \end{exmp}
  Please notice the difference between the Examples \ref{exmp:cond4}
  and \ref{exmp:cond7}.  In both sentences, the judged object is
  \textit{camping}.  But in the former case, the author would love the
  camping if the author's state would change (in this case, the
  author's state is an external object with regard to camping).  In
  the latter case, the author would love camping, if the camping's
  properties would change (which in that case is an internal change of
  the evaluated object).

  We admit, however, that not all cases of conditionals can be covered
  by the above rule of thumb.  So, if you see other evaluations in
  conditional sentences, you can also annotate additional sentiments
  or also annotate completely different than the way we have
  suggested.  We admit that multiple interpretations are possible in
  these cases and do not enforce you to agree with our view of these
  phenomena.

\item\textbf{Q: Should I annotate sentiments in irrealis
  sentences?}\label{qstn:irrealis}

  \textbf{A:} Most irrealis sentences which might express sentiment
  fall in two major categories:
  \begin{inparaenum}[\itshape 1)\upshape]
  \item implicit wishes and
  \item conditions.
  \end{inparaenum}

  Typical examples of implicit wishes are cases like \textit{Es w\"are
    sch\"on, wenn ....} (\textit{It would be nice, if ...}) or
  \textit{Es w\"are schrecklich, wenn ...} (\textit{It would be
    terrible, if ...}).  In both of these cases, you should proceed
  similarly as we did for explicit wishes (cf. Question
  \ref{qstn:wish}) and mark the whole wish expression as a
  \texttt{sentiment}, then annotate the wished property or the event
  as a \texttt{target}, and set the \texttt{polarity} value of that
  \texttt{sentiment} to \texttt{positive}, if the property or event is
  desirable, and to \texttt{negative} otherwise.

  For conditional sentences, please refer to Question
  \ref{qstn:condition} for instructions.

\item\textbf{Q: Should I mark sentiments in insults?}

  \textbf{A:} If you can locate the target, then yes.  For example, in
  sentence \textit{Du bist ein Idiot! (You are an idiot!)}, \textit{Du
    (You)} is the target of a negative evaluation.  On the other hand,
  curses like \textit{Idiot!  (Idiot!)} do not have any explicit
  target and, therefore, should not be annotated as a
  \texttt{sentiment} according to our definition.

\item\textbf{Q: Should I annotate sentiments in defenses?}

  \textbf{A:} Usually not.  If a soldier defends his position or a PhD
  student defends her thesis, it does not necessarily imply that he or
  she likes it.  The same is true in cases when someone defends
  another person in a dispute.

\item\textbf{Q: Should I annotate as sentiments sentences which do not
  have any explicit emotions except for the smiley at the end,
  e.g. ``kam bis heute nichts an :(''?}

\textbf{A:} If the smiley shows author's attitude to the object or
event described in the tweet, then yes, you should annotate such cases
as sentiments.  If, on the contrary, the emoticon only serves
politeness or phatic purposes, then you should not annotate it.  We
should, however, note that many examples are boundary cases and it
will often depend on your interpretation.  As a possible help for
making decisions on such tweets, we suggest you to look at the type of
the emoticon in use, because certain types are more often associated
with judgements.  Negative smileys like ``:(`` or ``\frownie{}'', for
example, usually tend to appear with negative sentiments (cf. example
in question); positive smileys, on the contrary, are much more
ambiguous and typically only express an evaluative judgement if they
show satisfaction or dissatisfaction of the writer with the facts
stated in tweets; the winking smiley (e.g. \textit{;-)}) is by far the
most ambiguous emoticon and it is only rarely involved in a sentiment
relation.

\item\textbf{Q: Should I annotate as sentiments cases like
  \textit{etw. zustimmen} (\textit{to agree with sth.}), \textit{etw.
    unterst\"utzen} (\textit{to support sth.}), \textit{sich f\"ur
    etw. entscheiden} (\textit{to opt for sth.}), and \textit{j-m
    etw. vorwerfen} (\textit{to accuse so. of sth.})?}

  \textbf{A:} These cases are a little bit tricky because subjective
  and objective information are mixed here.  But we would rather say
  ``yes'' unless the context strongly suggests that the expressed
  information is purely objective.  For example, if an attorney
  accused a defendant of a crime in the court, she would basically do
  her job and it would not necessarily be true that she had any
  personal attitude to the defendant.  On the contrary, if I accused
  someone of mean behavior, it would usually be my subjective
  judgement and, therefore, a sentiment.  The same is true for support
  feelings: if a person supports someone's opinion, she is usually
  judging positively about it.  This, however, may not always be the
  case.

\item\textbf{Q: How would you annotate the following cases of comparisons?}
  \begin{itemize}
  \item\textbf{\textit{Seehofer hat die Gr\"unen ausgeschlossen , aber
      die Linke nicht (Seehofer has excluded the Greens, but not the
      Left)}};

    \textbf{A:} Without any further context, I cannot see any
    sentiment relation here.  So, I would probably not annotate
    anything.

  \item\textbf{\textit{Lieber starke Mitte statt linker Rand (Better
      strong middle than left edge)}};

    \textbf{A:} This is a comparison with \textit{starke Mitte (strong
      middle)} as the preferred target, and \textit{linker Rand (left
      edge)} as the dispreferred one;

  \item\textbf{\textit{Die \#spd wird lieber mit den rechten von
      \#cdu, \#csu koalieren als mit der \#linke (The \#spd will
      better form a coalition with rightists from the \#cdu, \#csu
      than with the \#linke)}};

    \textbf{A:} Here again is a comparison with the \textit{\#spd} as
    a source, the \textit{\#cdu}, \textit{\#csu} as the preferred
    targets, and the \textit{\#linke} as the dispreferred target;

  \item\textbf{\textit{Die \#AfD + vereinigt mehr \"okonomische
      Kompetenz als alle etabl. Parteien + Bunde... (The \#AfD +
      combines more economic expertise than all established parties +
      federal...)}};

    \textbf{A:} Again, a comparison with \textit{\#AFD} as the
    preferred target and \textit{established parties} and
    \textit{federal} as the dispreferred ones;

  \item\textbf{\textit{Freiheit statt Bevormundung (Freedom instead of
      paternalism)}};

    \textbf{A:} Comparison, with \textit{freedom} as the preferred
    target and \textit{paternalism} as the dispreferred one;

  \item\textbf{\textit{Fettarme Milch hat mittlerweile mehr Prozent
      wie die FDP (lowfat milk has meanwhile more percents than the
      FDP)}};

    \textbf{A:} I would rather say that this is a sarcasm about the
    FDP.  Because we usually cannot compare a bottle of milk with a
    political party.  If we do so, then usually in order to kid about
    this party;

  \item\textbf{\textit{Was ist der Unterschied zwischen einem Smart
      und der FDP ? Der Smart hat wenigstens 2 Sitze :) (What is the
      difference between a Smart and the FDP? The Smart has at least
      two seats)}};

    \textbf{A:} The same as the previous question -- sarcasm about the
    FDP;
  \end{itemize}

\item\textbf{Q: How should I determine the intensity of a comparison?}

  \textbf{A:} As for the other type of sentiments, you should estimate
  the stylistic expressivity of the sentence.  If a sentence expresses
  a strong emotional evaluation, then you should set the
  \texttt{intensity} attribute of that \texttt{sentiment} to
  \texttt{high}.  If, on the other hand, the sentence rather makes an
  objective statement of facts, then you should mark the
  \texttt{intensity} of such \texttt{sentiment} (if it ever should be
  extracted) to \texttt{medium} or \texttt{weak}.

  For example, in the sentence \textit{this lousy Telekom is waaaaaay
    less reliable than O2}, the strength and the stylistic
  expressiveness of the sentence are much higher than in the sentence
  \textit{Telekom has a less reliable connection than 02}.
  Consequently, we should set the value of the \texttt{intensity}
  attribute in the former case to \texttt{strong} and in the latter
  case to \texttt{medium}.

\item\textbf{Q: It is said that we should disregard negations when
  determining the polarity of an emo-expression.  What about
  sentiments, shall we take into account negations there when
  determining their polarity?}

  \textbf{A:} Yes.  The polarity of an \texttt{emo-expression}
  represents the polar sense of that single lexical item.  The
  polarity of a \texttt{sentiment}, on the contrary, shows the joint
  meaning of the whole phrase, so negations should be taken into
  account if they affect this polarity.

\item\textbf{Q: Is it possible that sources and targets are expressed
  by other means than the ones described in these guidelines?}

  \textbf{A:} Yes. These guidelines are in no way exhaustive, they
  should only give you a better intuition of how sources or targets
  might typically look like.

\item\textbf{Q: What is target in the example \textit{a really nice
    weekend}.  The whole phrase?}

  \textbf{A:} No, it is only the word
  \textit{weekend}. \textit{really} is an intensifier and
  \textit{nice} is an emo-expression.  The whole noun phrase \textit{a
    really nice weekend} forms a sentiment with positive polarity.

\item\textbf{Q: What is target in cases like \textit{Now, we will
    begin with hair coloring.  Coool!!!}.  The whole sentence or only
  ``hair coloring''?}

\textbf{A:} In this case, both the whole sentence and the noun phrase
could be considered as targets, because the verb phrase is in fact
semantically tantamount to the noun phrase.  If one is happy about
\textit{hair coloring}, then she is also happy about the beginning of
tha hair coloring.  The same is true, for example, about a postal
package and the arrival of that package.  Since information in the
verb phrase is usually more elaborate and specific than in the noun
phrase, we would recommend you to annotate the whole verb phrases,
i.e. clauses, in such cases.

\item\textbf{Q: How should I judge if a word is an emo-expression?}

\textbf{A:} Emotional expressions are usually abstract concepts that
are strongly associated with some subjective polar feelings.  These
expressions will serve as our primary anchors for automatically
finding sentiments in texts.  Thus, if you can imagine the a given
abtract word can be used in some context where it would positively or
negatively characterize something, or if you have a strong polar
subjective feeling associated with that word or phrase then you should
mark it as an emo-expression.

For example, we usually associate negative emotions with words like
\textit{Betrug} (\textit{fraud}), \textit{Schuld} (\textit{guilt}), or
\textit{vorwerfen} (\textit{accuse of}).  Moreover, one can say that
someone \textit{begeht einen Betrug} (\textit{commits a fraud}),
\textit{hat die Schuld f\"ur den Eklat} (\textit{is at fault for the
  scandal}), or \textit{wird Unehrlichkeit vorgeworfen} (\textit{is
  accused of dishonesty}), and it would negatively characterize the
target person.  Therefore, we should regard these words as
emo-expressions.  On the other hand, concrete terms like
\textit{Stein} (\textit{stone}) or \textit{Krebs} (\textit{cancer})
should not be considered as emo-expressions since their primary
meaning is concrete and objective.  Expressions which are associated
with emotions but do not have a distinct polarity, like exclamation
marks, for example, should not be marked as emo-expressions either.

\item\textbf{Q: A sentiment is always related to some target, does the
  same apply to emo-expressions?}

\textbf{A:} No.  While it is surely true that a sentiment always
requires a target according to our definition, emotional expressions
can be any words or phrases that have a positive or negative
connotation.  So, for example, words like \textit{Erfolg}
(\textit{success}), \textit{Missgeschick} (\textit{misfortune}),
\textit{ver\"argert} (\textit{upset}) all can be considered as
emo-expressions if you think that there are positive or negative
feelings associated with them.

\item\textbf{Q: What should we do when target and its emo-expression
  are represented by one nominal compound, e.g. Ausl\"anderhass
  (\textit{hatred of foreigners})?}

In such cases, the compound words should be split and you should
annotate their constituents separately.  But, unfortunately, it is
difficult in \texttt{MMAX} to split words, but still possible.
Because you and other annotators will use the same word splitting
files and your markables will also be bound to that specific
splittings, the word-splitting operation should be performed
simultaneously for all annotators.  Therefore, it would be best
practice if you would send an e-mail to the author of these guidelines
with the name of the file, the tweet, and the word which in your
opinion is a compound which requires splitting.  I then would
centrally split these words and ask all other annotators to get a new
version of the files.

\item\textbf{Q: How should I annotate chains of
  intensifiers/diminishers -- each separately or the whole chain with
  one tag?}

  \textbf{A:} Each element should be tagged separately,
  e.g. \textit{Du bist die '\intensifier{aller} \intensifier{aller}'
    Beste! (You are the \intensifier{very} \intensifier{very} best!)}

\begin{exmp}
  @DenisQuadt : Wer meint , die \#piraten h\"atten noch nichts
  umgesetzt, m\"oge das bitte lesen :)

{\footnotesize(@DenisQuadt : Wer meint , die \#piraten h\"atten noch
  nichts umgesetzt, m\"oge das bitte lesen :)}\label{exmp:no-emoexpr}
\end{exmp}

\end{enumerate}

\section{Examples}\label{sec:examples}

In this section, we provide a couple of real-world examples of
complete annotations with explanations of our decisions.  In cases
when we do not specify an attribute for a markable, it is assumed that
this attribute takes its default value.

\begin{exmp}
  \footnotesize WAS HABEN ALLE MIT
  \sentiment[polarity=negative,intensity=strong,sarcasm=false]{IHREN
    \emoexpression[polarity=negative,intensity=strong,sarcasm=false]{VERF*CKTEN}
    \target{GR\"UNEN AUGEN}}

  {\scriptsize(WHAT DO THEY ALL HAVE WITH
    \sentiment[polarity=negative,intensity=strong,sarcasm=false]{THEIR
      \emoexpression[polarity=negative,intensity=strong,sarcasm=false]{F*CKED}
      \target{GREEN EYES}} )}\label{exmp:sarcasm}
\end{exmp}

\textbf{Explanation:} In this case, we have only an evaluative opinion
about the eyes.  It is not clear whether the author has any emotions
towards the people with green eyes, only that she finds these eyes
\textit{verf*ckt} (\textit{f*cked}).  \textit{verf*ckt} is an intense
abusive word with clear negative meaning, so we set the polarity of
this emo-expression and the sentiment it perains to to
\texttt{negative} and set the intensity of both elements to
\texttt{strong}.

\begin{exmp}
  \footnotesize\sentiment[polarity=negative,intensity=medium,sarcasm=true]{Wo
    ist der
    \emoexpression[polarity=positive,intensity=strong,sarcasm=true]{\#Jubel}
    von \target{\#CDU} \target{\#CSU} \& \target{\#FDP} \"uber den Tod
    der Mieterin nach \#Zwangsr\"aumung?}

  {\scriptsize(
    \sentiment[polarity=negative,intensity=medium,sarcasm=true]{Where
      is the
      \emoexpression[polarity=positive,intensity=strong,sarcasm=true]{\#exultation}
      of \target{\#CDU} \target{\#CSU} \& \target{\#FDP} about the
      death of the renter after forced
      \#eviction?})}\label{exmp:sarcasm}
\end{exmp}

\textbf{Explanation:} First of all, we do not mark \textit{Jubel von
  \#CDU,... \"uber den Tod von ...} (\textit{the exultation of the
  \#CDU ... about the death of ...}) as sentiment, because the
existence of this feeling is raised to question (cf. explanation for
Question \ref{qstn:interrogative} in Section \ref{sec:faq}).  On the
other hand, the mere hypothesis about the presence of such glee
feeling from the side of a political party, which should presumably
care about its potential voters, is already a sarcasm.  The
\texttt{emo-expression} which shows us that it is a sarcasm is the
word \textit{\#Jubel} (\textit{\#exultation}).  The primary sense of
this word has a positive polarity; the strength of the expressed
emotion is higher than that of the word \textit{Freude} (\textit{joy})
and we therefore set the value of the \texttt{intensity} attribute to
\texttt{high}.  Moreover, this word by itself is already meant
sarcastically in the given context, so, we accordingly set the value
of its \texttt{sarcasm} attribute to \texttt{true}.  But for all that
the word \textit{Jubel} (\textit{exultation}) has a strong intensity,
the overall way of expressing sentiment is rather subtle and does not
show high exaggeration of the author.  So, the total intensity of
that sentiment is set to \texttt{medium} rather than \texttt{high}.

A slightly more difficult case is represented by the following
example, which we will process step by step:
\begin{exmp}
  \footnotesize RT @JochenFlasbarth : Guter \#Spiegel-Titel , wie
  Welzer , Sloterdijk und andere Promi \#Nichtw\"ahler die Demokratie
  verspielen : Tr\"age , frustriert

  {\scriptsize(RT @JochenFlasbarth : A good \#Spiegel title , how
    Welzer , Sloterdijk, and other celebrity non-voters squander the
    democracy : Sluggish , frustrated)}\label{exmp:nested-2}
\end{exmp}

\textbf{Explanation:} First of all, we should look for words which
have an unambiguous lexical polarity, i.e. the emo-expressions, as
they are our primary cues when detecting a sentiment.  There is the
word \textit{guter} (\textit{good}) with an obvious positive polarity,
and there are words \textit{verspielen} (\textit{to squander}),
\textit{tr\"age} (\textit{sluggish}), \textit{frustriert}
(\textit{frustrated}), whose polarity is unequivocally negative.
Since we have two sets of emo-expressions with contradicting
polarities, it is most likely that we also have two types of
sentiments -- one with a positive evaluation and one with a negative.
The positive evaluation is made about the suggested \#Spiegel title
\textit{wie Welzer , Sloterdijk und andere Promi \#Nichtw\"ahler die
  Demokratie verspielen: Tr\"age , frustriert} (\textit{how Welzer ,
  Sloterdijk, and other celebrity non-voters squander the democracy:
  Sluggish , frustrated}).  The author thinks this title is good and
therefore has a positive attitude to the title as such.  The
annotation for that type of sentiment might look as follows:

\begin{exmp*}
  \footnotesize
  \sentiment[polarity=positive,intensity=medium,sarcasm=false,id=1]{RT
    \source[sentiment\_ref=1]{@JochenFlasbarth} :
    \emoexpression[polarity=positive,intensity=medium,sarcasm=false,
      sentiment\_ref=1]{Guter} \#Spiegel-Titel ,
    \target[sentiment\_ref=1]{wie Welzer , Sloterdijk und andere Promi
      \#Nichtw\"ahler die Demokratie verspielen : Tr\"age ,
      frustriert}}

  {\scriptsize(\sentiment[polarity=positive,intensity=medium,sarcasm=false,id=1]{RT
      \source[sentiment\_ref=1]{@JochenFlasbarth} : A
      \emoexpression[polarity=positive,intensity=medium,sarcasm=false,
        sentiment\_ref=1]{good} \#Spiegel title ,
      \target[sentiment\_ref=1]{how Welzer , Sloterdijk, and other
        celebrity non-voters squander the democracy : Sluggish ,
        frustrated}})}
\end{exmp*}

Another sentiment is instantiated by the set of emo-expressions with
negative polarity.  All the words \textit{verspielen} (\textit{to
  squander}), \textit{tr\"age} (\textit{sluggish}), and
\textit{frustriert} (\textit{frustrated}) seemingly relate to the
celebrity non-voters, including \textit{Welzer} and
\textit{Sloterdijk}, and author's attitude to these people is
obviously negative.  So, we mark this phrase as follows:

\begin{exmp*}
  \scriptsize
  \sentiment[{\tiny polarity=negative,intensity=medium,sarcasm=false,id=2}]{RT
    \source[sentiment\_ref=2]{@JochenFlasbarth} : Guter
    \#Spiegel-Titel , wie \target[sentiment\_ref=2]{Welzer} ,
    \target[sentiment\_ref=2]{Sloterdijk} und
    \target[sentiment\_ref=2]{andere Promi \#Nichtw\"ahler} die
    Demokratie\\
    \emoexpression[polarity=negative,intensity=medium,sarcasm=false,sentiment\_ref=2]{verspielen} :\\
    \emoexpression[polarity=negative,intensity=medium,sarcasm=false,sentiment\_ref=2]{Tr\"age}
    ,\\ \emoexpression[polarity=negative,intensity=medium,sarcasm=false,sentiment\_ref=2]{frustriert}\\
  }

  {\scriptsize(\sentiment[polarity=negative,intensity=medium,sarcasm=false,id=2]{RT
      \source[sentiment\_ref=2]{@JochenFlasbarth} : A good \#Spiegel
      title , how \target[sentiment\_ref=2]{Welzer} ,
      \target[sentiment\_ref=2]{Sloterdijk}, and
      \target[sentiment\_ref=2]{other celebrity non-voters}\\
      \emoexpression[polarity=negative,intensity=medium,sarcasm=false,sentiment\_ref=2]{squander}
      the democracy :\\
      \emoexpression[polarity=negative,intensity=medium,sarcasm=false,
        sentiment\_ref=2]{Sluggishly} ,\\
      \emoexpression[polarity=negative,intensity=medium,sarcasm=false,
        sentiment\_ref=2]{frustrated}\\
    })}
\end{exmp*}

In both cases, \textit{@JochenFlasbarth} is the original author of
cited opinion, so we mark it as \texttt{source}.  But since we have
two sentiment relations, we mark this word as \texttt{source} twice
and draw an edge (in our example denoted by attribute
\texttt{sentiment\_ref}) to the respective \texttt{sentiment} markable
in each case.
\end{document}

Example: sollte Haslauer mal Bundespr\"asident werden , bin ich der
erste der die Initiative " Kein Bundespr\"asidentenbild in
Klassenzimmern " startet.

If an irrealis sentence expresses a wish about a property of a
particular target, then you should mark the whole wish as a
\texttt{sentiment}, annotate the wished property as \texttt{target},
and set the \texttt{polarity} value of that \texttt{sentiment} to
\texttt{positive}, if the target property of the object is desired,
and to \texttt{negative} otherwise (cf. Example
\ref{exmp:irrealis-1}).
\begin{exmp}
  \sentiment[polarity=positive]{Ich \emoexpression{w\"unsche},
    \target{die Reaktion der deutschen Fu\ss{}ballfans w\"are\\
      gez\"ahmter}.}

  (\sentiment[polarity=positive]{I \emoexpression{wish}, \target{the
      reaction of German soccer fans would be more\\tamed}.})
  \label{exmp:irrealis-1}
\end{exmp}
Additionally, you should also mark the object which is currently
missing the desired property as \texttt{target} of another
\texttt{sentiment} markable and set the \texttt{polarity} value of
that \texttt{sentiment} to \texttt{negative} .
\begin{exmp}
  \sentiment[polarity=negative]{Ich \emoexpression{w\"unsche},
    \target{die Reaktion der deutschen Fu\ss{}ballfans}
    w\"are\\ gez\"ahmter.}

  (\sentiment[polarity=negative]{I \emoexpression{wish}, \target{the
      reaction of German soccer fans} would be more\\tamed.})
  \label{exmp:irrealis-2}
\end{exmp}
Such annotation would reflect the fact, that the author is
unsatisfied with the current state of the target object, but she
would be glad if that state would change in a certain way.

\end{document}
