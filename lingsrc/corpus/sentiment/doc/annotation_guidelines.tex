\documentclass[11pt,a4paper]{article}

%%%%%%%%%%%%%%%%%%%%%%%%%%%%%%%%%%%%%%%%%%%%%%%%%%%%%%%%%%%%%%%%%%
%% Libraries
\usepackage[usenames,dvipsnames]{xcolor}
\usepackage[driver=pdftex,vmargin=2cm,hmargin=2cm]{geometry}
\usepackage{amsthm}
\usepackage{array}
\usepackage{color}
\usepackage{comment}
\usepackage{changepage}
\usepackage{etoolbox}
\usepackage{eurosym}
\usepackage{framed}
\usepackage{gb4e}
\usepackage{hyperref}
\usepackage{multicol}
\usepackage{multirow}
\usepackage{nameref}
\usepackage{paralist}
\usepackage{url}
\usepackage{wasysym}   % smiley symbols
\usepackage{xparse}

\hypersetup{colorlinks, citecolor=Violet, linkcolor=Red,urlcolor=Blue}

%%%%%%%%%%%%%%%%%%%%%%%%%%%%%%%%%%%%%%%%%%%%%%%%%%%%%%%%%%%%%%%%%%
%% Lengths
\newlength{\cwidth}
\setlength{\cwidth}{0.18\textwidth}

\newlength{\clmnwidth}
\setlength{\clmnwidth}{0.65\textwidth}

\newlength{\exmpindent}
\setlength{\exmpindent}{70pt}

%%%%%%%%%%%%%%%%%%%%%%%%%%%%%%%%%%%%%%%%%%%%%%%%%%%%%%%%%%%%%%%%%%
%% Commands
\definecolor{gray80}{RGB}{204,204,204}
\definecolor{dodgerblue4}{RGB}{16,78,139}
\definecolor{orange3}{RGB}{205,133,0}
\definecolor{DarkSlateBlue}{RGB}{72,61,139}

\newcommand{\corpusDir}{\texttt{sentiment}}
\newcommand{\authorAddress}{\texttt{sidarenk@uni-potsdam.de}}
%% Examples Environment
\NewDocumentCommand{\Colorbox}{O{\dimexpr\linewidth-2\fboxsep} m m}{%
  \colorbox{#2}{\makebox[#1][l]{#3}}}
\newcommand{\code}[1]{

\smallskip\noindent\Colorbox{gray80}{\parbox{\textwidth}{\texttt{#1}}}\smallskip

\noindent}
%% Examples Environment
\newtheoremstyle{mytheoremstyle} % name
    {\topsep}                    % Space above
    {\topsep}                    % Space below
    {\parindent=\exmpindent\hangindent=\exmpindent} % Body font
    {}                           % Indent amount
    {\scshape}                   % Theorem head font
    {.}                          % Punctuation after theorem head
    {.5em}                       % Space after theorem head
    {}  % Theorem head spec (can be left empty, meaning `normal')

\theoremstyle{mytheoremstyle}
\newtheorem{exmp}{Example}[section]
%% Annotattion Tags
\newcommand{\mtag}[2]{{\upshape[\emph{#2}\upshape]$_{\textrm{\bfseries\emph{\tiny
        #1}}}$}}

\newcommand{\sentiment}[2][]{\mtag{sen\-ti\-ment\ifstrempty{#1}{}{:#1}}{#2}}
\newcommand{\source}[2][]{\mtag{source\ifstrempty{#1}{}{:#1}}{#2}}
\newcommand{\target}[2][]{\mtag{tar\-get\ifstrempty{#1}{}{:#1}}{#2}}
\newcommand{\emoexpression}[2][]{\mtag{emo-\-ex\-pression\ifstrempty{#1}{}{:#1}}{#2}}
\newcommand{\intensifier}[2][]{\mtag{in\-ten\-si\-fier\ifstrempty{#1}{}{:#1}}{#2}}
\newcommand{\diminisher}[2][]{\mtag{di\-mi\-ni\-sher\ifstrempty{#1}{}{:#1}}{#2}}
\newcommand{\negation}[2][]{\mtag{ne\-ga\-tion\ifstrempty{#1}{}{:#1}}{#2}}

%%%%%%%%%%%%%%%%%%%%%%%%%%%%%%%%%%%%%%%%%%%%%%%%%%%%%%%%%%%%%%%%%%
%% Title
\author{Wladimir Sidorenko}
\date{\today}
\title{Guidelines for the Annotation of the Sentiment Corpus}

%%%%%%%%%%%%%%%%%%%%%%%%%%%%%%%%%%%%%%%%%%%%%%%%%%%%%%%%%%%%%%%%%%
%% Main
\begin{document}
\maketitle{}
\section{Overview}
\subsection{Introduction}

In this assignment, your task is to annotate sentiments in a corpus of
Twitter messages.  We define \emph{sentiments} as polar (either
positive or negative) evaluative opinions about some persons,
subjects, or events.  In this assignment, you have to mark both --
text spans denoting the opinions (\emph{sentiments}) and text spans
denoting the subjects and events being evaluated (\emph{sentiment
  targets}).  Additionally, you should also annotate the holders of
the opinions (\emph{sentiment sources}) and lexical elements which
might significantly influence the polarity and intensity of
sentiments.  The latter elements include:
\begin{itemize}
  \item \emph{emotional expressions}, which are words or phrases that
    unequivocally possess some evaluative lexical meaning by
    themselves (these are typically words like \emph{hassen}
    (\emph{hate}), \emph{bewundern} (\emph{admire}), \emph{sch\"on}
    (\emph{nice}) etc.);
  \item \emph{negations}, which are words or expressions that might
    completely flip the polarity of an emotional expression or a
    sentiment to the opposite (e.g. \emph{nicht} gut (\emph{not}
    good), \emph{kein} Talent (\emph{not} a talent) etc.);
  \item \emph{intensifiers} and \emph{diminishers} (or
    \emph{downtoners}) which are words and expressions that might,
    respectively, increase or decrease the evaluative sense of
    emotional expressions.  Examples of intensifiers include words
    like \emph{sehr} (\emph{very}), \emph{besonders}
    (\emph{especially}), \emph{insbesondere} (\emph{particularly})
    etc.  Typical examples of diminishers include words like \emph{ein
      wenig} (\emph{a little}), \emph{ein bisschen} (\emph{a bit}),
    \emph{gewisserma\ss{}en} (\emph{to a certain degree}) etc.
\end{itemize}

After marking these elements, you should also specify the values of
their respective attributes.  A complete list of all elements to be
marked along with the description of their possible attributes is
given in Section \ref{sec:markables}.  In Section \ref{sec:faq}, we
also provide some annotation examples and answer some questions which
caused special difficulties during the previous annotation runs.

\subsection{Terminology and Format}
\textbf{Terminology.} Throughout this document, we will use the term
\emph{markable} to denote an annotated span of text.  The term
\emph{markable type} (or \emph{markable tag}) will refer to the tag
assigned to a markable.  Additional attributes associated with the
annotated text spans will be called \emph{markable attributes}.

We will not make a distinction between the terms \emph{opinions} and
\emph{sentiments} and will use both words interchangeably unless the
latter refers to the type of a markable.

\smallskip\noindent\textbf{Format.} In these guidelines, we rely
on the following conventions regarding the formatting of text.

We specify shell commands in gray boxes in \texttt{typesetting} font
as shown in the example below: \code{echo 'Hello world!'}  The
\texttt{typesetting} font is also used for literal mentions of
markable types, markable attributes, file names, directory paths, and
executable commands.

Examples of words and phrases are given in \textit{italics} and their
respective English translation is provided in parentheses.

Examples of sentence annotations are shown in regular font. Text
enclosed in markables is \emph{emphasized} and surrounded by square
brackets (e.g. [\emph{markable text}]).  The type of the markables is
given as a subscript after the closing right bracket; optional
markable attributes are specified after the type, separated from it by
a colon, e.g.:
\begin{exmp}
  \sentiment[polarity=positive]{\target{Der neue Papst} gilt als
    \emoexpression{bescheidener}, \emoexpression{zur\"uckgenommener}
    Typ}.

{\footnotesize(\sentiment[polarity=positive]{\target{The new Pope} is
    believed to be a \emoexpression{sober}, \emoexpression{modest}
    man}.)}\label{exmp:1}
\end{exmp}

\subsection{Annotation Tool}

For annotating text, you should use \texttt{MMAX2} -- a freely
available annotation tool -- which you can download under the
following link:

{\setlength{\parindent}{0pt}\small\url{http://sourceforge.net/projects/mmax2/files/mmax2/mmax2_1.13.003/MMAX2_1.13.003b.zip/download}}

After downloading this file, please unpack the received archive,
change to the newly created directory \texttt{1.13.003/MMAX2} in your
terminal shell and execute the following commands: \code{chmod u+x
  ./mmax2.sh\\ nohup ./mmax2.sh \&} An \texttt{MMAX2} window should
then appear on your screen.  If you have never used \texttt{MMAX2}
before, please read the document \texttt{mmax2quickstart.pdf}, which
you can find in the subdirectory \texttt{MMAX2/Docs} of the downloaded
archive.

\subsection{Corpus Files}

You should also have received a copy of corpus files either as a
tar-gzipped archive or via the version control system \texttt{Git}.

If you got a \texttt{.tgz} archive of the corpus, then unpack it using
the command: \code{tar -xzf archive-name.tgz} After that, a directory
called \corpusDir{} should appear in your current folder.

If you received access to the \texttt{Git} repository of the project,
you should clone the project to your local computer with the command:
\code{git clone repository-address path-to-trg-dir} Then, you should
find folders called \texttt{annotator-*} in the directory
\texttt{path-to-trg-dir/lingsrc/ corpus/}\corpusDir{}.

In order to load an annotation file into your \texttt{MMAX2} program,
you should change to the \texttt{MMAX2} window and click on the menu
\texttt{File -> Load}.  In the displayed popup window, select the path
to the newly created \corpusDir{}\texttt{/anno\-ta\-tor-x}
folder\footnote{Please, make sure that the path to the \corpusDir{}
  folder does not contain any white spaces.  Otherwise, \texttt{MMAX2}
  might fail to load the project.}, where \texttt{x} is the annotator
number that was assigned to you, and click on one of the
\texttt{*.mmax} files found there.  The chosen project should then be
loaded into your \texttt{MMAX2} editor.

If you have any difficulties with launching \texttt{MMAX2} or loading
the project files into your editor, please contact the author of these
guidelines via e-mail (\authorAddress{}).

%%%%%%%%%%%%%%%%%%%%%%%%%%%%%%%%%%%%%%%%%%%%%%%%%%%%%%%%%%%%%%%%%%

\section{Tags and Attributes}\label{sec:markables}
In the following, we provide a short list of all markables and their
possible attributes that will be used in this annotation task:

\begin{multicols}{2}
  \begin{enumerate}
  \item \texttt{sentiment}-markable with the attributes:
    \begin{enumerate}
    \item \texttt{polarity},
    \item \texttt{intensity},
    \item \texttt{sarcasm};
    \end{enumerate}
  \item \texttt{target}-markable with the attributes:
    \begin{enumerate}
    \item \texttt{preferred},
    \item \texttt{anaph-ref},
    \item \texttt{sentiment-ref};
    \end{enumerate}
  \item \texttt{source}-markable with the attributes:
    \begin{enumerate}
    \item \texttt{anaph-ref},
    \item \texttt{sentiment-ref};
    \end{enumerate}
  \item \texttt{emo-expression}-markable with the attributes:
    \begin{enumerate}
    \item \texttt{polarity},
    \item \texttt{intensity},
    \item \texttt{sarcasm},
    \item \texttt{sentiment-ref};
    \end{enumerate}
  \item \texttt{intensifier}-markable with the attributes:
    \begin{enumerate}
    \item \texttt{degree},
    \item \texttt{emo-expression-ref};
    \end{enumerate}
  \item \texttt{diminisher}-markable with the attributes:
    \begin{enumerate}
    \item \texttt{degree},
    \item \texttt{emo-expression-ref};
    \end{enumerate}
  \item and, finally, the \texttt{negation}-markable with the
    attribute:
    \begin{enumerate}
    \item \texttt{emo-expression-ref}.
    \end{enumerate}
  \end{enumerate}
\end{multicols}
\noindent{}A more detailed description o these markables and the
values of their respective attributes is given below.

\subsection{sentiment}\label{sec:sentiment}
With the \texttt{sentiment} tags, you should mark polar evaluative
opinions about people, subjects or events.  Typical examples of
sentiment statements are sentences like the following one:
\begin{exmp}
  \sentiment{Ich mag den neuen James Bond Film nicht}.

  (\sentiment{I don't like the new James Bond
    movie}.)\label{ex:sentiment}
\end{exmp}

According to our defintion, a sentiment must always fulfill the
following conditions:
\begin{itemize}
\item it should be \textbf{subjective}, i.e. you should not mark as
  \texttt{sentiment}s mere statements of facts like, for example,
  \textit{Beim Angriff wurden 14 Glasscheiben besch\"adigt}
  (\textit{14 glass plates were broken during the attack}); there
  should always be a personal opionion of the author or some person or
  institution being cited;

\item it should be \textbf{polar}, i.e. a sentiment should always
  reflect either positive or negative attitude to its respective
  target.  Cases like \textit{Ich glaube, er wird heute fr\"uher
    kommen} (\textit{I think he will be earlier today}) should not be
  marked as sentiments because the attitutde of the author is unclear;

\item it should be \textbf{evaluative}, which means that a sentiment
  should always refer to an explicit target.  You should not mark as
  sentiments cases like \textit{Ich bin heute so gl\"ucklich}
  (\textit{I am so happy today}), because these statements do not
  evaluate anything particular.
\end{itemize}

You should not mark as sentiments polar opinions whose validity status
is unknown.  This includes cases like \textit{Ich wei\ss{} nicht, ob
  ich meinen Bruder mag} (\textit{I don't know if I like my brother})
which should not be enclosed in the \texttt{sentiment} tags because
neither we nor the author actually know if the author likes or
dislikes her brother.  Exceptions from these are cases like
\textit{Ich zweifle, dass er ein guter Mensch ist} (\textit{I doubt
  that he is a good man}) or \textit{Ich glaube nicht, dass er diesen
  Preis verdient hat} (\textit{I don't think that he has deserved this
  award}) which act as negations and express a negative attitude.
Special care should be taken when dealing with irrealis and
interrogative clauses (cf. Questions \ref{qstn:interrogative} and
\ref{qstn:irrealis} in the \nameref{sec:faq} Section of this
document).

We also consider comparisons to be a special type of sentiments.  But
in contrast to regular evaluative opinions, comparisons typically
express a relative polarity, i.e. an object is regarded to be either
better or worse than another compared object, but we usually do not
know if the author actually likes or dislikes any of them.  So,
comparisons should also be marked with the \texttt{sentiment} tags,
but you should set the value of their \texttt{polarity} attribute to
the special type \texttt{comparison}.

When determining the boundaries of a \texttt{sentiment} span, you
should always enclose in the \texttt{sentiment} tags \emph{the minimal
  complete syntactic or discourse-level units that express an
  evaluative opinion}.  A sentiment unit should simultaneously
encompass the subject being evaluated (the target) and the actual
expression which is showing sentiment's attitude .  This is typically
done by a single noun or verb phrase (syntactic units) or by means of
one or multiple sentences (discourse-level units).  Minimal, in this
case, means that if both target and evaluative expression are
expressed within a single noun phrase, you should not include any
parent elements of this phrase in the markable.  Complete means that
if a sentiment is expressed by a single verb phrase, for example; then
you should enclose this verb phrase in the \texttt{sentiment} tags but
you should not mark any of its syntactic parents unless they are also
immediately related to the evaluative opinion being expressed.

Below we give a couple of annotation examples for sentiments together
with explanations.
\begin{exmp}
  Auf dem Tisch lag \sentiment{ein langweiliges Buch}.

  (There was \sentiment{a boring book} on the table.)\label{exmp:book}
\end{exmp}

In Example \ref{exmp:book}, you can see a personal, negative statement
made about a book.  This opinion is subjective (since it expresses a
personal judgement of the author), polar (because this statement is
judging negatively), and evaluative (as the judgement is made about
the book).  So, this opinion should be definitely marked as a
\texttt{sentiment}.  Next, we look at the syntactic and discourse
constituents by which this sentiment is expressed.  Both target
(\textit{book}) and evaluative expression (\textit{boring}) occur in a
single noun phrase (\textit{a boring book}), so we mark this noun
phrase completely including the indefinite article but we do not mark
anything else except for this phrase since other information are
unrelated to the judgement.

\begin{exmp}
  Wir akzeptieren das, weil \sentiment{wir alle ein bisschen in
    Petterson verliebt sind}.

  (We accept this because \sentiment{we all are a little bit in love
    with Petterson}.)\label{exmp:petterson}
\end{exmp}

In Example \ref{exmp:petterson}, the evaluative statement is made
about \textit{Petterson}.  The author says that they all \textit{in
  ihn verliebt sind} (\textit{are in love with him}) which is her
subjective positive opinion.  The target of this sentiment is
\textit{Petterson}, the evaluative expression is \textit{verliebt sein
  in} (\textit{to be in love with}) which both are included in the
verb phrase with the head verb \textit{be}.  So, we mark the whole
verb phrase including the subject \textit{wir} (\textit{we}).

%% \begin{itemize}
%% \item single noun phrases, possibly with their prepositional
%%   attributes, e.g. \textit{Auf dem Tisch lag \sentiment{ ein
%%       langweiliges Buch} (There was \sentiment{a boring book} on the
%%     table)};
%% \item clauses, e.g. \textit{\sentiment{Ich hasse B\"ucher ohne
%%   Inhaltsangabe}. (\sentiment{I hate books without table of
%%   contents})};
%% \item multiple sentences in cases when these sentences jointly form a
%%   sentiment relation, e.g.\\\textit{\sentiment{Sie denken, reden,
%%       riechen, lieben, schmecken, f*cken Plastik. Sie haben\\das so
%%       gelernt in der Plastik-Werbewelt}.\\ (\sentiment{They think and
%%       speak about, smell at, love, taste, f*ck plastic.  They
%%       have\\ learned it so in the advertising world of plastic}.)}.
%% \end{itemize}

After you have marked the \texttt{sentiment} span, you should next set
the values of its attributes.  Acceptable attributes, their meaning
and values are given in Table \ref{tbl:sentiment}.
\begin{center}
  \begin{table}[ht]
    \caption{Attributes and values of \texttt{sentiment}s.}
    \begin{tabular}{|l|c|p{\clmnwidth}|}\hline
      Attribute & Value & Value's Meaning\\\hline
      %%%%%%%%%%%%%%%%%%%

      & \textit{positive} & sentiment expresses positive attitude about
      its respective target, e.g. \textit{Es war ein fantastischer Abend
        (It was a fantastic evening)};\\\cline{2-3}

      & \textit{negative\newline(default)} & sentiment expresses
      negative attitude about its respective target, e.g. \textit{Seine
        Schwester ist einfach unausstehlich (His sister is simply
        obnoxious)}\\\cline{2-3}

      \multirow{-3}{*}{polarity} & \textit{comparison} & sentiment
      expresses a comparison of two objects with preference given to one
      of them, e.g. \textit{Mir gef\"allt das rote Kleid mehr als das
        blaue (I like the red dress more than the blue one)}\\\hline

      %%%%%%%%%%%%%%%%%%%

      & \textit{weak} & sentiment expresses a weak evaluative opinion,
      e.g. \textit{Der Auftritt war mehr oder weniger gut (The
        appearance was more or less good)}\\\cline{2-3}

      & \textit{medium\newline(default)} & sentiment has a middle
      emotional expressivity, e.g. \textit{Mir hat das neue Album gut
        gefallen (I enjoyed the new album)}\\\cline{2-3}

      \multirow{-3}{*}{intensity} & \textit{strong} & this sentiment
      expresses a very emotional polar statement, e.g. \textit{Dieses
        Festival war einfach umwerfend!!! (This festival was simply
        terrific!!!)}\\\hline
      %%%%%%%%%%%%%%%%%%%

      \multirow{2}{*}{sarcasm} & \textit{true} & this polar attitude is
      derisive, i.e. its actual polarity is the opposite of its apparent
      form. (An apparent praise, for example, could be meant as a rebuke
      and vice versa. The actual sense, however, can only be inferred on
      the basis of world knowledge or reasoning.)  An example of a
      sarcastic sentiment is the following passage: \textit{Mein
        J\"ungerer ist in der Pr\"ufung durchgefallen.  Klasse! (My
        youngest has failed his exam.  Well done!)}  In this case, you
      should set the polarity attribute of the sentiment to
      \texttt{negative} and the value of the \texttt{sarcasm} attribute
      to \textit{true}.\\\cline{2-3}

      & \textit{false\newline(default)} & no sarcasm is present -- the
      polar attitude has its literal meaning; this is the default.
      setting\\\hline
    \end{tabular}
    \label{tbl:sentiment}
  \end{table}
\end{center}

%%%%%%%%%%%%%%%%%%%%%%%%%%%%%%%%%%%%%%%%%%%%%%%%%%%%%%%%%%%%%%%%%%%%%%%%%%%%%%%%%%%%%%%%%%
\subsection{target}
You should mark with \texttt{target} tags objects or events which are
being evaluated by sentiment expressions.  An example of a sentiment
target is given in the following sentence:
\begin{exmp}
Mein Bruder ist nicht begeistert von \target{dem neuen Call of Duty}.

(My brother is not impressed by \target{the new Call of Duty}.)
\end{exmp}
\noindent{}Because sentiments are required to be evaluative, there
MUST always be at least one target for each sentiment relation.

Similar to sentiments, you should always mark with the \texttt{target}
tags minimal complete syntactic or discourse-level units which denote
the evaluated objects or events.  These are usually expressed by noun
phrases (e.g. \textit{Mir wird's schlecht, wenn ich \target{diese
    Werbung} im Fernsehen sehe} (\textit{I feel sick when I see this
  \target{ad} on TV})) or complete verb phrases (clauses)
(e.g. \textit{Ich hasse es wenn \target{Voldemort mein Shampoo
    benutzt.}} (\textit{I hate it when \target{Voldemort is using my
    shampoo}})).

If sentiment has multiple targets which are conjoined by commas or
conjunctions, you should mark each of these targets separately
(cf. Example \ref{exmp:trg-conj}).
\begin{exmp}
  Meiner Mutter haben immer \target{Nelken} und \target{Dahlien}
  gefallen.

  (My mother has always liked \target{carnations} and
  \target{dahlias}.)\label{exmp:trg-conj}
\end{exmp}

In cases, when a sentiment is expressed by a comparison, you should
mark each of the compared objects with separate tags.  Additionally,
for the object which is being dispreferred in the comparison, you
should also set the \texttt{preferred} attribute to the value
\texttt{false} (cf. Example \ref{exmp:trg-comp}).
\begin{exmp}
  Ich mag \target{Domino-Eis} mehr als
  \target[preferred=false]{Magnum}.

  (I like the \target{Domino ice cream} more than
  \target[preferred=false]{Magnum}.)\label{exmp:trg-comp}
\end{exmp}

Further possible attributes of the \texttt{target} tags along with
their allowed are shown in Table \ref{tbl:target}.
\begin{center}
  \begin{table}[h]
    \caption{Attributes and values of \texttt{target}s.}
    \begin{tabular}{|l|c|p{0.94\clmnwidth}|}\hline
      Attribute & Value & Value's Meaning\\\hline

      & \textit{true} (default) & in comparisons, this value means that
      the respective target is being considered better than another
      compared object, e.g. \textit{\emph{Die neue Frisur} passt ihr
        garantiert besser als die alte (\emph{The new hairstyle} suits
        her definitely better than the old one)};\\\cline{2-3}

      \multirow{-2}{*}{preferred} & \textit{false} & in comparisons,
      this value marks the target element which is being considered
      worse than its counterpart, e.g. \textit{Die zweite Saison von
        Breaking Bad war viel spannender als \emph{die dritte} (The
        second season of Breaking Bad was much more exciting than
        \emph{the third one})};\\\hline

      sentiment-ref & \textit{$\longrightarrow$\newline(directed
        edge)} & directed edge pointing from \texttt{target} to its
      respective \texttt{sentiment}.  You need to draw this edge in
      two cases:
      \begin{itemize}
      \item when the \texttt{target} is located at intersection of two
        different \texttt{sentiment}s (in this case, you should draw
        an edge from \texttt{target} to \texttt{sentiment}, which this
        \texttt{target} actually belongs to),

      \item when the target of an opinion is expressed outside the
        \texttt{sentiment} span;
      \end{itemize}\\\hline

      anaph-ref & \textit{$\longrightarrow$\newline(directed edge)} &
      directed edge pointing from \texttt{target} expressed by a
      pronoun or pronominal adverb to its respective non-pronomial
      antecedent (in order to draw this edge, you also need to mark
      the antecedent as \texttt{target})\\\hline
    \end{tabular}
    \label{tbl:target}
  \end{table}
\end{center}

%%%%%%%%%%%%%%%%%%%%%%%%%%%%%%%%%%%%%%%%%%%%%%%%%%%%%%%%%%%%%%%%%%
%% Source
\subsection{source}
With the \texttt{source} tag, you should mark the immediate author(s)
or holder(s) of an evaluative opinion.  These are typically writers of
messages or persons or institutions whose opinion is being cited.  If
sentiment's source of is not explicitly mentioned in a message, we
then assume that it is the author of the tweet. You need not mark
anything with the \texttt{source} tag in this case.  An example of an
explicit sentiment \texttt{source} is the pronoun \textit{Sie}
(\textit{she}) in Example \ref{exmp:source}.
\begin{exmp}
  \source{Sie} mag die neue Farbe nicht

  (\source{She} doesn't like the new color)\label{exmp:source}
\end{exmp}

In case of citations, you should only mark the person or institution
as a \texttt{source} whose original opinion is being cited.  Do not
mark the citing persons (cf. Example \ref{exmp:source-citation}).
\begin{exmp}
  Laut Staatsanwalt soll die \source{Angeklagte} sich missbilligend \"uber
  ihren Vorgesetzten ge\"au\ss{}ert haben.

  (According to the attorney, the \source{defendant} had made
  disapproving remarks about her boss.)\label{exmp:source-citation}
\end{exmp}

For determining the boundaries of the \texttt{source}, you should
proceed in similar fashion as we did for \texttt{sentiment}s and
\texttt{target}s and mark only complete minimal syntactic units.
Sources are most commonly expressed by noun phrases.  And, similar to
\texttt{target}s, if the source of a sentiment is expressed by
multiple coordinatively joined noun phrases, you should mark each of
them separately (cf. Example \ref{exmp:source2}).

\begin{exmp}
  \source{Ihr} und \source{ihrer Mutter} gef\"allt die neue Farbe
  nicht.\\ (Neither \source{she} and \source{her mother} likes the new
  color)\label{exmp:source2}
\end{exmp}

The attributes of the \texttt{source} tag are listed in Table
\ref{tbl:source}.  They are fully identical to the corresponding
attributes of \texttt{target}s.
\begin{center}
  \begin{table}[h]
    \caption{Attributes and values of \texttt{source}s.}
    \begin{tabular}{|l|c|p{0.935\clmnwidth}|}\hline
      Attribute & Value & Value's Meaning\\\hline

      sentiment-ref & \textit{$\longrightarrow$\newline(directed
        edge)} & cf. Table \ref{tbl:target}\\\hline

      anaph-ref & \textit{$\longrightarrow$\newline(directed edge)} &
      cf. Table \ref{tbl:target}\\\hline
    \end{tabular}\label{tbl:source}
  \end{table}
\end{center}

%%%%%%%%%%%%%%%%%%%%%%%%%%%%%%%%%%%%%%%%%%%%%%%%%%%%%%%%%%%%%%%%%%
%% Emo-expression
\subsection{emo-expression}
With the \texttt{emo-expression} tags, you should annotate words and
phrases which have a polar evaluative lexical meaning by themselves.
An example of emotional expression is the word \textit{ekelhaft}
(\textit{disgusting}) in Example \ref{exmp:emo-expr1}.
\begin{exmp}
  Beim Aufr\"aumen des Zimmers haben wir einen
  \emoexpression{ekelhaften} Teller mit verschimmeltem Essen unter dem
  Bett gefunden.

  (When we cleaned the room, we found a \emoexpression{disgusting}
  plate with moldy food under the bed.)\label{exmp:emo-expr1}
\end{exmp}

\noindent{}In contrast to \texttt{source}s and \texttt{target}s which
should only be marked in the presence of a \texttt{sentiment}, you
should always annotate emotional expressions in text no matter if an
evaluative \texttt{sentiment} is present or not in their vicinity.

Since many words and idioms are ambiguous and can express many
different lexical meanings, it is often the case that only some of
these lexical senses are evaluative and subjective.  In such cases,
you should only mark as \texttt{emo-expression}s words whose actual
meaning in the given context is polar.  If these words express an
objective sense in other messages, you should not annotate them
(cf. Example \ref{exmp:emo-expression-jewel}).
\begin{exmp}
  Dieser Wein ist ein echtes \emoexpression{Juwel} in meiner
  Kollektion.

  (This wine is a real \emoexpression{jewel} in my collection.)

  Koh-i-Noor ist das teuerste Juwel heutzutage.

  (Koh-i-Noor is the most expensive jewel nowadays.)\label{exmp:emo-expression-jewel}
\end{exmp}
\noindent{}In this example, the meaning of the word \textit{Juwel}
(\textit{jewel}) is metaphoric and subjective in the first sentence,
but literal and objective in the second.  So you should only annotate
this word in the former case but not in the latter.

\texttt{emo-expression}s are typically expressed by:
\begin{itemize}
  \item nouns, e.g. \textit{Held (hero)}, \textit{Ideal (ideal)},
    \textit{Betr\"uger (fraudster)} etc.;

  \item adjectives or adverbs, e.g. \textit{sch\"on (nice)},
    \textit{zuverl\"assig (reliably)}, \textit{hinterh\"altig
      (devious)}, \textit{heimt\"uckisch (insidiously)} etc.;

  \item verbs, e.g. \textit{lieben (to love)}, \textit{bewundern (to
    admire)}, \textit{hassen (to hate)} etc.;

  \item idioms, e.g. \textit{auf die Nerven gehen (to get on one's
    nerves)} etc.;

  \item smileys, e.g. :), :-(, \smiley{}, \frownie{} etc.;
\end{itemize}
In case of idiomatic expressions, you should always annotate complete
idioms.  For verbs, if they only take on an evaluative sense when used
together with some prepositions (e.g. \textit{to go for sth.} in the
sense of \textit{to like}), you should annotate both the verb and the
preposition as a single markable (please refer to the \texttt{MMAX}
manual to see how to annotate discontinuous spans).

When determining the value of the \texttt{polarity} attribute of an
\texttt{emo-expression}, you should disregard any possible contextual
modifiers like intensifiers or negations, however, and set the value
of this attribute to the lexical (or also called \emph{prior})
polarity of the \texttt{emo-expression} word (the one which it
expresses without negations) (cf. Example
\ref{exmp:emo-expression-polarity}).
\begin{exmp}
Es war keine \emoexpression[polarity=positive]{gute} Idee.

(It was not a \emoexpression[polarity=positive]{good} idea.)\label{exmp:emo-expression-polarity}
\end{exmp}

Also, when you are determining the value of the \texttt{polarity}
attribute, you should analyze this polarity from the perspective of
the subject or event which is being evaluated by the
\texttt{emo-expression} (if this subject is present in the context).
It means that if we are speaking about \textit{a nice book}, the
polarity expressed towards the book is positive, conversely \textit{a
  boring book} expresses a negative polarity.  Note, however, that in
cases like \textit{Ich vermisse meine Freundin} (\textit{I miss my
  girlfriend}), the polarity is still positive because the author
evidently has a positive attitude to his girlfriend despite (or better
to say proven by) the fact that he is suffering from her absence.

Further attributes of \texttt{emo-expression}s include
\texttt{intensity}, \texttt{sarcasm}, and \texttt{sentiment-ref}.  The
meaning and the values of these attributes are listed in Table
\ref{tbl:emo-expression}.
\begin{center}
  \begin{table}[ht]
    \caption{Attributes and values of \texttt{emo-expression}s.}
    \begin{tabular}{|l|c|p{0.935\clmnwidth}|}\hline
      Attribute & Value & Value's Meaning\\\hline
      %%%%%%%%%%%%%%%%%%%

      & \textit{positive} & emotional expression has a positive
      evaluative meaning, e.g. \textit{gut (good), verhimmeln (to
        ensky), Prachtkerl (corker)} etc.\\\cline{2-3}

      \multirow{-2}{*}{polarity} & \textit{negative\newline(default)}
      & emotional expression has a negative evaluative meaning towards
      its target, e.g. \textit{versauen (to botch up), rotzig
        (snotty), Dreckskerl (scum)} etc.\\\hline

      %%%%%%%%%%%%%%%%%%%

      & \textit{weak} & emo-expression has a weak evaluative sense,
      e.g. \textit{solala (so-so), nullachtf\"unfzehn (vanilla),
        durchschnittlich (mediocre)} etc.\\\cline{2-3}

      & \textit{medium\newline(default)} & emo-expression has middle
      stylistic expressivity, e.g. \textit{gut (good), schlecht (bad),
        robust (tough)} etc.\\\cline{2-3}

      \multirow{-3}{*}{intensity} & \textit{strong} & emo-expression
      expresses a very strong positive or negative evaluation,
      e.g. \textit{allerbeste (bettermost), zum Kotzen (to make one
        puke), Kacke (shit)} etc.\\\hline

      %%%%%%%%%%%%%%%%%%%%

      \multirow{2}{*}{sarcasm} & \textit{true} & emo-expression is
      derisive, i.e. its actual polarity is the opposite of its
      apparent form. (This means that an apparent praise which appears
      in text is in fact meant as a rebuke and vice versa. The actual
      sense, however, can only be inferred on the basis of world
      knowledge or reasoning.)\\\cline{2-3}

      & \textit{false\newline(default)} & no sarcasm is present -- the
      polar attitude has its literal meaning; this is the default
      setting\\\hline

      %%%%%%%%%%%%%%%%%%%

      sentiment-ref & \textit{$\longrightarrow$\newline(directed
        edge)} & arrow pointing to the \texttt{sentiment} which this
      \texttt{emo-expression} belongs to.  You should only draw this
      edge if an \texttt{emo-expression} is located in the overlapping
      of two \texttt{sentiment} spans or outside of the
      \texttt{sentiment} span which it belongs to\\\hline
    \end{tabular}
    \label{tbl:emo-expression}
  \end{table}
\end{center}

%%%%%%%%%%%%%%%%%%%%%%%%%%%%%%%%%%%%%%%%%%%%%%%%%%%%%%%%%%%%%%%%%%
%% Intensifier
\subsection{intensifier}
Intensifiers are elements which increase the expressivity and the
polar sense of an emotional expression.  An example of intensifier is
the word \textit{sehr (very)} in Example \ref{exmp:intensifier}.
\begin{exmp}
  Wir suchen eine \intensifier{sehr} zuverl\"assige Polin als
  Haushaltshilfe.

  (We are looking for a \intensifier{very} reliable Polish woman as
  domestic help.)\label{exmp:intensifier}
\end{exmp}
Intensifiers are usually expressed by adverbs or adjectives like
\textit{sehr (very), sicherlich (certainly)} etc.  But other ways of
expressions are still possible (cf. Example
\ref{exmp:intensifier-comp}).
\begin{exmp}
  Dieser Junge ist stark \intensifier{wie ein Pferd}.

  (This boy is trong \intensifier{as a
    horse}.)\label{exmp:intensifier-comp}
\end{exmp}

An \texttt{intensifier} should always relate to some
\texttt{emo-expression} and you should also always explicitly show
that relation by drawing an edge attribute from \texttt{intensifier}
to its respective \texttt{emo-expression} markable.

Further possible attributes of \texttt{intensifier}s are shown in
Table \ref{tbl:intensifier}.
\begin{center}
  \begin{table}[htb]
    \caption{Attributes and values of \texttt{intensifier}s.}
    \begin{tabular}{|l|c|p{0.875\clmnwidth}|}\hline

      & \textit{medium (default)} & the intensifier moderately
      increases the polar sense of the emotional expression,
      e.g. \textit{ziemlich (quite), recht (fairly)} etc.\\\cline{2-3}

      \multirow{-2}{*}{degree} & \textit{strong} & the intensifier
      strongly increases the polar sense and stylistic markedness of the
      emotional expression, e.g. \textit{sehr (very), super (super),
        stark (strongly)} etc.\\\hline

      %%%%%%

      emo-expression-ref & \textit{$\longrightarrow$\newline(directed
        edge)} & a directed edge pointing from the intensifier to the
      \texttt{emo-expression} whose meaning is being intensified\\\hline
    \end{tabular}
    \label{tbl:intensifier}
  \end{table}
\end{center}

\subsection{diminisher}
With the \texttt{diminisher} tag, you should annotate words and
phrases that decrease the polar lexical sense of an
\texttt{emo-expression}.  In Example \ref{exmp:diminisher}, the
diminisher is expressed by the adverb \textit{weniger} (\textit{less}.
\begin{exmp}
  \diminisher{Weniger} erfolgreiche Unternehmen verzichten auf externe
  Berater.\label{exmp:diminisher}

  The \diminisher{less} successful companies do not use external
  consultants.
\end{exmp}
Just like intensifiers, diminishers should always relate to some
emotional expression and you should also explicitly show this relation
by drawing an edge.

The attributes of the diminishers mainly correspond to that of
intensifiers.  The only difference concerns the \textit{degree}
attribute which shows how strong an \texttt{intensifier}
\emph{increases} but a \texttt{diminisher} \emph{decreases} the
lexical sense of an \texttt{emo-expression}.  A list of possible
attributes for the \texttt{diminisher}s is provided in Table
\ref{tbl:diminisher}.
\begin{center}
  \begin{table}[hb]
    \caption{Attributes and values of \texttt{diminisher}s.}
    \begin{tabular}{|l|c|p{0.88\clmnwidth}|}\hline

      & \textit{medium (default)} & diminisher moderately decreases
      the polar sense of its respective \texttt{emo-expression},
      e.g. \textit{wenig (few), bisschen (little)} etc.\\\cline{2-3}

      \multirow{-2}{*}{degree} & \textit{strong} & diminisher strongly
      decreases the polar sense of the \texttt{emo-expression},
      e.g. \textit{kaum (hardly)} etc.\\\hline

      emo-expression-ref & \textit{$\longrightarrow$\newline(directed
        edge)} & see Table \ref{tbl:intensifier}\\\hline
    \end{tabular}
    \label{tbl:diminisher}
  \end{table}
\end{center}

\subsection{negation}
With the \texttt{negation} tag, you should mark elements which turn
the polarity of an \texttt{emo-expression} to its complete opposite.
In Example \ref{exmp:negation}, for instance, the negative article
\textit{kein} (\textit{not}) makes the \emph{contextual} polarity of
the word \textit{interessant} (\textit{interesting}) to be negative
despite that the prior polarity of this word is unequivocally
positive.
\begin{exmp}
Diese Geschichte war \"uberhaupt nicht \negation{interessant}!

This story was \negation{not} interesting at all!\label{exmp:negation}
\end{exmp}
Like intensifiers and diminishers, negations should always relate to
some \texttt{emo-expression} and you should also show this relation
with an edge.

The role and the meaning of negations are closely related to that of
diminishers.  In order to help you better differentiate between these
elements, we have listed the most obvious differences between the two
classes:
\begin{itemize}
  \item\textit{Semantic differences}.  While the diminishers only
    decrease the lexical sense of an emo-expression, a part of this
    original sense still remains active (i.e. \textit{a hardly
      understandable speech} is still understandable); negations, on
    the contrary, fully deny that meaning and turn it to the complete
    opposite (\textit{a not understandable speech} is absolutely
    unintelligible);

  \item\textit{Part-of-speech differences}.  While diminishers are
    usually expressed by adjectives or adverbs, negations are
    typically represented by the negative article \textit{kein (no)},
    the negation particle \textit{nicht (not)}, or verbs or
    adjectives, e.g.  \textit{Es ist sehr zweifelhaft, dass die neue
      Version von Windows besser wird} (\textit{It is very doubtful
      that the new Windows version will be any better})

  %% \item\textit{Syntactic differences}.  While diminishers are usually
  %%   expressed by either adjectives or adverbs; negations, on the contrary,
  %%   are commonly represented by either the negative article \textit{kein
  %%   (no)} or the negation particle \textit{nicht (not)}, or even complete
  %%   clauses (\textit{Er ist nicht einverstanden, dass die neue Version von
  %%   Windows besser ist (He disagrees that the new Windows version is any
  %%   better)}).
\end{itemize}

The only attribute of negations is the mandatory edge
\texttt{emo-expression-ref}.  We describe the meaning of this
attribute in Table \ref{tbl:negation}.
\begin{center}
  \begin{table}
    \caption{Attributes and values of \texttt{negation}s.}
    \begin{tabular}{|l|c|p{0.875\clmnwidth}|}\hline
      emo-expression-ref & \textit{$\longrightarrow$\newline(directed
        edge)} & an edge from \texttt{negation} to the
      \texttt{emo-expression} being negated\\\hline
    \end{tabular}
    \label{tbl:negation}
  \end{table}
\end{center}

\subsection{Summary}\label{subsec:summary}
To summarize, your task in this assignment is to find subjective
evaluative opinions about some subjects or events.  You should
annotate these opinions with the \texttt{sentiment} tags and determine
the polarity and the intensity of the expressed attitudes.  After
that, you should annotate subjects and events which are being
evaluated and mark them as \texttt{target}s.  The holders of the
opinions should be annotated as \texttt{source}s.  Both,
\texttt{source}s and \texttt{target}s can only exist in the presence
of a \texttt{sentiment}.

Another important task is to annotate words and phrases which convey a
polar evaluative meaning by themselves.  We call these words
\texttt{emo-expression}s and you should always annotate them
regardless of whether a sentiment relation is present or not.  If an
\texttt{emo-expression} is intensified, diminished, or negated by
another word or phrase, you should also annotate this modifying
element as \texttt{intensifier}, \texttt{diminisher}, or
\texttt{negation}, respectively.

\section{FAQ}\label{sec:faq}
This section provides some examples of difficult and controversial
annotation cases and gives possible solutions to them.  Please read it
carefully before starting the annotation.

\begin{enumerate}
\item\textbf{Q: Should I annotate sentiments in questions?}\label{qstn:interrogative}

  \textbf{A:} It primarily depends on the type of the question.  You
  should typically distinguish two cases:
  \begin{itemize}
    \item If it is a \textit{yes-no-question}, which asks whether a
      particular sentiment statement is true or not, you should not
      annotate any sentiment relation in this sentence.  In example,
      in the sentence \textit{Gef\"allt dir der neue Rock?  (Do you
        like the new skirt?)}, we do not know whether the asked person
      likes or dislikes the new skirt, so we do not annotate a
      sentiment in this case;
      \ref{exm:sentiment-question2}).
      \begin{exmp}
        \sentiment{Gef\"allt dir der neue Rock}?

        (\sentiment{Do you like the new
          skirt}?)\label{exm:sentiment-question1})
      \end{exmp}
    \item If it is a \textit{wh-question}, which asks about the
      reasons or some other aspects of a polar opinion, but does not
      raise the truth of this opinion to question, the sentiment
      relation should be marked (cf. Example
      \ref{exm:sentiment-question2}).
      \begin{exmp}
        \sentiment{Warum hasst du deine Schwester}?

        (\sentiment{Why do you hate your
          sister}?)\label{exm:sentiment-question2})
      \end{exmp}
  \end{itemize}

\item\textbf{Q: How should I mark sentiments in irrealis sentences?}\label{qstn:irrealis}

  If an irrealis sentence expresses a wish about a particular target,
  then you should mark the whole wish expression as a
  \texttt{sentiment}, annotate the wished property as a
  \texttt{target}, and set the value of the sentiment to
  \texttt{positive}, if the target is desired, and to negative
  otherwise.
  \begin{exmp}
    \sentiment{Es w\"are \negation{nicht} \emoexpression{schlecht},
      wenn dieses Kino \target{mehr Sitzpl\"atze} h\"atte.}

    (\sentiment{It would \negation{not} be \emoexpression{bad} if, if
      this cinema had \target{more seats}.})
  \label{exmp:irrealis-1}
\end{exmp}

  If an irrealis sentence expresses an opinion about a target, which
  could possibly be true in the future, but which currently isn't,
  then you should annotate this sentence as a \texttt{sentiment}, but
  set it \texttt{polarity} value to the opposite of what the target
  object could possibly be (cf. Example \ref{exmp:irrealis-2}).

  \begin{exmp}
    \sentiment[polarity=negative]{\target{Die Kartoffelchips}
      k\"onnten \emoexpression[polarity=positive]{besser} sein}.

    (\sentiment[polarity=negative]{\target{The potato chips} could be
      \emoexpression[polarity=positive]{better}}.)\label{exmp:irrealis-2}
  \end{exmp}

\item\textbf{Q: How would you annotate the following cases of comparisons?}
  \begin{itemize}
  \item\textbf{\textit{Seehofer hat die Gr\"unen ausgeschlossen , aber
      die Linke nicht (Seehofer has excluded the Greens, but not the
      Left)}};

    \textbf{A:} Without any further context, I cannot see any
    sentiment relation here.  So, I would probably not annotate
    anything.

  \item\textbf{\textit{Lieber starke Mitte statt linker Rand (Better
      strong middle than left edge)}};

    \textbf{A:} This is a comparison with \textit{starke Mitte (strong
      middle)} as the preferred target, and \textit{linker Rand (left
      edge)} as the dispreferred one;

  \item\textbf{\textit{Die \#spd wird lieber mit den rechten von
      \#cdu, \#csu koalieren als mit der \#linke (The \#spd will
      better form a coalition with rightists from the \#cdu, \#csu
      than with the \#linke)}};

    \textbf{A:} Here again is a comparison with the \textit{\#spd} as
    a source, the \textit{\#cdu}, \textit{\#csu} as the preferred
    targets, and the \textit{\#linke} as the dispreferred target;

  \item\textbf{\textit{Die \#AfD + vereinigt mehr \"okonomische
      Kompetenz als alle etabl. Parteien + Bunde... (The \#AfD +
      combines more economic expertise than all established parties +
      federal...)}};

    \textbf{A:} Again, a comparison with \textit{\#AFD} as the
    preferred target and \textit{established parties} and
    \textit{federal} as the dispreferred ones;

  \item\textbf{\textit{Freiheit statt Bevormundung (Freedom instead of
      paternalism)}};

    \textbf{A:} Comparison, with \textit{freedom} as the preferred
    target and \textit{paternalism} as the dispreferred one;

  \item\textbf{\textit{Fettarme Milch hat mittlerweile mehr Prozent
      wie die FDP (lowfat milk has meanwhile more percents than the
      FDP)}};

    \textbf{A:} I would rather say that this is a sarcasm about the
    FDP.  Because we usually cannot compare a bottle of milk with a
    political party.  If we do so, then usually in order to kid about
    this party;

  \item\textbf{\textit{Was ist der Unterschied zwischen einem Smart
      und der FDP ? Der Smart hat wenigstens 2 Sitze :) (What is the
      difference between a Smart and the FDP? The Smart has at least
      two seats)}};

    \textbf{A:} The same as the previous question -- sarcasm about the
    FDP;
  \end{itemize}

\item\textbf{Q: How should I determine the intensity of a comparison?}

  \textbf{A:} As for the other type of sentiments, you should estimate
  the stylistic expressivity of the sentence.  If a sentence makes a
  strong emotional statement, its intensity should be high.  Vice
  versa, non-exclamatory sentences which rather resemble objective
  statements of facts should be marked with the \texttt{medium} or
  \texttt{weak} polarity.  For example, in the sentence \textit{this
    lousy Telekom waaaaaay less reliable than O2}, the strength and
  the stylistic expressiveness are much higher than in the sentence
  \textit{Telekom has a less reliable connection than 02}.
  Consequently, you should set the value of the \texttt{intensity}
  attribute to \texttt{strong} in the former case and to
  \texttt{medium} in the latter.

\item\textbf{Q: Should I annotate as sentiments cases like
  \textit{etw. zustimmen} (\textit{to agree with sth.}), \textit{etw.
    unterst\"utzen} (\textit{to support sth.}), \textit{sich f\"ur
    etw. entscheiden} (\textit{to opt for sth.}), and \textit{j-m
    etw. vorwerfen} (\textit{to accuse so. of sth.})?}

  \textbf{A:} These cases are a bit tricky because subjective and
  objective information are mixed here.  But we would rather say
  ``yes'' unless the context strongly suggests that the information
  expressed is purely objective.  For example, if an attorney accuses
  a defendant of a crime in the court, she is basically doing her job
  and it is not necessarily true that she has any personal attitude to
  the defendant.  On the contrary, if I accuse someone of mean tricks,
  it is usually my subjective opinion.  The same applies for the
  support: if a person supports someone's opinion, she is usually
  judging positively about it.  This, however, may not always be the
  case.

\item\textbf{Q: Should I annotate sentiments in recommendations?}

  \textbf{A:} Usually yes.  If this recommendation represents a
  personal opinion and is not a spam, and you can unambiguously
  determine the target of this opinion, then you should annotate
  sentiment here.

\item\textbf{Q: Should I annotate sentiments in defenses?}

  \textbf{A:} Usually not.  If a soldier defends his position or a PhD
  student defends her thesis, it does not necessarily imply that he or
  she likes it.  The same is true in cases when someone defenses
  another person in a dispute.

\item\textbf{Q: Should I mark sentiments in insults?}

  \textbf{A:} If you can locate the target, then yes.  For example, in
  the sentence \textit{Du bist ein Idiot! (You are an idiot!)},
  \textit{Du (You)} is the target of a negative sentiment.  On the
  contrary, in cases like \textit{Idiot!  (Idiot!)}, no target is
  expressed and therefore, according to our definition, no sentiment
  can exist.

\item\textbf{Q: Is it possible that sources and targets are expressed
  by other means than the ones described in these guidelines?}

  \textbf{A:} Yes. These guidelines are in no way exhaustive, they
  should just provide you with a better intuition of how sources or
  targets might typically look like.

\item\textbf{Q: It is said that we should disregard negations when
  determining the polarity of an emo-expression.  What about
  sentiments, shall we take into account negations there when
  determining their polarity?}

  \textbf{A:} Yes.  The polarity of an \texttt{emo-expression}
  represents the polar sense of that single lexical item.  The
  polarity of a \texttt{sentiment}, on the contrary, shows the joint
  meaning of the whole phrase, so negations shoukd be taken into
  account if they affect the polarity.

\item\textbf{Q: How should I annotate chains of
  intensifiers/diminishers -- each separately or the whole chain with
  one tag?}

  \textbf{A:} Each element should be tagged separately,
  e.g. \textit{Du bist die '\intensifier{aller} \intensifier{aller}'
    Beste! (You are the \intensifier{very} \intensifier{very} best!)}

\item\textbf{Q: What is the target in, for example, \textit{a really nice weekend}.
	The whole phrase?}

  \textbf{A:} No it is not the whole phrase but only the
  \textit{weekend}. \textit{really} is an intensifier and
  \textit{nice} is an emo-expression.

%% \item F: Dem Tweet aus politics1 ``Wo ist der \#Jubel von \#CDU \#CSU
%%   \& \#FDP \"uber den Tod der Mieterin nach \#Zwangsr\"aumung?'' habe
%%   ich als sarkastisch und mit target=''Tod der Mieterin nach
%%   \#Zwangsr\"aumung'' annotiert. Wei\ss aber nicht, ob das so richtig
%%   ist.

%% A: Mhm, aus meiner Sicht wird hier eher \"uber die CDU/CSU und FDP
%% gespottet, man w\"urde also annehmen, dass diese Parteien sich \"uber
%% den Tod der Mieterin freuen w\"urden, was nat\"urlich dem Image einer
%% Partei eher schaden w\"urde.  Ich w\"urde demzufolge den ganzen Satz
%% als negatives Sentiment mit den Parteien als Targets und dem Attribut
%% sarcasm gesetzt auf true annotieren.
\end{enumerate}
\end{document}

%%%%%%%%%%%%%%%%%%%%%%%%%%%%%%%%%%%%%%%%%%%%%%%%%%%%%%%%%%%%%%%%%%%%%%%%%%%%%
%%%%%%%%%%%%%%%%%%%%%%%%%%%%%%%%%%%%%%%%%%%%%%%%%%%%%%%%%%%%%%%%%%%%%%%%%%%%%

\subsection{sentiment}
\begin{itemize}
\item I have difficulties determining the overall sentiment of a
  tweet.  Is there a common modus operandi?

Please keep in mind that sentiments should always be annotated
target-oriented. Look at this tweet: \textit{Ich hab Maria ja schon so
  vermisst und so , ne ? (I kind of did miss Maria, right?)}. Here you
might think the tweet has a negative sentiment, because the author
missed Maria which is not a good thing. But we are always only
concerned about the sentiment relation between the source and the
target! And here it is a positive sentiment, because the author missed
Maria because he likes her.

\item How to annotate sarcasm?

If the apparent form of a sentiment is positive (and therefore should
be annotated as a positive sentiment) but you can detect sarcasm
please annotate the polarity as negative and additionally set
\texttt{sarcasm} to \texttt{true} in the attribute window. This
applies to the sentiment tag as well as to emo-expressions.

\item Should subjunctives be annotated?

Yes. Have a look on the following tweet: \textit{W\"are toll wenn das
  der n\"achste " Call of Duty " - Teil sein w\"urde , liebe
  @GameStar. (Dear @gamestar, it would be great if this would be the
  sequel of "Call of Duty".)}. Although this is a subjunctive
sentiment and is not "reality" right now you should annotate this,
too.

\item Should multiple sentiment layers be annotated?

Yes. For example in the tweet \textit{72 j\"ahriger Leser regt sich
  bei mir \"uber YouTube und Gema auf :) (72 year old reader troubles
  over YouTube and Gema at me :) )} you can find two layers of
sentiments. The first layer is the fact that the reader troubles over
YouTube and Gema. The second layer is the fact that the author of the
tweet is amused about the first fact. So here you have to annotate two
different sentiments. The first is negative and has "YouTube" and
"Gema" as target (with "troubles over" as emo-expression). The second
is positive and has "72 jähriger Leser" as target and "mir" as
source. Do not forget to use the \texttt{sentiment\_ref} attribute to
annotate which sentiment refers to which target.
\end{itemize}

\subsection{source}
\begin{itemize}
\item I can find the same source two times in one tweet. What to do?

Look at this tweet: \texttt{Ich hab \"Ubrigens ne 1 in Englisch.  Find
  ich gut ! (By the way, I got an A- in English. I Like it!)}  Here
the "Ich" in the first sentence needs not to be annotated. It is
sufficient to annotate just the "Ich" in the second sentence. Since
there is no anaphora present you will not need to mark an anaphoric
antecedent as well. \\ Here you should annotate both sentences as one
sentiment with "ne 1- in Englisch" as target, the "I" in the second
sentence as source and "gut" as emo-expression.
\end{itemize}

\subsection{target}

\subsection{emo-expression}
\begin{itemize}
\item What about English emo-expressions?

Although we annotate German tweets, it is not unusual to find English
emo-expressions. For example: \textit{DSDS ist fancy ! (DSDS is fancy
  !)}.  These expressions should be annotated too, if they are common
in German.

\item Should intejections be marked as emo-expressions?

Sollen Interjektionen (oh, aha, achso, OMG etc.) auch als
emo-expressions annotiert werden? Ich habe das n\"amlich nicht
gemacht, weil sie in so gut wie allen F\"allen nicht als wirklich
positiv bzw. negativ zu identifizieren waren. Wie
z.B. \textit{Lichgestaaaaaaaaaaalt !!!!!! in deren Schatten ich mich
  drehe uuuooooooh oooooh uuuuuooohhhh oooooohhhhh xD} oder
\textit{ooh du hasts gut , h\"att auch gern urlaub ...}. Ich finde
Interjektionen auch grenzwertig, da sie unter gewissen Umst\"anden
auch als intensifier dienen k\"onnen, z.B. oh wie gut, oh ja
etc. . Falls sie doch annotiert werden sollen, bitte Bescheid sagen
und/oder eine entsprechende Anmerkung in den Guidelines machen. W\"are
gut explizit zu schreiben ob sie annotiert werden sollen oder nicht.

\item Do we need sarcasm attribute for emo-expressions?
\end{itemize}

\subsection{intensifier}
\subsection{diminisher}
\subsection{negation}
\end{document}

%%%%%%%%%%%%%%%%%%%%%%%%%%%%%%%%%%%%%%%%%%%%%%%%%%%%%%%%%%%%%%%%%%

\section{Examples}
Please DO NOT mark as sentiments following cases:
\begin{itemize}
\item Statements describing some emotional states for which no
  target can be derived or found (e.g. \textit{Ich f\"uhle mich so
    traurig :(} - \textit{I am feeling so blue :(});
\item Statements describing some objective facts, even if possible
  consequences of these facts can be assumed to have negative
  influence on the author (e.g. \textit{Wenn ein Floh einen Menschen
    bei\ss{}t und ihn mit erbrochenem Blut infiziert, werden die
    Pestbakterien ins Gewebe \"ubertragen.} - \textit{When a flea
    bites a human and contaminates the wound with regurgitated
    blood, the plague carrying bacteria are passed into the
    tissue.});
\item Interrogative clauses in case they ask whether some polar
  opinion is true or not. (e.g. \textit{Findet die Mehrheit
    der Bev\"olkerung CDU toll?} - \textit{Does the majority
    of population consider CDU great?});
\end{itemize}

\subsection{negation}
\texttt{negation}s are lexical or syntactic elements that reverse the
primary polar meaning of emo-expressions to the opposite so that the
overall polarity of the whole sentiment is different to the polarity
of emo-expressions belonging to it. A typical example of negation is
\textit{nicht} in sentences like \textit{Ein guter Schritt war diese
  Entscheidung \textbf{nicht}.} (\textit{This decision was
  \textbf{not} a good move.}).

\vspace{0.5cm}
You SHOULD only mark negative elements that have an impact on the
sentiment polarity. Negating elements having no such impact
shold not be marked. Negation elements are usually represented by:
\begin{itemize}
\item Negation particle \textit{nicht} (\textit{not}),
  e.g. \textit{Klug ist dieser Hund sicherlich \textbf{nicht}}
  (\textit{This dog is certainly \textbf{not} clever});

\item Negative article \textit{kein} e.g. \textit{Er war
  \textbf{kein} Vorbild f\"ur seine Kinder} (\textit{He was
  \textbf{not} a good role model for his children});

\item Indefinite pronouns like \textit{niemand}, \textit{keiner}
  etc., e.g. \textit{\textbf{Niemand} hielt ihn f\"ur einen
    ehrlichen Menschen.} (\textit{\textbf{Nobody} considered him an
    honest man});

\item Any lexical or idiomatic unit in case they turn the sentiment
  polarity to the opposite, e.g. \textit{Ich \textbf{zweifle}, dass
    das neue iPhone ein besseres Display hat.} (\textit{I
    \textbf{doubt} the new iPhone has a better display}).
\end{itemize}

DO NOT mark elements as negations that have no effect on the sentiment
polarity. For example, in the sentence \textit{Ich mag Leute, die nicht
  nur an sich selbst denken.} (\textit{I like people who not only care
  about themselves.}) \textit{nicht} (\textit{not}) should \underline{not} be
marked as negation since it does not change the positive polarity
expressed by \textit{m\"ogen} (\textit{like}).
\vspace{0.5cm}

Negations only have one possible attribute, namely
\textit{sentiment-ref} that is a directed edge pointing from a negation
to the sentiment it belongs to. You should only draw this edge in cases multiple sentiment 
relations overlap and it is not obvious to which of these sentiments the negation belongs to,
i.e. in cases the negation is used for both sentiment spans.

%%%%%%%%%%%%%%%%%%%%%%%%%%%%%%%%%%%%%%%%%%%%%%%%%%%%%%%%%%%%%%%%%%%%%%%%%%%%%%%%%%

\subsection{intensifier}
\texttt{intensifier}s are elements that increase the polar
meaning of emotional expressions. Intensifiers are usually expressed
by adjectives or adverbs like \textit{sehr} (\textit{very}),
\textit{ziemlich} (\textit{rather}) and the like. In intensifier-chains 
\textit{(e.g. sehr, sehr gut)} please annotate each intensifier separately.
As mentioned above you should annotate every single emo-expression you can 
find regardless of a corresponding sentiment span. 
Please do not forget to annotate the corresponding intensifier if there is one. \newline

Intensifiers have the following attributes and values: \newline

\begin{tabular}{|l|c|p{\clmnwidth}|}\hline
  \multirow{2}{*}{degree} & \textit{1 (default)} & intensifier 
  slightly increases polarity of emotional
  expression, e.g. \textit{ziemlich},
  \textit{recht} etc.\\\cline{2-3}

  & \textit{2} & intensifier strongly increases polarity  of
  emotional expression, e.g. \textit{sehr}, \textit{super},
  \textit{stark} etc.\\\hline

  %%%%%%

  sentiment-ref & \textit{$->$\newline(directed edge)} & Directed
  edge pointing from the intensifier to the sentiment it belongs
  to. By analogy to negations, you should only draw this edge in cases
  multiple sentiment relations overlap and it is not obvious
  to which of these sentiments the intensifier belongs to,
  i.e. in cases when the intensifier is used for both
  sentiment spans.\\\hline
\end{tabular}

%%%%%%%%%%%%%%%%%%%%%%%%%%%%%%%%%%%%%%%%%%%%%%%%%%%%%%%%%%%%%%%%%%

\subsection{diminisher}
\texttt{diminisher}s are the counterpart to \texttt{intensifier}s.
These are elements that decrease the polar meaning of emotional
expressions.  Like intensifiers diminishers are usually expressed by
adjectives or adverbs. Typical examples of such adverbs are
\textit{wenig}, \textit{kaum}, \textit{ein bisschen} etc..

Diminishers have the same attributes as intensifiers. The only
difference are the values for the attribute. Instead of positive
values you have negative ones.

Here is a table of diminisher's attributes:

\begin{tabular}{|l|c|p{\clmnwidth}|}\hline

  \multirow{2}{*}{degree} & \textit{weak (default)} & diminisher
  slightly decreases polarity of emotional expression,
  e.g. \textit{wenig}, \textit{bisschen} etc.\\\cline{2-3}

  & \textit{strong} & diminisher strongly decreases polarity of
  emotional expression, e.g. \textit{kaum} etc.\\\hline


  sentiment-ref & \textit{$->$\newline(directed edge)} & Directed
  edge pointing from the diminisher to the sentiment it belongs
  to. You should only draw this edge in cases
  multiple sentiment relations overlap and it is not obvious
  to which of these sentiments the diminisher belongs to,
  i.e. in cases when the diminisher is used for both
  sentiment spans.\\\hline

\end{tabular}

%%%%%%%%%%%%%%%%%%%%%%%%%%%%%%%%%%%%%%%%%%%%%%%%%%%%%%%%%%%%%%%%%%%%%%%%%%%%

\section{MMAX Techniques}
In this section the most important techniques for annotating our corpus in MMAX will be described.

\subsection{Turn on Auto-Save and Auto-apply}
After setting up MMAX load a .mmax file as described in the beginning
of this document. You will be asked to validate the
annotations. Please always do that.  Before you start the annotation
it is recommended to turn on "Auto-Save" and "Auto-apply". Go to
\texttt{File -> Auto-Save -> Every} and select a time value.  Then
change to the window for the attributes (the window in the upper left
corner of your screen) and go to \texttt{Settings} and check the
\texttt{Auto-apply} box.  Your choice will be confirmed by a red
"Auto-apply is ON" at the bottom of the window. \newline

\subsection{Annotate Markables and Attributes}
The window in the middle of your screen now should show the tweets. To annotate a markable choose a word, phrase, clause or sentence by clicking the left mouse-button and mark the respective span while holding the mouse-button. After releasing the left mouse-button a menu will pop up showing the list of markables. Left-click on the corresponding markable and the span will turn to the respective color of the markable. Then change to the attribute window in the upper left corner, click on the tab for the markable you just annotated and set the value(s) for the attributes. (In case you did not turn on "Auto-apply" press "Apply".)

\subsection{Deleting Markables}
Just do a \underline{right-click} on a colored markable and left-click \texttt{Delete this markable}.

\subsection{Annoating Discontinuous Markables}
Sometimes you will find discontinuous emo-expressions. The most prominent example are German particle verbs (e.g. weh tun). To annotate them as one single element first annotate one element (it will turn to the respective markable color) as usual, left-click to highlight it and then mark the other element. A menu will pop up saying "Add to this markable". Left-click on that.

\subsection{Annotating anaphref}
Imaginge you have a sentence like: \textit{Peter ist schlau und gut aussehen tut er auch noch. (Peter is clever and he looks good, too.)}. Here you should annotate \textit{er} as anaphref to \textit{Peter}. To do so first annotate \textit{Peter} and \textit{er} as single targets. Then left-click on \textit{Peter} to highlight the markable and then do a \underline{right-click} on \textit{er}. A menu will pop up saying \texttt{"Mark as anaphric antecendent of target"}. Left-click on that.

\subsection{Annotating sentiment\_ref}
The same technique as above applies. First mark a sentiment span as usual. Second annotate the other markable (whatever it may be) as usual. Left-click on the first markable to highlight it then \underline{right-click} on the second markable and in the pop-up menu left-click on "Mark sentiment which this span belongs to".

%%%%%%%%%%%%%%%%%%%%%%%%%%%%%%%%%%%%%%%%%%%%%%%%%%%%%%%%%%%%%%%%%%%%%%%%%%%%
\section{FAQ}

\underline{Things you SHOULD mark as sentiments:}
\begin{itemize}
  \item Phrases, clauses, sentences and statements expressing a polar opinion
    (e.g. \textit{Der neue Film \"uber Superman war knorke!} -
    \textit{The new superman movie was fantastic});
  \item Interrogative clauses containing a polar opinion
    and do not call this opinion into question but rather ask for its
    reasons or other related aspects (e.g. \textit{Warum findet die
      Mehrheit der Bev\"olkerung CDU so toll?} - \textit{Why does the
      majority of the population consider CDU great?});
  \item Exclamations expressing support or disapproval of
    something (e.g. \textit{Der Sommer ist da. Super!} -
    \textit{Summer has come. Great!}).
  \item Comparisons expressing some preference of one subject/object over another 
      (e.g. \textit{Canon EOS 550d macht bessere Aufnahmen als 600d} - \textit{Canon EOS 550d takes
      better pictures than 600d}).
\end{itemize}


\underline{Things you SHOULD NOT mark as sentiments:}
\begin{itemize}
  \item Statements that do not have an annotatable target:
      (e.g. \textit{F\"uhle mich so traurig :(} - \textit{Feeling so blue :(});
  \item Statements describing some objective facts, even if possible
    consequences of these facts can be assumed to have negative
    influence on the author (e.g. \textit{Wenn ein Floh einen Menschen
      bei\ss{}t und ihn mit erbrochenem Blut infiziert, werden die
      Pestbakterien ins Gewebe \"ubertragen.} - \textit{When a flea
      bites a human and contaminates the wound with regurgitated
      blood, the plague carrying bacteria are passed into the
      tissue.});
  \item Interrogative clauses in case they ask whether some polar
    opinion is true or not. (e.g. \textit{Findet die Mehrheit
      der Bev\"olkerung CDU toll?} - \textit{Does the majority
      of the population consider CDU great?});
\end{itemize}

\underline{Things you SHOULD mark as sources:}
\begin{itemize}
  \item Original author(s) of polar opinions, i.e. persons,
    groups or officials who express sympathy or antipathy for
    something or somebody e.g. \textit{\textbf{Michael} meinte, das w\"are die
      beste L\"osung.}  (\textit{\textbf{Michael} thought, it would be the
      best solution.});
\end{itemize}


\underline{Things you SHOULD NOT mark as sources:}
\begin{itemize}
  \item Persons how are neutrally citing someone else's opinion,
    e.g. in the sentence \textit{Nach Tatjanas Worten war Michael sehr
      dar\"uber ver\"argert.} (\textit{According to Tatjana, Michael
      was very angry about that.}) you should only mark
    \textit{Michael} as source and not \textit{Tatjana}. Please note, that in case an author
    supports or contradicts someone else's opinion, two sentiment
    relations should be made - one with the original author as source
    and one with the supporter/opponent of the opinion as the other
    source.\footnote{In that case sentiments will overlap, and
      you also should mark these sources with \textit{sentiment-ref}-tag.}
\end{itemize}

\underline{Things you SHOULD mark as emo-expressions:}
\begin{itemize}
  \item Adjectives and adverbs bearing polar attitudes \textit{Peter
    hatte \textbf{bessere} Noten in der Schule als sein Bruder}
    (\textit{Peter had \textbf{better} grades at school than his
    brother.});

  \item Verbs expressing attitude of a speaker to target,
    e.g. \textit{Mir \textbf{gefiel} die neue House-Staffel}
    (\textit{I \textbf{liked} the new House series});

  \item Idiomatic expression including support verbs
    e.g. \textit{\textbf{Zum Teufel} soll die neue Regierung
      \textbf{gehen}} (\textit{The new government should \textbf{go to
        hell}}).

  \item Smileys in case they really express an emotional attitude and
    are not used for politeness or without any particular meaning
    e.g. \textit{Gleich in Braunschweig mit Kameraden treffen
        \textbf{:)}} (\textit{Will soon meet friends in Braunschweig
        \textbf{:)}}).
\end{itemize}
%%%%%%%%%%%%%%%%%%%%%%%%%%%%%%%%%%%%%%%%%%%%%%%%%%%%%%%%%%%%%%%%%%%%%%%%
