\documentclass[11pt,a4paper]{article}

%%%%%%%%%%%%%%%%%%%%%%%%%%%%%%%%%%%%%%%%%%%%%%%%%%%%%%%%%%%%%%%%%%
%% Libraries
\usepackage[driver=pdftex,vmargin=2cm,hmargin=2cm]{geometry}
\usepackage[usenames,dvipsnames]{xcolor}
%% \usepackage{amsmath}
\usepackage{array}
\usepackage{color}
\usepackage{hyperref}
\usepackage{multirow}
\usepackage{multicol}
\usepackage{paralist}
\usepackage{url}
\usepackage{eurosym}
\usepackage{gb4e}
\usepackage{wasysym}   % smiley symbols

\hypersetup{
  colorlinks,
  citecolor=Violet,
  linkcolor=Red,
  urlcolor=Blue}

%%%%%%%%%%%%%%%%%%%%%%%%%%%%%%%%%%%%%%%%%%%%%%%%%%%%%%%%%%%%%%%%%%
%% Commands
\definecolor{dodgerblue4}{RGB}{16,78,139}
\definecolor{orange3}{RGB}{205,133,0}
\definecolor{DarkSlateBlue}{RGB}{72,61,139}

\newcommand{\xmltag}[1]{{\textbf{\small$<$#1$>$}}}
\newcommand{\sentiment}[1]{\xmltag{sentiment}#1\xmltag{/sentiment}}
\newcommand{\source}[1]{\xmltag{source}#1\xmltag{/source}}
\newcommand{\target}[1]{\xmltag{target}#1\xmltag{/target}}
\newcommand{\emoexpression}[1]{\xmltag{emo-expression}#1\xmltag{/emo-expression}}
\newcommand{\intensifier}[1]{\xmltag{intensifier}#1\xmltag{/intensifier}}
\newcommand{\diminisher}[1]{\xmltag{diminisher}#1\xmltag{/diminisher}}
\newcommand{\negation}[1]{\xmltag{negation}#1\xmltag{/negation}}

%%%%%%%%%%%%%%%%%%%%%%%%%%%%%%%%%%%%%%%%%%%%%%%%%%%%%%%%%%%%%%%%%%
%% Lengths
\newlength{\cwidth} \setlength{\cwidth}{0.18\textwidth}
\newlength\clmnwidth
\setlength{\clmnwidth}{0.6\textwidth}

%%%%%%%%%%%%%%%%%%%%%%%%%%%%%%%%%%%%%%%%%%%%%%%%%%%%%%%%%%%%%%%%%%
%% Environments
\newenvironment{myexe}{
  \begin{exe}
    \ex\begin{center}
    \itshape
}{
    \end{center}
  \end{exe}
}

%%%%%%%%%%%%%%%%%%%%%%%%%%%%%%%%%%%%%%%%%%%%%%%%%%%%%%%%%%%%%%%%%%
%% Title
\author{Wladimir Sidorenko}
\date{\today}
\title{Guidelines for Sentiment Corpus Annotation}

%%%%%%%%%%%%%%%%%%%%%%%%%%%%%%%%%%%%%%%%%%%%%%%%%%%%%%%%%%%%%%%%%%
%% Main
\begin{document}
\maketitle{}
\section{Overview}
\subsection{Introduction}

Welcome.  In this assignment, your task is to annotate sentiments in a
corpus of Twitter messages.

As sentiments, we regard polar (either positive or negative) opinions
about persons, objects, or events.  In this task, you have to annotate
both the text spans expressing the opinions (\textit{sentiments}) and
the text spans expressing the objects and events being assessed
(\textit{sentiment targets}). In addition to that, you also have to
annotate the opinion-holders (also called \textit{sentiment sources}),
words and phrases that bear some polar assessing meaning on their own
(\textit{emotional expressions}), and elements which might decrease or
increase the lexical sense of a polar word (\textit{diminishers} and
\textit{intensifiers}) or turn that sense to the opposite
(\textit{negations}).

A complete list of elements to be annotated along with their
description and a description of their attributes is provided in
Section \ref{sec:markables}.  Section \ref{sec:summary} gives you a
short summary of the most important annotation elements. In Section
\ref{sec:faq}, you can also find some examples of difficult and
controversial annotation cases and some possible solutions that we
have suggested for them.

\subsection{Annotation Tool}

For the annotation task, you will use \texttt{MMAX2} -- a freely
available annotation tool.  You can download this program under the
following link:\\\newline
{\setlength{\parindent}{0pt}\small\url{http://sourceforge.net/projects/mmax2/files/mmax2/mmax2_1.13.003/MMAX2_1.13.003b.zip/download}}\\\newline
After the download has finished, please unpack the received archive,
change to the newly created directory \texttt{1.13.003/MMAX2} in your
terminal shell and execute the following commands:\\

\texttt{> chmod u+x ./mmax2.sh}

\texttt{> nohup ./mmax2.sh \&} \\\newline
       {\setlength{\parindent}{0pt}An \texttt{MMAX} window should then
         appear on your screen.  If you have never used \texttt{MMAX2}
         before, please read the document
         \texttt{mmax2quickstart.pdf}, which you can find in the
         subdirectory \texttt{MMAX2/Docs} of the downloaded archive.}

\subsection{Corpus Files}

You should also have received a copy of corpus files as a tar-gzipped
archive.  Please unpack this archive by using the command:\\\newline
\texttt{> tar -xzf twitter-sentiment.tgz}\\\newline
       {\setlength{\parindent}{0pt} A directory called
         \texttt{mmax-prj} should then appear in your current folder.
         Change to your \texttt{MMAX2} window and click on the menu
         \texttt{File -> Load}.  Select the path to the unpacked
         \texttt{mmax-prj} folder in the displayed popup
         window\footnote{Make sure that the path to \texttt{mmax-prj}
           does not contain any white spaces.  Otherwise,
           \texttt{MMAX2} will fail to load the project file.} and
         select one of the \texttt{*.mmax} files there. The source
         data for the annotation file will then be loaded into your
         \texttt{MMAX2} editor.}

%%%%%%%%%%%%%%%%%%%%%%%%%%%%%%%%%%%%%%%%%%%%%%%%%%%%%%%%%%%%%%%%%%

\section{Tags and Attributes}\label{sec:markables}
Below, you can find a short list of all tags and their possible
attributes, that will be used in this annotation task:

\begin{multicols}{2}
  \begin{enumerate}
  \item \textit{sentiment}-tag with the attributes:
    \begin{enumerate}
    \item polarity,
    \item intensity,
    \item sarcasm;
    \end{enumerate}
  \item \textit{target}-tag with the attributes:
    \begin{enumerate}
    \item preferred,
    \item anaph-ref,
    \item sentiment-ref;
    \end{enumerate}
  \item \textit{source}-tag with the attributes:
    \begin{enumerate}
    \item anaph-ref,
    \item sentiment-ref;
    \end{enumerate}
  \item \textit{emo-expression}-tag with the attributes:
    \begin{enumerate}
    \item polarity,
    \item intensity,
    \item sarcasm,
    \item sentiment-ref;
    \end{enumerate}
  \item \textit{intensifier}-tag with the attributes:
    \begin{enumerate}
    \item degree,
    \item emo-expression-ref;
    \end{enumerate}
  \item \textit{diminisher}-tag with the attributes:
    \begin{enumerate}
    \item degree,
    \item emo-expression-ref;
    \end{enumerate}
  \item and the \textit{negation}-tag.
    \begin{enumerate}
    \item emo-expression-ref;
    \end{enumerate}
  \end{enumerate}
\end{multicols}
    {\setlength{\parindent}{0pt} A more detailed description of the
      meaning of these tags and the values of their attributes is
      provided in the next subsections.}

\subsection{sentiment}\label{sec:sentiment}
With \texttt{sentiment} tags, you should mark any polar opinion
statements made about people, objects or events.  Typical examples of
sentiment statements are sentences like the following one:
\begin{myexe}
  \sentiment{Ich mag den neuen James Bond Film nicht}.

  (\sentiment{I don't like the new James Bond
    movie}.)\label{ex:sentiment}
\end{myexe}
{\setlength{\parindent}{0pt} in which a \textit{James Bond movie} is
  being subjectively assessed.}

Sentiment expressions, however, are not only constrained to sentences
-- they can also be longer or shorter than these.  In general, polar
opinions can usually be expressed by:
\begin{itemize}
\item single noun phrases, possibly with their prepositional
  attributes, e.g. \textit{Auf dem Tisch lag \xmltag{sentiment}\\ ein
    langweiliges Buch\xmltag{/sentiment} (There was a
    \sentiment{boring book} on the table)};
\item clauses, e.g. \textit{\sentiment{Ich hasse B\"ucher ohne
  Inhaltsangabe}. (\sentiment{I hate books without table of
  contents})};
\item multiple sentences in cases when these sentences jointly form a
  sentiment relation, e.g.\\\textit{\sentiment{Sie denken, reden,
      riechen, lieben, schmecken, f*cken Plastik. Sie haben\\das so
      gelernt in der Plastik-Werbewelt}.\\ (\sentiment{They think and
      speak about, smell at, love, taste, f*ck plastic.  They
      have\\ learned it so in the advertising world of plastic}.)}.
\end{itemize}
This list is however not limited.

In order to decide whether a given statement forms a sentiment or not,
you should check whether it satisfies the following three conditions:
\begin{enumerate}
  \item The sentiment statement has to be \textit{subjective}, i.e. it has to
    express or imply someone's opinion and should not be a mere declaration of
    objective facts.  Therefore, cases like \textit{CDU hat die Wahl gewonnen
      (CDU has won the election)} should not be marked as sentiments because
    they do not contain any subjective information;

  \item The sentiment statement has to be \textit{polar}, i.e. it
    should clearly express either a positive or a negative judgement
    about something.  According to this rule, cases like
    \textit{Meiner Meinung nach k\"onnte man den Tisch zuerst
      raustragen (I think we could take out the table first)} should
    not be marked as sentiments since they do not express any polarity
    (i.e a positive or negative attitude) even though a subjective
    opinion is expressed;

  \item The judgement of the statement should always be
    \textit{directed at a particular target} which can be any object,
    including people, institutions, properties, events etc.  It means
    that cases like \textit{Ich bin so froh! (I am so happy!)} also
    should not be marked as sentiments as defined here since they do
    not assess anything in particular.
\end{enumerate}

A special case of evaluative opinions which also usually form a
sentiment are comparisons.  In contrast to generally positive or
negative evaluations, comparisons normally express a relative
assessment of a person or an object, i.e. it is considered either good
or bad only with regard to another object of the same class.  If you
can clearly distinguish a comparison in a sentence and determine which
of the compared objects is being preferred, you should also annotate
such cases as sentiments.

Sentiment tags can accept the following attributes with the following
possible values:
\begin{center}
  \begin{tabular}{|l|c|p{\clmnwidth}|}\hline
    Attribute & Value & Value's Meaning\\\hline
    %%%%%%%%%%%%%%%%%%%

    & \textit{positive} & sentiment expresses positive attitude about
    its respective target, e.g. \textit{Es war ein fantastischer Abend
      (It was a fantastic evening)};\\\cline{2-3}

    & \textit{negative\newline(default)} & sentiment expresses
    negative attitude about its respective target, e.g. \textit{Seine
      Schwester ist einfach unausstehlich (His sister is simply
      obnoxious)}\\\cline{2-3}

    \multirow{-3}{*}{polarity} & \textit{comparison} & sentiment
    expresses a comparison of two targets with preference given to one
    of them, e.g. \textit{Mir gef\"allt das rote Kleid mehr als das
      blaue (I like the red dress more than the blue one)}\\\hline

    %%%%%%%%%%%%%%%%%%%

    & \textit{weak} & weak positive or
    negative assessments, e.g. \textit{Der Auftritt war mehr oder
      weniger gut (The appearance was more or less good)}\\\cline{2-3}

    & \textit{medium\newline(default)} & sentiments with middle
    emotional expressivity, e.g. \textit{Mir hat das neue Album gut
      gefallen (I enjoyed the new album)}\\\cline{2-3}

    \multirow{-3}{*}{intensity} & \textit{strong} & very emotional
    polar statements, e.g. \textit{Dieses Festival war einfach
      umwerfend!!! (This festival was simply terrific!!!)}\\\hline
    %%%%%%%%%%%%%%%%%%%

    \multirow{2}{*}{sarcasm} & \textit{true} & this polar attitude is
    derisive, i.e. its actual polarity is the opposite of its apparent
    form. (This means that an apparent praise which appears in text is
    in fact meant as a rebuke and vice versa. The actual sense,
    however, can only be inferred on the basis of world knowledge or
    reasoning.)  An example of a sarcastic sentiment is the following
    passage: \textit{Mein J\"ungerer ist in der Pr\"ufung
      durchgefallen.  Gut gemacht! (My youngest has failed his exam.
      Well done!)}  In this case, you should set the polarity attribute
    of the sentiment to \textit{negative} and the value of the
    \textit{sarcasm} attribute to \textit{true}.\\\cline{2-3}

    & \textit{false\newline(default)} & no sarcasm is present -- the
    polar attitude has its literal meaning; this is the default
    setting\\\hline
  \end{tabular}
\end{center}

%%%%%%%%%%%%%%%%%%%%%%%%%%%%%%%%%%%%%%%%%%%%%%%%%%%%%%%%%%%%%%%%%%%%%%%%%%%%%%%%%%%%%%%%%%
\subsection{target}
With the \texttt{target} tags, you should mark objects or events which
are being assessed by sentiment expressions.  There should always be
at least one target element for one sentiment relation.

An example of a sentiment target is given in the following sentence:
\begin{myexe}
Mein Bruder ist nicht begeistert von \target{dem neuen Call of Duty}.

(My brother is not impressed by \target{the new Call of Duty}.)
\end{myexe}

The target element is usually expressed by:
\begin{itemize}
\item single noun phrases, possibly with their prepositional attributes,
  e.g. \textit{Mir wird's schlecht, wenn ich \target{diese Werbung} im
    Fernsehen sehe (I feel sick when I see this \target{ad} on TV)};

\item clauses, e.g. \textit{Ich hasse, \target{wenn jemand sich in die dritte
    Spur einordnet und dann wie eine Schnecke kriecht} (I hate \target{when
    someone changes to the leftmost lane and then crawls at a snail's pace})};
\end{itemize}

If the given sentiment relation has multiple targets, which are independent
from each other and conjoined by commas or coordinative conjunctions, you
should mark each of such targets with a separate \texttt{target} tag.
%% \begin{myexe}
%%   Meiner Mutter haben immer \target{Nelken} und \target{Dahlien}
%%   gefallen.

%%   (My mother has always liked \target{carnations} and
%%   \target{dahlias}.)
%% \end{myexe}

In cases, when the sentiment relation is represented by a comparison,
you should also mark each of the compared objects as a separate
target.  Additionally, for the object which is being preferred in the
comparison, you also should set the \textit{preferred} attribute of
that target to \textit{true}:
\begin{myexe}
  Ich mag $<$\textbf{target
    preferred=``true''}$>$Domino-Eis$<$\textbf{/target}$>$ mehr als
  \target{Magnum}.

  (I like the $<$\textbf{target preferred=``true''}$>$Domino ice
  cream$<$\textbf{/target}$>$ more than \target{Magnum}.)
\end{myexe}

Possible attributes and attribute values for the \texttt{target} tag
are listed in Table \ref{tbl-target}:
\begin{center}
  \begin{tabular}{|l|c|p{\clmnwidth}|}\hline
    Attribute & Value & Value's Meaning\\\hline\label{tbl-target}

    & \textit{true} (default) & in comparisons, this value means that
    the respective target is being considered better than another
    compared object, e.g. \textit{\textbf{Die neue Frisur} passt ihr
      garantiert besser als die alte (\textbf{The new hairstyle} suits
      her definitely better than the old one)};\\\cline{2-3}

    \multirow{-2}{*}{preferred} & \textit{false} & in comparisons,
    this value marks the target element which is being considered
    worse than its counterpart, e.g. \textit{Die zweite Saison von
      Breaking Bad war viel spannender als \textbf{die dritte} (The
      second season of Breaking Bad was much more exciting than
      \textbf{the third one})};\\\hline

    sentiment-ref & \textit{$\longrightarrow$\newline(directed edge)}
    & directed edge pointing from the target element to its respective
    sentiment span.  You need to draw this edge in two cases:
    \begin{itemize}
    \item if the target element is located in an intersection of two
      different sentiments (in this case, you should draw an edge from
      the target to the sentiment relation, which this target actually
      belongs to)

    \item when the target of a sentiment opinion is expressed outside
      the sentiment span
    \end{itemize}\\\hline

    anaph-ref & \textit{$\longrightarrow$\newline(directed edge)} &
    directed edge pointing from the target element expressed by a
    pronoun or pronominal adverb to its respective non-pronomial
    antecedent (in order to draw this edge, you will also need to mark
    the antecedent as target)\\\hline
  \end{tabular}
\end{center}

%%%%%%%%%%%%%%%%%%%%%%%%%%%%%%%%%%%%%%%%%%%%%%%%%%%%%%%%%%%%%%%%%%
%% Source
\subsection{source}
The \texttt{source} tag is used to mark the immediate author(s) or
holder(s) of a sentiment opinion.  These are usually speakers or
writers of an assessment or persons whose evaluative attitude is being
cited.  In cases, when the sentiment source is not explicitly present
in the message, it is implicitly assumed to be the author of the
tweet.

An example of a \texttt{source} expression is the pronoun \textit{Sie
  (she)} in the following sentence:
\begin{myexe}
  \source{Sie} mag die neue Farbe nicht

  (\source{She} doesn't like the new color)
\end{myexe}

%% You should usually mark the source element after you have determined
%% and annotated the sentiment relation and its target.  The source
%% should denote the immediate author or holder of the determined
%% sentiment opinion.
In case of citations, only the person or institution, who originally
expressed the assessment, should be marked as \texttt{source} (see
Example \ref{ex:source-requote}).
\begin{myexe}
  Laut Staatsanwalt soll die \source{Angeklagte} sich missbilligend \"uber
  ihren Vorgesetzten ge\"au\ss{}ert haben.

  (According to the attorney, the \source{defendant} had made disapproving
  remarks about her boss.)\label{ex:source-requote}
\end{myexe}
The citing persons should not be considered as sources unless it is
explicitly stated that they also share the opinion they cite.

A \texttt{source} tag can usually include:
\begin{itemize}
  \item Pronouns, e.g. \textit{\source{Er} genoss die Aussicht von
    seinem Balkon (\source{He } enjoyed the view from his balcony)}.
    In this case, you should also draw an edge from the pronoun to its
    respective antecedent if this antecedent can be observed in the
    immediate context (see the \texttt{anaph-ref} attribute in Table
    \ref{tbl-target});

  \item Noun phrases formed by nouns and their attributes,
    e.g. \textit{\source{Sein Onkel} regte sich \"uber die neuen
      Steuern auf (\source{His uncle} got worked up over the new
      taxes)};
\end{itemize}
In cases, when the source of a sentiment is expressed by multiple
coordinatively joined independent noun phrases, you should mark each such noun
phrase with a separate \texttt{source} tag, e.g.:
\begin{myexe}\label{ex:2source}
  \source{Ihr} und \source{ihrer Mutter} gef\"allt die neue Farbe
  nicht.\\ (\source{She} and \source{her mother} do not like the new
  color)
\end{myexe}

The attributes of the \texttt{source} tag are listed in Table
\ref{tbl-source}.  The meaning of these attributes is fully identical
to the meaning of the corresponding attributes for \texttt{target}s.
\begin{center}
  \begin{tabular}{|l|c|p{\clmnwidth}|}\hline
    Attribute & Value & Value's Meaning\\\hline\label{tbl-source}

    sentiment-ref & \textit{$\longrightarrow$\newline(directed edge)}
    & see Table \ref{tbl-target}\\\hline

    anaph-ref & \textit{$\longrightarrow$\newline(directed edge)} &
    see Table \ref{tbl-target}\\\hline
  \end{tabular}
\end{center}

%%%%%%%%%%%%%%%%%%%%%%%%%%%%%%%%%%%%%%%%%%%%%%%%%%%%%%%%%%%%%%%%%%
%% Emo-expression
\subsection{emo-expression}
With the \texttt{emo-expression} tag, you should mark words and
phrases that have some polar subjective sense in their lexical
meaning.  A possible example of an emo-expression is the word
\textit{ekelhaft} (\textit{disgusting}) the sentence below:
\begin{myexe}
  Beim Aufr\"aumen des Zimmers haben wir einen
  \emoexpression{ekelhaften } Teller mit verschimmeltem Essen unter
  dem Bett gefunden.

  (When we cleaned the room, we found a \emoexpression{disgusting}
  plate of moldy food under the bed.)
\end{myexe}

Emotional expressions must always be marked regardless of whether a
sentiment is present or not. This is in contrast to sources and
targets which should only be annotated in the context of a sentiment
relation.

Please note, that due to lexical ambiguity, it can also sometimes
happen that the same word or expression, that appears to have an
emotional lexical sense in one context, will have a neutral meaning in
another.  You should mark an element as an \texttt{emo-expression}
only in the former case, i.e. when its immediately activated sense in
the given text is evaluative (see the following examples):
\begin{myexe}
  Dieser Wein ist ein echtes \emoexpression{Juwel} in meiner
  Kollektion.

  This wine is a real \emoexpression{jewel} in my collection.
\end{myexe}
\begin{myexe}
  Koh-i-Noor ist das teuerste Juwel heutzutage.

  Koh-i-Noor is the most expensive jewel nowadays.
\end{myexe}
In the first example, the meaning of \textit{Juwel (jewel)} is
metaphoric and only this metaphoric sense has a polar connotation.  In
the second example, the meaning of \textit{Juwel (jewel)} is literal
and does not express any appraisal.

The \texttt{emo-expression}s are usually represented by:
\begin{itemize}
  \item nouns, e.g. \textit{Held (hero)}, \textit{Ideal (ideal)},
    \textit{Betr\"uger (fraudster)} etc.;

  \item adjectives or adverbs, e.g. \textit{sch\"on (nice)},
    \textit{zuverl\"assig (reliably)}, \textit{hinterh\"altig
      (devious)}, \textit{heimt\"uckisch (insidiously)} etc.;

  \item verbs, e.g. \textit{lieben (to love)}, \textit{bewundern (to
    admire)}, \textit{hassen (to hate)} etc.;

  \item idioms, e.g. \textit{auf die Nerven gehen (to get on one's
    nerves)} etc.;

  \item smileys, e.g. :), :-(, \smiley{}, \frownie{} etc.;
\end{itemize}
This list is however neither complete nor limited.

The \texttt{emo-expression} elements can accept the following
attributes with the following possible values:
\begin{center}
  \begin{tabular}{|l|c|p{\clmnwidth}|}\hline\label{tbl-emo}
    Attribute & Value & Value's Meaning\\\hline
    %%%%%%%%%%%%%%%%%%%

    & \textit{positive} & the given emotional expression has a
    positive appraising meaning, e.g. \textit{gut (good), verhimmeln
      (to ensky), Prachtkerl (corker)} etc.\\\cline{2-3}

    \multirow{-2}{*}{polarity} & \textit{negative\newline(default)} &
    the given emotional expression has a negative polar sense,
    e.g. \textit{versauen (to botch up), rotzig (snotty), Dreckskerl
      (scum)} etc.\\\hline

    %%%%%%%%%%%%%%%%%%%

    & \textit{weak} & the given emo-expression has a weak appraising
    sense, e.g. \textit{solala (so-so), nullachtf\"unfzehn (vanilla),
      durchschnittlich (mediocre)} etc.\\\cline{2-3}

    & \textit{medium\newline(default)} & the given emo-expression has
    a regular appraising sense and middle stylistic expressivity,
    e.g. \textit{gut (good), schlecht (bad), robust (tough)}
    etc.\\\cline{2-3}

    \multirow{-3}{*}{intensity} & \textit{strong} & the given
    emo-expression forms a very strong positive or negative
    assessment, e.g. \textit{allerbeste (bettermost), zum Kotzen (to
      make one puke), Kacke (shit)} etc.\\\hline

    %%%%%%%%%%%%%%%%%%%

    sentiment-ref & \textit{$\longrightarrow$\newline(directed edge)}
    & This attribute has the same meaning as its equivalents for
    targets and sources.  In ambiguous usage cases, it should show to
    which sentiment the given emo-expression belongs.  Just like for
    targets, a usage case should be considered ambiguous if this
    expression is located in the intersection of two sentiment spans
    or if it is expressed outside of any marked sentiment relation but
    has an immediate relation to it.  If an emo-expression does not
    belong to any sentiment or if it is unambigously embedded in just
    one sentiment relation, this attribute edge needs not to be
    drawn.\\\hline

    %%%%%%%%%%%%%%%%%%%%

    \multirow{2}{*}{sarcasm} & \textit{true} & the given emo-expression is
    derisive, i.e. its actual polarity is the opposite of its apparent
    form. (This means that an apparent praise which appears in text is
    in fact meant as a rebuke and vice versa. The actual sense,
    however, can only be inferred on the basis of world knowledge or
    reasoning.)\\\cline{2-3}

    & \textit{false\newline(default)} & no sarcasm is present -- the
    polar attitude has its literal meaning; this is the default
    setting\\\hline
    
\end{tabular}
\end{center}


When determining the polarity of an emotional expression, you should keep in
mind the following two aspects:
\begin{itemize}
  \item The annotated polarity of an \texttt{emo-expression} should be
    its \textit{prior} (or sometimes also called \textit{lexical})
    polarity.  This type of polarity is the polar sense of a word or
    expression, independent of its possible contextual changes.  In
    simple words, this usually means that you should disregard any
    negations when determining and setting the value of the polarity
    attribute for an \texttt{emo-expression}. So, for example, in the
    sentence \textit{Es war keine gute Idee (It was not a good idea)},
    the polarity of the \texttt{emo-expression} \textit{gute (good)}
    should still be positive despite the fact that in the end, this
    polarity is negated;

  \item Some emo-expressions can have an asymmetric polarity for their
    arguments.  A typical example of such an expression is the word
    \textit{vermissen (miss)}.  If we say that \textit{Alice vermisst
      Bob (Alice misses Bob)}, it usually means something negative for
    Alice (because she experiences a discomfort feeling) but at the
    same time something positive for Bob (because it is implied that
    Alice would be happier if he joined her).  In these cases, if one
    of the arguments is the sentiment's target, you should determine
    the polarity of the emo-expression from the perspective of the
    target.  So, in this case, the target of the sentiment is
    \textit{Bob} (it is his absence which is being assessed) and the
    polarity of the emo-expression \textit{vermissen (miss)} should
    therefore be positive.
\end{itemize}

%%%%%%%%%%%%%%%%%%%%%%%%%%%%%%%%%%%%%%%%%%%%%%%%%%%%%%%%%%%%%%%%%%
%% Intensifier
\subsection{intensifier}
Intensifiers are elements which increase the stylistic expressivity
and the polar sense of an emotional expression.  An example of
intensifier is the word \textit{sehr (very)} in the following example:
\begin{myexe}
  wir suchen eine \intensifier{sehr} zuverl\"assige Polin als
  Haushaltshilfe.

  (we are looking for a \intensifier{very} reliable Polish woman as
  domestic help.)
\end{myexe}
An intensifier should always relate to an emo-expression and you
should always explicitly show this relation by drawing an attribute
edge from the intensifier to the respective expression whose sense is
being intensified (see the \texttt{emo-expression-ref} attribute in
Table \ref{tbl-intens}).

Intensifiers are usually expressed by adverbs or adjectives like
\textit{sehr (very), sicherlich (certainly)} etc.  Other ways of
expressing these elements are still possible, though.

The intensifier elements can have the following attributes with the
following possible values:
\begin{center}
  \begin{tabular}{|l|c|p{\clmnwidth}|}\hline\label{tbl-intens}

    & \textit{medium (default)} & the intensifier moderately increases
    the polar sense of the emotional expression, e.g. \textit{ziemlich
      (quite), recht (fairly)} etc.\\\cline{2-3}

    \multirow{-2}{*}{degree} & \textit{strong} & the intensifier
    strongly increases the polar sense and stylistic markedness of the
    emotional expression, e.g. \textit{sehr (very), super (super),
      stark (strongly)} etc.\\\hline

    %%%%%%

    emo-expression-ref & \textit{$\longrightarrow$\newline(directed
      edge)} & a directed edge pointing from the intensifier to the
    \texttt{emo-expression} whose meaning is being intensified\\\hline
  \end{tabular}
\end{center}

\subsection{diminisher}
With the \texttt{diminisher} tag, you should mark elements that
decrease the polar lexical sense of an emotional expression.  In
Example \ref{ex:diminisher}, you can see one possible usage of a
diminisher:
\begin{myexe}
  \diminisher{Weniger} erfolgreiche Unternehmen verzichten auf externe
  Berater.\label{ex:diminisher}

  The \diminisher{less} successful companies do not use external
  consultants.
\end{myexe}
Just like intensifiers, diminishers should always relate to some emotional
expression and you should always explicitly show this relation by drawing the
attribute edge \texttt{emo-expression-ref} (see Table \ref{tbl-intens}).

Also like intensifieres, diminishers are usually expressed by
adjectives or adverbs, e.g. \textit{wenig (few), kaum (hardly), ein
  bisschen (little)}.  But other means of expressing these elements
are possible too.

Diminishers have the same set of attributes as intensifiers with the only
difference in the interpretation of the \textit{degree} attribute:
\begin{center}
  \begin{tabular}{|l|c|p{\clmnwidth}|}\hline\label{tbl-dimin}

    & \textit{medium (default)} & the given diminisher moderately
    decreases the polar sense of its respective emotional expression,
    e.g. \textit{wenig (few), bisschen (little)} etc.\\\cline{2-3}

    \multirow{-2}{*}{degree} & \textit{strong} & the given diminisher
    strongly decreases the polar sense of the emotional expression,
    e.g. \textit{kaum (hardly)} etc.\\\hline

    emo-expression-ref & \textit{$\longrightarrow$\newline(directed
      edge)} & see Table \ref{tbl-intens}\\\hline
  \end{tabular}
\end{center}

\subsection{negation}
With the \texttt{negation} tag, you should mark elements which turn
the polarity of a sentiment relation to its complete opposite.  For
example, in the following sentence, the negative article \textit{kein
  (not)} renders the sentiment's polarity negative, but without this
article, it would have been positive:
\begin{myexe}
Einstein war \negation{kein} kluger Mensch!\label{ex:negation}

Einstein was \negation{not} a clever person!
\end{myexe}

The role and the meaning of negations are closely related to that of
the diminishers.  In order to help you better differentiate between
these two elements, we have listed the most obvious differences
between the classes:
\begin{itemize}
  \item\textit{Semantic differences}.  While diminishers still allow
    for some portion of the original meaning of an emo-expression to
    remain (\textit{a hardly understandable speech} is still
    understandable), negations fully deny this meaning and change it
    to a completely antonymous one (\textit{a not understandable
      speech} is completely unintelligible);

  \item\textit{Part-of-speech differences}.  While diminishers are usually
    expressed by either adjectives or adverbs; negations, on the contrary, are
    commonly represented by either the negative article \textit{kein (no)} or
    the negation particle \textit{nicht (not)}, or complete clauses, e.g.
    \textit{Es ist sehr zweifelhaft, dass die neue Version von Windows besser
      wird (It is very doubtful that the new Windows version will be any
      better)}.

  %% \item\textit{Syntactic differences}.  While diminishers are usually
  %%   expressed by either adjectives or adverbs; negations, on the contrary,
  %%   are commonly represented by either the negative article \textit{kein
  %%   (no)} or the negation particle \textit{nicht (not)}, or even complete
  %%   clauses (\textit{Er ist nicht einverstanden, dass die neue Version von
  %%   Windows besser ist (He disagrees that the new Windows version is any
  %%   better)}).
\end{itemize}
All three elements -- diminishers, intensifiers, and negations --
should only be annotated in the presence of an \texttt{emo-expression}
or a \texttt{sentiment}.  If neither the former nor the latter is
present in the text message, you should not annotate these elements
too.

As already mentioned, negations can usually be expressed by:
\begin{itemize}
  \item the negative article \textit{kein (no)}, e.g. \textit{Es war keine
    Vergn\"ugungsreise f\"ur australische Machthaber (It was no pleasure trip
    for some of Australia's most powerful leaders)};
  \item the negation particle \textit{nicht (not)}, e.g. \textit{Er hat seine
    Studienzeit ganz und gar nicht genossen (He did not enjoy his college days
    at all)};
  \item clauses, e.g. \textit{Es ist absolut unm\"oglich, dass aus ihm am Ende
    noch ein guter Mensch wird (It is absolutely impossible that he will
    eventually become a good man)}.
\end{itemize}

Like diminishers and intensifiers, negations also can have the pointer
attribute \texttt{emo-expression-ref} (see Table \ref{tbl-negation}).
You should draw this edge in cases when you can clearly indentify
which \texttt{emo-expression} is being negated by a negation.  In
cases, when you can identify a sentiment and a negation but no
emotional expression is present, you must not draw this edge.
\begin{center}
  \begin{tabular}{|l|c|p{\clmnwidth}|}\hline\label{tbl-negation}
    emo-expression-ref & \textit{$\longrightarrow$\newline(directed
      edge)} & an edge from the negation to the
    \texttt{emo-expression} being negated\\\hline
  \end{tabular}
\end{center}

\section{Summary}\label{sec:summary}
To summarize, your task in this assignment is to find subjective
appraising opinions, that judge about some particular things or events
from a personal perspective.  You should annotate these assessments as
\texttt{sentiment}s, determine whether the judgements they represent
are negative or positive, and also decide how strong the judgements
are.  Next, you should separately mark the objects being assessed as
sentiment \texttt{target}s.  The authors of the opinions (or the
persons whose appraising opinions are being cited) should be marked as
sentiment \texttt{source}s.  Both, sources and targets should only be
marked if a sentiment relation is present in message.

Another important task is to annotate lexical expressions whose
meaning contains some polar assessing sense.  These elements are
called \texttt{emo-expression}s and they should always be annotated
regardless of whether a sentiment relation is present or not.  In case
when an emo-expression is being intensified, diminished, or negated by
some elements, you should also annotate these modifying elements as
\texttt{intensifier}, \texttt{diminisher}, and \texttt{negation},
respectively.

\section{FAQ}\label{sec:faq}
This section provides possible solutions to some difficult and
controversial annotation cases.  Please read through this list
carefully before starting to annotate.

\begin{itemize}
\item\textbf{Q: Should I annotate as sentiments cases like
  \textit{einer Sache zustimmen (to agree with sth.)}, \textit{sich
    f\"ur etwas entscheiden (to opt for sth.)} or \textit{etwas
    unterst\"utzen (to support sth.)}?}

  \textbf{A:} Well, these cases are a little bit tricky and,
  unfortunately, we will not be able to give you a clean and elegant
  solution for them.  We would rather say that both variants,
  i.e. annotating and not annotating sentences with these elements as
  sentiments, would be legitimate.  In fact, after all annotators
  finish their work, we are going to conduct a study on the
  inter-annotator agreement for such expressions and make our final
  decision based on the majority of choices made by different
  annotators.

\item\textbf{Q: How would you annotate the following cases of comparisons?}
  \begin{itemize}
  \item\textbf{\textit{Seehofer hat die Gr\"unen ausgeschlossen , aber
      die Linke nicht (Seehofer has excluded the Greens, but not the
      Left)}};

    \textbf{A:} Without any further context, I cannot see any
    sentiment relation here.  So, I would probably not annotate
    anything;

  \item\textbf{\textit{Lieber starke Mitte statt linker Rand (Better
      strong middle than left edge)}};

    \textbf{A:} Comparison with \textit{starke Mitte (strong middle)}
    as preferred target, and \textit{linker Rand (left edge)} as
    dispreferred one;

  \item\textbf{\textit{Die \#spd wird lieber mit den rechten von \#cdu
      , \#csu koalieren als mit der \#linke (The \#spd will better for
      a coalition with the rightists from \#cdu, \#csu than with the
      \#left)}};

    \textbf{A:} Comparison, with \textit{\#spd} as source,
    \textit{\#cdu} and \textit{\#csu} preferred targets, and
    \textit{\#linke (\#left)} as dispreferred one;

  \item\textbf{\textit{Die \#AfD + vereinigt mehr \"okonomische
      Kompetenz als alle etabl. Parteien + Bunde... (The \#AfD +
      combines more economic expertise than all established parties +
      federal...)}};

    \textbf{A:} Comparison, with \textit{\#AFD} as preferred target
    and \textit{established parties} and \textit{federal} as
    dispreferred ones;

  \item\textbf{\textit{Freiheit statt Bevormundung (Freedom instead of
      paternalism)}};

    \textbf{A:} Comparison, with \textit{Freiheit (freedom)} as
    preferred target, and \textit{Bevormundung (paternalism)} as
    dispreferred one;

  \item\textbf{\textit{Fettarme Milch hat mittlerweile mehr Prozent
      wie die FDP (lowfat milk has meanwhile more percents than the
      FDP)}};

    \textbf{A:} I'd rather say that this is sarcasm about the FDP.
    Because we usually cannot compare a bottle of milk with a
    political party.  If we do so then usually with the purpose of
    tutting at this party;

  \item\textbf{\textit{Was ist der Unterschied zwischen einem Smart
      und der FDP ? Der Smart hat wenigstens 2 Sitze :) (What is the
      difference between a Smart and the FDP? The Smart has at least
      two seats)}};

    \textbf{A:} The same as the previous question -- sarcasm about the
    FDP;
  \end{itemize}

\item\textbf{Q: How can I determine the intensity of a comparison?}

  \textbf{A:} Well, comparisons are, in fact, assessments too.  But in
  contrast to other sentiment relations, these assessments are
  relative rather than absolute.  It means that we do not claim that
  something is bad or good, but we just say that something is worse or
  better than something else.  The more vivid this difference between
  two compared objects is, the higher is the intensity of the
  comparison.  So, if we find \textit{this lousy Telekom waaaaaay less
    reliable than O2}, then the intensity should probably be higher
  than in \textit{Telekom has a less reliable connection than 02}.

\item\textbf{Q: Should I annotate sentiments in recommendations?}

  \textbf{A:} Usually yes.  If this recommendation represents a
  personal opinion and is not a spam, and you can unambiguously
  determine the target of this opinion, then you should annotate
  sentiment here.

\item\textbf{Q: Should I annotate sentiments in defenses?}

  \textbf{A:} Usually not.  If a soldier defends his position or a PhD
  student defends her thesis, it does not necessarily imply that he or
  she likes it.  The same is true in cases when someone defenses
  another person in a dispute.

\item\textbf{Q: Should I annotate sentiments in questions?}

  \textbf{A:} Well, it primarily depends on the type of the question.
  Here, you should distinguish two cases:
  \begin{itemize}
    \item If it is a \textit{yes-no-question}, which asks whether a
      particular sentiment opinion is true or not, you should not
      annotate any sentiment relation in such sentence.  For example,
      in \textit{Gef\"allt dir der neue Rock?  (Do you like the new
        skirt?)}, we do not know whether the asked person likes or
      dislikes the new skirt, thus we should not annotate a sentiment
      in this case;
    \item If it is a \textit{wh-question}, which asks about the
      reasons or some other aspects of a polar opinion but does not
      raise the truth of this opinion to question, the sentiment
      relation should be marked.  For example, in \textit{Warum hasst
        du deine Schwester? (Why do you hate your sister?)}, the fact
      that someone hates her sister is not in doubt and we should
      annotate sentiment in this sentence.
  \end{itemize}

\item\textbf{Q: Should I mark sentiments in insults?}

  \textbf{A:} If you can locate the target, then yes.  For example, in
  \textit{Du bist ein Idiot! (You are an idiot!)}, \textit{Du (You)}
  is the target of a negative sentiment relation.  On the contrary, in
  cases like \textit{Idiot!  (Idiot!)}, no target is detectable and
  therefore, according to our definition, no sentiment can exist.

\item\textbf{Q: Is it possible that sources and targets are expressed
  by other means than the ones described in these guidelines?}

  \textbf{A:} Yes. These guidelines are in no way exhaustive, they
  should just give you a better sense of how a typical source or
  target usually might look like.

\item\textbf{Q: It is said that we should disregard negations when
  determining the polarity of an emo-expression.  What about
  sentiments, shall we take into account negations there?}

  \textbf{A:} Yes.  The polarity of an emo-expression represents the
  polar sense of a single lexical item.  The polarity of a sentiment,
  on the contrary, should show the joint sense of the whole phrase,
  which forms a sentiment relation, so you should take into account
  negations there.

\item\textbf{Q: How should I annotate chains of
  intensifiers/diminishers -- each separately or the whole chain with
  one tag?}

  \textbf{A:} In case of multiple intensifiers/diminishers, each
  element should be tagged separately, e.g. \textit{Du bist die
    '\intensifier{aller} \intensifier{aller}' Beste! (You are the
    \intensifier{very} \intensifier{very} best!)}

\item\textbf{Q: What is the target in, for example, \textit{a really nice weekend}. 
	The whole phrase?}

  \textbf{A:} No it is not the whole phrase but only \textit{weekend}. \textit{really} 
	is an intensifier and \textit{nice} is an emo-expression.

%% \item F: Dem Tweet aus politics1 ``Wo ist der \#Jubel von \#CDU \#CSU
%%   \& \#FDP \"uber den Tod der Mieterin nach \#Zwangsr\"aumung?'' habe
%%   ich als sarkastisch und mit target=''Tod der Mieterin nach
%%   \#Zwangsr\"aumung'' annotiert. Wei\ss aber nicht, ob das so richtig
%%   ist.

%% A: Mhm, aus meiner Sicht wird hier eher \"uber die CDU/CSU und FDP
%% gespottet, man w\"urde also annehmen, dass diese Parteien sich \"uber
%% den Tod der Mieterin freuen w\"urden, was nat\"urlich dem Image einer
%% Partei eher schaden w\"urde.  Ich w\"urde demzufolge den ganzen Satz
%% als negatives Sentiment mit den Parteien als Targets und dem Attribut
%% sarcasm gesetzt auf true annotieren.
\end{itemize}
\end{document}


%%%%%%%%%%%%%%%%%%%%%%%%%%%%%%%%%%%%%%%%%%%%%%%%%%%%%%%%%%%%%%%%%%%%%%%%%%%%%
%%%%%%%%%%%%%%%%%%%%%%%%%%%%%%%%%%%%%%%%%%%%%%%%%%%%%%%%%%%%%%%%%%%%%%%%%%%%%

\subsection{sentiment}
\begin{itemize}
\item I have difficulties determining the overall sentiment of a tweet.  Is
  there a common modus operandi?

Please keep in mind that sentiments should always be annotated
target-oriented. Look at this tweet: \textit{Ich hab Maria ja schon so
  vermisst und so , ne ? (I kind of did miss Maria, right?)}. Here you
might think the tweet has a negative sentiment, because the author
missed Maria which is not a good thing. But we are always only
concerned about the sentiment relation between the source and the
target! And here it is a positive sentiment, because the author missed
Maria because he likes her.

\item How to annotate sarcasm?

If the apparent form of a sentiment is positive (and therefore should
be annotated as a positive sentiment) but you can detect sarcasm
please annotate the polarity as negative and additionally set
\texttt{sarcasm} to \texttt{true} in the attribute window. This
applies to the sentiment tag as well as to emo-expressions.

\item Should subjunctives be annotated?

Yes. Have a look on the following tweet: \textit{W\"are toll wenn das
  der n\"achste " Call of Duty " - Teil sein w\"urde , liebe
  @GameStar. (Dear @gamestar, it would be great if this would be the
  sequel of "Call of Duty".)}. Although this is a subjunctive
sentiment and is not "reality" right now you should annotate this,
too.

\item Should multiple sentiment layers be annotated?

Yes. For example in the tweet \textit{72 j\"ahriger Leser regt sich
  bei mir \"uber YouTube und Gema auf :) (72 year old reader troubles
  over YouTube and Gema at me :) )} you can find two layers of
sentiments. The first layer is the fact that the reader troubles over
YouTube and Gema. The second layer is the fact that the author of the
tweet is amused about the first fact. So here you have to annotate two
different sentiments. The first is negative and has "YouTube" and
"Gema" as target (with "troubles over" as emo-expression). The second
is positive and has "72 jähriger Leser" as target and "mir" as
source. Do not forget to use the \texttt{sentiment\_ref} attribute to
annotate which sentiment refers to which target.
\end{itemize}

\subsection{source}
\begin{itemize}
\item I can find the same source two times in one tweet. What to do?

Look at this tweet: \texttt{Ich hab \"Ubrigens ne 1 in Englisch.  Find
  ich gut ! (By the way, I got an A- in English. I Like it!)}  Here
the "Ich" in the first sentence needs not to be annotated. It is
sufficient to annotate just the "Ich" in the second sentence. Since
there is no anaphora present you will not need to mark an anaphoric
antecedent as well. \\ Here you should annotate both sentences as one
sentiment with "ne 1- in Englisch" as target, the "I" in the second
sentence as source and "gut" as emo-expression.
\end{itemize}

\subsection{target}

\subsection{emo-expression}
\begin{itemize}
\item What about English emo-expressions?

Although we annotate German tweets, it is not unusual to find English
emo-expressions. For example: \textit{DSDS ist fancy ! (DSDS is fancy
  !)}.  These expressions should be annotated too, if they are common
in German.

\item Should intejections be marked as emo-expressions?

Sollen Interjektionen (oh, aha, achso, OMG etc.) auch als
emo-expressions annotiert werden? Ich habe das n\"amlich nicht
gemacht, weil sie in so gut wie allen F\"allen nicht als wirklich
positiv bzw. negativ zu identifizieren waren. Wie
z.B. \textit{Lichgestaaaaaaaaaaalt !!!!!! in deren Schatten ich mich
  drehe uuuooooooh oooooh uuuuuooohhhh oooooohhhhh xD} oder
\textit{ooh du hasts gut , h\"att auch gern urlaub ...}. Ich finde
Interjektionen auch grenzwertig, da sie unter gewissen Umst\"anden
auch als intensifier dienen k\"onnen, z.B. oh wie gut, oh ja
etc. . Falls sie doch annotiert werden sollen, bitte Bescheid sagen
und/oder eine entsprechende Anmerkung in den Guidelines machen. W\"are
gut explizit zu schreiben ob sie annotiert werden sollen oder nicht.

\item Do we need sarcasm attribute for emo-expressions?
\end{itemize}

\subsection{intensifier}
\subsection{diminisher}
\subsection{negation}
\end{document}

%%%%%%%%%%%%%%%%%%%%%%%%%%%%%%%%%%%%%%%%%%%%%%%%%%%%%%%%%%%%%%%%%%

\section{Examples}
Please DO NOT mark as sentiments following cases:
\begin{itemize}
\item Statements describing some emotional states for which no
  target can be derived or found (e.g. \textit{Ich f\"uhle mich so
    traurig :(} - \textit{I am feeling so blue :(});
\item Statements describing some objective facts, even if possible
  consequences of these facts can be assumed to have negative
  influence on the author (e.g. \textit{Wenn ein Floh einen Menschen
    bei\ss{}t und ihn mit erbrochenem Blut infiziert, werden die
    Pestbakterien ins Gewebe \"ubertragen.} - \textit{When a flea
    bites a human and contaminates the wound with regurgitated
    blood, the plague carrying bacteria are passed into the
    tissue.});
\item Interrogative clauses in case they ask whether some polar
  opinion is true or not. (e.g. \textit{Findet die Mehrheit
    der Bev\"olkerung CDU toll?} - \textit{Does the majority
    of population consider CDU great?});
\end{itemize}

\subsection{negation}
\texttt{negation}s are lexical or syntactic elements that reverse the
primary polar meaning of emo-expressions to the opposite so that the
overall polarity of the whole sentiment is different to the polarity
of emo-expressions belonging to it. A typical example of negation is
\textit{nicht} in sentences like \textit{Ein guter Schritt war diese
  Entscheidung \textbf{nicht}.} (\textit{This decision was
  \textbf{not} a good move.}).

\vspace{0.5cm}
You SHOULD only mark negative elements that have an impact on the
sentiment polarity. Negating elements having no such impact
shold not be marked. Negation elements are usually represented by:
\begin{itemize}
\item Negation particle \textit{nicht} (\textit{not}),
  e.g. \textit{Klug ist dieser Hund sicherlich \textbf{nicht}}
  (\textit{This dog is certainly \textbf{not} clever});

\item Negative article \textit{kein} e.g. \textit{Er war
  \textbf{kein} Vorbild f\"ur seine Kinder} (\textit{He was
  \textbf{not} a good role model for his children});

\item Indefinite pronouns like \textit{niemand}, \textit{keiner}
  etc., e.g. \textit{\textbf{Niemand} hielt ihn f\"ur einen
    ehrlichen Menschen.} (\textit{\textbf{Nobody} considered him an
    honest man});

\item Any lexical or idiomatic unit in case they turn the sentiment
  polarity to the opposite, e.g. \textit{Ich \textbf{zweifle}, dass
    das neue iPhone ein besseres Display hat.} (\textit{I
    \textbf{doubt} the new iPhone has a better display}).
\end{itemize}

DO NOT mark elements as negations that have no effect on the sentiment
polarity. For example, in the sentence \textit{Ich mag Leute, die nicht
  nur an sich selbst denken.} (\textit{I like people who not only care
  about themselves.}) \textit{nicht} (\textit{not}) should \underline{not} be
marked as negation since it does not change the positive polarity
expressed by \textit{m\"ogen} (\textit{like}).
\vspace{0.5cm}

Negations only have one possible attribute, namely
\textit{sentiment-ref} that is a directed edge pointing from a negation
to the sentiment it belongs to. You should only draw this edge in cases multiple sentiment 
relations overlap and it is not obvious to which of these sentiments the negation belongs to,
i.e. in cases the negation is used for both sentiment spans.

%%%%%%%%%%%%%%%%%%%%%%%%%%%%%%%%%%%%%%%%%%%%%%%%%%%%%%%%%%%%%%%%%%%%%%%%%%%%%%%%%%

\subsection{intensifier}
\texttt{intensifier}s are elements that increase the polar
meaning of emotional expressions. Intensifiers are usually expressed
by adjectives or adverbs like \textit{sehr} (\textit{very}),
\textit{ziemlich} (\textit{rather}) and the like. In intensifier-chains 
\textit{(e.g. sehr, sehr gut)} please annotate each intensifier separately.
As mentioned above you should annotate every single emo-expression you can 
find regardless of a corresponding sentiment span. 
Please do not forget to annotate the corresponding intensifier if there is one. \newline

Intensifiers have the following attributes and values: \newline

\begin{tabular}{|l|c|p{\clmnwidth}|}\hline
  \multirow{2}{*}{degree} & \textit{1 (default)} & intensifier 
  slightly increases polarity of emotional
  expression, e.g. \textit{ziemlich},
  \textit{recht} etc.\\\cline{2-3}

  & \textit{2} & intensifier strongly increases polarity  of
  emotional expression, e.g. \textit{sehr}, \textit{super},
  \textit{stark} etc.\\\hline

  %%%%%%

  sentiment-ref & \textit{$->$\newline(directed edge)} & Directed
  edge pointing from the intensifier to the sentiment it belongs
  to. By analogy to negations, you should only draw this edge in cases
  multiple sentiment relations overlap and it is not obvious
  to which of these sentiments the intensifier belongs to,
  i.e. in cases when the intensifier is used for both
  sentiment spans.\\\hline
\end{tabular}

%%%%%%%%%%%%%%%%%%%%%%%%%%%%%%%%%%%%%%%%%%%%%%%%%%%%%%%%%%%%%%%%%%

\subsection{diminisher}
\texttt{diminisher}s are the counterpart to \texttt{intensifier}s.
These are elements that decrease the polar meaning of emotional
expressions.  Like intensifiers diminishers are usually expressed by
adjectives or adverbs. Typical examples of such adverbs are
\textit{wenig}, \textit{kaum}, \textit{ein bisschen} etc..

Diminishers have the same attributes as intensifiers. The only
difference are the values for the attribute. Instead of positive
values you have negative ones.

Here is a table of diminisher's attributes:

\begin{tabular}{|l|c|p{\clmnwidth}|}\hline

  \multirow{2}{*}{degree} & \textit{weak (default)} & diminisher
  slightly decreases polarity of emotional expression,
  e.g. \textit{wenig}, \textit{bisschen} etc.\\\cline{2-3}

  & \textit{strong} & diminisher strongly decreases polarity of
  emotional expression, e.g. \textit{kaum} etc.\\\hline


  sentiment-ref & \textit{$->$\newline(directed edge)} & Directed
  edge pointing from the diminisher to the sentiment it belongs
  to. You should only draw this edge in cases
  multiple sentiment relations overlap and it is not obvious
  to which of these sentiments the diminisher belongs to,
  i.e. in cases when the diminisher is used for both
  sentiment spans.\\\hline

\end{tabular}

%%%%%%%%%%%%%%%%%%%%%%%%%%%%%%%%%%%%%%%%%%%%%%%%%%%%%%%%%%%%%%%%%%%%%%%%%%%%

\section{MMAX Techniques}
In this section the most important techniques for annotating our corpus in MMAX will be described.

\subsection{Turn on Auto-Save and Auto-apply}
After setting up MMAX load a .mmax file as described in the beginning
of this document. You will be asked to validate the
annotations. Please always do that.  Before you start the annotation
it is recommended to turn on "Auto-Save" and "Auto-apply". Go to
\texttt{File -> Auto-Save -> Every} and select a time value.  Then
change to the window for the attributes (the window in the upper left
corner of your screen) and go to \texttt{Settings} and check the
\texttt{Auto-apply} box.  Your choice will be confirmed by a red
"Auto-apply is ON" at the bottom of the window. \newline

\subsection{Annotate Markables and Attributes}
The window in the middle of your screen now should show the tweets. To annotate a markable choose a word, phrase, clause or sentence by clicking the left mouse-button and mark the respective span while holding the mouse-button. After releasing the left mouse-button a menu will pop up showing the list of markables. Left-click on the corresponding markable and the span will turn to the respective color of the markable. Then change to the attribute window in the upper left corner, click on the tab for the markable you just annotated and set the value(s) for the attributes. (In case you did not turn on "Auto-apply" press "Apply".)

\subsection{Deleting Markables}
Just do a \underline{right-click} on a colored markable and left-click \texttt{Delete this markable}.

\subsection{Annoating Discontinuous Markables}
Sometimes you will find discontinuous emo-expressions. The most prominent example are German particle verbs (e.g. weh tun). To annotate them as one single element first annotate one element (it will turn to the respective markable color) as usual, left-click to highlight it and then mark the other element. A menu will pop up saying "Add to this markable". Left-click on that.

\subsection{Annotating anaphref}
Imaginge you have a sentence like: \textit{Peter ist schlau und gut aussehen tut er auch noch. (Peter is clever and he looks good, too.)}. Here you should annotate \textit{er} as anaphref to \textit{Peter}. To do so first annotate \textit{Peter} and \textit{er} as single targets. Then left-click on \textit{Peter} to highlight the markable and then do a \underline{right-click} on \textit{er}. A menu will pop up saying \texttt{"Mark as anaphric antecendent of target"}. Left-click on that.

\subsection{Annotating sentiment\_ref}
The same technique as above applies. First mark a sentiment span as usual. Second annotate the other markable (whatever it may be) as usual. Left-click on the first markable to highlight it then \underline{right-click} on the second markable and in the pop-up menu left-click on "Mark sentiment which this span belongs to".

%%%%%%%%%%%%%%%%%%%%%%%%%%%%%%%%%%%%%%%%%%%%%%%%%%%%%%%%%%%%%%%%%%%%%%%%%%%%
\section{FAQ}

\underline{Things you SHOULD mark as sentiments:}
\begin{itemize}
  \item Phrases, clauses, sentences and statements expressing a polar opinion
    (e.g. \textit{Der neue Film \"uber Superman war knorke!} -
    \textit{The new superman movie was fantastic});
  \item Interrogative clauses containing a polar opinion
    and do not call this opinion into question but rather ask for its
    reasons or other related aspects (e.g. \textit{Warum findet die
      Mehrheit der Bev\"olkerung CDU so toll?} - \textit{Why does the
      majority of the population consider CDU great?});
  \item Exclamations expressing support or disapproval of
    something (e.g. \textit{Der Sommer ist da. Super!} -
    \textit{Summer has come. Great!}).
  \item Comparisons expressing some preference of one subject/object over another 
      (e.g. \textit{Canon EOS 550d macht bessere Aufnahmen als 600d} - \textit{Canon EOS 550d takes
      better pictures than 600d}).
\end{itemize}


\underline{Things you SHOULD NOT mark as sentiments:}
\begin{itemize}
  \item Statements that do not have an annotatable target:
      (e.g. \textit{F\"uhle mich so traurig :(} - \textit{Feeling so blue :(});
  \item Statements describing some objective facts, even if possible
    consequences of these facts can be assumed to have negative
    influence on the author (e.g. \textit{Wenn ein Floh einen Menschen
      bei\ss{}t und ihn mit erbrochenem Blut infiziert, werden die
      Pestbakterien ins Gewebe \"ubertragen.} - \textit{When a flea
      bites a human and contaminates the wound with regurgitated
      blood, the plague carrying bacteria are passed into the
      tissue.});
  \item Interrogative clauses in case they ask whether some polar
    opinion is true or not. (e.g. \textit{Findet die Mehrheit
      der Bev\"olkerung CDU toll?} - \textit{Does the majority
      of the population consider CDU great?});
\end{itemize}

\underline{Things you SHOULD mark as sources:}
\begin{itemize}
  \item Original author(s) of polar opinions, i.e. persons,
    groups or officials who express sympathy or antipathy for
    something or somebody e.g. \textit{\textbf{Michael} meinte, das w\"are die
      beste L\"osung.}  (\textit{\textbf{Michael} thought, it would be the
      best solution.});
\end{itemize}


\underline{Things you SHOULD NOT mark as sources:}
\begin{itemize}
  \item Persons how are neutrally citing someone else's opinion,
    e.g. in the sentence \textit{Nach Tatjanas Worten war Michael sehr
      dar\"uber ver\"argert.} (\textit{According to Tatjana, Michael
      was very angry about that.}) you should only mark
    \textit{Michael} as source and not \textit{Tatjana}. Please note, that in case an author
    supports or contradicts someone else's opinion, two sentiment
    relations should be made - one with the original author as source
    and one with the supporter/opponent of the opinion as the other
    source.\footnote{In that case sentiments will overlap, and
      you also should mark these sources with \textit{sentiment-ref}-tag.}
\end{itemize}

\underline{Things you SHOULD mark as emo-expressions:}
\begin{itemize}
  \item Adjectives and adverbs bearing polar attitudes \textit{Peter
    hatte \textbf{bessere} Noten in der Schule als sein Bruder}
    (\textit{Peter had \textbf{better} grades at school than his
    brother.});

  \item Verbs expressing attitude of a speaker to target,
    e.g. \textit{Mir \textbf{gefiel} die neue House-Staffel}
    (\textit{I \textbf{liked} the new House series});

  \item Idiomatic expression including support verbs
    e.g. \textit{\textbf{Zum Teufel} soll die neue Regierung
      \textbf{gehen}} (\textit{The new government should \textbf{go to
        hell}}).

  \item Smileys in case they really express an emotional attitude and
    are not used for politeness or without any particular meaning
    e.g. \textit{Gleich in Braunschweig mit Kameraden treffen
        \textbf{:)}} (\textit{Will soon meet friends in Braunschweig
        \textbf{:)}}).
\end{itemize}
%%%%%%%%%%%%%%%%%%%%%%%%%%%%%%%%%%%%%%%%%%%%%%%%%%%%%%%%%%%%%%%%%%%%%%%%
