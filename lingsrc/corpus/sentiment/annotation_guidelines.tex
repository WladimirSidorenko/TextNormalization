\documentclass[11pt,a4paper]{article}

%%%%%%%%%%%%%%%%%%%%%%%%%%%%%%%%%%%%%%%%%%%%%%%%%%%%%%%%%%%%%%%%%%
%% Libraries
\usepackage[driver=pdftex,vmargin=2cm,hmargin=2cm]{geometry}
\usepackage[usenames,dvipsnames]{xcolor}
\usepackage{amsmath}
\usepackage{array}
\usepackage{booktabs}
\usepackage{color}
\usepackage{framed}
\usepackage[colorlinks=true]{hyperref}
\usepackage{multirow}
\usepackage{natbib}
\usepackage{paralist}
\usepackage{url}
\usepackage{tikz}
\usepackage{xargs}
\usepackage{eurosym}

\hypersetup{
  colorlinks,
  citecolor=Violet,
  linkcolor=Red,
  urlcolor=Blue}

%%%%%%%%%%%%%%%%%%%%%%%%%%%%%%%%%%%%%%%%%%%%%%%%%%%%%%%%%%%%%%%%%%
%% Commands
\definecolor{dodgerblue4}{RGB}{16,78,139}
\definecolor{orange3}{RGB}{205,133,0}
\definecolor{DarkSlateBlue}{RGB}{72,61,139}

\newlength{\cwidth} \setlength{\cwidth}{0.18\textwidth}
\newenvironment{example}{\begin{center}\begin{exe}\ex}{\end{exe}\end{center}}
\newcommand{\xmltag}[1]{\textcolor{black}{{\small$<$#1$>$}}}
\newcommand{\sentiment}[2][negative]{$<$sentiment
  sentiment\_polarity="#1"$>$\textcolor{dodgerblue4}{#2}$<$/sentiment$>$}
\newcommand{\source}[1]{\xmltag{source}\textcolor{orange3}{#1}\xmltag{/source}}
\newcommand{\target}[1]{\xmltag{target}\textbf{#1}\xmltag{/target}}
\newcommand{\negation}[1]{\xmltag{negation}\textcolor{red}{#1}\xmltag{/negation}}
\newcommand{\intensifier}[2][1]{\xmltag{intensifier
    intensity="#1"}\textcolor{DarkSlateBlue}{#2}\xmltag{/intensifier}}
\newcommandx{\emoexpression}[3][1=negative, 2=1]{
  \xmltag{emo-expression polarity="#1" intensity="#2"}
  \textcolor{green}{#3}\xmltag{/emo-expression}}

\renewenvironment{example}{\begin{center}\itshape}{\upshape\end{center}}

\newlength\clmnwidth
\setlength{\clmnwidth}{0.55\textwidth}

%%%%%%%%%%%%%%%%%%%%%%%%%%%%%%%%%%%%%%%%%%%%%%%%%%%%%%%%%%%%%%%%%%
%% Title
\author{Wladimir Sidorenko}
\date{\today}
\title{Guidelines for the Annotation of the Sentiment Corpus}

%%%%%%%%%%%%%%%%%%%%%%%%%%%%%%%%%%%%%%%%%%%%%%%%%%%%%%%%%%%%%%%%%%
%% Main
\begin{document}
\maketitle{}
\section{Introduction}
Welcome. In this assignment your task is to annotate sentiments in a
corpus of German Twitter messages.

\subsection{Annotation Tool}

For annotation, you will use \texttt{MMAX2} -- a freely available
annotation tool that can be downloaded under the following link:

\url{http://sourceforge.net/projects/mmax2/files/mmax2/mmax2_1.13.003/MMAX2_1.13.003b.zip/download}

After you have downloaded and unpacked the archive, change to the
newly created directory \texttt{1.13.003/MMAX2} in your shell and
execute the following commands:

\texttt{> chmod u+x ./mmax2.sh}

\texttt{> nohup ./mmax2.sh \&} \newline
{\setlength{\parindent}{0pt}An \texttt{MMAX} window should then appear
  on your screen.}

If you have never used \texttt{MMAX2} before, please read the document
\texttt{mmax2quickstart.pdf} first, which you can find in the
subdirectory \texttt{MMAX2/Docs}.

\subsection{Corpus Files}

You should also have received a copy of corpus files as a tar-gzipped
archive.  Please unpack this archive using the command:

\texttt{> tar -xzf twitter-sentiment.tgz}

{\setlength{\parindent}{0pt} After that, a directory called
  \texttt{mmax-prj} will appear in your current folder.  Change to
  your \texttt{MMAX2} window and click on the menu \texttt{File ->
    Load}.  Select the path to the unpacked \texttt{mmax-prj} folder
  in the displayed popup menu\footnote{Make sure that the path does
    not contain any white spaces.  Otherwise, \texttt{MMAX2} will fail
    to load the project file with the error message
    \emph{java.netMalformedURLException.no protocol: words.dtd}} and
  select one of the \texttt{*.mmax} files there.  Now, source data will
  be loaded into \texttt{MMAX} program.}

%%%%%%%%%%%%%%%%%%%%%%%%%%%%%%%%%%%%%%%%%%%%%%%%%%%%%%%%%%%%%%%%%%

\section{Tags and Attributes}
Here is a short list of all tags along with their possible attributes
that you will use in this annotation task:
\begin{enumerate}
\item \textit{sentiment}-tag with attributes:
  \begin{enumerate}
  \item polarity,
  \item intensity,
  \item sarcasm,
  \item sentiment-ref;
  \end{enumerate}
\item \textit{source}-tag with attributes:
  \begin{enumerate}
  \item anaph-ref,
  \item sentiment-ref;
  \end{enumerate}
\item \textit{target}-tag with attributes:
  \begin{enumerate}
  \item anaph-ref,
  \item sentiment-ref;
  \end{enumerate}
\item \textit{emo-expression}-tag with attributes:
  \begin{enumerate}
  \item polarity,
  \item sarcasm,
  \item intensity,
  \item emo-expression-ref,
  \item sentiment-ref;
  \end{enumerate}
\item \textit{intensifier}-tag with attributes:
  \begin{enumerate}
  \item degree,
  \item sentiment-ref;
  \end{enumerate}
\item \textit{diminisher}-tag with attributes:
  \begin{enumerate}
  \item degree,
  \item sentiment-ref;
  \end{enumerate}
\item and the \textit{negation}-tag.
\end{enumerate}
A more detailed description of the meaning of these tags and
attributes is given in the next subsections.

\subsection{sentiment}
\texttt{sentiment}s are our most important markables.  With these tags
you should mark statements which express some polar assessing opinions
about particular subjects or events.  Opinions which are polar but
which are not assessing anything like, for example, greetings or vague
emotional statements, should not be marked as sentiments.

Sentiment tags can include:
\begin{itemize}
\item single noun phrases, possibly with their prepositional
  attributes, e.g. ``\textit{Auf dem Tresor lag \xmltag{sentiment}ein
    langweiliges Buch\xmltag{/sentiment}}.'';
\item clauses, e.g. ``\textit{\xmltag{sentiment}Ich hasse B\"ucher
  ohne Inhaltsangabe.\xmltag{/sentiment}}'';
\item multiple sentences in cases when these sentences jointly form a
  sentiment relation, e.g. ``\textit{\xmltag{sentiment}Sie denken,
    reden, riechen, lieben, schmecken, ficken Plastik. Sie haben das
    so gelernt in der
    Plastik-Werbewelt.\xmltag{/sentiment}}''\footnote{In this example,
    \textit{Plastik-Werbewelt} is being assessed.  It is, however,
    impossible to state that a sentiment is expressed by looking at
    only one of these two sentences.}.
\end{itemize}

Sentiment tags can have the following attributes with the following
possible values:\\
\begin{tabular}[t]{|l|c|p{\clmnwidth}|}\hline
  Attribute & Value & Value's Meaning\\\hline
  %%%%%%%%%%%%%%%%%%%

  \multirow{3}{*}{polarity} & \textit{positive} & this sentiment
  expresses positive attitude to its respective target\\\cline{2-3}

  & \textit{negative\newline(default)} & sentiment
  expresses a negative attitude to its respective target\\\cline{2-3}

  & \textit{comparison} & sentiment expresses a comparison of two
  targets with a preference for one of them\\\hline

  %%%%%%%%%%%%%%%%%%%
  \multirow{3}{*}{intensity} & \textit{weak} & stylistically weak
  expression tending to be a neutral phrase/clause/sentence\\\cline{2-3}

  & \textit{medium\newline(default)} & middle stylistic
  expressivity\\\cline{2-3}

  & \textit{strong} & stylistically very expressive statement\\\hline
  %%%%%%%%%%%%%%%%%%%

  \multirow{2}{*}{sarcasm} & \textit{true} & this polar attitude is
  derisive, i.e. its actual polarity is the opposite of its apparent
  form (that means that an apparent praise is in fact meant as rebuke
  and vice versa - but this actual sense can only be derived on the
  basis of outside context or common knowledge)\\\cline{2-3}

  & \textit{false\newline(default)} & no sarcasm present - polar
  attitude has literal meaning\\\hline

  %%%%%%%%%%%%%%%%%%%
  sentiment-ref & \textit{$->$\newline(directed edge)} & in case a
  single sentiment is not a contiguous span of text but is rather split into
  parts, you can draw an edge with this label from additional parts of
  a sentiment to its main part\\\hline
\end{tabular}

\subsection{source}
The \texttt{source} tag is used to mark the immediate author(s) or
experiencer(s) of a sentiment expression.  These are usually speakers
or writers of an assessment or persons whose assessing attitude is
being cited.  In cases when a source is not explicitly present in a
message, it is implicitly assumed to be the author who posted the
tweet.

An example of a \texttt{source} expression is the pronoun ``Sie''
(``she'') in the following sentence:
\begin{example}
  \textit{\source{Sie} mag die neue Farbe
    nicht}\\ (\textit{\textbf{She} doesn't like the new color})
\end{example}
Sources are usually expressed by noun phrases.  In cases when a
sentiment contains multiple sources joined by coordinative
conjunctions, you should mark each conjoined source with a separate
tag (see the example below):
\begin{example}
  \textit{\source{Ihr} und \source{ihrer Mutter} gef\"allt die neue
    Farbe nicht}\\ (\textit{\textbf{She} and \textbf{her mother} do not
    like the new color})
\end{example}

The attributes of the \texttt{source} tag with their possible values
and meanings are listed in the table below:\\
\begin{tabular}{|l|c|p{\clmnwidth}|}\hline
  Attribute & Value & Value's Meaning\\\hline sentiment-ref &
  \textit{$->$\newline(directed edge)} & directed edge pointing from
  source element to its respective sentiment span.  You only need to
  draw this edge if two different sentiment relations are overlapping
  on the same span and we need to disambiguate to which sentiment
  relation given source belongs.  Another case when you should draw
  this edge is when source is expressed outside the sentiment
  span.\\\hline

  anaph-ref & \textit{$->$\newline(directed edge)} & directed edge
  pointing from a source element expressed by a pronoun or pronominal
  adverb to its respective antecedent\\\hline
\end{tabular}
\vspace{0.5cm}

%%%%%%%%%%%%%%%%%%%%%%%%%%%%%%%%%%%%%%%%%%%%%%%%%%%%%%%%%%%%%%%%%%%%%%%%%%%%%%%%%%%%%%%%%%
\subsection{target}
With \texttt{target}-tags you should mark objects or events that are
assessed by a sentiment expression.  One sentiment relation should
always have at least one target.  In cases when multiple targets are
conjoined in a single expression, you should mark each of these
targets separately in the same way as it is done for sources.

An example of a sentiment target is given in the sentence below:
\begin{example}
  \textit{Mein Bruder ist nicht gerade begeistert von \target{Call of
      Duty}.}\\ (\textit{My brother is not exactly impressed by
    \textbf{Call of Duty}.})
\end{example}

In cases when a sentiment is represented by a comparison, you should
mark each of the compared objects with a separate target tag.
Additionally, for object which is being preferred in the comparison,
you should set the \textit{preferred} attribute of the tag to
\textit{true}:
\begin{example}
  \textit{Ich mag $<$target
    preferred=``true''$>$\textbf{Domino-Eis}$<$/target$>$ mehr als
    \target{Magnum}.}\\ (\textit{I like \textbf{Domino ice cream} more
    than \textbf{Magnum}.})
\end{example}

The attributes and attribute values of the \texttt{target} tags fully
correspond to the respective attributes of the \texttt{source} tag.
The only additional attribute for targets is the aforementioned
\textit{preferred} attribute.

\subsection{emo-expression}
With the \texttt{emo-expression} tag, you should mark words or phrases
which have some polar connotation in their lexical meaning in given
context.  Typical examples of emotional expressions are adjectives and
adverbs (e.g. ``gut'', ``sch\"on'', ``traurig'' etc.), nouns
(e.g. ``Held'', ``Vorbild'', ``Schurke'' etc.), verbs
(e.g. ``lieben'', ``hassen'', ``beschimpfen'' etc.), idiomatic
expressions (e.g. ``auf die Nerven gehen'' etc.), smileys,
interjections etc.

Please note that, depending on the context in which it is used, one
and the same word can have different lexical polarities or also loose
its polar meaning completely in certain settings.  For example, the
word ``Edelstein'' (``jewel'') in the following two sentences only
becomes a polar term and, correspondingly, an \texttt{emo-expression}
in the second example when its meaning is metaphoric:
\begin{example}
  \textit{Koh-i-Noor ist das teuerste Juwel heutzutage.}\\
  \textit{Koh-i-Noor is the most expensive jewel nowadays.}
\end{example}
\begin{example}
  \textit{Dieser Wein ist ein wahres \emoexpression{Juwel} in meiner Kollektion.}\\
  \textit{This wine is a true \textbf{jewel} in my collection.}
\end{example}

You should only mark a term as an emo-expression, if the meaning of
this term in given context is polar.  The polarity attribute of the
tag should reflect the \textit{lexical} or also called \textit{prior}
polarity of the marked term disregarding possible negations in the
context.  It means that in a sentence like ``\textit{Das war keine
  gute Idee}'' (``\textit{It was not a good idea}''), the polarity of
the emotional expression ``\textit{gute}'' (``\textit{good}'') still
has to be positive even though the negative article ``\textit{keine}''
turns its polarity to the opposite.

Furthermore, you should also mark emotional expressions in cases when
these expressions do not evoke any sentiment.  If an emo-expression is
a part of a sentiment however, its prior polarity should be determined
from the perspective of sentiment's target.  It means, that in a
sentence like ``\textit{Ich vermisse meinen Freund.}''  (``\textit{I
  am missing my boy friend.}''), the polarity of the term
``\textit{vermisse}'' (``\textit{missing}'') should be positive
because it expresses a positive attitude to the target
``\textit{Freund}'' (``\textit{boy friend}'') whose absence is
perceived as discomfort.

A complete list of possible attributes for emo-expressions is listed
in the table below:

\begin{tabular}{|l|c|p{\clmnwidth}|}\hline
  Attribute & Value & Value's Meaning\\\hline
  %%%%%%%%%%%%%%%%%%%

  \multirow{2}{*}{polarity} & \textit{positive} & this emotional
  expression has positive polar meaning\\\cline{2-3}

  & \textit{negative\newline(default)} & this emotional expression has
  negative polar meaning\\\hline

  %%%%%%%%%%%%%%%%%%%

  \multirow{3}{*}{intensity} & \textit{weak} & stylistically weakly
  marked emotional expression\\\cline{2-3}

  & \textit{medium\newline(default)} & middle stylistic
  expressivity\\\cline{2-3}

  & \textit{strong} & stylistically very expressive emotional
  expression\\\hline

  %%%%%%%%%%%%%%%%%%%

  emo-expression-ref & \textit{$->$\newline(directed edge)} & in cases
  when a single emotional expression is split into parts this edge
  should point from the separated parts to the main part of the
  emo-expression\\\hline
\end{tabular}

\subsection{intensifier}
Intensifiers are elements that increase the polar meaning of emotional
expressions.  Intensifiers are usually expressed by adverbs or
adjectives like \textit{sehr} (\textit{very}), \textit{ziemlich}
(\textit{rather}) etc.  Other ways of expressing intensifiers are
however still possible.

Intensifier elements can have following attributes with following
values:

\begin{tabular}{|l|c|p{\clmnwidth}|}\hline
  \multirow{2}{*}{degree} & \textit{medium (default)} & intensifier
  moderately increases the polar sense of emotional expressions,
  e.g. \textit{ziemlich}, \textit{recht} etc.\\\cline{2-3}

  & \textit{strong} & intensifier strongly increases polar sense of
  emotional expression, e.g. \textit{sehr}, \textit{super},
  \textit{stark} etc.\\\hline

  %%%%%%

  sentiment-ref\footnote{TODO: do we need it?} &
  \textit{$->$\newline(directed edge)} & a directed edge pointing from
  the intensifier to the sentiment it belongs to. By analogy to
  negations, you should only draw this edge in cases multiple
  sentiment relations overlap each other and it's not obvious to which
  of these sentiments the given intensifier belongs to, i.e. in cases
  when the given intensifier is included in both sentiment
  spans\\\hline
\end{tabular}
\vspace{0.5cm}

\subsection{diminisher}
Diminishers are elements that decrease the polar lexical meaning of an
emotional expression.  Like intensifiers, diminishers are usually
expressed by adjectives or adverbs (e.g. ``\textit{wenig}'',
``\textit{kaum}'', ``\textit{ein bisschen}''), however other means of
expressing diminishing notions are still possible.

Diminishers have the same set of attributes as intensifiers with the
only difference in the interpretation of the attribute
\textit{degree}:

\begin{tabular}{|l|c|p{\clmnwidth}|}\hline
  \multirow{2}{*}{degree} & \textit{medium (default)} & degree by which
  this diminisher decreases polar sense of respective emotional
  expression (1 means slight decrease), e.g. \textit{wenig},
  \textit{bisschen} etc.\\\cline{2-3}

  & \textit{strong} & this diminisher strongly decreases polar sense
  of emotional expression, e.g. \textit{kaum} etc.\\\hline

  sentiment-ref\footnote{TODO: Do we need it???} &
  \textit{$->$\newline(directed edge)} & a directed edge pointing from
  diminisher to sentiment it pertains to. You should only draw this
  edge in cases when multiple sentiment relations overlap with each
  other and it's not obvious to which of these sentiments given
  diminisher belongs to, i.e. in cases when given diminisher is
  included in both sentiment spans\\\hline
\end{tabular}

\subsection{negation}
Negations are elements which change the polarity of an emotional
expression to the opposite.  Negations can, for example, be
represented by the negation particle ``\textit{nicht}'', the negative
article ``\textit{kein}'' or any other means.

Negations are closely related to the diminishers in that they reduce
the prior polar sense of an emotional expression.  The difference
between these two elements is that negations reduce this polar sense
completely, turning it to the opposite (e.g. ``\textit{nicht
  verst\"andlich}'' -- ``\textit{not comprehensible}''), whereas
diminishers still allow some part of the prior polar meaning to remain
(compare ``\textit{weniger verst\"andlich}'' -- ``\textit{less
  comprehensible}'').

The only attribute that a negation element can take on is
\textit{sentiment-ref} described below:\footnote{TODO: Do we need it?}

\begin{tabular}{|l|c|p{\clmnwidth}|}\hline
  sentiment-ref\footnote{TODO: Do we need it???} &
  \textit{$->$\newline(directed edge)} & a directed edge pointing from
  negation to the sentiment it pertains to. You should only draw this
  edge in cases when multiple sentiment relations overlap each other
  and it is not obvious to which of these sentiments given negation
  belongs to, i.e. in cases when given negation is included in both
  sentiment spans\\\hline
\end{tabular}

Please note that negations, diminishers, and intensifiers should only
be marked as such if they have an immediate impact on the intensity or
polarity of a sentiment.  This also means that none of these elements
should be marked if no sentiment relation is present in a tweet.

\section{FAQ}
This section provides some examples and more elaborate descriptions of
difficult and controversial annotation cases.%%   It is subdivided into
%% subsection depending on which annotation element may cause the
%% difficulties:

\begin{itemize}
\item F: Smilies oder andere emo-expressions immer als solche annotieren,
  egal ob es ein zugeh\"origes sentiment, target, source gibt? [j/n]

A: Ja. Man sollte keine Sentiment-Spanne markieren, wenn kein Target
in Sicht ist, aber Smilies und emo-expressions brauchen wir immer, um
sp\"ater mal zu gucken, wie verschiedene Sentiment-Lexika diese
Entit\"aten abdecken.

\item F: Emo-expressions k\"onnen sowohl Adjektive und Verben als auch Substantive sein? [j/n]

A: Substantive auch. Beliebige Wortarten generell, wenn du meinst,
dass sie eine Polarit\"at ausdr\"ucken.

\item F: In den Guidelines steht ``\texttt{emo-expression}s are
  lexical elements that bear some polarity meaning on their own.'' Wie
  sieht es mit Kollokationen, wie z.B.: ``knapp bei Kasse'', aus?
  Sowas auch als emo-expression annotieren?

A: Ja. Feste Redewendungen auch, und zwar m\"oglicht als ein Markable,
damit wir sehen k\"onnten, dass diese eine Einheit sind.

\item F: Soll ich intensifier-Ketten wie in z.B. "Du bist die 'aller
  aller aller' Beste !" als einen ganzen intensifier mit h\"ochster
  Intensit\"at, oder als 3 einzelne mit h\"ochster Intensit\"at annotieren?
  Ich habe es als einzelne annotiert. Richtig so? [j/n]

A: Richtig. Alles einzeln.

\item F: Probleme habe ich solche Tweets zu annotieren, z.B.: "Sie
  denken , reden , riechen , lieben , schmecken , ficken Plastik . Sie
  haben das so gelernt in der Plastik - Werbewelt ." Hier ist
  eindeutig eine Abneigung des Autors gegen\"uber Werbetreibenden heraus
  zu lesen. Das sarcasm sentiment l\"asst sich hier auch nicht so
  richtig anwenden, da (Zitat aus den Guidelines): "this polar
  attitutde is meant as a derision, i.e. its actual polarity is the
  opposite of its apparent form (that means that an apparent praise is
  in fact meant as rebuke and vice versa - but this actual sense can
  only be derived on the basis of outside context or )" dies auch
  nicht so richtig auf den Tweet zutrifft. Wie soll ich mit solchen
  Tweets verfahren? Unannotiert lassen? [j/n]

A: Ich w\"urde beide S\"atze als ein Sentiment markieren, mit
Plastik-Werbewelt als Target und negativer Polarit\"at.

\item F: Eine Frage zu negation in Kombination zu sentiment. Wenn ich
  einen Tweet wie z.B.: " Es geht auch ohne @ArminWolf - ich finde
  nicht ;-)" habe, dann dr\"uckt dieser ja eine positive Einstellung
  gegen\"uber Armin Wolf aus. Das "nicht" ist eine Negation. Die
  Grundstimmung des Tweets w\"urde ich aber als positiv ansehen. Wird
  dann der Tweet durch die Negation als insgesamt negativ berechnet?
  Weil eigtl. ist es ja Negativ (erster Teilsatz) mit nochmal Negativ
  (zweiter Teilsatz) wieder positiv. Berechnest Du einzelne
  Teils\"atze oder wird ein Tweet der positiv als Grundstimmung hat
  durch eine einzelne Negation im Tweet dann auch negativ?

A: Alles richtig. Die Stimmung des Sentiments ist positiv. Du setzt
das auch so im polarity-Attribut des Sentiments. Diese Polarit\"at,
die du setzt, betrachten wir auch als die endg\"ultige des gesamten
Ausdrucks. Negations helfen uns dabei lediglich besser zu verstehen,
wie es zu dieser endg\"ultigen Polarit\"at gekommen ist. D.h. es
werden unsere St\"utzpunkte sein, um zu verstehen, warum ein Ausdruck,
der aus lauter W\"ortern mit an sich negativer Polarit\"at besteht,
pl\"otzlich eine positive Gesamtpolarit\"at hat.


\item F: Problematisch finde ich auch solche Tweets: "Ich hab Maria ja
  schon so vermisst und so , ne ?" dessen Overall-Sentiment eher als
  negativ zu bewerten ist im Grunde aber eine positive Einstellung
  ausdr\"uckt (hier dass der Poster die Maria gerne hat, was ja positiv
  ist und sie deshalb vermisst hat, was wiederum eher negativ
  ist). Also das Grundsentiment zwischen source und target ist
  positiv, der Tweet m\"usste aber v.a. aufgrund der negativen
  emo-expression "vermisst" als negativ gewertet werden.

A: In diesem Fall ist die Polarit\"at des Sentiments ``positiv'', da
wir das Target-orientiert machen. D.h. wir beurteilen stets danach,
wie die Einstellung gegen\"uber Target ist, und achten nicht darauf,
dass die Source eines Sentiments unter Umst\"anden selbst etwas
Anderes f\"uhlen mag. Wichtig f\"ur uns ist nur, was die Source vom
Target h\"alt.

\item F: Mehrere targets oder alles als ein target annotieren:
  "@JKR\_hb ich mag essen 2 oder die Farafelle in der GW2" soll ich
  "essen 2" und "die Farafelle" einzeln oder als ganzes target
  annotieren "essen 2 oder die Farafelle in der GW2"? Habs erstmal
  einzeln gemacht.

A: Alles einzeln.

\item F: Schwierig finde ich auch ohne Kontext Sarkasmus
  aufzusp\"uren. "@VictoriousJump man das nervt mich aber grad . :D"
  Dieser Tweet k\"onnte aufgrund des Smileys am Ende durchaus
  sarkastisch gemeint sein. Nur l\"asst sich das ohne vorangegangenen
  Kontext schwer einsch\"atzen. Habe diesen Tweet daher als nicht
  sarkastisch eingestuft.  Beschimpfungen habe ich auch nicht
  annotiert (analog zu Entschuldigungen; steht ja auch so in den
  Guidelines, dass man das nicht annotieren soll. Ich habe "insults"
  noch mit in die Guidelines aufgenommen).

A: Ob im Beispielsatz Sarkasmus vorliegt oder nicht, kann man hier
ohne weiteren Kontext leider nicht genau sagen.  Jeder Annotator soll
in diesem Fall seine eigene Entscheidung treffen, ausgehend von seinem
Sprachgef\"uhl.  Wir werden uns in diesem Fall auf die Entscheidung
der meisten Annotatoren festlegen m\"ussen.

Was Beschimpfungen und Lob angeht, so w\"urde ich beide als Sentiment
betrachten.  Letztendlich, dr\"ucken sie auch eine polare Menung aus.

\item F: Im Allgemeinen ist eine (meist sinnlose) \"Uberverwendung von
  Smileys sowohl in den general- als auch den politics-Tweets zu
  beobachten, die weder zum Inhalt/Sentiment beitragen noch sonst
  irgendeine Bedeutung haben. Ich glaube das ist schon eine Marotte
  vieler Twitter-User geworden einfach nach jedem Tweet einen (oder
  mehrere) Smileys zu setzen. Der einzige Sinn, den ich darin erkennen
  kann ist, dass die User die schriftliche Kommunikation menschlicher
  gestalten wollen, sprich Gesichtsausdr\"ucke und Gesten simulieren
  wollen. Siehe:

  \textit{Seufz ich muss noch mehr putzen}

  \textit{@jamievaerle ik weet niet wie je bent maar neuhh @Youarenotme2 was
  steht denn zur Option?}

A: Der erste Satz ist m.E. ein negatives Sentiment mit "putzen"
bzw. "mehr putzen" als Target, alles andere keine Sentiments.

\item F: Eine Frage h\"atte ich noch wie man Interrogativs\"atze
  annotieren soll. In den Guidelines steht unter "Things you SHOULD
  mark as sentiment": Interrogative clauses in case they express some
  polar opinion and don�t call this opinion into question but rather
  ask for its reasons or other related aspects (e.g. Warum findet die
  Mehrheit der Bev\"olkerung CDU so toll? - Why does the majority of
  population consider CDU great? ); Was soll ich da annotieren?
  ``toll'' w\"urde ich als emo-expression sehen, CDU als target; aber
  was ist das Overall-Sentiment? Positiv, negativ, sarkastisch? Ich
  kann hier keines von den dreien erkennen. Ich w\"urde $<$sentiment
  positiv, intensity=0$>$ findet die Mehrheit der Bev\"olkerung CDU so
  toll?$<$/sentiment$>$ annotieren, aber was ist mit dem ``Warum'' vom
  Satzanfang? Es stellt ja das Sentiment in Frage. Also
  Overall-Sentiment eher negativ?

A: In diesem Fall wird nicht bezweifelt, dass die Bev\"olkerung die
CDU toll findet - es wird lediglich nach den Gr\"unden daf\"ur
gefragt. Warum man nach diesen Gr\"unden fragt, kann ich z.B. nicht
eindeutig erkennen, ob man diese Zuneigung der Bev\"olkerung nicht
versteht oder einfach nur sich informieren m\"ochte. Etwas mehr als
das positive Sentiment w\"urde ich also nicht annotieren.

\item F: Dem Tweet aus politics1 ``Wo ist der \#Jubel von \#CDU \#CSU
  \& \#FDP \"uber den Tod der Mieterin nach \#Zwangsr\"aumung?'' habe
  ich als sarkastisch und mit target=''Tod der Mieterin nach
  \#Zwangsr\"aumung'' annotiert. Wei\ss aber nicht, ob das so richtig
  ist.

A: Mhm, aus meiner Sicht wird hier eher \"uber die CDU/CSU und FDP
gespottet, man w\"urde also annehmen, dass diese Parteien sich \"uber
den Tod der Mieterin freuen w\"urden, was nat\"urlich dem Image einer
Partei eher schaden w\"urde.  Ich w\"urde demzufolge den ganzen Satz
als negatives Sentiment mit den Parteien als Targets und dem Attribut
sarcasm gesetzt auf true annotieren.

\item F: Should we introduce a separate tag for emoticons?

A:

\item F: Should we remove the sentiment-ref attribute for all elements
  except sources and targets?

A:
\end{itemize}
\end{document}

\subsection{sentiment}
\begin{itemize}
\item I have difficulties determining the overall sentiment of a tweet.
Is there a common modus operandi?

Please keep in mind that sentiments should always be annotated
target-oriented. Look at this tweet: \textit{Ich hab Maria ja schon so
  vermisst und so , ne ? (I kind of did miss Maria, right?)}. Here you
might think the tweet has a negative sentiment, because the author
missed Maria which is not a good thing. But we are always only
concerned about the sentiment relation between the source and the
target! And here it is a positive sentiment, because the author missed
Maria because he likes her.

\item How to annotate sarcasm?

If the apparent form of a sentiment is positive (and therefore should
be annotated as a positive sentiment) but you can detect sarcasm
please annotate the polarity as negative and additionally set
\texttt{sarcasm} to \texttt{true} in the attribute window. This
applies to the sentiment tag as well as to emo-expressions.

\item Should subjunctives be annotated?

Yes. Have a look on the following tweet: \textit{W\"are toll wenn das
  der n\"achste " Call of Duty " - Teil sein w\"urde , liebe
  @GameStar. (Dear @gamestar, it would be great if this would be the
  sequel of "Call of Duty".)}. Although this is a subjunctive
sentiment and is not "reality" right now you should annotate this,
too.

\item Should multiple sentiment layers be annotated?

Yes. For example in the tweet \textit{72 j\"ahriger Leser regt sich
  bei mir \"uber YouTube und Gema auf :) (72 year old reader troubles
  over YouTube and Gema at me :) )} you can find two layers of
sentiments. The first layer is the fact that the reader troubles over
YouTube and Gema. The second layer is the fact that the author of the
tweet is amused about the first fact. So here you have to annotate two
different sentiments. The first is negative and has "YouTube" and
"Gema" as target (with "troubles over" as emo-expression). The second
is positive and has "72 jähriger Leser" as target and "mir" as
source. Do not forget to use the \texttt{sentiment\_ref} attribute to
annotate which sentiment refers to which target.
\end{itemize}

\subsection{source}
\begin{itemize}
\item I can find the same source two times in one tweet. What to do?

Look at this tweet: \texttt{Ich hab \"Ubrigens ne 1 in Englisch.  Find
  ich gut ! (By the way, I got an A- in English. I Like it!)}  Here
the "Ich" in the first sentence needs not to be annotated. It is
sufficient to annotate just the "Ich" in the second sentence. Since
there is no anaphora present you will not need to mark an anaphoric
antecedent as well. \\ Here you should annotate both sentences as one
sentiment with "ne 1- in Englisch" as target, the "I" in the second
sentence as source and "gut" as emo-expression.
\end{itemize}

\subsection{target}

\subsection{emo-expression}
\begin{itemize}
\item What about English emo-expressions?

Although we annotate German tweets, it is not unusual to find English
emo-expressions. For example: \textit{DSDS ist fancy ! (DSDS is fancy
  !)}.  These expressions should be annotated too, if they are common
in German.

\item Should intejections be marked as emo-expressions?

Sollen Interjektionen (oh, aha, achso, OMG etc.) auch als
emo-expressions annotiert werden? Ich habe das n\"amlich nicht
gemacht, weil sie in so gut wie allen F\"allen nicht als wirklich
positiv bzw. negativ zu identifizieren waren. Wie
z.B. \textit{Lichgestaaaaaaaaaaalt !!!!!! in deren Schatten ich mich
  drehe uuuooooooh oooooh uuuuuooohhhh oooooohhhhh xD} oder
\textit{ooh du hasts gut , h\"att auch gern urlaub ...}. Ich finde
Interjektionen auch grenzwertig, da sie unter gewissen Umst\"anden
auch als intensifier dienen k\"onnen, z.B. oh wie gut, oh ja
etc. . Falls sie doch annotiert werden sollen, bitte Bescheid sagen
und/oder eine entsprechende Anmerkung in den Guidelines machen. W\"are
gut explizit zu schreiben ob sie annotiert werden sollen oder nicht.

\item Do we need sarcasm attribute for emo-expressions?
\end{itemize}

\subsection{intensifier}
\subsection{diminisher}
\subsection{negation}
\end{document}

%%%%%%%%%%%%%%%%%%%%%%%%%%%%%%%%%%%%%%%%%%%%%%%%%%%%%%%%%%%%%%%%%%

\section{Examples}
Please DO NOT mark as sentiments following cases:
\begin{itemize}
\item Statements describing some emotional states for which no
  target can be derived or found (e.g. \textit{Ich f\"uhle mich so
    traurig :(} - \textit{I am feeling so blue :(});
\item Statements describing some objective facts, even if possible
  consequences of these facts can be assumed to have negative
  influence on the author (e.g. \textit{Wenn ein Floh einen Menschen
    bei\ss{}t und ihn mit erbrochenem Blut infiziert, werden die
    Pestbakterien ins Gewebe \"ubertragen.} - \textit{When a flea
    bites a human and contaminates the wound with regurgitated
    blood, the plague carrying bacteria are passed into the
    tissue.});
\item Interrogative clauses in case they ask whether some polar
  opinion is true or not. (e.g. \textit{Findet die Mehrheit
    der Bev\"olkerung CDU toll?} - \textit{Does the majority
    of population consider CDU great?});
\end{itemize}

\subsection{negation}
\texttt{negation}s are lexical or syntactic elements that reverse the
primary polar meaning of emo-expressions to the opposite so that the
overall polarity of the whole sentiment is different to the polarity
of emo-expressions belonging to it. A typical example of negation is
\textit{nicht} in sentences like \textit{Ein guter Schritt war diese
  Entscheidung \textbf{nicht}.} (\textit{This decision was
  \textbf{not} a good move.}).

\vspace{0.5cm}
You SHOULD only mark negative elements that have an impact on the
sentiment polarity. Negating elements having no such impact
shold not be marked. Negation elements are usually represented by:
\begin{itemize}
\item Negation particle \textit{nicht} (\textit{not}),
  e.g. \textit{Klug ist dieser Hund sicherlich \textbf{nicht}}
  (\textit{This dog is certainly \textbf{not} clever});

\item Negative article \textit{kein} e.g. \textit{Er war
  \textbf{kein} Vorbild f\"ur seine Kinder} (\textit{He was
  \textbf{not} a good role model for his children});

\item Indefinite pronouns like \textit{niemand}, \textit{keiner}
  etc., e.g. \textit{\textbf{Niemand} hielt ihn f\"ur einen
    ehrlichen Menschen.} (\textit{\textbf{Nobody} considered him an
    honest man});

\item Any lexical or idiomatic unit in case they turn the sentiment
  polarity to the opposite, e.g. \textit{Ich \textbf{zweifle}, dass
    das neue iPhone ein besseres Display hat.} (\textit{I
    \textbf{doubt} the new iPhone has a better display}).
\end{itemize}

DO NOT mark elements as negations that have no effect on the sentiment
polarity. For example, in the sentence \textit{Ich mag Leute, die nicht
  nur an sich selbst denken.} (\textit{I like people who not only care
  about themselves.}) \textit{nicht} (\textit{not}) should \underline{not} be
marked as negation since it does not change the positive polarity
expressed by \textit{m\"ogen} (\textit{like}).
\vspace{0.5cm}

Negations only have one possible attribute, namely
\textit{sentiment-ref} that is a directed edge pointing from a negation
to the sentiment it belongs to. You should only draw this edge in cases multiple sentiment 
relations overlap and it is not obvious to which of these sentiments the negation belongs to,
i.e. in cases the negation is used for both sentiment spans.

%%%%%%%%%%%%%%%%%%%%%%%%%%%%%%%%%%%%%%%%%%%%%%%%%%%%%%%%%%%%%%%%%%%%%%%%%%%%%%%%%%

\subsection{intensifier}
\texttt{intensifier}s are elements that increase the polar
meaning of emotional expressions. Intensifiers are usually expressed
by adjectives or adverbs like \textit{sehr} (\textit{very}),
\textit{ziemlich} (\textit{rather}) and the like. In intensifier-chains 
\textit{(e.g. sehr, sehr gut)} please annotate each intensifier separately.
As mentioned above you should annotate every single emo-expression you can 
find regardless of a corresponding sentiment span. 
Please do not forget to annotate the corresponding intensifier if there is one. \newline

Intensifiers have the following attributes and values: \newline

\begin{tabular}{|l|c|p{\clmnwidth}|}\hline
  \multirow{2}{*}{degree} & \textit{1 (default)} & intensifier 
  slightly increases polarity of emotional
  expression, e.g. \textit{ziemlich},
  \textit{recht} etc.\\\cline{2-3}

  & \textit{2} & intensifier strongly increases polarity  of
  emotional expression, e.g. \textit{sehr}, \textit{super},
  \textit{stark} etc.\\\hline

  %%%%%%

  sentiment-ref & \textit{$->$\newline(directed edge)} & Directed
  edge pointing from the intensifier to the sentiment it belongs
  to. By analogy to negations, you should only draw this edge in cases
  multiple sentiment relations overlap and it is not obvious
  to which of these sentiments the intensifier belongs to,
  i.e. in cases when the intensifier is used for both
  sentiment spans.\\\hline
\end{tabular}

%%%%%%%%%%%%%%%%%%%%%%%%%%%%%%%%%%%%%%%%%%%%%%%%%%%%%%%%%%%%%%%%%%

\subsection{diminisher}
\texttt{diminisher}s are the counterpart to \texttt{intensifier}s.
These are elements that decrease the polar meaning of emotional
expressions.  Like intensifiers diminishers are usually expressed by
adjectives or adverbs. Typical examples of such adverbs are
\textit{wenig}, \textit{kaum}, \textit{ein bisschen} etc..

Diminishers have the same attributes as intensifiers. The only
difference are the values for the attribute. Instead of positive
values you have negative ones.

Here is a table of diminisher's attributes:

\begin{tabular}{|l|c|p{\clmnwidth}|}\hline

  \multirow{2}{*}{degree} & \textit{weak (default)} & diminisher
  slightly decreases polarity of emotional expression,
  e.g. \textit{wenig}, \textit{bisschen} etc.\\\cline{2-3}

  & \textit{strong} & diminisher strongly decreases polarity of
  emotional expression, e.g. \textit{kaum} etc.\\\hline


  sentiment-ref & \textit{$->$\newline(directed edge)} & Directed
  edge pointing from the diminisher to the sentiment it belongs
  to. You should only draw this edge in cases
  multiple sentiment relations overlap and it is not obvious
  to which of these sentiments the diminisher belongs to,
  i.e. in cases when the diminisher is used for both
  sentiment spans.\\\hline

\end{tabular}

%%%%%%%%%%%%%%%%%%%%%%%%%%%%%%%%%%%%%%%%%%%%%%%%%%%%%%%%%%%%%%%%%%%%%%%%%%%%

\section{MMAX Techniques}
In this section the most important techniques for annotating our corpus in MMAX will be described.

\subsection{Turn on Auto-Save and Auto-apply}
After setting up MMAX load a .mmax file as described in the beginning
of this document. You will be asked to validate the
annotations. Please always do that.  Before you start the annotation
it is recommended to turn on "Auto-Save" and "Auto-apply". Go to
\texttt{File -> Auto-Save -> Every} and select a time value.  Then
change to the window for the attributes (the window in the upper left
corner of your screen) and go to \texttt{Settings} and check the
\texttt{Auto-apply} box.  Your choice will be confirmed by a red
"Auto-apply is ON" at the bottom of the window. \newline

\subsection{Annotate Markables and Attributes}
The window in the middle of your screen now should show the tweets. To annotate a markable choose a word, phrase, clause or sentence by clicking the left mouse-button and mark the respective span while holding the mouse-button. After releasing the left mouse-button a menu will pop up showing the list of markables. Left-click on the corresponding markable and the span will turn to the respective color of the markable. Then change to the attribute window in the upper left corner, click on the tab for the markable you just annotated and set the value(s) for the attributes. (In case you did not turn on "Auto-apply" press "Apply".)

\subsection{Deleting Markables}
Just do a \underline{right-click} on a colored markable and left-click \texttt{Delete this markable}.

\subsection{Annoating Discontinuous Markables}
Sometimes you will find discontinuous emo-expressions. The most prominent example are German particle verbs (e.g. weh tun). To annotate them as one single element first annotate one element (it will turn to the respective markable color) as usual, left-click to highlight it and then mark the other element. A menu will pop up saying "Add to this markable". Left-click on that.

\subsection{Annotating anaphref}
Imaginge you have a sentence like: \textit{Peter ist schlau und gut aussehen tut er auch noch. (Peter is clever and he looks good, too.)}. Here you should annotate \textit{er} as anaphref to \textit{Peter}. To do so first annotate \textit{Peter} and \textit{er} as single targets. Then left-click on \textit{Peter} to highlight the markable and then do a \underline{right-click} on \textit{er}. A menu will pop up saying \texttt{"Mark as anaphric antecendent of target"}. Left-click on that.

\subsection{Annotating sentiment\_ref}
The same technique as above applies. First mark a sentiment span as usual. Second annotate the other markable (whatever it may be) as usual. Left-click on the first markable to highlight it then \underline{right-click} on the second markable and in the pop-up menu left-click on "Mark sentiment which this span belongs to".

%%%%%%%%%%%%%%%%%%%%%%%%%%%%%%%%%%%%%%%%%%%%%%%%%%%%%%%%%%%%%%%%%%%%%%%%%%%%
\section{FAQ}

\underline{Things you SHOULD mark as sentiments:}
\begin{itemize}
  \item Phrases, clauses, sentences and statements expressing a polar opinion
    (e.g. \textit{Der neue Film \"uber Superman war knorke!} -
    \textit{The new superman movie was fantastic});
  \item Interrogative clauses containing a polar opinion
    and do not call this opinion into question but rather ask for its
    reasons or other related aspects (e.g. \textit{Warum findet die
      Mehrheit der Bev\"olkerung CDU so toll?} - \textit{Why does the
      majority of the population consider CDU great?});
  \item Exclamations expressing support or disapproval of
    something (e.g. \textit{Der Sommer ist da. Super!} -
    \textit{Summer has come. Great!}).
  \item Comparisons expressing some preference of one subject/object over another 
      (e.g. \textit{Canon EOS 550d macht bessere Aufnahmen als 600d} - \textit{Canon EOS 550d takes
      better pictures than 600d}).
\end{itemize}


\underline{Things you SHOULD NOT mark as sentiments:}
\begin{itemize}
  \item Statements that do not have an annotatable target:
      (e.g. \textit{F\"uhle mich so traurig :(} - \textit{Feeling so blue :(});
  \item Statements describing some objective facts, even if possible
    consequences of these facts can be assumed to have negative
    influence on the author (e.g. \textit{Wenn ein Floh einen Menschen
      bei\ss{}t und ihn mit erbrochenem Blut infiziert, werden die
      Pestbakterien ins Gewebe \"ubertragen.} - \textit{When a flea
      bites a human and contaminates the wound with regurgitated
      blood, the plague carrying bacteria are passed into the
      tissue.});
  \item Interrogative clauses in case they ask whether some polar
    opinion is true or not. (e.g. \textit{Findet die Mehrheit
      der Bev\"olkerung CDU toll?} - \textit{Does the majority
      of the population consider CDU great?});
\end{itemize}

\underline{Things you SHOULD mark as sources:}
\begin{itemize}
  \item Original author(s) of polar opinions, i.e. persons,
    groups or officials who express sympathy or antipathy for
    something or somebody e.g. \textit{\textbf{Michael} meinte, das w\"are die
      beste L\"osung.}  (\textit{\textbf{Michael} thought, it would be the
      best solution.});
\end{itemize}


\underline{Things you SHOULD NOT mark as sources:}
\begin{itemize}
  \item Persons how are neutrally citing someone else's opinion,
    e.g. in the sentence \textit{Nach Tatjanas Worten war Michael sehr
      dar\"uber ver\"argert.} (\textit{According to Tatjana, Michael
      was very angry about that.}) you should only mark
    \textit{Michael} as source and not \textit{Tatjana}. Please note, that in case an author
    supports or contradicts someone else's opinion, two sentiment
    relations should be made - one with the original author as source
    and one with the supporter/opponent of the opinion as the other
    source.\footnote{In that case sentiments will overlap, and
      you also should mark these sources with \textit{sentiment-ref}-tag.}
\end{itemize}

\underline{Things you SHOULD mark as emo-expressions:}
\begin{itemize}
  \item Adjectives and adverbs bearing polar attitudes \textit{Peter
    hatte \textbf{bessere} Noten in der Schule als sein Bruder}
    (\textit{Peter had \textbf{better} grades at school than his
    brother.});

  \item Verbs expressing attitude of a speaker to target,
    e.g. \textit{Mir \textbf{gefiel} die neue House-Staffel}
    (\textit{I \textbf{liked} the new House series});

  \item Idiomatic expression including support verbs
    e.g. \textit{\textbf{Zum Teufel} soll die neue Regierung
      \textbf{gehen}} (\textit{The new government should \textbf{go to
        hell}}).

  \item Smileys in case they really express an emotional attitude and
    are not used for politeness or without any particular meaning
    e.g. \textit{Gleich in Braunschweig mit Kameraden treffen
        \textbf{:)}} (\textit{Will soon meet friends in Braunschweig
        \textbf{:)}}).
\end{itemize}
%%%%%%%%%%%%%%%%%%%%%%%%%%%%%%%%%%%%%%%%%%%%%%%%%%%%%%%%%%%%%%%%%%%%%%%%
